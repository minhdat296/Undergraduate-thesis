Let us firstly collect certain useful facts about $\ell$-adic sheaves on a suitable nice scheme. In particular, we are interested in how categories of $\ell$-adic local systems are constructed as well as whether or not the category of $\ell$-adic local systems over a suitably reasonable scheme (cf. convention \ref{conv: l_adic_sheaves_conventions}) is compatible with Grothendieck's Six Functor Formalism. Our main references are \cite{conrad_etale_cohomology}, \cite[Chapter 1]{behrend_l_adic_sheaves_for_algebraic_stacks}, and \cite{laszlo_olsson_adic_sheaves_on_artin_stacks_2}.
        
        \begin{convention}[Homological conventions]
            If $\calA$ is an additive (respectively abelian) category then we shall write $\Ho(\calA)$ (respectively $\rmD(\calA)$) for the homotopy (respectively derived) category of $\Ch(\calA)$, the category of chain complexes in $\calA$; both shall be assumed to come equipped with its canonical triangulated structure that arises via mapping cones. Additionally, we shall be grading \textit{cohomologically}\footnote{\cite[Section 1.4]{conrad_etale_cohomology}, on the other hand, grades \textit{homologically}.}.
        \end{convention}
        
        We begin with the notion of \textbf{Artin-Rees categories} (cf. definition \ref{def: artin_rees_categories}), which will help us define $\ell$-adic sheaves in very convenient terms. 
        \begin{definition}[Artin-Rees categories] \label{def: artin_rees_categories}
            The \textbf{Artin-Rees category} associated to an abelian category $\calA$ is the full subcategory of $\Pro(\calA)$ spanned by cofiltered diagrams $\{M_n\}_{n \in \Z}$; we denote it by $\AR(\calA)$. Of particular interest are the so-called \textbf{null systems}, which are objects $\{M_n\}_{n \in \Z} \in \AR(\calA)$ such that there exists $\nu \in \N$ so that for all $n \in \Z$ the morphism $M_n \to M_{n + \nu}$ is zero.
        \end{definition}
        \begin{proposition}[Artin-Rees categories are abelian] \label{prop: artin_rees_categories_are_abelian}
            For any abelian category $\calA$, the associated Artin-Rees category $\AR(\calA)$ is also abelian, with zero objects being the null systems.
        \end{proposition}
        \begin{corollary}[AR-isomorphisms] \label{coro: AR_isomorphisms}
            For any abelian category $\calA$, an isomorphism in $\AR(\calA)$ (henceforth referred to as an \textbf{AR-isomorphism}) is a morphism in $\Pro(\calA)$ whose kernel and cokernel are null.
        \end{corollary}
        \begin{proposition}[Artin-Rees categories are linear] \label{prop: artin_rees_categories_are_linear}
            If a given abelian category $\calA$ is $\Lambda$-linear (i.e. if hom-sets of $\calA$ are $\Lambda$-modules) for some commutative ring $\Lambda$ (e.g. $\calA \cong \Lambda\mod$) then $\AR(\calA)$ will also be $\Lambda$-linear.
        \end{proposition}
        \begin{convention} \label{conv: complete_artinian_ring}
            For now, suppose that $\Lambda$ is an Artinian commutative ring (hence Noetherian \textit{a priori}) that is adically separated and complete with respect to some ideal $\m$ therein (i.e. $\Lambda$ is its own $\m$-adic completion and $\bigcap_{n \in \N} \m^n = 0$). Note that we need the Noetherian hypothesis on $\Lambda$ in order to be able to endow it with the $\m$-adic topology. In addition, let us write $M_n \cong M/\m^n$ for all $\Lambda$-modules $M$.
        \end{convention}
        \begin{remark}[Adic projective systems] \label{remark: adic_projective_systems}
            Observe that for every finite $\Lambda$-module $M$, the corresponding cofiltered diagram of finite $\Lambda_n$-modules $\{M_n\}_{n \leq 0}$ enjoys an interesting property, which is that the transition maps $M_{n + 1} \tensor_{\Lambda_{n + 1}} \Lambda_n \to M_n$ are actually isomorphisms of (finite) $\Lambda_n$-modules. Any such cofiltered diagram of $\Lambda_n$-modules is said to be \textbf{strictly $\m$-adic} (cf. \cite[Definition 1.4.1.1]{conrad_etale_cohomology}). Now, because $\Lambda$ is $\m$-adically complete and separated, it can be shown that the aforementioned process of associating a strictly $\m$-adic projective system to a given finite $\Lambda$-module induces an equivalence between the category $\Lambda\mod^{\fin}$ of finite $\Lambda$-modules and the category ${}^{\leq 0}(\Lambda, \m)\mod^{\fin, \ad}$ of strictly $\m$-adic projective systems of finite $\Lambda_n$-modules: in one direction, the functor is $M \mapsto \{M_n\}_{n \leq 0}$ and in the converse direction, the functor is $\{M_n\}_{n \leq 0} \mapsto \underset{n \leq 0}{\lim} M_n$ (cf. \cite[\href{https://stacks.math.columbia.edu/tag/031D}{Tag 031D}]{stacks}). In doing so, we have endowed ${}^{\leq 0}(\Lambda, \m)\mod^{\fin, \ad}$ with the structure of an abelian category (namely that of $\Lambda\mod^{\fin}$), and so we might consider its associated Artin-Rees category.
        \end{remark}
        Let us now take a short detour and discuss the notion of scalar extension in the context of general linear abelian categories, wherein one might not have access to a notion of tensor products. This is a direct generalisation of the same notion for categories of modules over rings. 
        \begin{remark}[Extension of scalars for linear abelian categories] \label{remark: extension_of_scalar_for_linear_abelian_categories}
            For any commutative ring $A$, an $A$-linear category $\calA$ is enriched over $A\mod$ by definition, so while one might not have a good notion of tensor products of objects in $\calA$ (i.e. $\calA$ might not be monoidal), one always has tensor products at the level of hom-sets, as these are just $A$-modules. If $B$ is a commutative and unital $A$-algebra, then one can define the extension of scalars of $\calA$, which we denote by $\calA \tensor_A B$, to be the category whose hom-sets are the $B$-modules of the form $\calA(M, N) \tensor_A B$ (for pairs of objects $M, N \in \calA$). Now, the question is that should $\calA$ be abelian in addition, would $\calA \tensor_A B$ also be abelian ? 
        \end{remark}
        \begin{definition}[Torsion and adic objects] \label{def: torsion_objects_and_adic_objects}
            From any $\Lambda$-linear abelian category $\calA$, one can construct an cofiltered diagram of $\Lambda_n$-linear categories $\{- \tensor_{\Lambda_{n + 1}} \Lambda_n: \calA_{n + 1} \to \calA_n\}_{n \in \N}$ which we call subcategories of \textbf{$\m^n$-torsion objects} (or the \textbf{$\m^n$-torsion subcategories} for short).
        \end{definition}
        \begin{proposition}[Artin-Rees categories of adic projective systems] \label{prop: artin_rees_categories_of_adic_projective_systems}
            Let $\calA$ be a $\Lambda$-linear abelian category and let $\calA^{\fin}$ be a $\Lambda$-linear finite abelian subcategory of $\calA$ (i.e. one whose objects are all simultaneously Noetherian and Artinian\footnote{$\Lambda\mod^{\fin}$ is one such subcategories of $\Lambda\mod$, as a module over an Artinian ring is finitely generated if and only if it is Noetherian, which itself is the case if and only if said module is Artinian (cf. \cite[Theorem 4.15]{lam_first_course_in_noncommutative_rings}).}). Then, (cf. \cite[Proposition 2.2.4]{behrend_l_adic_sheaves_for_algebraic_stacks})
        \end{proposition}
        
        \begin{convention}[The setting for adic sheaves] \label{conv: l_adic_sheaves_conventions}
            \noindent
            \begin{itemize}
                \item For our purposes, $\calX$ shall be a scheme that is locally of finite type\footnote{Althought $\calX$ might actually be an algebraic stack of finite type over $S$ (for details, see \cite{laszlo_olsson_adic_sheaves_on_artin_stacks_1} and \cite{laszlo_olsson_adic_sheaves_on_artin_stacks_2}).} over a base scheme $S$, which we take to be is affine, regular, Noetherian and of dimension $\leq 1$, and of characteristic $p \geq 0$ (e.g. $\calX \cong \Bun_G(X)$, with $X$ as in convention \ref{conv: automorphic_side_conventions} and $G$ some connected reductive group); moreoever, we would like to work under the assumption that every finite-type $S$-scheme $T$ is also of finite cohomological dimension. At the same time, $\Lambda$ shall be a Gorenstein local ring of dimension $0$ with maximal ideal $\m$ and residue  characteristic $\ell \not = p$ (e.g. $\Lambda \cong \Z_{\ell}$). This is the same as in \cite{laszlo_olsson_adic_sheaves_on_artin_stacks_1}.
                \item For $\calX$ and $\Lambda$ as above, write $\rmD_{\Lambda, c}(\calX_{\lisse\-\et})$ for the \textit{unbounded} derived category of \href{https://stacks.math.columbia.edu/tag/03RW}{\underline{constructible}} sheaves of $\Lambda$-modules on the lisse-\'etale site of $\calX$, which is implicitly equipped with its natural t-structure.
            \end{itemize}
        \end{convention}
        \begin{remark}[About the assumptions on $\Lambda$]
            It should be noted that we have required $\Lambda$ to be Noetherian because otherwise, we can not guarantee that $\Shv_{\Lambda}^c(\calX_{\lisse\-\et})$ would be a thick subcategory of the abelian category $\Lambda\mod(\calX_{\lisse\-\et})$ of sheaves of $\Lambda$-modules (the relevance of this fact will become clear shortly). In addition, we require that $\Lambda$ is Gorenstein so that as a Noetherian ring, it would admit a dualising complex (in fact, a Noetherian ring has a dualising complex if and only if it is a quotient of a finite-dimensional Gorenstein ring; cf. \cite[Corollary 1.4]{kawasaki_macaulayfication_of_noetherian_rings}). Lastly, $\dim \Lambda = 0$ because otherwise, we would have to keep track of cohomological shifts (cf. \cite[\href{https://stacks.math.columbia.edu/tag/0AWS}{Tag 0AWS} and \href{https://stacks.math.columbia.edu/tag/0B5A}{Tag 0B5A}]{stacks}), which are inessential technicalities.
        \end{remark}
        
        \begin{convention}[The base scheme] \label{conv: base_schemes_for_adic_sheaves}
            
        \end{convention}
        \begin{definition}[$\m$-adic sheaves] \label{def: m_adic_sheaves}
            
        \end{definition}
        
        \begin{definition}[$\ell$-adic sheaves] \label{def: l_adic_sheaves}
            
        \end{definition}
        
        \begin{lemma}
            \cite[\href{https://stacks.math.columbia.edu/tag/0GIY}{Tag 0GIY}]{stacks} Fix a geometric point $\bar{x}$ of a connected and normal Noetherian scheme $X$. Then, there exists a canonical equivalence of categories $\Shv_{\underline{\Lambda}}^{\fin}(X) \cong \Rep_{\Lambda}^{\fin}(\pi_1(X, \bar{x}))^{\cont}$ given by $\calL \mapsto \calL_{\bar{x}}$.
        \end{lemma}
        \begin{theorem}[Galois representations are lisse adic sheaves] \label{theorem: galois_representations_are_local_systems}
            Fix a geometric point $\bar{x}$ of a connected and normal Noetherian scheme $X$. Then, there is a canonical equivalence $\Shv_{\underline{\bar{\Q}_{\ell}}}^{\ad}(X) \cong \Rep_{\bar{\Q}_{\ell}}^{\fin}(\pi_1(X_{\fet}, \bar{x}))^{\cont}$ given by $\calL \mapsto \calL_{\bar{x}}$, which happens to also be a monoidal functor.
        \end{theorem}
            \begin{proof}
                
            \end{proof}
        \begin{lemma}[Abelianising continuous characters] \label{lemma: abelianising_continuous_characters}
            Let $G$ be a topological group and $E$ a topological field. Then, there is a group isomorphism $\Rep^1_E(G)^{\cont} \cong \Rep^1_E(G^{\ab})^{\cont}$ given by $\chi \mapsto \chi^{\ab}$, with $\chi^{\ab}$ being the composition of $\chi$ with the canonical quotient map $G \to G^{\ab}$.
        \end{lemma}
            \begin{proof}
                Let us show that the homomorphism $\chi \to \chi^{\ab}$ is surjective. For this, it shall suffice to show that the group $\Rep^1_E([G, G])^{\cont}$ is trivial: but this is evident from the fact that $\GL_1(E)$ is abelian and from the definition of the commutator subgroup $[G, G]$, namely that $[G, G] := \<ghg^{-1}h^{-1} \mid \forall g, h \in G\>$ (so for all $x \in [G, G]$ and all $\chi \in \Rep^1_E([G, G])$, $\chi(x) = 1$), so we are done.
            \end{proof}
        \begin{corollary} \label{coro: abelian_galois_characters_are_local_systems}
            Fix a geometric point $\bar{x}$ of a connected Noetherian scheme $X$. Then, on obtains an equivalence $\Shv_{\underline{\bar{\Q}_{\ell}}}^{\ad, 1}(X) \cong \Rep^1_{\bar{\Q}_{\ell}}(\pi_1^{\ab}(X_{\fet}))^{\cont}$ via $\calL \mapsto \calL_{\bar{x}}$ by combining theorem \ref{theorem: galois_representations_are_local_systems} and lemma \ref{lemma: abelianising_continuous_characters}.
        \end{corollary}