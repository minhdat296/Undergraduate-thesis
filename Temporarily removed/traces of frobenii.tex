\begin{definition}[Traces of Frobenii] \label{def: traces_of_frobenii}
            Let $Z$ be a scheme of finite type over $\Spec k$. For each geometric point $\bar{z} \in Z(\bar{\F}_q)$, we define the trace of Frobenius on an $\ell$-adic local system $\calF \in \Shv_{\underline{\bar{\Q}_{\ell}}}^{\ad, 1}(Z)$ at $\bar{z}$ to be the trace of the Frobenius endomorphism on the stalk $\calF_{\bar{z}}$, i.e. $\trace(\Frob_{\bar{z}}^{\flat}, \calF_{\bar{z}})$. Since $\calF$ is an $\ell$-adic local system of rank $1$, one obtains a function as follows in doing so:
                $$\frob(-, \calF): Z(\bar{\F}_q) \to \bar{\Q}_{\ell}$$
        \end{definition}
        \begin{remark}[Traces of Frobenii as functions on rational points] \label{remark: traces_of_frobenii_as_functions_on_rational_points}
            Let us first remark that we do not define functions associated to $\ell$-adic local systems $\calF \in \Shv_{\underline{\bar{\Q}_{\ell}}}^{\ad, 1}(Z)$ over the set $Z(\bar{\F}_q)$ instead of $Z(\F_q)$ out of necessity, not convenience, and this is because stalks of \'etale sheaves can only be computed over geometric points (cf. \cite[\href{https://stacks.math.columbia.edu/tag/03PN}{Tag 03PN}]{stacks}). But we do want functions over $Z(\F_q)$ as well, and for this, recall that $Z(\F_q) \cong Z(\bar{\F}_q)^{\Gal(\bar{\F}_q/\F_q)}$. Defining a function:
                $$\frob(-, \calF): Z(\F_q) \to \bar{\Q}_{\ell}$$
            coming from the trace of the Frobenius endomorphism on $\calF$ then becomes a simple matter of domain restriction.
        \end{remark}
        \begin{remark}[Basic properties of traces of Frobenii] \label{remark: basic_properties_of_traces_of_frobenii}
            The following properties are trivial consequences of definition \ref{def: traces_of_frobenii}:
                \begin{itemize}
                    \item For any pair of rank-$1$ $\ell$-adic local systems $\calF, \calF' \in \Shv_{\underline{\bar{\Q}_{\ell}}}^{\ad, 1}(Z)$ and any fixed geometric point $\bar{z} \in Z(\bar{\F}_q)$, one has:
                        $$\frob(\bar{z}, \calF \tensor \calF') = \frob(\bar{z}, \calF) \frob(\bar{z}, \calF')$$
                    \item Let $f: Y \to Z$ be a morphism between $\F_q$-schemes that are locally of finite type and let $\calF$ be an $\ell$-adic local system of rank $1$ on $Z$. Then the following diagram commutes\footnote{To prove that the diagram indeed commutes, recall also the basic sheaf-theoretic fact that for any geometric point $\bar{y} \in Y(\bar{\F}_q)$ over a fixed geometric point $\bar{z} \in Z(\bar{F}_q)$ (i.e. such that $f_{\bar{\F}_q}(\bar{y}) = \bar{z}$), one has an isomorphism $(f^*\calF)_{f_{\bar{\F}_q}(\bar{y})} \cong \calF_{\bar{z}}$ of stalks.}:
                        $$
                            \begin{tikzcd}
                            	{Y(\bar{\F}_q)} && {Z(\bar{\F}_q)} \\
                            	& {\bar{\Q}_{\ell}}
                            	\arrow["{\frob(-, \calF)}", from=1-3, to=2-2]
                            	\arrow["{\frob(-, f^*\calF)}"', from=1-1, to=2-2]
                            	\arrow["{f_{\bar{\F}_q}}", from=1-1, to=1-3]
                            \end{tikzcd}
                        $$
                \end{itemize}
        \end{remark}
        
        \begin{lemma}[The Lang isogeny] \label{lemma: lang_isogeny}
            If $G$ is a connected and smooth algebraic group over $\Spec \F_q$ then there will exist a surjective topological group homomorphism $q_G: \pi_1(G_{\fet}) \to G(\F_q)$.
        \end{lemma}
            \begin{proof}
                The strategy is to make use of corollary \ref{remark: geometric_galois_correspondence}, since $G(\F_q)$ is finite and therefore corresponds uniquely to the quotient of $\pi_1(G_{\fet})$ by a normal subgroup of finite index. For that, we shall need to construct a Galois covering of $G$ with which to compute $\pi_1(G_{\fet})$; this turns out to be the \textbf{Lang isogeny} $l_G: G \to G$, which is defined by $l_G(g) := \Frob_G(g)g^{-1}$. For more details, see \cite[Theorem 4.4.17]{springer_linear_algebraic_groups}.
            \end{proof}
        \begin{proposition}
            For each $\xi \in \Rep^1_{\bar{\Q}_{\ell}}(G(\F_q))^{\cont}$ there exists $\tilde{\xi} \in \Rep^1_{\bar{\Q}_{\ell}}(\pi_1(G_{\fet}))^{\cont}$ such that $\tilde{\xi} := \xi \circ q_G$ (with $q_G$ as in lemma \ref{lemma: lang_isogeny}), which corresponds to some $\calF^{\xi} \in \Shv_{\underline{\bar{\Q}_{\ell}}}^{\ad, 1}(G)$ by theorem \ref{theorem: galois_representations_are_local_systems}. We claim that $\frob(-, \calF^{\xi}) = \xi$ as continuous $\ell$-adic characters of $G(\F_q)$.
        \end{proposition}
            \begin{proof}
                
            \end{proof}
            
    \begin{convention} \label{conv: derived_categories_of_constructible_sheaves}
                If $\calY$ is a scheme and $\Lambda$ is a Noetherian ring then we write $\rmD_{\Lambda, \cons}(\calY_{\et})$ for the \textit{unbounded} derived category of \href{https://stacks.math.columbia.edu/tag/03RW}{\underline{constructible}} sheaves of $\Lambda$-modules on the \'etale site of $\calY$, which is implicitly equipped with its natural t-structure. This category has a natural triangulated full subcategory spanned by objects that are locally of finite $\Tor$-dimension; we denote this subcategory by $\rmD_{\Lambda, \cons}^{\tor\fin}(\calY_{\et})$
            \end{convention}
            \begin{remark}[$\Tor$-finiteness for constructible sheaves] \label{remark: tor_finiteness_for_constructible_sheaves}
                
            \end{remark}
            \begin{convention} \label{conv: trace_formula_conventions}
                For our purposes, $\calY$ shall be a scheme that is locally of finite type\footnote{Althought $\calY$ might actually be an algebraic stack of finite type over a base scheme $S$ that is affine, regular, Noetherian and of dimension $\leq 1$, and of characteristic $p \geq 0$ (for details, see \cite{laszlo_olsson_adic_sheaves_on_artin_stacks_1} and \cite{laszlo_olsson_adic_sheaves_on_artin_stacks_2}), e.g. $\calY \cong \Bun_G(X)$, with $X$ as in convention \ref{conv: automorphic_side_conventions} and $G$ some connected reductive group (such as $\GL_1$).} and stratifies into countably many connected components, each of which is of finite type over $\Spec \F_q$. Moreoever, we would like to work under the assumption that every finite-type $\F_q$-scheme $T$ is also of finite cohomological dimension. At the same time, $\Lambda$ shall be a Gorenstein local ring of dimension $0$ with maximal ideal $\m$ and residue  characteristic $\ell \not = p$ (e.g. $\Lambda \cong \Z_{\ell}$); this is the same as in \cite{laszlo_olsson_adic_sheaves_on_artin_stacks_1}.
            \end{convention}
            
            \begin{lemma}[The Grothendieck-Lefschetz trace formula] \label{lemma: the_trace_formula}
                \cite[\href{https://stacks.math.columbia.edu/tag/03UG}{Tag 03UG}]{stacks} Let $\calY$ as in convention \ref{conv: trace_formula_conventions} be separated also, and let $\Lambda$ as in convention \ref{conv: trace_formula_conventions} be finite. Then, the following trace formula holds for any $\calF \in \rmD_{\Lambda, \cons}^{\tor\fin}(\calY_{\et})$:
                    $$\trace(\Frob_{\calY}^* \mid \calF) = \sum_{y \in \calY(\F_q)} \frob^{\calF}(\bar{y})$$
            \end{lemma}
            
            
    