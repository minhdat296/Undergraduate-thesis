\begin{remark}[Groupoid structure on character groups] \label{remark: groupoids_of_characters}
            Let $G$ be a topological group and $E$ an algebraically closed topological field. Observe that because characters - by being identically $1$-dimensional - are necessarily irreducible as linear representations, one gets via Schur's Lemma (cf. \cite[Lemma 3.6, pp. 35]{lam_first_course_in_noncommutative_rings}) that all discrete $E$-characters of $G$ are isomorphic. Additionally, since $E$ is an algebraically closed field, we get through another application of Schur's Lemma (or rather, the fact that invertible matrices over an algebraically closed field are diagonal) that the isomorphism between any pair of discrete $\ell$-adic characters $\chi_1, \chi_2 \in \Rep^1_E(G)$ are group homomorphisms $\varphi_{\lambda}: \GL_1(E) \to \GL_1(E)$ (for some $\lambda \in E^{\x}$) given by:
                $$\forall g \in G: \forall v \in \GL_1(E): \chi_2(g)(v) = (\chi_1 \circ \varphi_{\lambda})(g)(v) = \lambda \cdot \chi_1(g)(v)$$
            Evidently, if $\chi_1(g)$ is continuous for all $g \in G$ then so is $\lambda \cdot \chi_1(g)$. Thus, all the continuous $E$-characters of $G$ are also isomorphic or in other words, $\Rep^1_E(G)^{\cont}$ is a groupoid. Furthermore, $\Rep^1_E(G)^{\cont}$ has an underlying group structure, namely that of the group whose elements are continuous $E$-characters $\chi: G \to \GL_1(E)$, on which the group structure is pointwise multiplication (note that this is compatible with the previous interpretation of $\Rep^1_E(G)^{\cont}$ as a groupoid, since $\chi(g) \in F^{\x}$ for all $g \in G$ and all $\chi \in \Rep^1_E(G)^{\cont}$, and hence the characters do in fact differ by non-zero scalar multiples).
        \end{remark}
        
Furthermore, if $\pi$ is a profinite group\footnote{Which need not be strictly profinite; e.g. $\Gal(\Q^{\alg}/\Q)$ will have many finite-index normal subgroups which are not open if one assumes the Axiom of Choice (cf. \cite[Proposition 7.26]{milne_field_theory}).} such that $\calG \cong \pi\-\Fin$ then $\pi_1(\calG, F) \cong \pi$.

\begin{remark}[The \'etale fundamental group satisfies descent] \label{remark: the_etale_fundamental_group_satisfies_descent}
            In proposition \ref{prop: etale_eckmann_hilton_duality}, we demonstrated that for any connected qcqs base scheme $X$ and for all profinite groups $G$, there is a natural bijection:
                $$\Grp(\Pro\Fin)(\pi_1(X_{\fet}), G) \cong \Sch_{/\Spec k}(X, \underline{G})$$
            From this and form the fact that schemes are representable \'etale sheaves, one infers that for all profinite groups $G$, the functor:
                $$\Grp(\Pro\Fin)(\pi_1(-), G): (\Sch_{/X})_{\fet}^{\op} \to \Grp(\Pro\Fin)$$
            satisfies \'etale descent. As a consequence, an appropriate choice of $G$ (e.g. we shall be concerned with $G \cong \GL_n(\Z_{\ell})$) can help us compute $\pi_1(X_{\fet})$ using descent-theoretic techniques. In particular, should $\{Y_i \to X\}_{i \in \calI}$ be a Galois cover of $X$, then:
                $$\Grp(\Pro\Fin)(\pi_1(X_{\fet}), G) \cong \underset{i \in I}{\lim} \Grp(\Pro\Fin)(\pi_1((Y_i)_{\fet}), G)$$
        \end{remark}
        
        If $f: Y \to X$ is a \href{https://stacks.math.columbia.edu/tag/04DC}{\underline{universal homeomorphism}} between connected schemes and if $\bar{y} \in Y$ is a geometric point lying over a fixed geometric point $\bar{x} \in X$, then not only is the base change functor:
                    $$(\Sch_{/X})_{\fet} \to (\Sch_{/Y})_{\fet}$$
                    $$X' \mapsto X' \x_X Y$$
                an equivalence of Galois categories, but also, one has an isomorphism of \'etale fundamental groups $\pi_1(X_{\fet}, \bar{x}) \cong \pi_1(Y_{\fet}, \bar{y})$. 
                
        \begin{lemma} \label{lemma: base_change_of_thickenings}
            If $X \subset X'$ be a \href{https://stacks.math.columbia.edu/tag/04EW}{\underline{thickening}} of schemes. Then, the following base change functor is an equivalence of Galois categories:
                $$(\Sch_{/X'})_{\fet} \to (\Sch_{/X})_{\fet}$$
                $$T \mapsto T \x_{X'} X$$
        \end{lemma}
            \begin{proof}
                
            \end{proof}
            
    We begin by checking whether or not \'etale fundamental group is dual - in some sense - to the construction of an Eilenberg-MacLane space, thereby having concrete connections to the $1^{st}$ \'etale cohomology group and in turn, admitting descriptions in terms of torsors.
        \begin{remark}[Points of the moduli stack of $G$-torsors]
            For proposition \ref{prop: etale_eckmann_hilton_duality}, a basic fact one should keep in mind is that should $G$ be a constant group\footnote{As opposed to say, an algebraic group or more general group schemes}, then the groupoid $\Bun_{\underline{G}}(X) := \Sch_{/\Spec k}(X, \underline{G})$ of \'etale $\underline{G}$-torsors\footnote{Here, $\underline{G}$ denotes the group scheme represented by $\coprod_{g \in G} \Spec k$. Note how it is \'etale over $\Spec k$.} on a scheme $X$ over a field $k$ is equivalent to the groupoid $\Bun_{\underline{G}}(\Spec k)(X) := \Sch_{/\Spec k}(X, \underline{G(k)})$ of $X$-points of the moduli stack of $\underline{G}$-torsors on $\Spec k$.
        \end{remark}
        \begin{proposition}[The \'etale Eckmann-Hilton Duality] \label{prop: etale_eckmann_hilton_duality}
            For any smooth projective connected curve $X$\footnote{Actually, this proposition holds also when $X$ is any irreducible geometrically unibranch scheme (which can be thought of as analogues of path-connected spaces), but we are not interested in such generalities.} over a field $k$ and any constant profinite group $G$, there exists the following adjunction:
                $$
                    \begin{tikzcd}
                    	{\Grp(\Fin)} & {(\Sch_{/X})_{\fet}}
                    	\arrow[""{name=0, anchor=center, inner sep=0}, "{\Sch_{/\Spec k}(X, -)}"', bend right, from=1-1, to=1-2]
                    	\arrow[""{name=1, anchor=center, inner sep=0}, "{\pi_1^{\fet}}"', bend right, from=1-2, to=1-1]
                    	\arrow["\dashv"{anchor=center, rotate=-90}, draw=none, from=1, to=0]
                    \end{tikzcd}
                $$
            and in addition, a canonical equivalence:
                $$\Grp(\Pro\Fin)(\pi_1(X_{\fet}), G) \cong \Bun_{\underline{G}}(X)$$
            between the groupoid of continuous homomorphisms $\pi_1(X_{\fet}) \to G$ of profinite groups and that of $\underline{G}$-torsors on $X$.
        \end{proposition}
            \begin{proof}
                Each continuous homomorphism $\pi_1(X_{\fet}) \to G$ determines a unique $G$-torsor in $\pi_1(X_{\fet})\-\Pro\Fin$. Because there is an equivalence $\pi_1(X_{\fet})\-\Pro\Fin \cong (\Sch_{/X})_{\profet}$ (cf. proposition \ref{prop: categorical_galois_correspondence}) and because schemes are representable \'etale sheaves, each such $G$-torsor in $\pi_1(X_{\fet})\-\Pro\Fin$ corresponds to a unique $\underline{G}$-torsor on $X$. Such a $\underline{G}$-torsor on $X$, in turn, is an $X$-point of the classifying stack $\Bun_{\underline{G}}((\Spec k))$, i.e. a morphism $X \to \underline{G}$ of $k$-schemes, and the proposition follows suite.
            \end{proof}
        \begin{convention}
            From now on, if $E$ is a non-archimedean normed field then its subring of power-bounded elements shall be denoted by $\scrO_E$. 
        \end{convention}
        \begin{corollary}[Continuous representations of the \'etale fundamental group are torsors] \label{coro: continuous_representations_of_the_etale_fundamental_group_are_torsors}
            Let $\ell$ be a prime, let $E$ be an $\ell$-adic number field (i.e. a finite extension of $\Q_{\ell}$). For any smooth projective connected curve over a field $k$ along with any choice of discrete group $G$, there exists a canonical equivalence:
                $$\Rep^n_{\scrO_E}(\pi_1(X_{\fet}))^{\cont} \cong \Bun_{\underline{\GL_n(\scrO_E)}}(X)$$
            between the groupoid of continuous $n$-dimensional $\scrO_E$-linear representations of $\pi_1(X_{\fet})$ and that of $\GL_n(\scrO_E)$-torsors on $X$.
        \end{corollary}
        \begin{remark}[What about higher-dimensional representations ?]
            Corollary \ref{coro: continuous_representations_of_the_etale_fundamental_group_are_torsors} does not hold for $\GL_n(E)$ (for any $n \geq 1$), as these groups are only locally profinite, as opposed to being globally profinite like $\GL_n(\scrO_E)$.
        \end{remark}
    
        \begin{convention}[The Curve] \label{conv: base_curve}
            Henceforth, $X$ shall be a smooth projective \textit{connected} curve over $\Spec k$ (with $k$ some field).
        \end{convention}
        
        \begin{proposition}[$\ell$-adic representations are $\ell$-adic torsors] \label{prop: E_representations_are_E_local_systems}
            Let $\ell$ be a prime and $E$ be an $\ell$-adic number field. Then there exists a canonical equivalence of groupoids as follows, for all $n \geq 1$:
                $$\Rep_E^n(\pi_1(X_{\fet}))^{\cont} \cong \Bun_{\underline{\GL_n(E)}}(X)$$
        \end{proposition}
            \begin{proof}
                Let $\rho: \pi_1(X_{\fet}) \to \GL_n(E)$ be a continuous representation. $\GL_n(\scrO_E)$ is an open subgroup of $\GL_n(E)$, so the preimage $H := \rho^{-1}(\GL_n(\scrO_E))$ must be an open subgroup of $\pi_1(X_{\fet})$; since $\pi_1(X_{\fet})$ is profinite, $H$ is furthermore normal. Consequently, there exists a Galois $X$-scheme $X^H$ such that $\pi_1(X^H_{\fet}) \cong H$; as a result, we obtain the following pullback of topological groups:
                    $$
                        \begin{tikzcd}
                        	{\pi_1(X^H_{\fet})} & {\GL_n(\scrO_E)} \\
                        	{\pi_1(X_{\fet})} & {\GL_n(E)}
                        	\arrow[hook, from=1-2, to=2-2]
                        	\arrow["\rho", from=2-1, to=2-2]
                        	\arrow[from=1-1, to=2-1]
                        	\arrow["{(\rho^H)^{\circ}}", from=1-1, to=1-2]
                        	\arrow["\lrcorner"{anchor=center, pos=0.125}, draw=none, from=1-1, to=2-2]
                        \end{tikzcd}
                    $$
                Now, due to corollary \ref{coro: continuous_representations_of_the_etale_fundamental_group_are_torsors}, each representation $(\rho^H)^{\circ}: \pi_1(X^H_{\fet}) \to \GL_n(\scrO_E)$ corresponds to a unique $\underline{\GL_n(\scrO_E)}$-torsor on $X^H$, and since $\Bun_{\underline{\GL_n(\scrO_E)}}$ satisfies (profinite-)\'etale descent, one thus obtains in addition a unique $\underline{\GL_n(\scrO_E)}$-torsor on $X$, i.e. a representation $\rho^{\circ}: \pi_1(X_{\fet}) \to \GL_n(\scrO_E)$. There is thus a fully faithful embedding of $\Rep_E^n(\pi_1(X_{\fet}))^{\cont}$ into $\Bun_{\underline{\GL_n(E)}}(X)$. Now, to show that this embedding is also essentially surjective, observer that because $(\Sch_{/X})_{\profet} \cong \pi_1(X_{\fet})\-\Pro\Fin$, each $\underline{\GL_n(E)}$-torsor on $X$ corresponds to a unique (continuous) $\GL_n(E)$-torsor in $\pi_1(X_{\fet})\-\Pro\Fin$. But such a torsor is nothing but a continuous representation $\pi_1(X_{\fet}) \to \GL_n(E)$, so we are done.
            \end{proof}
        \begin{corollary}[Representations of the \'etale fundamental group are local systems] \label{coro: representations_of_the_etale_fundamental_group}
            Let $\ell$ be a prime and $E$ be an $\ell$-adic number field. In addition, fix a geometric point $\bar{x} \in X$. Then there exists a canonical equivalence as follows, for all $n \geq 1$:
                $$\Shv^n_{\underline{E}}(X) \cong \Rep_E^n(\pi_1(X_{\fet}))^{\cont}$$
                $$\calL \mapsto \calL_{\bar{x}}$$
        \end{corollary}
            
        The following result crucially exploits the fact that $\GL_1(E)$ is abelian, unlike $\GL_n(E)$ for $n \geq 2$.
        \begin{lemma}[Abelianising continuous characters] \label{lemma: abelianising_continuous_characters}
            Let $G$ be a topological group and $E$ a topological field. Then, there is a group isomorphism:
                $$\Rep^1_E(G)^{\cont} \cong \Rep^1_E(G^{\ab})^{\cont}$$
        \end{lemma}
            \begin{proof}
                There is a natural injective group homomorphism:
                    $$\Rep^1_E(G)^{\cont} \to \Rep^1_E(G^{\ab})^{\cont}$$
                    $$\chi \mapsto \chi^{\ab}$$
                coming from the canonical quotient map $G \to G^{\ab}$, so the only thing to do is to show that this homomorphism is surjective. For this, it shall suffice to show that the group $\Rep^1_E([G, G])^{\cont}$ is trivial: but this is evident from the fact that $\GL_1(E)$ is abelian and from the definition of the commutator subgroup $[G, G]$, namely that $[G, G] := \<ghg^{-1}h^{-1} \mid \forall g, h \in G\>$ (so for all $x \in [G, G]$ and all $\chi \in \Rep^1_E([G, G])$, $\chi(x) = 1$), so we are done.
            \end{proof}
            
    \begin{proposition}[Gluing adic sheaves along simplicial covers]
            Consider $(\scrX_{\bullet}, \calO_{\scrX_{\bullet}})$ a ringed simplicial topos along with an ordinary (i.e. $0$-coconnective) ringed topos $(X_{\lisse\-\et}, \calO_X)$ and a flat geometric morphism:
                $$
                    \begin{tikzcd}
                    	{(\scrX_{\bullet}, \calO_{\scrX_{\bullet}})} & {(X, \calO_X)}
                    	\arrow[""{name=0, anchor=center, inner sep=0}, "{\e_*}"', bend right, from=1-1, to=1-2]
                    	\arrow[""{name=1, anchor=center, inner sep=0}, "{\e^*}"', bend right, from=1-2, to=1-1]
                    	\arrow["\dashv"{anchor=center, rotate=-90}, draw=none, from=1, to=0]
                    \end{tikzcd}
                $$
            If $\S$ is a thick subcategory of the triangulated category $\calO_X\mod$ and if $\S^{\bullet}$ is the essential image of $\L\e^*: \calO_X\mod \to \calO_{\scrX_{\bullet}}\mod$, then not only is $\S^{\bullet}$ a thick subcategory of $\calO_{\scrX_{\bullet}}\mod$ but also, the pair $\L\e^* \ladjoint \R\e_*$ is an adjoint equivalence between $\S^{\bullet}$ and $\S$.
        \end{proposition}
            \begin{proof}
                \cite[Lemma 2.2.2 and Theorem 2.2.3]{laszlo_olsson_adic_sheaves_on_artin_stacks_1}.
            \end{proof}
        \begin{example}[Gluing constructible sheaves] \label{example: gluing_constructible_sheaves}
            Consider a smooth (hence flat) simplicial cover $U_{\bullet} \to \calX$ of the algebraic stack $\calX$ from convention \ref{conv: l_adic_sheaves_conventions} by simplicial algebraic spaces and assume furthermore that $\Lambda$ is Artinian (e..g $\Lambda \cong \Z/\ell^n\Z$). Additionally, recall that the construction of lisse-\'etale sites is functorial with respect to smooth morphisms and let the distinguished ring object of the topos $\calX_{\lisse\-\et}$ be the constant sheaf $\underline{\Lambda}$ (so that the category of modules on $\calX_{\lisse\-\et}$ that we shall be working with will be $\Lambda\mod$). If $\S \cong \Shv_{\Lambda}^c(\calX_{\lisse\-\et})$ - and recall that $\Shv_{\Lambda}^c(\calX_{\lisse\-\et})$ is a thick subcategory of the triangulated category $\Lambda\mod(\calX_{\lisse\-\et})$ \textit{a priori} (cf. \cite[\href{https://stacks.math.columbia.edu/tag/03RZ}{Tag 03RZ}]{stacks}) - then $\S^{\bullet} \cong \Shv_{\Lambda}^c((U_{\bullet})_{\lisse\-\et}) \cong \Shv_{\Lambda}^c((U_{\bullet})_{\et})$, and so:
                $$\Shv_{\Lambda}^c(\calX_{\lisse\-\et}) \cong \Shv_{\Lambda}^c((U_{\bullet})_{\et})$$
            Since (simplicial) algebraic spaces are \'etale-locally affine, this equivalence allows one to replace the (simplicial) algebraic spaces $U_{\bullet}$ by (simplicial) affine schemes. 
        \end{example}
        
        \begin{remark}[Strictly adic projective systems] \label{remark: strictly_adic_projective_systems}
            \noindent
            \begin{enumerate}
                \item Observe that for every finite $\Lambda$-module $M$, the corresponding cofiltered diagram of finite $\Lambda_n$-modules $\{M_n\}_{n \in \N}$ enjoys an interesting property, which is that the transition maps $M_{n + 1} \tensor_{\Lambda_{n + 1}} \Lambda_n \to M_n$ are actually isomorphisms of (finite) $\Lambda_n$-modules. Any such cofiltered diagram of $\Lambda_n$-modules is said to be \textbf{strictly $\m$-adic} (cf. \cite[Definition 1.4.1.1]{conrad_etale_cohomology}). 
            
                Because $\Lambda$ is $\m$-adically complete and separated, it can be shown that the aforementioned process of associating a strictly $\m$-adic projective system to a given finite $\Lambda$-module induces an equivalence between the category $\Lambda\mod^{\fin}$ of finite $\Lambda$-modules and the category ${}^{\geq 0}(\Lambda, \m)\mod^{\fin, \ad}$ of strictly $\m$-adic projective systems of finite $\Lambda_n$-modules: in one direction, the functor is $M \mapsto \{M_n\}_{n \in \N}$ and in the converse direction, the functor is $\{M_n\}_{n \in \N} \mapsto \underset{n \in \N}{\lim} M_n$ (cf. \cite[\href{https://stacks.math.columbia.edu/tag/031D}{Tag 031D}]{stacks}).
                \item As a category, however, ${}^{\geq 0}(\Lambda, \m)\mod^{\fin, \ad}$ turns out to not be too well-behaved: for instance, if $\{\varphi_n: M'_n \to M_n\}_{n \in \N}$ is a morphism of strictly $\m$-adic projective systems of finite $\Lambda_n$-modules then there will be no guarantee that $\{\ker \varphi_n\}_{n \in \N} \cong \ker \{\varphi_n\}_{n \in \N}$, as the left-hand side might even fail at being a strictly $\m$-adic projective system, let alone at being the sought-for kernel in this situation (cf. \cite[Example 1.4.1.3]{conrad_etale_cohomology}). Modifications are therefore in order, which are inspired by the Artin-Rees Lemma (cf. \cite[Example 1.4.1.4]{conrad_etale_cohomology}). In particular, we shall be enlarging the category ${}^{\geq 0}(\Lambda, \m)\mod^{\fin, \ad}$ to $(\Lambda, \m)\mod^{\ad}$, keeping only the strict $\m$-adicity condition. 
            \end{enumerate}
        \end{remark}
        
        \begin{proposition}[\'Etale fundamental group of products]
                \cite[Corollaire 1.7]{SGA1} Let $k$ be an algebraically closed field and let $X, Y$ be two connected schemes which are, respectively, proper and locally noetherian over $\Spec k$. In addition, fix two geometric points $x \in X$ and $y \in Y$. Then, there is a canonical isomorphism:
                    $$\pi_1((X \x_{\Spec k} Y)_{\fet}, (x, y)) \cong \pi_1(X_{\fet}, x) \x \pi_1(Y_{\fet}, y)$$
            \end{proposition}
            
            
            \begin{example}[\'Etale fundamental group of the affine line] \label{example: etale_fundamental_group_of_the_affine_line}
                If $k$ is a field then $\pi_1((\A^1_k)_{\fet}) \cong \Gal(\bar{k}/k)$ when $\chara k = 0$, and hence $\pi_1((\A^1_k)_{\fet}) \cong 1$ if $k$ is furthermore algebraically closed. If however $\chara k > 0$, then $\pi_1((\A^1_k)_{\fet})$ can fail to be trivial even when $k$ is algebraically closed. For now, see \cite[Theorem 6.13, Remark 6.23, and Exercises 6.28 and 6.29]{lenstra_1985_galois_theory_for_schemes}.
            \end{example}
            
    \begin{remark}[Pullbacks, pushforwards, and tensor products] \label{remark: pullbacks_pushforwards_and_tensor_products}
            We are going to be liberally employing the commutativity of pullbacks and pushforwards of abelian sheaves with (external) tensor products in the proofs of theorem \ref{theorem: unramified_abelian_geometric_class_field_theory} and results that follow, so it shall be worthwhile to recall the details of these commutativity identities. To that end, let $S$ be a base scheme, and let $f: X' \to X$ and $g: Y' \to Y$ be morphisms of $S$-schemes. In addition, denote by $\Shv$ an abelian sheaf theory, by which we mean one such that for all $S$-schemes, the category $\Shv(Z)$ is abelian (e.g. $\ell$-adic sheaves or quasi-coherent sheaves).
            \begin{itemize}
                \item \textbf{(Pullbacks and tensor products):} Recall that for all morphisms of $S$-schemes $\theta: Z' \to Z$, the pullback functor $\theta^*: \Shv(Z) \to \Shv(Z')$ commutes with tensor products of objects in $\Shv(Z)$. 
                \item \textbf{(Pushforwards and tensor products):}
            \end{itemize}
        \end{remark}