\section{Abelian varieties}
    Since the theory of abelian variety is too rich for us to present in any amount of substantial details, we shall recommend that the reader consult \cite{bhatt_abelian_varieties} instead. Regardless, we shall collect here a list of necessary machineries that shall help us understand the role that abelian varieties (particularly, Jacobians) play in the establishment of geometric class field theory.
    
    Let us begin with the definition of abelian varieties.
    \begin{definition}[Abelian varieties] \label{def: abelian_varieties}
        Let $S$ be a base scheme. An \textbf{abelian $S$-scheme} is thus a group $S$-scheme that is smooth, proper and has geometrically connected fibres, which we refer to as \textbf{abelian varieties}.
    \end{definition}
    \begin{remark}[Some basic properties of abelian varieties] \label{remark: basic_properties_of_abelian_varieties}
        The following are important basic properties of abelian varieties that the reader should keep in mind:
        \begin{itemize}
            \item \textbf{(Stability under base change):} Smoothness, properness, and geometric connectedness are all preserved by pullbacks, so pullbacks of abelian schemes are also abelian schemes. In fact, for each fixed base $S$, the category of abelian $S$-schemes is a symmetric monoidal full subcategory of the category of $S$-schemes that are smooth, proper, and with geometrically connected fibres.  
            \item \textbf{(Abelian group structure):} Abelian schemes, like their name suggests, are actually abelian group schemes, although this is a somewhat non-trivial phenomenon (see \cite[Proposition 2.1 and Corollaries 2.2, 2.3, and 2.4]{bhatt_abelian_varieties}). 
            \item \textbf{(Projectivity):} Abelian schemes over fields (i.e. abelian varieties) are not just proper, but actually projective. This is one more a non-trivial fact, and one proof is \cite[Theorem 11.1]{bhatt_abelian_varieties}.
        \end{itemize}
    \end{remark}
    \begin{example}[Some instances of abelian varieties]
        \noindent
        \begin{itemize}
            \item \textbf{(Elliptic curves):} Elliptic curves are nothing but abelian varieties of (pure) dimension $1$. 
            \item \textbf{(Abelian varieties of higher dimensions):} Algebraic groups over characteristic $0$ are smooth \textit{a priori} (cf. \cite[\href{https://stacks.math.columbia.edu/tag/047N}{Tag 047N}]{stacks}), so any proper characteristic-$0$ group scheme with geometrically connected fibres (e.g. $\Proj\left(\frac{\Z\left[x, y, z, \frac{1}{\Delta}\right]}{(y^2z - x^3 - axz^2 - bz^3)}\right)$, where $a, b \in \Z$ are such that $\Delta := -16(4a^3 + 27b^2)$ is non-zero) is automatically an abelian variety. In particular, a complex abelian variety of dimension $n$, through Serre's GAGA theorem, corresponds to a compact and connected complex Lie group of the form $\bbC^n/\Lambda$ for some full-rank lattice $\Lambda$ (cf. \cite[Remark 1.9]{bhatt_abelian_varieties}); the converse is not necessarily true.
            
            Over positive characteristics, one can use the fact that an algebraic group over a perfect field is smooth if it is geometrically reduced (cf. \cite[\href{https://stacks.math.columbia.edu/tag/047P}{Tag 047P}]{stacks}) to single out the class of abelian varieties from the class of algebraic groups. Similar to above, one might consider $\Proj\left(\frac{\F_p\left[x, y, z, \frac{1}{\Delta}\right]}{(y^2z - x^3 - axz^2 - bz^3)}\right)$ for $p \not = 2, 3$ and $\Delta \not \equiv 0 \pmod{p}$.
        \end{itemize}
    \end{example}
    
    Like abelian locally compact groups, abelian varieties and their duals are related to one another by an analogue of the Fourier transform for said topological groups, namely the Fourier-Mukai transform. In fact, if one were to think of the various derived categories of sheaves over abelian varieties as generalisations of the algebras of, say, $L^2$-functions on abelian locally compact groups, then the notion of Fourier-Mukai transforms can be thought of as a direct categorification of the notion of Fourier transforms as integral transforms. Before we can establish a link between abelian varieties and their duals, we must first know how dual abelian varieties are constructed at the level of objects. 
    \begin{proposition}[Existence and uniqueness of dual abelian varieties] \label{prop: dual_abelian_varieties}
        Let $A$ be an abelian variety over some algebraically closed field $k$. Then, there exists a unique abelian variety over $\Spec k$ (up to natural isomorphisms, of course), commonly denoted by $A^{\vee}$ and known as the \textbf{dual} of $A$\footnote{In the sense of dual abelian group.} whose functor of points is the presheaf which assigns to each $S \in \Sch_{/\Spec k}$ the data of:
            \begin{itemize}
                \item a line bundle $\calL \in \Bun_{\GL_1}(S \x_{\Spec k} A)$ such that for every (geometric) point $s \in S$, $(s \x \id_A)^*\calL \in \Bun_{\GL_1}(A)$, and
                \item a \say{rigidifying} isomorphism $(\id_S \x 0_A)^*\calL \cong \calO_{S/k}$.
            \end{itemize}
        Furthermore, the assignment of abelian varieties to their duals is functorial, and any abelian variety $A$ is canonically isomorphic to its double dual $A^{\vee \vee}$.
    \end{proposition}
        \begin{proof}
            \cite[Sections 8 and 13]{mumford_1970_abelian_varieties} and \cite[Subsection 15.2 and Theorem 16.2]{bhatt_abelian_varieties}.
        \end{proof}
    Now, let us define the Fourier-Mukai transform and utilise it for the sake of establishing a categorical relationship between abelian varieties and their duals.
    \begin{convention}
        From now on, if $k$ is a field and then $\Sch_{/\Spec k}^{\ft}$ shall denote the category of schemes that are of finite type over $\Spec k$. Futhermore, by $\QCoh, \Coh$, etc. we shall actually mean the corresponding unbounded derived categories (perhaps with decorations such as $b, \leq 0, \geq 0$, etc.), and likewise for the various functors .
    \end{convention}
    \begin{definition}[Integral transforms] \label{def: integral_transforms}
        For any $X, Y \in \Sch_{/\Spec k}^{\ft}$, the \textbf{integral transform} with \textbf{kernel} $K \in \QCoh(X \x Y)$ the functor:
            $$\Phi_K: \QCoh(X) \to \QCoh(Y)$$
            $$\calF \mapsto (\pr_2)_*\left( \pr_1^* \calF \tensor_{\calO_{X \x Y}} K \right)$$
    \end{definition}
    \begin{proposition}[Convolution of sheaves] \label{prop: convolution_of_sheaves}
        
    \end{proposition}