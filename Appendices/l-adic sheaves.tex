\section{\texorpdfstring{$\ell$}{}-adic sheaves}
    We collect here certain useful facts about adic sheaves on a suitable nice algebraic stack; in particular, we are interested in the properties enjoyed by the category of $\ell$-adic local systems.
    
    \begin{convention}[The setting for adic sheaves] \label{conv: l_adic_sheaves_conventions}
        \noindent
        \begin{itemize}
            \item For the purposes of this appendix, $\calX$ shall be an algebraic stack that is locally of finite type over a base scheme $S$ that is affine, regular, Noetherian and of dimension $\leq 1$, and of characteristic $p \geq 0$ (e.g. $\calX \cong \Bun_G(X)$, with $X$ as in convention \ref{conv: automorphic_side_conventions} and $G$ some connected reductive group); moreoever, we would like to work under the assumption that every finite-type $S$-scheme $T$ is also of finite cohomological dimension. At the same time, $\Lambda$ shall be a Gorenstein local ring of dimension $0$\footnote{Hence Noetherian.} and characteristic $\ell \not = p$ (e.g. $\Lambda \cong \Z_{\ell}$). This is the same as in \cite{laszlo_olsson_adic_sheaves_on_artin_stacks_1}.
            \item For $\calX$ and $\Lambda$ as above, write $\Shv_{\Lambda}^c(\calX_{\lisse\-\et})$ for the \textit{unbounded} derived category of (\'etale-)\href{https://stacks.math.columbia.edu/tag/03RW}{\underline{constructible}} sheaves of $\Lambda$-modules on the lisse-\'etale site of $\calX$, which is implicitly equipped with its natural t-structure.
        \end{itemize}
    \end{convention}
    \begin{remark}[About the assumptions on $\Lambda$]
        It should be noted that we have required $\Lambda$ to be Noetherian because otherwise, we can not guarantee that $\Shv_{\Lambda}^c(\calX_{\lisse\-\et})$ would be a thick subcategory of the abelian category $\Lambda\mod(\calX_{\lisse\-\et})$ of sheaves of $\Lambda$-modules (the relevance of this fact will become clear shortly). In addition, we require that $\Lambda$ is Gorenstein so that as a Noetherian ring, it would admit a dualising complex (in fact, a Noetherian ring has a dualising complex if and only if it is a quotient of a finite-dimensional Gorenstein ring; cf. \cite[Corollary 1.4]{kawasaki_macaulayfication_of_noetherian_rings}). Lastly, $\dim \Lambda = 0$ because otherwise, we would have to keep track of cohomological shifts (cf. \cite[\href{https://stacks.math.columbia.edu/tag/0AWS}{Tag 0AWS} and \href{https://stacks.math.columbia.edu/tag/0B5A}{Tag 0B5A}]{stacks}), which are inessential technicalities.
    \end{remark}
    
    \begin{proposition}[Gluing adic sheaves along simplicial covers]
        Consider $(\scrX_{\bullet}, \calO_{\scrX_{\bullet}})$ a ringed simplicial topos along with an ordinary (i.e. $0$-coconnective) ringed topos $(X_{\lisse\-\et}, \calO_X)$ and a flat geometric morphism:
            $$
                \begin{tikzcd}
                	{(\scrX_{\bullet}, \calO_{\scrX_{\bullet}})} & {(X, \calO_X)}
                	\arrow[""{name=0, anchor=center, inner sep=0}, "{\e_*}"', bend right, from=1-1, to=1-2]
                	\arrow[""{name=1, anchor=center, inner sep=0}, "{\e^*}"', bend right, from=1-2, to=1-1]
                	\arrow["\dashv"{anchor=center, rotate=-90}, draw=none, from=1, to=0]
                \end{tikzcd}
            $$
        If $\S$ is a thick subcategory of the triangulated category $\calO_X\mod$ and if $\S^{\bullet}$ is the essential image of $\L\e^*: \calO_X\mod \to \calO_{\scrX_{\bullet}}\mod$, then not only is $\S^{\bullet}$ a thick subcategory of $\calO_{\scrX_{\bullet}}\mod$ but also, the pair $\L\e^* \ladjoint \R\e_*$ is an adjoint equivalence between $\S^{\bullet}$ and $\S$.
    \end{proposition}
        \begin{proof}
            \cite[Lemma 2.2.2 and Theorem 2.2.3]{laszlo_olsson_adic_sheaves_on_artin_stacks_1}.
        \end{proof}
    \begin{example}[Gluing constructible sheaves] \label{example: gluing_constructible_sheaves}
        Consider a smooth (hence flat) simplicial cover $U_{\bullet} \to \calX$ of the algebraic stack $\calX$ from convention \ref{conv: l_adic_sheaves_conventions} by simplicial algebraic spaces and assume furthermore that $\Lambda$ is Artinian (e..g $\Lambda \cong \Z/\ell^n\Z$). Additionally, recall that the construction of lisse-\'etale sites is functorial with respect to smooth morphisms and let the distinguished ring object of the topos $\calX_{\lisse\-\et}$ be the constant sheaf $\underline{\Lambda}$ (so that the category of modules on $\calX_{\lisse\-\et}$ that we shall be working with will be $\Lambda\mod$). If $\S \cong \Shv_{\Lambda}^c(\calX_{\lisse\-\et})$ - and recall that $\Shv_{\Lambda}^c(\calX_{\lisse\-\et})$ is a thick subcategory of the triangulated category $\Lambda\mod(\calX_{\lisse\-\et})$ \textit{a priori} (cf. \cite[\href{https://stacks.math.columbia.edu/tag/03RZ}{Tag 03RZ}]{stacks}) - then $\S^{\bullet} \cong \Shv_{\Lambda}^c((U_{\bullet})_{\lisse\-\et}) \cong \Shv_{\Lambda}^c((U_{\bullet})_{\et})$, and so:
            $$\Shv_{\Lambda}^c(\calX_{\lisse\-\et}) \cong \Shv_{\Lambda}^c((U_{\bullet})_{\et})$$
        Since (simplicial) algebraic spaces are \'etale-locally affine, this equivalence allows one to replace the (simplicial) algebraic spaces $U_{\bullet}$ by (simplicial) affine schemes. 
    \end{example}
    
    Throughout the paper (particularly the second half), we consider pullbacks, pushforwards, as well as tensor products of $\ell$-adic local systems. Therefore, let us write down the results that precisely state that these functors are well-defined constructions for adic sheaves. 
        
        