\section{Introduction}
    The paper will be organised into two main sections, detailing what we shall call the \textbf{Galois Side} and the \textbf{Automorphic Side} of geometric class field theory. This introductory section will be dedicated to the outlining of our approaches to these sections, as well as for laying down some conventions that we will be following until the end of the paper.

    \subsection{The Galois Side}
        The goal of this section is to define the \'etale fundamental group, as well as investigate its relationship with representations of absolute Galois groups. Our starting point is the following result, which explains why one might even suspect any sort of involvement of algebraic geometry in the first place:
        \begin{lemma}[Varieties and field extensions] \label{lemma: varieties_and_field_extensions}
            \cite[\href{https://stacks.math.columbia.edu/tag/0BXN}{Tag 0BXN}]{stacks} Let $k$ be a field. Then, $\trdeg K_X = \dim X$ for all varieties $X/k$, and there exists a canonical equivalence of categories as follows:
                $$\{\text{Finite-type field extensions $K/k$ and $k$-algebra homomorphisms}\}^{\op}$$
                $$\cong$$
                $$\{\text{Varieties $X/k$ and \href{https://stacks.math.columbia.edu/tag/01RI}{\underline{dominant}} \href{https://stacks.math.columbia.edu/tag/01RR}{\underline{rational}} maps}\}$$
        \end{lemma}
            \begin{proof}
                If $f: X \to Y$ is a dominant map of varieties and if $\eta_X$ and $\eta_Y$ are the unique generic points of $X$ and $Y$ (unique because varieties are integral by definition, and every integral scheme \textit{a priori} has a unique generic point), then $f(\eta_X) = \eta_Y$ per the definition of dominant morphisms. The residue field at generic points are precisely the function fields (consider the stalk of the structure sheaves at the generic points to see why this is the case), so we have obtained a map of function fields $K_Y \to K_X$. We leave the proof of finiteness up to the reader.
            \end{proof}
        Through lemma \ref{lemma: varieties_and_field_extensions}, one obtains the following regarding the relationship between curves (i.e. algebraic varieties of dimension $1$) and their function fields (which \textit{a priori} are of transcedence degree $1$ over the ground field) with little difficulty:
        \begin{proposition}[Curves and function fields] \label{prop: curves_and_function_fields}
            \cite[\href{https://stacks.math.columbia.edu/tag/0BY1}{Tag 0BY1}]{stacks} For any field $k$, one has the following canonical equivalences of categories:
                $$\{\text{Field extensions $K/k$ of transcendence degree $1$ and $k$-algebra homomorphisms}\}^{\op}$$
                $$\cong$$
                $$\{\text{Curves $X/k$ and dominant rational maps}\}$$
                $$\cong$$
                $$\{\text{Non-singular projective curves $X/k$ and dominant rational maps}\}$$
        \end{proposition}
            \begin{proof}
                The first equivalence is an obvious consequence of lemma \ref{lemma: varieties_and_field_extensions}. To show that the second equivalence holds, note firstly that there is an evident fully faithful functor from the third category to the second; we shall need to show that this functor is also essentially surjective. For this, simply recall that for each curve $X/k$, there exists a non-singular projective curve $\tilde{X}/k$ that is birational to $X/k$, namely $X^{\nu} \cup \{\infty_1, ..., \infty_n\}$, the normalisation $X^{\nu}/k$ of $X/k$ with finitely many extra points (recall also that any normal Noetherian scheme of dimension $\leq 1$ is \textit{a priori} non-singular; cf. \cite[\href{https://stacks.math.columbia.edu/tag/0BX2}{Tag 0BX2}]{stacks}).
            \end{proof}
            
        Through proposition \ref{prop: curves_and_function_fields}, we obtain the first crucial tool for the geometrisation of class field theory.
        \begin{corollary}[Galois covers of curves and Galois extensions] \label{coro: galois_covers_of_curves_and_galois_extensions}
            Let $k$ be a field. If $X$ is a connected non-singular projective curve over $\Spec k$ with function field $K$, then there is a canonical equivalence:
                $$({}^{K/}\Fld^{\fin, \Gal})^{\op} \cong (\Sch_{/X})_{\fet}^{\Gal}$$
            between the category of finite Galois extensions of $K$ and finite \'etale-Galois covers of $X$ (cf. remark \ref{remark: geometric_galois_correspondence}). 
        \end{corollary}
        The importance of corollary \ref{coro: galois_covers_of_curves_and_galois_extensions} can not be understated: what it tells us is essentially that the Galois group $\Gal(K^{\ab}/K)$ is nothing but the \'etale fundamental group of $(\Sch_{/X})_{\fet}^{\Gal}$ (more on this later, after we have introduced the \'etale fundamental group; cf. definition \ref{def: etale_fundamental_groups}). This, already, is one side of the Artin's Reciprocity Law, which we shall refer to as \say{\textbf{The Galois Side}} per popular conventions. Actually, this is a bit of a lie: instead of formulating geometric class field theory directly in terms of the \'etale fundamental group $\pi_1^{\ab}(X_{\fet})$ (or rather, its continuous $\ell$-adic characters), we will be phrasing things in terms of $\ell$-adic local systems of rank $1$ on $X$; the main result on the Galois Side shall be theorem \ref{theorem: unramified_representations_are_sheaves_on_X}, which rigorously and precisely establishes the categorification of continuous representations of $\pi_1^{\ab}(X_{\fet})$ to $\ell$-adic local systems on $X$ via a canonical equivalence of categories:
            $$\Rep^1_{\overline{\Q_{\ell}}}(\pi_1^{\ab}(X_{\fet}))^{\cont} \cong \LocSys^1_{\overline{\Q_{\ell}}}(X)$$
    
    \subsection{The Automorphic Side}
        \begin{convention}[The setting of the main theorem]
            In what follows, $k$ shall be algebraically closed field and $X$ shall be a smooth projective \textit{connected} curve over $\Spec k$. Additionally, $\Bun_{\GL_1}(X)$ shall denote the moduli stack of line bundles on $X$.
        \end{convention}
    
        Let us now move on to what is known as the \say{\textbf{Automorphic Side}}, and we shall begin with the notion of \textbf{Hecke eigensheaves}. To introduce these gadgets, however, we will first need to discuss the \textbf{Hecke correspondence}, the categorification of the action of the Hecke algebra on the space of functions satisfying certain smoothness and growth conditions on the double ad\`elic quotient $\GL_1(K) \backslash \GL_1(\A_K) / \GL_1(\scrO_K)$ (i.e. automorphic forms). The details will be spelled out in definition \ref{def: hecke_correspondences}, but for now, let us think of the Hecke correspondence as a span, i.e. a diagram of the form:
            $$
                \begin{tikzcd}
                	& {\Hecke_{\GL_1}(X)} \\
                	{\Bun_{\GL_1}(X)} && {X \x \Bun_{\GL_1}(X)}
                	\arrow["{\cev{h}_X}"', from=1-2, to=2-1]
                	\arrow["{\supp_X \x \vec{h}_X}", from=1-2, to=2-3]
                \end{tikzcd}
            $$
        If we were to generically denote a given category of \say{good sheaf theory} by $\Shv(-)$\footnote{Eventually, we will be interested particularly in $\ell$-adic sheaves of rank $1$, which shall be denoted by $\Shv_{\overline{\Q_{\ell}}}^1(-)$, but more on this later. In the wider context of the Geometric Langlands Programme, $\Shv(-)$ might mean perverse sheaves, or when we are working over $\bbC$, D-modules; we shall not touch on these sheaf theories.}, then the Hecke correspondence induces the following sheaf pull-push diagram:
            $$
                \begin{tikzcd}
                	& {\Shv(\Hecke_{\GL_1}(X))} \\
                	{\Shv(\Bun_{\GL_1}(X))} && {\Shv(X \x \Bun_{\GL_1}(X))}
                	\arrow["{(\cev{h}_X)^*}", from=2-1, to=1-2]
                	\arrow["{(\supp_X \x \vec{h}_X)_*}", from=1-2, to=2-3]
                \end{tikzcd}
            $$
        and by composing the two functors in the obvious manner, one gets a new functor:
            $$\scrH_X: \Shv(\Bun_{\GL_1}(X)) \to \Shv(X \x \Bun_{\GL_1}(X))$$
        This is commonly known as the \textbf{Hecke functor} or the \textbf{Hecke operator} (should we want to put emphasis on the spectral nature of $\scrH_X$), and it is of central importance to us. However, before we can explain why this is the case, observe that the Hecke operator $\scrH_X$ is actually \say{global} in a sense: the fibre of $\supp_X \x \vec{h}$ over any given point $x \in X$ is the \say{local} Hecke correspondence:
            $$
                \begin{tikzcd}
                	& {\Hecke_{\GL_1}(x)} \\
                	{\Bun_{\GL_1}(X)} && {\Bun_{\GL_1}(X)}
                	\arrow["{\cev{h}_x}"', from=1-2, to=2-1]
                	\arrow["{\vec{h}_x}", from=1-2, to=2-3]
                \end{tikzcd}
            $$
        and one can thus define the local Hecke operator at $x \in X$ as:
            $$\scrH_x := (\vec{h}_x)_* (\cev{h}_x)^*$$
        Arguably, this is more akin to the classical Hecke operator, as it is an endofunctor on $\Shv(\Bun_{\GL_1}(X))$ as opposed to the global operator $\scrH_X$, which has differing domain and codomain; henceforth, we will be thinking of the global Hecke operator $\scrH_X$ as a family $\{\scrH_x\}_{x \in X}$ of local Hecke operators parametrised by points $x \in X$. 
        
        It is now an essential technicality that we work with $\ell$-adic local systems of rank $1$ (for which we shall write $\Shv_{\overline{\Q_{\ell}}}^1(-)$) instead of simply with a generic sheaf theory $\Shv(-)$ as we have until this moment. Via the Hecke operators, one can define the Hecke eigensheaves that we eluded to earlier: as the name suggests, these are nothing but sheaves $\E \in \Shv_{\overline{\Q_{\ell}}}^1(\Bun_{\GL_1}(X))$ that are \say{eigenvectors} of the \textit{global} Hecke operators $\scrH_X$, i.e. for each such $\E$, there exists $\calL \in \Shv_{\overline{\Q_{\ell}}}^1(X)$ such that\footnote{Here, $\boxtimes$ denotes the Deligne tensor product in the $1$-category of finite abelian categories and right-exact functors; in this particular case, we shall be content with $\calL \boxtimes \E \cong \pr_1^*\calL \tensor \pr_2^*\E$.}:
            $$\scrH_X(\E) \cong \calL \boxtimes \E$$
        Now, because $\calL$ is an $\ell$-adic local system of rank $1$ on $X$, its stalk $\calL_x$ at any point $x \in X$ is nothing but $\overline{\Q_{\ell}}$. Consequently, the corresponding local Hecke operators $\scrH_x$ admit $(\overline{\Q_{\ell}})_x \in \Shv(X)$ - the skyscraper sheaf with value $\overline{\Q_{\ell}}$ and supported at $x \in X$ - as an \say{eigenvalue}:
            $$\scrH_x(\E) \cong (\overline{\Q_{\ell}})_x \boxtimes \E$$
        (and thus one may think of $\calL$ as a family of eigenvalues of $\E$ parametrised by points $x \in X$). It is easy to see that Hecke eigensheaves form a full symmetric monoidal subcategory of $\Shv_{\overline{\Q_{\ell}}}^1(\Bun_{\GL_1}(X))$, which we shall denote by $\Eig^1_{\overline{\Q_{\ell}}}(\Bun_{\GL_1}(X))$. 
        
        At this point, we can state and prove the main theorem of this section, which establishes a canonical equivalence between the category of rank-$1$ $\ell$-adic local systems on $X$ and the category of ($\ell$-adic) Hecke eigensheaves of rank $1$ on $\Bun_{\GL_1(X)}$:
            $$\LocSys_{\overline{\Q_{\ell}}}^1(X) \cong \Eig^1_{\overline{\Q_{\ell}}}(\Bun_{\GL_1}(X))$$
        which maps each local system $\calL \in \LocSys_{\overline{\Q_{\ell}}}^1(X)$ to a Hecke eigensheaf $\Autom_X(\calL) \in \Eig^1_{\overline{\Q_{\ell}}}(\Bun_{\GL_1}(X))$ with eigenvalue $\calL$. Finally, by putting theorem \ref{theorem: unramified_representations_are_sheaves_on_X} and theorem \ref{theorem: unramified_abelian_geometric_class_field_theory} together, one obtains a canonical equivalence of categories as follows:
            $$\Rep_{\overline{\Q_{\ell}}}^1(\pi_1^{\ab}(X_{\fet}))^{\cont} \cong \Eig^1_{\overline{\Q_{\ell}}}(\Bun_{\GL_1}(X))$$
        This is the version of global class field theory that we seek, and it tells us that $1$-dimensional continuous $\ell$-adic Galois representations are the same as automorphic forms associated to $\GL_1$.
        
    \subsection{Conventions}
        \begin{convention}[Category theory] \label{conv: category_theory}
            \noindent
            \begin{itemize}
                \item \textbf{(Fundamentals):} We assume familiarity with the notion of categories along with fundamental categorical concepts such as universal properties, (co)slice categories, (co)limits, and adjunctions, etc. \cite{maclane} and \cite[\href{https://stacks.math.columbia.edu/tag/0011}{Tag 0011}]{stacks} shall be our main references regarding these notions.
                \item \textbf{(Sheaf theory):} We shall assume to have theory of sheaves of sets (i.e. the theory of sheaf topoi) and of sheaves taking values in tensor categories such as module categories (cf. \cite{EGNO}) at our disposal. Readers who are not too familiar with the former are encouraged to consult \cite{sga4} and \cite[\href{https://stacks.math.columbia.edu/tag/00UZ}{Tag 00UZ}]{stacks}; for the latter, we recommend \cite[\href{https://stacks.math.columbia.edu/tag/006A}{Tag 006A}, \href{https://stacks.math.columbia.edu/tag/01AC}{Tag 01AC}, and \href{https://stacks.math.columbia.edu/tag/03A4}{Tag 03A4}]{stacks}. Also related is descent theory, for which we shall refer to \cite{vistoli_descent}, \cite[section C2.1]{elephant1}, \cite[Chapter III]{sheaves_in_geometry_and_logic}, and \cite[\href{https://stacks.math.columbia.edu/tag/0266}{Tag 0266} and \href{https://stacks.math.columbia.edu/tag/0238}{Tag 0238}]{stacks}.
                \item \textbf{(Monoidal categories):} For information on monoidal categories, we refer the readers to \cite{EGNO}.
            \end{itemize}
        \end{convention}
        
        \begin{convention}[Algebraic geometry] \label{conv: algebraic_geometry}
            \noindent
            \begin{itemize}
                \item \textbf{(Schemes):} For generalities on schemes, we refer the reader to \cite[Chapters II and III]{hartshorne}, as well as \cite[\href{https://stacks.math.columbia.edu/tag/01H8}{Tag 01H8}, \href{https://stacks.math.columbia.edu/tag/01QL}{Tag 01QL}, and \href{https://stacks.math.columbia.edu/tag/0209}{Tag 0209}]{stacks}.
                \item \textbf{(Topologies on schemes):} For information on topologies on the category of schemes, we refer the reader to \cite[\href{https://stacks.math.columbia.edu/tag/0214}{Tag 0214}]{stacks}. For a general categorical treatment of descent theory, we refer the reader to \cite{vistoli_descent}, \cite[section C2.1]{elephant1}, \cite[Chapter III]{sheaves_in_geometry_and_logic}, and \cite[\href{https://stacks.math.columbia.edu/tag/0266}{Tag 0266} and \href{https://stacks.math.columbia.edu/tag/0238}{Tag 0238}]{stacks}.
            \end{itemize}
        \end{convention}