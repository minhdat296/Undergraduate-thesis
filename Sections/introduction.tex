\section{Introduction}
    \subsection{Using algebraic geometry to understand class field theory}
        Our starting point is the classical following result, which hints at why one might even suspect any sort of involvement of algebraic geometry in the first place:
        \begin{proposition}[Curves and function fields] \label{prop: curves_and_function_fields}
            \cite[\href{https://stacks.math.columbia.edu/tag/0BY1}{Tag 0BY1}]{stacks} For any perfect field $k$, one has an equivalences of categories:
                $$\{\text{Field extensions $K/k$ of transcendence degree $1$ and $k$-algebra homomorphisms}\}^{\op}$$
                $$\cong$$
                $$\{\text{Smooth projective curves $X/k$ and \href{https://stacks.math.columbia.edu/tag/01RI}{\underline{dominant}} \href{https://stacks.math.columbia.edu/tag/01RR}{\underline{rational}} maps}\}$$
        \end{proposition}
        From this proposition, one infers that the study of global function fields over perfect fields $k$ (i.e. field extensions $K/k$ of transcendence degree $1$, such as $k(t)$) is actually the same as the study of smooth projective curves\footnote{The reader might also wish to consult \cite[Sections I.13 and I.14]{neukirch_2010_algebraic_number_theory}.} over $\Spec k$, such as $\P^1_k$. As a result, one might believe that at least for global function fields, there should exist a purely geometric version of class field theory. The idea is that such a version of global class field theory would be much more intuitive than the classical formulation of the theory, as various relevant notions from algebraic number theory - such as places, ramification, or even what it means to be \say{global} itself - have natural interpretations in the language of schemes. 
        
        Now, while such a version of global class field theory indeed exists, it is not without its shortcomings. Most prominently is the fact that due to technical inadequacies (e.g. the lack of a Frobenius endomorphism in characteristic $0$), our ground field $k$ can only be allowed to be a finite field should we want to recover the classical version of class field theory from the geometric version. However, for curves over finite fields, we can manage to obtain a rather complete story, neatly encapsulated in the following commutative diagram wherin the arrows are all isomorphisms of (abelian) groups:
            $$
                \begin{tikzcd}
                	{\Shv_{\bar{\Q}_{\ell}}^{\lisse, 1}(X)} && {\Eig\Shv_{\bar{\Q}_{\ell}}^1(\Bun_{\GL_1}(X))} \\
                	{\Rep^1_{\bar{\Q}_{\ell}}(\pi_1(X_{\fet}))} && {\scrA_{\GL_1}(X, \bar{\Q}_{\ell})}
                	\arrow["{\text{Theorem \ref{theorem: unramified_abelian_geometric_class_field_theory}}}", from=1-1, to=1-3]
                	\arrow["{\text{Theorem \ref{theorem: galois_representations_are_lisse_sheaves}}}"', from=1-1, to=2-1]
                	\arrow["{{\text{Proposition \ref{prop: hecke_characters_from_hecke_eigensheaves}}}}", from=1-3, to=2-3]
                	\arrow["{\text{Theorem \ref{theorem: artin_reciprocity_for_function_fields_over_finite_fields}}}", from=2-1, to=2-3]
                \end{tikzcd}
            $$
        In the diagram above, the top arrow is the geometric version of global class field theory, and the two side arrows tells us how the \say{decategorification} process through which we might obtain the classical version is done. As for the bottom arrow, it expresses Artin's Reciprocity Law for global function fields $K_X$ over finite fields (which we know to arise from smooth projective curves $X$ over finite fields, thanks to proposition \ref{prop: curves_and_function_fields}). This in turn implies the classical statement of class field theory (again, for global function fields over finite fields), which asserts that there is an equivalence of categories as below, relating finite unramified abelian extensions of $K_X$ and finite-index subgroups of the so-called id\`ele class group $\GL_1(K_X)\backslash\GL_1(\A_X)$ that contain $\GL_1(\bbO_K)$:
            $$
                \begin{tikzcd}
                	{\{\text{Finite unramified abelian extensions $L/K_X$ and $K_X$-algebra homomorphisms}\}^{\op}} \\
                	{\{\text{Finite-index subgroups of $\GL_1(K_X)\backslash\GL_1(\A_X)/\GL_1(\bbO_{X})$}\}}
                	\arrow["{\text{Theorem \ref{theorem: class_field_theory_for_global_function_fields_over_finite_fields}}}", from=1-1, to=2-1]
                \end{tikzcd}
            $$ 
        Now, in addition to the fact that it is possible to obtain the classical version of global class field theory via purely geometric methods, it should also be noted that the geometric formulation is arguably more fundamental, and that the classical version ought to be thought of as a corollary thereof and not the other way around (evident from the fact that theorem \ref{theorem: unramified_abelian_geometric_class_field_theory} implies theorem \ref{theorem: artin_reciprocity_for_function_fields_over_finite_fields}, which in turn implies theorem \ref{theorem: class_field_theory_for_global_function_fields_over_finite_fields}). Furthermore, this version of global class field theory, as it is naturally phrased in tems of sheaves instead of the underlying spaces, demonstrates to us the importance of representation theory in the study of Galois groups.
        
    \subsection{Conventions}
        \begin{convention}[Category theory] \label{conv: category_theory}
            \noindent
            \begin{itemize}
                \item \textbf{(General category theory):} We assume familiarity with the notion of categories along with fundamental categorical concepts such as universal properties, (co)slice categories, (co)limits, and adjunctions, etc. \cite{maclane} and \cite[\href{https://stacks.math.columbia.edu/tag/0011}{Tag 0011}]{stacks} shall be our main references regarding these notions. We will also be making use of monoidal categories, and for information on them, we refer the readers to \cite[Chapter VII]{maclane} and \cite[Chapters 2 and 4]{EGNO}.
                \item \textbf{(Sheaf theory):} We shall assume to have theory of sheaves of sets (i.e. the theory of sheaf topoi) and of sheaves taking values in tensor categories such as module categories (cf. \cite[Chapter 4]{EGNO}) at our disposal. Readers who are not too familiar with the former are encouraged to consult \cite{sga4} and \cite[\href{https://stacks.math.columbia.edu/tag/00UZ}{Tag 00UZ}]{stacks}; for the latter, we recommend \cite[\href{https://stacks.math.columbia.edu/tag/006A}{Tag 006A}, \href{https://stacks.math.columbia.edu/tag/01AC}{Tag 01AC}, and \href{https://stacks.math.columbia.edu/tag/03A4}{Tag 03A4}]{stacks}. Also related is descent theory, for which we shall refer to \cite{vistoli_descent}, \cite[\href{https://stacks.math.columbia.edu/tag/0266}{Tag 0266}]{stacks}, and \cite[\href{https://stacks.math.columbia.edu/tag/0238}{Tag 0238}]{stacks}.
            \end{itemize}
        \end{convention}
        
        \begin{convention}[Algebra and algebraic geometry] \label{conv: algebraic_geometry}
            \noindent
            \begin{itemize}
                \item \textbf{(Group theory):} In addition to basic group theory, we assume some familiarity with (locally) compact topological groups and in particular, profinite groups (see \cite[Section I.1]{neukirch_1999_cohomology_of_number_field}). Understanding of some basic group cohomology is also be useful (see \cite[Chapter VI]{hilton_stammbach_homological_algebra}), though not necessarily required.
                \item \textbf{(Rings, modules, and homological algebra):} The reader is expected to be familiar with notions from the theory of rings and modules such as such as what it means for a module over a ring to be Noetherian, of finite type, along with some basic homological algebra, such as the notions of projectivity and injectivity for modules and how they relate to exactness of (co)representable functors between categories of modules. For details, the reader may consult \cite{chapter0}.
                \item \textbf{(Commutative algebra and schemes):} For generalities on schemes, we refer the reader to \cite[Chapters II and III]{hartshorne}, as well as \cite[\href{https://stacks.math.columbia.edu/tag/01H8}{Tag 01H8}, \href{https://stacks.math.columbia.edu/tag/01QL}{Tag 01QL}, and \href{https://stacks.math.columbia.edu/tag/0209}{Tag 0209}]{stacks}. As the theory of schemes relies heavily on the theory of commutative rings, we shall also assume that the reader is familiar with notions such as smoothness, regularity, locality, etc. for commutative rings (see \cite[\href{https://stacks.math.columbia.edu/tag/00AO}{Tag 00AO}]{stacks}).
                
                \textit{We do not assume that our readers are familiar with, nor are we going to make use of algebraic spaces and algebraic stacks.} In particular, whenever the word \say{moduli space} or \say{moduli functor} is used, it will only mean a presheaf $F: \Sch^{\op} \to \Sets$, and will be for putting emphasis on the geometric nature of presheaves of sets on the category of schemes; in fact, all moduli functors that we make use of, such as $\Bun_{\GL_1}(X)$ and $\Div_{X/k}^{\eff, (d)}$, will turn out to be representable by schemes. 
                \item \textbf{(Topologies on schemes):} For information on the various kinds of commonly used Grothendieck topologies on the category of schemes, we refer the reader to \cite[\href{https://stacks.math.columbia.edu/tag/020K}{Tag 020K} and \href{https://stacks.math.columbia.edu/tag/0238}{Tag 0238}]{stacks}.
                \item \textbf{(Group schemes):} It shall be convenient for the reader to have some familiarity with group schemes (and in particular, group schemes over fields; see \cite[\href{https://stacks.math.columbia.edu/tag/047J}{Tag 047J} and \href{https://stacks.math.columbia.edu/tag/0BF6}{Tag 0BF6}]{stacks}), but this is not a requirement.
            \end{itemize}
        \end{convention}
        
        \begin{convention}[Number theory] \label{conv: number_theory}
            \noindent
            \begin{itemize}
                \item \textbf{(Galois theory):} We assume understanding of Galois theory for fields, and since we will be using the \'etale fundamental group, some familiarity with the Galois Correspondence for Covering Spaces from algebraic topology (cf. \cite{MATH525_covering_spaces_and_fundamental_group_project} and \cite[Theorem 1.38 and Proposition 1.39]{hatcher2002algebraic}) might also be useful, though this is not necessary. In particular, the reader should be familiar with the Fundamental Theorem of Galois Theory (cf. \cite[\href{https://stacks.math.columbia.edu/tag/0BML}{Tag 0BML}]{stacks}), and they should also have an understanding of how to compute certain common instances of (absolute) Galois groups, such as those of finite fields.
                \item \textbf{(Local and global fields):} Aside from the definitions of local and global fields themselves (and related notions such as the ring of integers of a global/local field, or the ring of rational ad\`eles of a global field), the reader is expected to have some familiarity with important results in the theory of valuations, such as Ostrowski's Theorem. For more details, see \cite[Chapter II]{neukirch_2010_algebraic_number_theory}.
                \item \textbf{(Classical global class field theory):} Although it is absolutely not necessary for the reader to have any prior knowledge of this subject, readers who wish to understand how the classical approach to global class field theory differs from our geometric version are enouraged to consult \cite[Chapters IV and VI]{neukirch_2010_algebraic_number_theory} and \cite[Chapters III, VIII, and IX]{neukirch_1999_cohomology_of_number_field}.
            \end{itemize}
        \end{convention}
        
    \subsection{Acknowledgements}
        My thanks go out to my thesis supervisor, Dr. Jerrod Smith, who has been among the most supportive of teachers, mentors, and friends all throughout my undergraduate career. Thank you for all the time you have dedicated to me, for all the invaluable advices, for always being understanding, for all the conversations, and for having helped me pursue my mathematical curiosity to my heart's content. And thank you for introducing me to the Langlands Programme: words can not describe how much this means to me.
        
        I would also like to thank Dr. Clifton Cunningham, who was very kind to give this thesis an additional quality assurance read, and who has also been a great professor upon all the occasions that I have had the privilege to learn from. My appreciation also extends to the the members of the $p$-attic - Dr. Cunningham's graduate students, both former and current - along with Deni Salja, for their ability to humour my fascination with arithmetic geometry, categorical techniques, as well as my horrendous sense of humour.
        
        Finally, I would like to thank Dr. Ryan Hamilton. Without his initial encouragement, I would never have thought of pursuing a degree in mathematics. It truly has been a wonderful journey.