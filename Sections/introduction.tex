\section{Introduction}
    The paper will be organised into two main sections, detailing what we shall call the \textbf{Galois Side} and the \textbf{Automorphic Side} of geometric class field theory. This introductory section will be dedicated to the outlining of our approaches to these sections, as well as for laying down some conventions that we will be following until the end of the paper.

    \subsection{The Galois Side}
        The goal of this section is to define the \'etale fundamental group, as well as investigate its relationship with representations of absolute Galois groups. Our starting point is the following result, which explains why one might even suspect any sort of involvement of algebraic geometry in the first place:
        \begin{lemma}[Varieties and field extensions] \label{lemma: varieties_and_field_extensions}
            \cite[\href{https://stacks.math.columbia.edu/tag/0BXN}{Tag 0BXN}]{stacks} Let $k$ be a field. Then, $\trdeg K_X = \dim X$ for all varieties $X/k$, and there exists a canonical equivalence of categories as follows:
                $$\{\text{Finite-type field extensions $K/k$ and $k$-algebra homomorphisms}\}^{\op}$$
                $$\cong$$
                $$\{\text{Varieties $X/k$ and \href{https://stacks.math.columbia.edu/tag/01RI}{\underline{dominant}} \href{https://stacks.math.columbia.edu/tag/01RR}{\underline{rational}} maps}\}$$
        \end{lemma}
            \begin{proof}
                If $f: X \to Y$ is a dominant map of varieties and if $\eta_X$ and $\eta_Y$ are the unique generic points of $X$ and $Y$ (unique because varieties are integral by definition, and every integral scheme \textit{a priori} has a unique generic point), then $f(\eta_X) = \eta_Y$ per the definition of dominant morphisms. The residue field at generic points are precisely the function fields (consider the stalk of the structure sheaves at the generic points to see why this is the case), so we have obtained a map of function fields $K_Y \to K_X$. We leave the proof of finiteness up to the reader.
            \end{proof}
        Through lemma \ref{lemma: varieties_and_field_extensions}, one obtains the following regarding the relationship between curves (i.e. algebraic varieties of dimension $1$) and their function fields (which \textit{a priori} are of transcedence degree $1$ over the ground field) with little difficulty:
        \begin{proposition}[Curves and function fields] \label{prop: curves_and_function_fields}
            \cite[\href{https://stacks.math.columbia.edu/tag/0BY1}{Tag 0BY1}]{stacks} For any field $k$, one has the following canonical equivalences of categories:
                $$\{\text{Field extensions $K/k$ of transcendence degree $1$ and $k$-algebra homomorphisms}\}^{\op}$$
                $$\cong$$
                $$\{\text{Curves $X/k$ and dominant rational maps}\}$$
                $$\cong$$
                $$\{\text{Non-singular projective curves $X/k$ and dominant rational maps}\}$$
        \end{proposition}
            \begin{proof}
                The first equivalence is an obvious consequence of lemma \ref{lemma: varieties_and_field_extensions}. To show that the second equivalence holds, note firstly that there is an evident fully faithful functor from the third category to the second; we shall need to show that this functor is also essentially surjective. For this, simply recall that for each curve $X/k$, there exists a non-singular projective curve $\tilde{X}/k$ that is birational to $X/k$, namely $X^{\nu} \cup \{\infty_1, ..., \infty_n\}$, the normalisation $X^{\nu}/k$ of $X/k$ with finitely many extra points (recall also that any normal Noetherian scheme of dimension $\leq 1$ is \textit{a priori} non-singular; cf. \cite[\href{https://stacks.math.columbia.edu/tag/0BX2}{Tag 0BX2}]{stacks}).
            \end{proof}
        Proposition \ref{prop: curves_and_function_fields} implies that the \'etale fundamental group of a curve $X$ is precisely the same as the absolute Galois group of its function field $K$, and thanks to it, absolute Galois groups of global fields can be studied using algebraic curves. In fact, this is the content of the main theorem of this section (theorem \ref{theorem: galois_representations_are_local_systems}), which states that there is an equivalence:
            $$\Shv^{\ad, 1}_{\underline{\bar{\Q}_{\ell}}}(X) \cong \Rep^{\ad, 1}_{\underline{\bar{\Q}_{\ell}}}(\pi_1^{\ab}(X_{\fet}))^{\cont}$$
        between the category of rank-$1$ $\ell$-adic local systems on $X$, and that of continuous $\ell$-adic characters of the abelianisation $\pi_1^{\ab}(X_{\fet})$ of the \'etale fundamental group of $X$. 
    
    \subsection{The Automorphic Side}
        \begin{convention}[The setting of the main theorem]
            In what follows, $k$ shall be algebraically closed field and $X$ shall be a smooth projective \textit{connected} curve over $\Spec k$. Additionally, $\Bun_{\GL_1}(X)$ shall denote the moduli stack of line bundles on $X$.
        \end{convention}
    
        Let us now move on to what is known as the \say{\textbf{Automorphic Side}}, and we shall begin with the notion of \textbf{Hecke eigensheaves}. To introduce these gadgets, however, we will first need to discuss the \textbf{Hecke correspondence}, the categorification of the action of the Hecke algebra on the space of automorphic forms, which are certain kinds of functions on the double ad\`elic quotient $\GL_1(K) \backslash \GL_1(\A_K) / \GL_1(\scrO_K)$ satisfying certain smoothness and growth conditions. Once that is done, we will be able to define Hecke operators:
            $$\scrH: \Shv(\Bun_{\GL_1}(X)) \to \Shv(X \x \Bun_{\GL_1}(X))$$
        The category $\Shv(\Bun_{\GL_1}(X))$ enjoys certain \say{linearity} properties (much like how function spaces are topological vector spaces) and we shall make use of these properties to define Hecke eigensheaves as \say{eigenvectors} of the Hecke operators. These Hecke eigensheaves form a category, whihc we shall denote by $\Eig\Shv^1_{\bar{\Q}_{\ell}}(\Bun_{\GL_1}(X))$. The main theorem concerning these objects is that there is an equivalence of categories:
            $$\Shv^{\ad, 1}_{\underline{\bar{\Q}_{\ell}}}(X) \cong \Eig\Shv^1_{\bar{\Q}_{\ell}}(\Bun_{\GL_1}(X))$$
        which maps each local system $\calL \in \Shv^{\ad, 1}_{\underline{\bar{\Q}_{\ell}}}(X)$ to a Hecke eigensheaf $\Autom(\calL) \in \Eig\Shv^1_{\bar{\Q}_{\ell}}(\Bun_{\GL_1}(X))$ with eigenvalue $\calL$. Finally, by putting the Galois Side and Automorphic Side together, one obtains an equivalence of categories:
            $$\Rep_{\bar{\Q}_{\ell}}^1(\pi_1^{\ab}(X_{\fet}))^{\cont} \cong \Eig\Shv^1_{\bar{\Q}_{\ell}}(\Bun_{\GL_1}(X))$$
        which relates Galois representations and automorphic forms in a concise manner. This is the version of global class field theory that we seek, and it tells us that $1$-dimensional continuous $\ell$-adic Galois representations are the same as automorphic forms associated to $\GL_1$.
        
    \subsection{Conventions}
        \begin{convention}[Category theory] \label{conv: category_theory}
            \noindent
            \begin{itemize}
                \item \textbf{(Fundamentals):} We assume familiarity with the notion of categories along with fundamental categorical concepts such as universal properties, (co)slice categories, (co)limits, and adjunctions, etc. \cite{maclane} and \cite[\href{https://stacks.math.columbia.edu/tag/0011}{Tag 0011}]{stacks} shall be our main references regarding these notions.
                \item \textbf{(Sheaf theory):} We shall assume to have theory of sheaves of sets (i.e. the theory of sheaf topoi) and of sheaves taking values in tensor categories such as module categories (cf. \cite{EGNO}) at our disposal. Readers who are not too familiar with the former are encouraged to consult \cite{sga4} and \cite[\href{https://stacks.math.columbia.edu/tag/00UZ}{Tag 00UZ}]{stacks}; for the latter, we recommend \cite[\href{https://stacks.math.columbia.edu/tag/006A}{Tag 006A}, \href{https://stacks.math.columbia.edu/tag/01AC}{Tag 01AC}, and \href{https://stacks.math.columbia.edu/tag/03A4}{Tag 03A4}]{stacks}. Also related is descent theory, for which we shall refer to \cite{vistoli_descent}, \cite[section C2.1]{elephant1}, \cite[Chapter III]{sheaves_in_geometry_and_logic}, and \cite[\href{https://stacks.math.columbia.edu/tag/0266}{Tag 0266} and \href{https://stacks.math.columbia.edu/tag/0238}{Tag 0238}]{stacks}.
                \item \textbf{(Monoidal categories):} For information on monoidal categories, we refer the readers to \cite{EGNO}.
            \end{itemize}
        \end{convention}
        
        \begin{convention}[Algebraic geometry] \label{conv: algebraic_geometry}
            \noindent
            \begin{itemize}
                \item \textbf{(Schemes):} For generalities on schemes, we refer the reader to \cite[Chapters II and III]{hartshorne}, as well as \cite[\href{https://stacks.math.columbia.edu/tag/01H8}{Tag 01H8}, \href{https://stacks.math.columbia.edu/tag/01QL}{Tag 01QL}, and \href{https://stacks.math.columbia.edu/tag/0209}{Tag 0209}]{stacks}.
                
                One notion that is indispensible for us is that of base change, for which we introduce the following short-hand: for any base scheme $S$, and any pair of $S$-schemes $X, T$, the base change $X \x_S T$ shall be denoted by $X_T$.
                \item \textbf{(Topologies on schemes):} For information on topologies on the category of schemes, we refer the reader to \cite[\href{https://stacks.math.columbia.edu/tag/0214}{Tag 0214}]{stacks}. For a general categorical treatment of descent theory, we refer the reader to \cite{vistoli_descent}, \cite[section C2.1]{elephant1}, \cite[Chapter III]{sheaves_in_geometry_and_logic}, and \cite[\href{https://stacks.math.columbia.edu/tag/0266}{Tag 0266} and \href{https://stacks.math.columbia.edu/tag/0238}{Tag 0238}]{stacks}.
                \item \textbf{(Homological categories):} If $\Lambda$ is a commutative ring and $X$ is a scheme then we shall write $\Shv_{\Lambda}(X)$ for the category of sheaves of $\Lambda$-modules on $X$ (a notably exceptional use of this notation is $\Shv_{\underline{\bar{\Q}_{\ell}}}(X)$, which means the category of $\ell$-adic local systems on $X$). $\Shv_{\underline{\Lambda}}(X)$ shall be used to denote the category of locally constant sheaves of $\Lambda$-modules on $X$, whereas $\Shv_{\Lambda}^c(X)$ shall be used to denote the category of \href{https://stacks.math.columbia.edu/tag/05BE}{\underline{constructible}} sheaves of $\Lambda$-modules on $X$, which admits $\Shv_{\underline{\Lambda}}(X)$ as a full subcategory.
            \end{itemize}
        \end{convention}