\section{Introduction}
    \subsection{Using algebraic geometry to understand class field theory}
        Our starting point is the classical following result, which hints at why one might even suspect any sort of involvement of algebraic geometry in the first place:
        \begin{lemma}[Varieties and field extensions] \label{lemma: varieties_and_field_extensions}
            \cite[\href{https://stacks.math.columbia.edu/tag/0BXN}{Tag 0BXN}]{stacks} Let $k$ be a field. Then, $\trdeg_k K_X = \dim X$ for all varieties $X/k$, and there exists a canonical equivalence of categories as follows:
                $$\{\text{Finite-type field extensions $K/k$ and $k$-algebra homomorphisms}\}^{\op}$$
                $$\cong$$
                $$\{\text{Varieties $X/k$ and \href{https://stacks.math.columbia.edu/tag/01RI}{\underline{dominant}} \href{https://stacks.math.columbia.edu/tag/01RR}{\underline{rational}} maps}\}$$
        \end{lemma}
            \begin{proof}
                If $f: X \to Y$ is a dominant map of varieties and if $\eta_X$ and $\eta_Y$ are the unique generic points of $X$ and $Y$ (unique because varieties are integral by definition, and every integral scheme \textit{a priori} has a unique generic point), then $f(\eta_X) = \eta_Y$ per the definition of dominant morphisms. The residue field at generic points are precisely the function fields (consider the stalk of the structure sheaves at the generic points to see why this is the case), so we have obtained a map of function fields $K_Y \to K_X$. We leave the proof of finiteness up to the reader.
            \end{proof}
        Through lemma \ref{lemma: varieties_and_field_extensions}, one obtains the following regarding the relationship between curves (i.e. algebraic varieties of dimension $1$) and their function fields (which \textit{a priori} are of transcedence degree $1$ over the ground field) with little difficulty:
        \begin{proposition}[Curves and function fields] \label{prop: curves_and_function_fields}
            \cite[\href{https://stacks.math.columbia.edu/tag/0BY1}{Tag 0BY1}]{stacks} For any field $k$, one has the following canonical equivalences of categories:
                $$\{\text{Field extensions $K/k$ of transcendence degree $1$ and $k$-algebra homomorphisms}\}^{\op}$$
                $$\cong$$
                $$\{\text{Curves $X/k$ and dominant rational maps}\}$$
                $$\cong$$
                $$\{\text{Non-singular projective curves $X/k$ and dominant rational maps}\}$$
        \end{proposition}
            \begin{proof}
                The first equivalence is an obvious consequence of lemma \ref{lemma: varieties_and_field_extensions}. To show that the second equivalence holds, note firstly that there is an evident fully faithful functor from the third category to the second; we shall need to show that this functor is also essentially surjective. For this, simply recall that for each curve $X/k$, there exists a non-singular projective curve $\tilde{X}/k$ that is birational to $X/k$, namely $X^{\nu} \cup \{\infty_1, ..., \infty_n\}$, the normalisation $X^{\nu}/k$ of $X/k$ with finitely many extra points (recall also that any normal Noetherian scheme of dimension $\leq 1$ is \textit{a priori} non-singular; cf. \cite[\href{https://stacks.math.columbia.edu/tag/0BX2}{Tag 0BX2}]{stacks}).
            \end{proof}
        
    \subsection{Conventions}
        \begin{convention}[Category theory] \label{conv: category_theory}
            \noindent
            \begin{itemize}
                \item \textbf{(Fundamentals):} We assume familiarity with the notion of categories along with fundamental categorical concepts such as universal properties, (co)slice categories, (co)limits, and adjunctions, etc. \cite{maclane} and \cite[\href{https://stacks.math.columbia.edu/tag/0011}{Tag 0011}]{stacks} shall be our main references regarding these notions.
                \item \textbf{(Sheaf theory):} We shall assume to have theory of sheaves of sets (i.e. the theory of sheaf topoi) and of sheaves taking values in tensor categories such as module categories (cf. \cite{EGNO}) at our disposal. Readers who are not too familiar with the former are encouraged to consult \cite{sga4} and \cite[\href{https://stacks.math.columbia.edu/tag/00UZ}{Tag 00UZ}]{stacks}; for the latter, we recommend \cite[\href{https://stacks.math.columbia.edu/tag/006A}{Tag 006A}, \href{https://stacks.math.columbia.edu/tag/01AC}{Tag 01AC}, and \href{https://stacks.math.columbia.edu/tag/03A4}{Tag 03A4}]{stacks}. Also related is descent theory, for which we shall refer to \cite{vistoli_descent} and \cite[\href{https://stacks.math.columbia.edu/tag/0266}{Tag 0266} and \href{https://stacks.math.columbia.edu/tag/0238}{Tag 0238}]{stacks}.
                \item \textbf{(Monoidal categories):} For information on monoidal categories, we refer the readers to \cite{EGNO}.
            \end{itemize}
        \end{convention}
        
        \begin{convention}[Algebraic geometry] \label{conv: algebraic_geometry}
            \noindent
            \begin{itemize}
                \item \textbf{(Schemes):} For generalities on schemes, we refer the reader to \cite[Chapters II and III]{hartshorne}, as well as \cite[\href{https://stacks.math.columbia.edu/tag/01H8}{Tag 01H8}, \href{https://stacks.math.columbia.edu/tag/01QL}{Tag 01QL}, and \href{https://stacks.math.columbia.edu/tag/0209}{Tag 0209}]{stacks}.
                
                One notion that is indispensible for us is that of base change, for which we introduce the following short-hand: for any base scheme $S$, and any pair of $S$-schemes $X, T$, the base change $X \x_S T$ shall be denoted by $X_T$.
                \item \textbf{(Topologies on schemes):} For information on the various kinds of commonly used Grothendieck topologies on the category of schemes, we refer the reader to \cite[\href{https://stacks.math.columbia.edu/tag/020K}{Tag 020K} and \href{https://stacks.math.columbia.edu/tag/0238}{Tag 0238}]{stacks}].
            \end{itemize}
        \end{convention}