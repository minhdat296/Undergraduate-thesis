\section{The Galois Side}
    \subsection{\'Etale fundamental groups}
        \subsubsection{Construction of \'etale fundamental groups}
            \'Etale fundamental groups of schemes as in \cite[Expos\'e V]{SGA1} are constructed using so-called \textbf{finite Galois categories}: in particular, for any fixed scheme $X$, the category of finite \'etale $X$-schemes is an instance of a finite Galois category, for whom the fundamental group is the group of natural automorphisms of the \textbf{fibre functor} based at a geometric point $\bar{x} \in X$:
                $$\Fib_{\bar{x}}: (\Sch_{/X})_{\fet} \to \Fin\Sets$$
                $$Y \mapsto Y(\bar{x})$$
            Below shall be a brief discussion of the construction (remark \ref{remark: finite_etale_schemes}) and key properties of \'etale fundamental groups of schemes (propositions \ref{prop: etale_fundamental_groups_do_not_depend_on_base_points} and \ref{prop: etale_homotopy_exact_sequence}), accompanied by certain relevant examples (cf. examples \ref{example: etale_fundamental_group_of_a_field} and \ref{example: etale_fundamental_group_of_a_curve}). For the sake of brevity, we have delegated the fine technicalities of this construction to \cite{MATH525_covering_spaces_and_fundamental_group_project}. The reader might also wish to consult \cite[Expos\'e V]{SGA1}, which was where \'etale fundamental groups of schemes were first discussed, as well as the more modern account in \cite[\href{https://stacks.math.columbia.edu/tag/0BQ6}{Tag 0BQ6}]{stacks}.
            
            \begin{definition}[Finite Galois categories] \label{def: finite_galois_categories}
                \noindent
                \begin{itemize}
                    \item \textbf{(Finite Galois categories):} A \textbf{finite Galois category} is defined via the data contained in a pair $(\calG, F)$ consisting of:
                    \begin{itemize}
                        \item a \textit{finitely complete and finitely cocomplete} small category $\calG$, wherein objects can all be written as finite coproducts of \textit{connected} objects\footnote{Objects $Y \in \calG$ such that the copresheaf $h^Y: \calG \to \Sets$ preserves all coproducts.}, and
                        \item a \textit{pro-representable} functor $F: \calG \to \Fin\Sets$, called the \textbf{fibre functor}, which we shall require to be exact and to reflect isomorphisms.
                    \end{itemize}
                    \item \textbf{(Galois objects):} An object $Y$ of a finite Galois category $\calG$ is a \textbf{Galois object} if and only if $Y/\Aut(Y)$ is terminal as an object of $\calG$ (which exists as $\calG$ is finitely complete by definition).
                    \item \textbf{(Galois functors):} A \textbf{Galois functor} is an exact functor $\Phi: \calG \to \calG'$ between finite Galois categories $(\calG, F), (\calG', F')$ which preserves connected objects and commute with the fibre functors in the following manner:
                        $$
                            \begin{tikzcd}
                            	\calG && {\calG'} \\
                            	& \Fin\Sets
                            	\arrow["F"', from=1-1, to=2-2]
                            	\arrow["{F'}", from=1-3, to=2-2]
                            	\arrow["\Phi", from=1-1, to=1-3]
                            \end{tikzcd}
                        $$
                \end{itemize}
            \end{definition}
            \begin{definition}[Fundamental groups of finite Galois categories] \label{def: fundamental_groups_of_finite_galois_categories}
                The \textbf{fundamental group} of a given finite Galois category $(\calG, F)$, denoted by $\pi_1(\calG, F)$, is defined to be $\Aut(F)$.
            \end{definition}
            \begin{remark}[Finite-\'etale schemes] \label{remark: finite_etale_schemes}
                For any given by scheme $X$, the small category $(\Sch_{/X})_{\fet}$ of finite-\'etale $X$-schemes is a category wherein:
                    \begin{itemize}
                        \item all finite limits and all finite colimits exist, and
                        \item all objects can be written as a (possibly empty) finite coproduct of connected objects, which happen to be schemes that are \'etale over $X$.  
                    \end{itemize}
                (for a detailed proof, see \cite[\href{https://stacks.math.columbia.edu/tag/0BN9}{Tag 0BN9}]{stacks}) so should we be able to define a fibre functor $(\Sch_{/X})_{\fet} \to \Fin\Sets$, we will have succeeded in putting a finite Galois category structure on $(\Sch_{/X})_{\fet}$. As a matter of fact, such a well-defined fibre functor has good reasons to exist: it is an easy consequence of \cite[\href{https://stacks.math.columbia.edu/tag/00U3}{Tag 00U3}]{stacks} that for any fixed geometric point $\bar{x} \in X$ (corresponding to an algebraic closure\footnote{Certain sources consider geometric points to correspond to separable closures. For us, however, geometric points are algebraically closed fields $K$ so that $\Spec K$ be a Galois object of $(\Sch_{/\Spec K})_{\fet}$. In practice this choice usually does not matter, since we will mostly work over perfect field, and separable closures of perfect fields are algebraically closed.} $\bar{\kappa}_x$ of the residue field of $x \in X$), one has:
                    $$(\Spec \bar{\kappa}_x)_{\fet} \cong \Fin\Sets$$
                (the forward direct simply involves taking the underlying set, and the inverse functors is given by $I \mapsto \coprod_{i = 1}^{|I|} \Spec \bar{\kappa}_x$) and so for any $k$-scheme $X$, one has the following canonical defined functor:
                    $$(\Sch_{/X})_{\fet} \to (\Sch_{/\Spec \bar{\kappa}_x})$$
                    $$Y \mapsto Y_{\bar{x}}$$
                where $Y_{\bar{x}} \cong Y \x_X \Spec \bar{\kappa}_x$; one can then take the underlying set of $Y_{\bar{x}}$ to get the following trivially left-exact\footnote{... and hence pro-representable (cf. \cite[Proposition 3.1]{grothendieck_fga_2}).} functor:
                    $$\Fib_{\bar{x}}: (\Sch_{/X})_{\fet} \to \Fin\Sets$$
                    $$Y \mapsto |Y_{\bar{x}}|$$
                We should also verify that the sets $|Y_{\bar{x}}|$ are indeed finite. To this end, let us first apply the fact that pullbacks of \'etale morphisms are \'etale to see that if $Y$ is affine over $X$ then $Y_{\bar{x}}$ will have to be the spectrum of an \'etale $\bar{\kappa}_x$-algebra; however, according to \cite[\href{https://stacks.math.columbia.edu/tag/00U3}{Tag 00U3}]{stacks}, this means that $Y_{\bar{x}} \cong \Spec (\bar{\kappa}_x)^{\oplus N}$ for some finite $N$. The locality of \'etale-ness and the finiteness of $Y$ as an $X$-scheme then tells us that in general, $Y_{\bar{x}}$ must be a finite disjoint union of affine schemes of the form $\Spec (\bar{\kappa}_x)^{\oplus N}$, meaning that $Y_{\bar{x}} \cong \Spec (\bar{\kappa}_x)^{\oplus N'}$ for some finite $N'$. The set $|Y_{\bar{x}}|$ is therefore always finite. One also sees that an immediate consequence of this proof is that $\Fib_{\bar{x}}$ necessarily \textit{reflects isomorphisms} and is \textit{right-exact}. 
            \end{remark}
            
            \begin{definition}[\'Etale fundamental groups] \label{def: etale_fundamental_groups}
                For any scheme $X$ with a fixed geometric point $\bar{x}$, the pair $((\Sch_{/X})_{\fet}, \Fib_{\bar{x}})$ as in remark \ref{remark: finite_etale_schemes} defines a finite Galois category in the sense of \cite[Section 1]{MATH525_covering_spaces_and_fundamental_group_project}. Its fundamental group is commonly denoted by $\pi_1(X_{\fet}, \bar{x})$ and called the \textbf{\'etale fundamental group} of $X$ based at $\bar{x}$.
            \end{definition}
            
            \begin{theorem}[The Geometric Galois Correspondence] \label{theorem: geometric_galois_correspondence}
                For $(X, \bar{x})$ a pointed connected scheme, there is an equivalence of categories:
                    $$\{\text{Finite \'etale $X$-schemes $Y$ with base points $\bar{y}$ lying over $\bar{x}$}\}$$
                    $$\cong$$
                    $$\{\text{Finite-index subgroups of $\pi_1(X_{\fet}, \bar{x})$}\}$$
                Furthermore, this equivalence restricts down to:
                    $$\{\text{Finite \'etale Galois $X$-schemes $Y$ with base points $\bar{y}$ lying over $\bar{x}$}\}$$
                    $$\cong$$
                    $$\{\text{Finite-index normal subgroups of $\pi_1(X_{\fet}, \bar{x})$}\}$$
            \end{theorem}
            \begin{corollary}
                Should $H$ be a finite-index normal subgroup of $\pi_1(X_{\fet}, \bar{x})$ and $(X^H, \bar{x}^H)$ be the corresponding Galois $X$-scheme with a choice of base point $\bar{x}^H$ lying over $\bar{x}$, then $\pi_1(X^H_{\fet}, \bar{x}^H) \cong H$. 
            \end{corollary}
            \begin{example}[The \'etale fundamental group of a field] \label{example: etale_fundamental_group_of_a_field}
                As a sanity check, note that if $K$ is a field then finite-\'etale Galois schemes over $\Spec K$ shall be of the form $\Spec L \to \Spec K$, where $L/K$ is a finite Galois extension, and as a consequence, there are there are the following equivalences of lattices, which demonstrate that theorem \ref{theorem: geometric_galois_correspondence} directly generalises the classical Galois Correspondence:
                    $$\{\text{Finite-index normal subgroups of $\pi_1((\Spec K)_{\fet})$}\}$$
                    $$\cong$$
                    $$\{\text{Finite \'etale Galois schemes over $\Spec K$}\}$$
                    $$\cong$$
                    $$\{\text{Finite Galois extensions of $K$}\}^{\op}$$
                    $$\cong$$
                    $$\{\text{Finite-index normal subgroups of $\Gal(\bar{K}/K)$}\}$$
            \end{example}
            \begin{example}[The \'etale fundamental group of a curve] \label{example: etale_fundamental_group_of_a_curve}
                Let $k$ be a field. If $X$ is a connected non-singular projective curve over $\Spec k$ with function field $K$, then there is a canonical equivalence $({}^{K/}\Fld^{\fin, \Gal})^{\op} \cong (\Sch_{/X})_{\fet}^{\Gal}$ between the lattice of finite Galois extensions of $K$ and Galois $X$-schemes (which are precisely dominant rational maps whose associated function field extensions are Galois). Through this, it is easy to see that:
                    $$\pi_1(X_{\fet}) \cong \Gal(\bar{K}/K)$$
                For instance, we have:
                    $$\pi_1((\P^1_k)_{\fet}) \cong \Gal(\bar{k}/k)$$
                (since the function field of $\P^1_k$ is $k(t)$), which tells us that $\P^1_k$ is simply \'etale-connected if and only if $k$ is algebraically closed (since $\Gal(\bar{k}/k)$ is \textit{a fortiori} trivial in that case). 
            \end{example}
        
        \subsubsection{Properties of \'etale fundamental groups}
            Now, let us make sure that the \'etale fundamental group $\pi_1(X_{\fet}, \bar{x})$ as defined in definition \ref{def: etale_fundamental_groups} is meaningful as a formal construction. Namely, we would like to know the behaviours of $\pi_1(X_{\fet}, \bar{x})$ when we change the base point and when we base-change (cf. proposition \ref{prop: etale_fundamental_groups_do_not_depend_on_base_points}), as well as whether or not \'etale fibrations induce homotopy exact sequences of fundamental groups (cf. proposition \ref{prop: etale_homotopy_exact_sequence}). 
            \begin{proposition}[\'Etale fundamental group do not depend on base points] \label{prop: etale_fundamental_groups_do_not_depend_on_base_points}
                \cite[\href{https://stacks.math.columbia.edu/tag/0BQA}{Tag 0BQA}]{stacks} Let $f: Y \to X$ be a morphism of connected qcqs\footnote{quasi-compact and quasi-separated} schemes such that the base change functor $X' \mapsto X' \x_X Y$ is an equivalence of Galois categories between $(\Sch_{/X})_{\fet}$ and $(\Sch_{/Y})_{\fet}$. Then, for any choice of geometric points $\bar{x} \in X$ and $\bar{y} \in Y$, one has the following isomorphism of \'etale fundamental groups $\pi_1(X_{\fet}, \bar{x}) \cong \pi_1(Y_{\fet}, \bar{y})$.
            \end{proposition}
            \begin{corollary}[Uniqueness of \'etale fundamental groups] \label{coro: etale_fundamental_group_uniqueness}
                For any connected qcqs scheme $X$ and any pair of possibly distinct geometric points $\bar{x}, \bar{x}' \in X$, one has any isomorphism of \'etale fundamental groups $\pi_1(X_{\fet}, \bar{x}) \cong \pi_1(X_{\fet}, \bar{x}')$, and therefore it makes sense to only speak of \textit{the} fundamental group of $X$, which we shall denote by $\pi_1(X_{\fet})$.
            \end{corollary}
            
            \begin{proposition}[The \'etale homotopy exact sequence] \label{prop: etale_homotopy_exact_sequence}
                \cite[\href{https://stacks.math.columbia.edu/tag/0C0J}{Tag 0C0J}]{stacks} Let $X$ be a connected scheme. If $f: Y \to X$ be a flat proper morphism of finite presentation whose geometric fibres $Y_{\bar{x}}$ are connected and reduced, then for any geometric point $\bar{x} \in X$, there exists a right-exact sequence of groups as follows:
                    $$\pi_1((Y_{\bar{x}})_{\fet}) \to \pi_1(Y_{\fet}) \to \pi_1(X_{\fet}) \to 1$$
            \end{proposition}
    
    \subsection{\texorpdfstring{$\ell$}{}-adic sheaves and Grothendieck's Galois Theory}
        The process of geometrising class field theory begins with the geometrisation of Galois representations, which thanks to a combination of proposition \ref{prop: curves_and_function_fields} and theorem \ref{theorem: geometric_galois_correspondence} are more or less the same as representations of \'etale fundamental groups of schemes. As such, we seek a geometrisation of representations of \'etale fundamental groups and this will be done via establishing a connection between said representations and a certain kind of abelian sheaves - called \textbf{lisse $\bar{\Q}_{\ell}$-adic sheaves} - on \say{the curve of global class field theory} (cf. convention \ref{conv: automorphic_side_conventions}); our efforts shall culminate in theorem \ref{theorem: galois_representations_are_lisse_sheaves}. In addition, we shall be collecting several facts about the algebra of lisse $\bar{\Q}_{\ell}$-adic sheaves for later use. In particular, we want to keep in mind that lisse $\bar{\Q}_{\ell}$-adic sheaves behave well around tensor products as well as pullbacks and pushforwards.  
    
        \subsubsection{Artin-Rees categories and adic sheaves}
            We begin by building our way up to the notion of \textbf{lisse $\bar{\Q}_{\ell}$-adic sheaves}, which are essential for the statement of the main theorem of this section, namely theorem \ref{theorem: galois_representations_are_lisse_sheaves}. For subtle technical reasons, this involves some abstraction in the form of so-called \textbf{Artin-Rees categories}, which are direct generalisations of categories of adically complete modules over Noetherian rings. Our main reference is \cite[Subsection 1.4]{conrad_etale_cohomology} and \cite[Expos\'e V]{sga5}, which was where the constructions below first appeared.
        
            \begin{definition}[Artin-Rees categories] \label{def: artin_rees_categories}
                The \textbf{Artin-Rees category} associated to an abelian category $\calA$ is the full subcategory of $\calA_{\bullet} := \Pro(\calA)$ spanned by cofiltered diagrams $\{M_n\}_{n \in \Z}$; we denote it by $\calA_{\bullet}^{\AR}$. Of particular interest are the so-called \textbf{null systems}, which are objects $\{M_n\}_{n \in \Z} \in \calA_{\bullet}^{\AR}$ such that there exists $\nu \in \N$ so that for all $n \in \Z$ the morphism $M_n \to M_{n + \nu}$ is zero.
            \end{definition}
            
            \begin{proposition}[Artin-Rees categories are linear and abelian] \label{prop: artin_rees_categories_are_linear_and_abelian}
                \cite[Expos\'e V, Propositions 2.2.2 et 2.4.1]{sga5} For any (locally finite) $\Lambda$-linear\footnote{I.e. if hom-sets of $\calA$ are (locally finite) $\Lambda$-modules (e.g. when $\calA \cong \Lambda\mod$).} abelian category $\calA$, the associated Artin-Rees category $\calA_{\bullet}^{\AR}$ is also a (locally finite) $\Lambda$-linear abelian category, with zero objects being the null systems. In fact, the Artin-Rees category $\calA_{\bullet}^{\AR}$ is the localisation\footnote{In the sense of \cite[\href{https://stacks.math.columbia.edu/tag/02MS}{Tag 02MS}]{stacks}.} of $\calA_{\bullet}$ at the thick\footnote{Cf. \cite[\href{https://stacks.math.columbia.edu/tag/02MO}{Tag 02MO}]{stacks}.} subcategory of null systems, meaning that an isomorphism in $\calA_{\bullet}^{\AR}$ (henceforth referred to as an \textbf{AR-isomorphism}) is a morphism in $\calA_{\bullet}$ whose kernel and cokernel are null. 
            \end{proposition}
            
            \begin{convention}[The setting for adic sheaves] \label{conv: l_adic_sheaves_conventions}
                For our purposes, $\calX$ shall be a scheme that is locally of finite type\footnote{Althought $\calX$ might actually be an algebraic stack of finite type over $S$; for details, see \cite{laszlo_olsson_adic_sheaves_on_artin_stacks_1} and \cite{laszlo_olsson_adic_sheaves_on_artin_stacks_2}. It should also be noted that in \cite[Subsection 1.4]{conrad_etale_cohomology}, it was only required that $\calX$ would be Noetherian, which is not sufficient for us, as $\Bun_{\GL_1}(X)$ is merely locally of finite type, and hence only locally Noetherian \textit{a priori}.} over a base scheme $S$ that is affine, regular, Noetherian\footnote{Note that this implies that $\calX$ is locally Noetherian (cf. \cite[\href{https://stacks.math.columbia.edu/tag/01T6}{Tag 01T6}]{stacks}).} and of dimension $\leq 1$, and of characteristic $p \geq 0$; moreoever, we would like to work under the assumption that every finite-type $S$-scheme $T$ is also of finite cohomological dimension. In addition, $\Lambda$ shall be a discrete valuation ring of mixed characteristic $(0, \ell)$ (for some auxiliary prime $\ell \not = p$) with maximal ideal $\m$, and fraction field $E$.
            \end{convention}
            
            \begin{definition}[Torsion objects in tensor categories] \label{def: torsion_objects_in_tensor_categories}
                Let $A$ be a commutative ring, $I$ be an ideal of $A$, and $\calA$ be an $A$-linear category. Then, the subcategory of $\calA$ spanned by $I$-torsion objects is the one wherein the hom-sets are $\Hom_{\calA/I}(M, N) \cong \Hom_\calA(M, N) \tensor_A A/I$.
            \end{definition}
            \begin{definition}[Adic objects and lisse objects of Artin-Rees categories] \label{def: adic_objects_and_lisse_objects_of_artin_rees_categories}
                Consider the Artin-Rees category $\calA_{\bullet}^{\AR}$ associated to a locally finite $\Lambda$-linear abelian category $\calA$. 
                    \begin{enumerate}
                        \item \textbf{(Adic objects):} $\calA_{\bullet}^{\AR}$ admits a full subcategory, denoted by $\calA_{\bullet}^{\ad}$, whose objects $\{M_n\}_{n \in \Z}$ are such that:
                            \begin{itemize}
                                \item $M_n \cong 0$ for all $n < 0$,
                                \item $M_n$ is $\m^{n + 1}$-torsion for all $n \geq 0$, and
                                \item the canonical maps $\Lambda/\m^{n + 2} \tensor_{\Lambda} \Hom_{\calA_{\bullet}^{\AR}}(M_{\bullet}, N_{\bullet}) \to \Hom_{\calA_{\bullet}^{\AR}}(M_{\bullet}, N_{\bullet})/\m^n$ are isomorphisms of $\Lambda/\m^{n + 1}$-modules for all $n \geq 0$.
                            \end{itemize}
                        Objects of this full subcategory are said to be \textbf{$\m$-adic}.
                        \item \textbf{(Lisse objects):} Let $\calA_{\bullet}^{\fin}$ denote the category of Artin-Rees projective systems of objects of $\calA$ which are simultaneously Artinian and Noetherian. Objects of the category\footnote{Here, the intersection is understood to be at both the level of objects and that of morphisms.} $\calA_{\bullet}^{\lisse} := \calA_{\bullet}^{\ad} \cap \calA_{\bullet}^{\fin}$ are then said to be \textbf{lisse}. 
                    \end{enumerate}
            \end{definition}
            \begin{proposition}[Adic categories are linear and abelian] \label{prop: adic_categories_are_linear_and_abelian}
                Let $\calA$ be a locally finite $\Lambda$-linear abelian category. Then, we shall have a tower $\calA_{\bullet}^{\lisse} \subset \calA_{\bullet}^{\ad} \subset \calA_{\bullet}^{\AR}$ of locally finite $\Lambda$-linear abelian categories, wherein the inclusions are fully faithful exact functors.
            \end{proposition}
            \begin{example}[Adic and lisse $\Lambda$-sheaves] \label{example: adic_sheaves}
                Recall first of all that the category of constructible \'etale sheaves of $\Lambda$-modules on a Noetherian scheme - of which the category $\Lambda\mod^{\cons}(\calX_{\et})$ of constructible sheaves of $\Lambda$-modules on $\calX_{\et}$ is a special case - is a locally finite $\Lambda$-linear closed monoidal category (to see why, combine \cite[Propositions 3.20 and 3.22]{behrend_l_adic_sheaves_for_algebraic_stacks}). Then, the category of adic constructible $\m$-adic sheaves on $\calX$ (also called constructible $\Lambda$-sheaves), commonly denoted by $\Shv_{\Lambda}^{\ad}(\calX)$, is nothing but $\Lambda\mod^{\cons}(\calX_{\et})_{\bullet}^{\ad}$, and the category of lisse $\m$-adic sheaves on $\calX$, denoted by $\Shv_{\Lambda}^{\lisse}(\calX)$ is simply $\Lambda\mod^{\cons}(\calX_{\et})_{\bullet}^{\lisse}$.
            \end{example}
            
            \begin{definition}[$E$-objects] \label{def: E_objects}
                Let $\calA$ be a locally finite $\Lambda$-linear abelian category. Then, we can define the associated category of \textbf{$E$-objects} to be the localisation $\calA \tensor_{\Lambda} E$ at $E$-linear morphisms, i.e. we define:
                    $$\Hom_{\calA \tensor_{\Lambda} E}(M, N) \cong \Hom_{\calA}(M, N) \tensor_{\Lambda} E$$
                It can be easily verified that $\calA \tensor_{\Lambda} E$ is locally finite $E$-linear and abelian.
            \end{definition}
            \begin{definition}[$\bar{E}$-objects] \label{def: bar_E_objects}
                Let $\calA$ be a locally finite $\Lambda$-linear abelian category. We define the associated category $\calA \tensor_{\Lambda} \bar{E}$ of \textbf{$\bar{E}$-objects} via:
                    $$\Hom_{\calA \tensor_{\Lambda} \bar{E}}(M, N) \cong \underset{\text{$E'/E$ finite extensions}}{\colim} \Hom_{\calA}(M, N) \tensor_{\Lambda} E'$$
                Using the fact that associated categories of $E$-objects are abelian and $E$-linear, one can show that associated categories of $\bar{E}$-objects are abelian and $\bar{E}$-linear\footnote{The one technicality to keep in mind is that for finite-dimensional vector spaces, filtered colimits commute with kernels.}. 
            \end{definition}
            \begin{example}[Adic and lisse $E$-sheaves and $\bar{E}$-sheaves] \label{example: E_sheaves}
                Because $\Shv_{\Lambda}^{\ad}(\calX)$ and $\Shv_{\Lambda}^{\lisse}(\calX)$ are locally finite $\Lambda$-linear and abelian (cf. proposition \ref{prop: adic_categories_are_linear_and_abelian}), one can define the categories of constructible adic $E$-sheaves and that of lisse $E$-sheaves on $\calX$ to be $\Shv_E^{\ad}(\calX) \cong \Shv_{\Lambda}^{\ad}(\calX) \tensor_{\Lambda} E$ and $\Shv_E^{\lisse}(\calX) \cong \Shv_{\Lambda}^{\lisse}(\calX) \tensor_{\Lambda} E$. Adic and lisse $\bar{E}$-sheaves can thus also be defined.
            \end{example}
            
        \subsubsection{Operations with adic sheaves}
            The following results shall be used throughout the rest of the paper without any explicit mention (that is, with the notable exception of theorem \ref{theorem: galois_representations_are_lisse_sheaves}). For details, we refer the reader to \cite[Sections 6-8]{laszlo_olsson_adic_sheaves_on_artin_stacks_2} (which is a sequel to \cite{laszlo_olsson_adic_sheaves_on_artin_stacks_1}) and \cite[Sections II.7-II.10]{kiehl_weissauer_weil_conjecture_perverse_sheaves_and_l_adic_fourier_transform}.
            
            \begin{proposition}[Tensor products of constructible adic objects] \label{prop: tensor_products_of_constructible_adic_objects}
                \noindent
                \begin{enumerate}
                    \item \cite[Proposition 6.1]{laszlo_olsson_adic_sheaves_on_artin_stacks_2} For any locally finite $\Lambda$-linear closed monoidal abelian category $(\calA, \tensor, \1)$, the associated categories $\calA_{\bullet}^{\ad}$ and $\calA_{\bullet}^{\lisse}$ of adic and lisse systems are monoidal with respect to term-wise tensor products $M_{\bullet} \tensor N_{\bullet} \cong \{M_n \tensor N_n\}_{n \in \Z}$. In fact, they both embed via fully faithful monoidal exact $\Lambda$-linear functors into $\calA_{\bullet}^{\AR}$, which is also monoidal with respect to term-wise tensor products. 
                    \item \cite[Theorem III.12.2 and Appendix A]{kiehl_weissauer_weil_conjecture_perverse_sheaves_and_l_adic_fourier_transform} Furthermore, $\calA_{\bullet}^{\ad}$ and $\calA_{\bullet}^{\lisse}$ (respectively, $\calA_{\bullet}^{\ad} \tensor_{\Lambda} \bar{E}$ and $\calA_{\bullet}^{\lisse} \tensor_{\Lambda} \bar{E}$ for any choice of algebraic closure $\bar{E}/E$) are closed monoidal categories with respect to these tensor products.
                \end{enumerate}
            \end{proposition}
            \begin{example}[Tensor products of constructible adic sheaves] \label{def: tensor_products_of_constructible_adic_sheaves}
                $\Lambda\mod^{\cons}(\calX_{\et})$ is a locally finite closed monoidal category with respect to tensor products over the constant sheaf $\underline{\Lambda}$ (cf. \cite[\href{https://stacks.math.columbia.edu/tag/093P}{Tag 093P}]{stacks}), so by proposition \ref{prop: tensor_products_of_constructible_adic_objects}, the category $\Shv_{\bar{\Q}_{\ell}}^{\lisse}(\calX)$ of lisse $\bar{\Q}_{\ell}$ sheaves on $\calX$ shall be closed monoidal with respect to the constant sheaf $\underline{\bar{\Q}_{\ell}}$.
            \end{example}
            
            \begin{proposition}[$*$-pullbacks and $*$-pushforwards] \label{prop: *_pullbacks_and_pushforwards_of_l_adic_sheaves}
                For $f: \calX \to \calY$ a morphism of finite type between $S$-schemes that are locally of finite type (with $S$ as in convention \ref{conv: l_adic_sheaves_conventions}), there is an adjoint equivalence whose components $f^*$ and $f_*$ are computed term-wise as in \cite[\href{https://stacks.math.columbia.edu/tag/03PZ}{Tag 03PZ}]{stacks} and  \cite[\href{https://stacks.math.columbia.edu/tag/03PV}{Tag 03PV}]{stacks} respectively:
                    $$
                        \begin{tikzcd}
                        	{\Shv_{\bar{\Q}_{\ell}}^{\lisse}(\calX)} & {\Shv_{\bar{\Q}_{\ell}}^{\lisse}(\calY)}
                        	\arrow[""{name=0, anchor=center, inner sep=0}, "{f_*}"', bend right, from=1-1, to=1-2]
                        	\arrow[""{name=1, anchor=center, inner sep=0}, "{f^*}"', bend right, from=1-2, to=1-1]
                        	\arrow["\dashv"{anchor=center, rotate=-90}, draw=none, from=1, to=0]
                        \end{tikzcd}
                    $$
                Furthermore, the funcotrs $f^*$ and $f_*$ both commute with (external) tensor products of lisse $\bar{\Q}_{\ell}$-sheaves.
            \end{proposition}
                \begin{proof}
                    Combine \cite[Proposition 8.3]{laszlo_olsson_adic_sheaves_on_artin_stacks_2} with \cite[Theorem II.7.1]{kiehl_weissauer_weil_conjecture_perverse_sheaves_and_l_adic_fourier_transform}
                \end{proof}
            
            \begin{definition}[Stalks of constructible adic sheaves] \label{def: stalks_of_constructible_adic_sheaves}
                \cite[Definition 1.4.4.3]{conrad_etale_cohomology} Let $\bar{x} \in \calX$ be a geometric point and $\calF_{\bullet} \in \Shv_{\bar{\Q}_{\ell}}^{\ad}(\calX)$ be a constructible $\bar{\Q}_{\ell}$-sheaf on $\calX$. Then, the \textbf{stalk} at $\bar{x}$ shall be given by $\underset{n \in \N}{\lim} (\calF_n)_{\bar{x}}$, with each term $(\calF_n)_{\bar{x}}$ being computed as stalks of \'etale sheaves (cf. \cite[\href{https://stacks.math.columbia.edu/tag/040R}{Tag 040R}]{stacks}).
            \end{definition}
            \begin{remark}
                As tensor products and $*$-pullbacks/pushforwards of lisse $\bar{\Q}_{\ell}$ are computed term-wise, taking stalks of lisse $\bar{\Q}_{\ell}$-sheaves commutes with those operations.
            \end{remark}
            
            \begin{convention}[Continuous linear representations] \label{conv: continuous_linear_representations}
                From now on, if $G$ is a topological group and $E$ is a topological field then we will be writing $\Rep_E(G)$ for the category of \textit{continuous} $E$-linear representations of $G$. In fact, all representations shall be implicitly assumed to be continuous. Also, for all $n \geq 1$, $\Rep_E^n(G)$ shall be used to denote the subcategory of $n$-dimensional $E$-linear representations of $G$.
            \end{convention}
            \begin{theorem}[Galois representations are lisse $\bar{\Q}_{\ell}$-sheaves] \label{theorem: galois_representations_are_lisse_sheaves}
                \cite[Theorem 1.4.5.7]{conrad_etale_cohomology} Let $\calX$ be a connected Noetherian scheme and fix a geometric point $\bar{x} \in \calX$. Then, there exists a monoidal equivalence given by $\calF \mapsto \calF_{\bar{x}}$, from the symmetric monoidal category $\Shv_{\bar{\Q}_{\ell}}^{\lisse}(\calX)$ of lisse $\bar{\Q}_{\ell}$-sheaves to the symmetric monoidal category $\Rep_{\bar{\Q}_{\ell}}^{\fin}(\pi_1(\calX_{\fet}))$ of finite-dimensional continuous $\bar{\Q}_{\ell}$-linear representations of $\pi_1(X_{\fet})$.
            \end{theorem}