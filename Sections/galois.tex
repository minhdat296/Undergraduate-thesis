\section{The Galois Side}
    \subsection{The \'etale fundamental group}
        \begin{definition}[Galois categories and their fundamental groups] \label{def: finite_galois_categories}
            \noindent
            \begin{enumerate}
                \item 
                    \begin{itemize}
                        \item \textbf{(Finite Galois categories \cite[\href{https://stacks.math.columbia.edu/tag/0BMY}{Tag 0BMY}]{stacks}):} A \textbf{finite Galois category} is defined via the data contained in a pair $(\calG, F)$ consisting of:
                        \begin{itemize}
                            \item a \textit{finitely complete and finitely cocomplete} base category $\calG$, wherein objects can all be written as finite coproducts of connected objects\footnote{Objects $X \in \calG$ such that the copresheaf $\calG(X, -)$ preserves all coproducts.}.
                            \item a functor $F: \calG \to \Fin$\footnote{With $\Fin$ denoting the category of finite sets}, called the \textbf{fibre functor}, which we shall require to be exact and to reflect isomorphisms (i.e. for all bijections $Fx \cong Fy$ between finite sets, one has an isomorphism $x \cong y$ in $\calG$).
                        \end{itemize}
                        \item \textbf{(Galois objects):} An object $X$ of a (locally small) Galois category $\calG$ is a \textbf{Galois object} if and only if it has no non-trivial automorphisms, i.e. if and only if there is a bijection $\calG(\pt, X) \cong \Aut_{\calG}(X)$ (with $\pt$ a terminal object of $\calG$\footnote{Note that Galois categories must have terminal objects, as they are finitely complete and terminal objects are nothing but the limit of the empty diagram (which is finite by virtue of containing no vertices and no edges).})
                        \item An exact functor $\Phi: \calG \to \calG'$ between Galois categories $(\calG, F), (\calG', F')$ which preserves connected objects and commute with the fibre functors in the following manner:
                            $$
                                \begin{tikzcd}
                                	& \Fin \\
                                	\calG && {\calG'}
                                	\arrow["F", from=2-1, to=1-2]
                                	\arrow["{F'}"', from=2-3, to=1-2]
                                	\arrow["\Phi", from=2-1, to=2-3]
                                \end{tikzcd}
                            $$
                        is said to be a Galois functor.
                    \end{itemize}
                \item \textbf{(Fundamental groups):} In the sense of \cite[Theorem 2.16]{noohi_fundamental_group}, the \textbf{fundamental group} of a given Galois category $(\calG, F)$ is the group $\Aut(F)$ of natural automorphisms on thefibre functor $F: \calG \to \Fin$. We shall suggestively denote the fundamental group of a given Galois category by $\pi_1(\calG, F)$. 
            \end{enumerate}
        \end{definition}
        
        Before we state the next definition, let us observe that for any given base scheme $X$, the category $\Sch_{/X}^{\fet}$ of schemes finite and \'etale over $X$ is a category wherein:
            \begin{itemize}
                \item all finite limits and all finite colimits exist, and
                \item all objects can be written as a (possibly empty) finite coproduct of connected objects, which happen to be schemes that are \'etale over $X$.  
            \end{itemize}
        In other words, the category spanned by (possibly empty) finite coproducts of schemes \'etale over $X$ can serve as the base category of a finite Galois category. For a detailed proof, see \cite[\href{https://stacks.math.columbia.edu/tag/0BN9}{Tag 0BN9}]{stacks}. 
        \begin{definition}[\'Etale fundamental group] \label{def: etale_fundamental_groups}
            Let $X$ be a scheme with a fixed geometric point $\bar{x}$ and define the following fibre functor:
                $$F_{\bar{x}}: \Sch_{/X}^{\fet} \to \Fin$$
                $$Y \mapsto |Y_{\bar{x}}|$$
            (where $|Y_{\bar{x}}|$ denotes the underlying set of the fibre of the $X$-scheme $Y$ over $\bar{x}$) The pair $(\Sch_{/X}^{\fet}, F_{\bar{x}})$ as above thus define a finite Galois category. Its fundamental group $\Aut(F_{\bar{x}})$ is commonly denoted by $\pi_1(X_{\fet}, \bar{x})$ and called the \textbf{\'etale fundamental group} of $X$ based at $\bar{x}$.
        \end{definition}
        \begin{remark}
            Definition \ref{def: etale_fundamental_groups} is actually a bit subtle and honestly, somewhat ill-founded. For instance, it is not entirely clear that $F_{\bar{x}}$ is an honest-to-Grothendieck fibre functor. It is certainly left-exact, by virtue of being defined via pullbacks, and it is right-exact because any \'etale algebra over a field can be written as a finite direct sum of finite extensions of that field \cite[\href{https://stacks.math.columbia.edu/tag/00U3}{Tag 00U3}]{stacks}, and direct sums are biproducts of vector spaces. 
            
            However, the fact that the sets $|Y_{\bar{x}}|$ are finite is not really trivial, although it is not too hard to prove either. It is also a consequence of \'etale algebras being isomorphic to finite direct sums of finite extensions: in our case, because the separable closure $\kappa_x^{\alg}$ of the residue field of $x$ does not admit any non-trivial finite extension (since finite extensions are \textit{a priori} algebraic), the underlying vector space of any \'etale $\kappa_x^{\alg}$-algebra must be isomorphic to a finite direct sum of $\kappa_x^{\alg}$ itself. In terms of schemes, this means that when both $Y$ and $X$ are affine, the fibre $Y_{\bar{x}}$ is nothing but a coproduct of finitely many copies of $\Spec \kappa_x^{\alg}$ (i.e. of the geometric point $\bar{x}$), and hence the set $|Y_{\bar{x}}|$ has to be finite. Then, by using the fact the \'etale-ness is a local property, we can deduce that the set $|Y_{\bar{x}}|$ must be finite regardless of whether $Y$ and $X$ are finite or not. This proof also demonstrates that $F_{\bar{x}}$ necessarily reflects isomorphisms.
            
            The functor from definition \ref{def: etale_fundamental_groups}:
                $$F_{\bar{x}}: \Sch_{/X}^{\fet} \to \Fin$$
                $$Y \mapsto |Y_{\bar{x}}|$$
            is therefore a well-defined fibre functor.
        \end{remark}
        
        Now, let us make sure that the \'etale fundamental group $\pi_1(X_{\fet}, \bar{x})$ as defined in definition \ref{def: etale_fundamental_groups} actually makes sense topologically and geometrically. Namely, we would like to know the behaviours of $\pi_1(X_{\fet}, \bar{x})$ when we change the base point and when we base-change, as well as whether or not it is \say{insensitive} to (universal) homeomorphisms.
        \begin{proposition}[The \'etale fundamental group as a topological invariance] \label{prop: the_etale_fundamental_group_as_a_topological_invariance}
            \noindent
            \begin{enumerate}
                \item Let $f: Y \to X$ be a morphism of connected schemes such that the base change functor:
                    $$f^*: \Sch_{/X}^{\fet} \to \Sch_{/Y}^{\fet}$$
                    $$X' \mapsto X' \x_X Y$$
                is an equivalence. Then, for any choice of geometric points $\bar{x} \in X$ and $\bar{y} \in Y$, one has the following isomorphism of \'etale fundamental groups $\pi_1(X_{\fet}, \bar{x}) \cong \pi_1(Y_{\fet}, \bar{y})$.
                \item If $f: Y \to X$ is a \href{https://stacks.math.columbia.edu/tag/04DC}{\underline{universal homeomorphism}} between connected schemes then not only is the base change functor $f^*$ an equivalence, but also, one has an isomorphism of \'etale fundamental groups $\pi_1(X_{\fet}) \cong \pi_1(Y_{\fet})$\footnote{See corollary \ref{coro: etale_fundamental_group_uniqueness} for why we have ommited the base points.}. 
            \end{enumerate}
        \end{proposition}
            \begin{proof}
                \noindent
                \begin{enumerate}
                    \item This is an immediate consequence of the assumption that the base change functor $f^*$ is an equivalence and from the definition of \'etale fundamental groups (cf. definition \ref{def: etale_fundamental_groups}).
                    \item 
                \end{enumerate}
            \end{proof}
        \begin{corollary}[Uniqueness of \'etale fundamental groups] \label{coro: etale_fundamental_group_uniqueness}
            For any connected scheme $X$ and any pair of possibly distinct geometric points $\bar{x}, \bar{x}' \in X$, one has any isomorphism of \'etale fundamental groups $\pi_1(X_{\fet}, \bar{x}) \cong \pi_1(X_{\fet}, \bar{x}')$, and therefore it makes sense to only speak of \textit{the} fundamental group of $X$, which we shall denote by $\pi_1(X_{\fet})$.
        \end{corollary}
    
    \subsection{\texorpdfstring{$\ell$}{}-adic sheaves and Grothendieck's Galois Theory}
        \begin{lemma}[The \'etale monodromy correspondence] \label{lemma: etale_monodromy}
            Let $\Lambda$ be a commutative ring and let $X$ be a connected scheme. Then, there exists a canonical equivalence of categories:
                $$\LocSys_{\Lambda}^{\fin}(X_{\et}) \cong \Rep_{\Lambda}^{\fin}(\pi_1(X_{\fet}))^{\cont}$$
            between the category of \'etale locally constant sheaves of finite type $\Lambda$-modules on $X$ and the category of finite-dimensional continuous $\Lambda$-linear representations of $\pi_1(X_{\fet})$
        \end{lemma}
            \begin{proof}
                
            \end{proof}
    
        \begin{convention} \label{conv: base_curve}
            From this point on, $k$ shall be a separably closed field and $X$ shall be a smooth projective \textit{connected} curve over $\Spec k$. Furthermore, by \say{local systems}, we shall always mean locally constant sheaves in the \'etale topology.
        \end{convention}
        
        \begin{theorem}[Unramified representations are sheaves on $X$] \label{theorem: unramified_representations_are_sheaves_on_X}
            There is a canonical equivalence of categories:
                $$\Rep^1_{\overline{\Q_{\ell}}}(\pi_1^{\ab}(X_{\fet}))^{\cont} \cong \LocSys^1_{\overline{\Q_{\ell}}}(X)$$
            between the category of continuous $\ell$-adic characters of $\pi_1^{\ab}(X_{\fet})$ and that of $\ell$-adic local systems of rank $1$ on $X$.
        \end{theorem}
            \begin{proof}
                Lemma \ref{lemma: etale_monodromy} tells us that we have an equivalence $\Rep^1_{\overline{\Q_{\ell}}}(\pi_1(X_{\fet}))^{\cont} \cong \LocSys^1_{\overline{\Q_{\ell}}}(X)$, so the only thing to show is that $\Rep^1_{\overline{\Q_{\ell}}}(\pi_1(X_{\fet}))^{\cont} \cong \Rep^1_{\overline{\Q_{\ell}}}(\pi_1^{\ab}(X_{\fet}))^{\cont}$. For this, observe that because characters are necessarily irreducible as linear representations, and since $\overline{\Q_{\ell}}$ is an algebraically closed field of charactersitic $0$, we get through Schur's Lemma (cf. \cite[Lemma 3.6, pp. 35]{lam_first_course_in_noncommutative_rings}) that any pair of continuous $\ell$-adic characters $\chi_1, \chi_2$ of $\pi_1(X_{\fet})$ are unique up to multiplication by units $\lambda \in \overline{\Q_{\ell}}^{\x}$, which can be easily shown to be homeomorphic group homomorphisms $\lambda: \GL_1(\overline{\Q_{\ell}}) \to \GL_1(\overline{\Q_{\ell}})$ such that:
                    $$\forall \sigma \in \pi_1(X_{\fet}): \forall v \in \GL_1(\overline{\Q_{\ell}}): \chi_2(\sigma)(v) = \lambda \cdot \chi_1(\sigma)(v)$$
                As a result, $\Rep^1_{\overline{\Q_{\ell}}}(\pi_1(X_{\fet}))^{\cont}$ is a groupoid. Now, thanks to the First Isomorphism Theorem for topological groups, any continuous $\ell$-adic character $\chi: \pi_1(X_{\fet}) \to \GL_1(\overline{\Q_{\ell}})$ necessarily factors through the canoncial quotient map $\pi_1(X_{\fet}) \to \pi_1^{\ab}(X_{\fet})$ and hence, any such character $\chi$ defines a \textit{unique} continuous \say{abelian $\ell$-adic character} $\chi^{\ab}: \pi_1^{\ab}(X_{\fet}) \to \GL_1(\overline{\Q_{\ell}})$. By combining the two observations, one sees that there exists a fully faithful and essentially surjective functor:
                    $$\Rep^1_{\overline{\Q_{\ell}}}(\pi_1(X_{\fet}))^{\cont} \cong \Rep^1_{\overline{\Q_{\ell}}}(\pi_1^{\ab}(X_{\fet}))^{\cont}$$
                    $$\chi \mapsto \chi^{\ab}$$
                making it an equivalence of categories\footnote{In fact, of groupoids. Note that $\LocSys_{\overline{\Q_{\ell}}}^1(X)$ is also a groupoid, and so the statement of theorem \ref{theorem: unramified_representations_are_sheaves_on_X} is actually an equivalence of groupoids.} by definition.
            \end{proof}
        \begin{example}
            \noindent
            \begin{itemize}
                \item \textbf{($X \cong \P^1$):}
                \item \textbf{(Elliptic curves):}
                \item \textbf{(Counter-example: $X \cong \A^1$):}
            \end{itemize}
        \end{example}