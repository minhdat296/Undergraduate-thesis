\section{The Galois Side}
    \subsection{The \'etale fundamental group}
        Let us begin with an auxiliary notion, that of pro-representable functors, which is necessary for our first important construction, that of Galois categories.
        \begin{definition}[Pro-representable functors] \label{def: pro_representable_functors}
            \noindent
            \begin{enumerate}
                \item \textbf{(Pro-completions):} Following \cite[Definition 2.1]{isaksen_2001_limits_and_colimits_in_pro_categories}, the \textbf{pro-completion} $\Pro(\C)$ of a small category $\C$ is the category whose objects are cofiltered diagrams in $\C$ and whose hom-sets are given by:
                    $$\Pro(\C)(\{X_i\}_{i \in \calI}, \{Y_j\}_{j \in \calJ}) \cong \underset{j \in \calJ}{\lim} \underset{i \in \calI}{\colim} \C(X_i, Y_j)$$
                The dual notion is that of ind-completions; we denote the ind-completion of $\C$ by $\Ind(\C)$.
                \item \textbf{(Pro-representable functors):} Let $\C$ be a small category, and suppose that $\C$ is enriched in some small \href{http://nlab-pages.s3.us-east-2.amazonaws.com/nlab/show/closed+monoidal+category}{\underline{closed monoidal category}} $\V$ (e.g. the category of finite sets or the category of sets where the monoidal structure is given by products). Then, a $\V$-presheaf:
                    $$F: \C \to \V$$
                on $\C^{\op}$ is said to be \textbf{pro-representable} if and only if its \textbf{pro-completion}:
                    $$\Pro(F): \Pro(\C) \to \Pro(\V)$$
                is representable as a $\Pro(\V)$-copresheaf on $\Pro(\C)^{\op}$.
            \end{enumerate}
        \end{definition}
        \begin{remark} \label{remark: pro_representable_functors_are_ind_objects}
            Observe that due to Yoneda's Lemma, for $\C$ any small category and $\V$ any small closed monoidal category, the category of pro-representable $\V$-presheaves on $\C^{\op}$ is equivalent to $\Pro(\C)^{\op}$.
            
            Additionally, note that any pro-completion of a finite complete small category is necessarily cofiltered, since every finite cone must therefore admit a cone. Furthermore, pro-completions are their own maximal cofinal cofiltered subdiagram.
        \end{remark}
        
        We now officially begin our discussion of Grothendieck's Galois Theory with the notion of Galois categories, axiomatic settings in which one can \say{do Galois theory}, in the sense of classifying subobjects of a given universal object by checking whether or not they remain stable under certain \say{Galois group} actions; the idea is that Galois categories behave similarly to the category of finite sets (which can be thought of as the prototypical Galois category), in the same manner that sheaf topoi resemble the category of sets. Do keep in mind that for the sake of convenience (although without loss of generality, at least for our purposes), definition \ref{def: galois_categories} is a combination of \cite[D\'efinition V.4.5.1]{SGA1} and \cite[\href{https://stacks.math.columbia.edu/tag/0BMY}{Tag 0BMY}]{stacks}; namely, we require that the fibre functor is \textit{pro-representable}, which the latter source does not.
        \begin{definition}[Galois categories and their fundamental groups] \label{def: galois_categories}
            \noindent
            \begin{itemize}
                \item \textbf{(Galois categories):} A \textbf{Galois category} is defined via the data contained in a pair $(\calG, F)$ consisting of:
                \begin{itemize}
                    \item a \textit{finitely complete and finitely cocomplete} small category $\calG$, wherein objects can all be written as finite coproducts of \textit{connected} objects\footnote{Objects $X \in \calG$ such that the copresheaf $\calG(X, -)$ preserves all coproducts.}.
                    \item a \textit{pro-representable} $\Fin$-presheaf on $\calG^{\op}$:
                        $$F: \calG \to \Fin$$
                    called the \textbf{fibre functor}, which we shall require to be exact and to reflect isomorphisms (i.e. for all bijections $Fx \cong Fy$ between finite sets, one has an isomorphism $x \cong y$ in $\calG$).
                \end{itemize}
                \item \textbf{(Galois objects):} An object $X$ of a Galois category $\calG$ is a \textbf{Galois object} if and only if it has no non-trivial automorphisms, i.e. if and only if $X/\Aut_{\calG}(X) \cong \pt$, with $\pt$ a terminal object of $\calG$.\footnote{Note that Galois categories must have terminal objects, as they are finitely complete and terminal objects are nothing but the limit of the empty diagram (which is finite by virtue of containing no vertices and no edges).}
                \item \textbf{(Galois functors):} A \textbf{Galois functor} is an exact functor $\Phi: \calG \to \calG'$ between Galois categories $(\calG, F), (\calG', F')$ which preserves connected objects and commute with the fibre functors in the following manner:
                    $$
                        \begin{tikzcd}
                        	\calG && {\calG'} \\
                        	& \Fin
                        	\arrow["F"', from=1-1, to=2-2]
                        	\arrow["{F'}", from=1-3, to=2-2]
                        	\arrow["\Phi", from=1-1, to=1-3]
                        \end{tikzcd}
                    $$
            \end{itemize}
        \end{definition}
        \begin{definition}[Fundamental groups of Galois categories] \label{def: fundamental_groups_of_galois_categories}
            The \textbf{fundamental group} of a given Galois category $(\calG, F)$, denoted by $\pi_1(\calG, F)$, is defined to be the automorphism group $\Aut(\Pro(F))$.
        \end{definition}
        
        \begin{proposition}[The Categorical Galois Correspondence] \label{prop: categorical_galois_correspondence}
            Fix a Galois category $(\calG, F)$ with terminal objects $\pt$. Then there are the following equivalences of categories:
                $$\calG \cong \pi_1(\calG, F)\-\Fin\Sets$$
                $$Y \mapsto F(Y)$$
                $$\{\text{Finite-index subgroups of $\pi_1(\calG, F)$}\} \cong \pi_1(\calG, F)\-\Fin\Sets$$
                $$H \mapsto \pi_1(\calG, F)/H$$
            Furthermore, there is the following restricted equivalence:
                $$\calG^{\Gal} \cong \{\text{Finite-index normal subgroups of $\pi_1(\calG, F)$}\}$$
        \end{proposition}
            \begin{proof}
                \todo{Cite a proof}
            \end{proof}
            
        \begin{definition}[Universal covers] \label{def: universal_covers}
            Let $(\calG, F)$ be a Galois category. A pro-object $\tilde{X} \in \Pro(\calG)$ is called a \textbf{universal cover} if and only if its fundamental group $\pi_1(\tilde{X}) \cong \Aut(\Pro(F)(\tilde{X}))$ is trivial (i.e. if and only if it is simply-connected).
        \end{definition}
        \begin{remark}[Fundamental groups are automorphism groups of universal covers] \label{remark: fundamental_groups_are_automorphism_groups_of_universal_covers}
            
        \end{remark}
        
        Let us now try to adapt definitions \ref{def: galois_categories} and \ref{def: fundamental_groups_of_galois_categories} to a appropriate categories of schemes, namely those spanned by schemes finite-\'etale over a given base.
        \begin{remark}[\'Etale vs. finite-\'etale] \label{remark: etale_vs_finite_etale}
            One crucial tehcnicality that we will need to keep in mind is that finite-\'etale morphisms are \'etale, but the converse need not be true (e.g. the affine line is \'etale but not at all finite). However, \'etale morphisms are indeed finite when the codomain is the spectrum of a field (this is not the only case where \'etale morphisms are finite-\'etale, but it is sufficient for us); a proof can easily derived from \cite[\href{https://stacks.math.columbia.edu/tag/00U3}{Tag 00U3}]{stacks}, which asserts that \'etale (commutative) algebras over a field $k$ are isomorphic to finite direct sums of finite separable extension of $k$. 
        \end{remark}
        \begin{remark}[Finite-\'etale schemes] \label{remark: finite_etale_schemes}
            For any given by scheme $X$, the small category $(\Sch_{/X})_{\fet}$ of finite-\'etale $X$-schemes is a category wherein:
                \begin{itemize}
                    \item all finite limits and all finite colimits exist, and
                    \item all objects can be written as a (possibly empty) finite coproduct of connected objects, which happen to be schemes that are \'etale over $X$.  
                \end{itemize}
            (for a detailed proof, see \cite[\href{https://stacks.math.columbia.edu/tag/0BN9}{Tag 0BN9}]{stacks}) so should we be able to define a fibre functor $(\Sch_{/X})_{\fet} \to \Fin$, we will have succeeded in putting a Galois category structure on $(\Sch_{/X})_{\fet}$. As a matter of fact, such a well-defined fibre functor has good reasons to exist: it is an easy consequence of \cite[\href{https://stacks.math.columbia.edu/tag/00U3}{Tag 00U3}]{stacks} that for any fixed geometric point $\bar{x} \in X$ (corresponding to an algebraic closure $\bar{\kappa}_x$ of the residue field of $x \in X$\footnote{Certain sources consider geometric points to correspond to separable closures. For us, however, geometric points are algebraically closed fields $K$ so that $\Spec K$ be a Galois object of $(\Sch_{/\Spec K})_{\fet}$ (cf. definition \ref{def: galois_categories}). In practice this choice usually does not matter, since we will mostly work over perfect field, and separable closures of perfect fields are algebraically closed (a notable exception is when we work over perfectoid fields; cf. \cite{scholze2011perfectoid}).}), one has:
                $$(\Spec \bar{\kappa}_x)_{\fet} \cong \Fin$$
            (the forward direct simply involves taking the underlying set, and the inverse functors is given by $I \mapsto \coprod_{i = 1}^{|I|} \Spec \bar{\kappa}_x$) and so for any $k$-scheme $X$, one has the following canonical defined functor:
                $$(\Sch_{/X})_{\fet} \to (\Sch_{/\Spec \bar{\kappa}_x})$$
                $$Y \mapsto Y_{\bar{x}}$$
            where $Y_{\bar{x}} \cong Y \x_X \Spec \bar{\kappa}_x$; one can then take the underlying set of $Y_{\bar{x}}$ to get the following trivially left-exact functor:
                $$F_{\bar{x}}: (\Sch_{/X})_{\fet} \to \Fin$$
                $$Y \mapsto |Y_{\bar{x}}|$$
            We should also verify that the sets $|Y_{\bar{x}}|$ are indeed finite. To this end, let us first apply the fact that pullbacks of \'etale morphisms are \'etale to see that if $Y$ is affine over $X$ then $Y_{\bar{x}}$ will have to be the spectrum of an \'etale $\bar{\kappa}_x$-algebra; however, according to \cite[\href{https://stacks.math.columbia.edu/tag/00U3}{Tag 00U3}]{stacks}, this means that $Y_{\bar{x}} \cong \Spec (\bar{\kappa}_x)^{\oplus N}$ for some finite $N$. The locality of \'etale-ness and the finiteness of $Y$ as an $X$-scheme then tells us that in general, $Y_{\bar{x}}$ must be a finite disjoint union of affine schemes of the form $\Spec (\bar{\kappa}_x)^{\oplus N}$, meaning that $Y_{\bar{x}} \cong \Spec (\bar{\kappa}_x)^{\oplus N'}$ for some finite $N'$. The set $|Y_{\bar{x}}|$ is therefore always finite. One also sees that an immediate consequence of this proof is that $F_{\bar{x}}$ necessarily \textit{reflects isomorphisms} and is \textit{right-exact}. 
            
            It thus remains to show that $F_{\bar{x}}$ is \textit{pro-representable}. For this, observe first of all that as a functor on $\Sch_{/X}$ (as opposed to a functor on $(\Sch_{/X})_{\fet}$), $F_{\bar{x}}$ is naturally isomorphic to $\Sch_{/X}(\bar{x}, -)$. Since $\bar{x}$ is a pro-object of $(\Sch_{/\Spec \kappa_x})_{\fet}$, 
            
            We have thus constructed a well-defined fibre functor, in the sense of definition \ref{def: galois_categories}:
                $$F_{\bar{x}}: (\Sch_{/X})_{\fet} \to \Fin$$
                $$Y \mapsto |Y_{\bar{x}}|$$
        \end{remark}
        \begin{remark}[Finite \'etale Galois schemes] \label{remark: galois_schemes}
            Fix a base scheme $X$, and thanks to the fact that objects of Galois categories ($(\Sch_{/X})_{\fet}$ in this instance) can be written as finite coproducts of connected objects, we can assume without loss of generality that $X$ is connected. By definition \ref{def: galois_categories}, a Galois object in $(\Sch_{/X})_{\fet}$ is a finite-\'etale $X$-scheme $Y$ such that $Y/\Aut_X(Y) \cong X$. 
        \end{remark}
        \begin{definition}[\'Etale fundamental groups] \label{def: etale_fundamental_groups}
            For any scheme $X$ with a fixed geometric point $\bar{x}$, the pair $((\Sch_{/X})_{\fet}, F_{\bar{x}})$ as in remark \ref{remark: finite_etale_schemes} defines a Galois category. Its fundamental group is commonly denoted by $\pi_1(X_{\fet}, \bar{x})$ and called the \textbf{\'etale fundamental group} of $X$ based at $\bar{x}$.
        \end{definition}
        \begin{remark}[\'Etale fundamental groups as automorphism groups of universal covers] \label{remark: etale_fundamental_groups_as_automorphism_groups_of_universal_covers}
            In remark \ref{remark: finite_etale_schemes}, we have implicitly shown that the full subcategory $(\Sch_{/X})_{\fet}^{\Gal}$ is a diagram in $(\Sch_{/X})_{\fet}$ such that  
        \end{remark}
        \begin{remark}[The Geometric Galois Correspondence] \label{remark: geometric_galois_correspondence}
            Let $(X, \bar{x})$ be a pointed connected scheme. Then by proposition \ref{prop: categorical_galois_correspondence}, there is an equivalence of categories:
                $$\{\text{Finite-index subgroups of $\pi_1(X_{\fet}, \bar{x})$}\}$$
                $$\cong$$
                $$\{\text{Finite \'etale $X$-schemes $Y$ with base points $\bar{y}$ lying over $\bar{x}$}\}$$
            Furthermore (and also thanks to proposition \ref{prop: categorical_galois_correspondence}), this equivalence restricts down to:
                $$\{\text{Finite-index normal subgroups of $\pi_1(X_{\fet}, \bar{x})$}\}$$
                $$\cong$$
                $$\{\text{Finite \'etale Galois $X$-schemes $Y$ with base points $\bar{y}$ lying over $\bar{x}$}\}$$
            As a consequence, should $H$ be a finite-index normal subgroup of $\pi_1(X_{\fet}, \bar{x})$ and $(X^H, \bar{x}^H)$ be the corresponding Galois $X$-scheme with a choice of base point $\bar{x}^H$ lying over $\bar{x}$, then $\pi_1(X^H_{\fet}, \bar{x}^H) \cong H$. 
        \end{remark}
        \begin{example}[The \'etale fundamental group of a field] \label{example: etale_fundamental_group_of_a_field}
            As a sanity check, note that if $K$ is a field then finite-\'etale Galois schemes over $\Spec K$ shall be of the form $\Spec L \to \Spec K$, where $L/K$ is a finite Galois extension, and as a consequence, there are there are the following equivalences of lattices, which demonstrate that remark \ref{remark: geometric_galois_correspondence} directly generalises the classical Galois Correspondence:
                $$\{\text{Finite-index normal subgroups of $\pi_1((\Spec K)_{\fet})$}\}$$
                $$\cong$$
                $$\{\text{Finite \'etale Galois schemes over $\Spec K$}\}$$
                $$\cong$$
                $$\{\text{Finite Galois extensions of $K$}\}^{\op}$$
                $$\cong$$
                $$\{\text{Finite-index normal subgroups of $\Gal(\bar{K}/K)$}\}$$
        \end{example}
        \begin{example}[The \'etale fundamental group of a curve] \label{example: etale_fundamental_group_of_a_curve}
            Let $k$ be a field. If $X$ is a connected non-singular projective curve over $\Spec k$ with function field $K$, then there is a canonical equivalence $({}^{K/}\Fld^{\fin, \Gal})^{\op} \cong (\Sch_{/X})_{\fet}^{\Gal}$ between the lattice of finite Galois extensions of $K$ and Galois $X$-schemes (which are precisely dominant rational maps whose associated function field extensions are Galois). Through this, it is easy to see that:
                $$\pi_1(X_{\fet}) \cong \Gal(\bar{K}/K)$$
            For instance, we have:
                $$\pi_1((\P^1_k)_{\fet}) \cong \Gal(\bar{k}/k)$$
            (since the function field of $\P^1_k$ is $k(t)$), which tells us that $\P^1_k$ is simply \'etale-connected if and only if $k$ is algebraically closed (since $\Gal(\bar{k}/k)$ is \textit{a fortiori} trivial in that case). 
            
            Another interesting case that one might wish to consider is that of elliptic curves; for the sake of simplicity, let us work with an elliptic curve $E$ over an algebraically closed field $k$ of characteristic $0$. First of all, because $k$ is algebraically closed, every $k$-rational point $x \in E(k)$ is automatically geometric; therefore, we might as well work with $x = 0$. Now, it can be shown without too much difficulty (cf. \cite[Proposition 5.11]{kundu_etale_fundamental_group_of_elliptic_curves}) that every scheme finite \'etale over an elliptic curve over any field is automatically Galois. Together with the definition of $\pi_1(E_{\fet})$, this implies that for any cofinal diagram $\{Y_i\}_{i \in \calI}$ in $(\Sch_{/E})_{\fet}$, one has:
                $$\pi_1(E_{\fet}) \cong \underset{i \in \calI}{\lim} \Aut(F_0(Y_i)) \cong \underset{i \in \calI}{\lim} \Aut(|(Y_i)_0|)$$
            Luckily, there is a canonical choice of such a cofinal diagram, namely $\{E[n]\}_{n \in \N}$ (cf. \cite[Proposition 3.8]{kundu_etale_fundamental_group_of_elliptic_curves}), which are nothing but the fibres over $0 \in E(k)$ (i.e. kernels) of the $n$-torsion maps $[n]: E \to E$ (this is why we chose the base point $x = 0$). It is well-known that:
                $$\Aut(|E[n]|) \cong (\Z/n\Z)^{\oplus 2}$$
            so by taking the limit, one obtains:
                $$\pi_1(E_{\fet}) \cong \hat{\Z}^{\oplus 2}$$
            Elliptic curves over (algebraically closed) fields of positive characteristics $p_0$ behave somewhat differently, but it is also difficult to compute their \'etale fundamental groups (provided). If $E$ is supersingular (i.e. if the $p_0$-torsion map $[p_0]: E \to E$ has trivial kernel) then one has the following description of the \'etale fundamental group of $E$ (cf. \cite[Proposition 5.13]{kundu_etale_fundamental_group_of_elliptic_curves}):
                $$\pi_1(E_{\fet}) \cong \bigoplus_{(p) \in |\Spec \Z| \setminus \{(0), (p_0)\}} \Z_p^{\oplus 2}$$
            and otherwise, if $E$ is an ordinary elliptic curve, one has the following (cf. \cite[Proposition 5.14]{kundu_etale_fundamental_group_of_elliptic_curves}):
                $$\pi_1(E_{\fet}) \cong \Z_{p_0} \oplus \bigoplus_{(p) \in |\Spec \Z| \setminus \{(0), (p_0)\}} \Z_p^{\oplus 2}$$
        \end{example}
        \begin{example}[\'Etale fundamental group of the affine line] \label{example: etale_fundamental_group_of_the_affine_line}
            If $k$ is a field then $\pi_1((\A^1_k)_{\fet}) \cong \Gal(\bar{k}/k)$ when $\chara k = 0$, and hence $\pi_1((\A^1_k)_{\fet}) \cong 1$ if $k$ is furthermore algebraically closed. If however $\chara k > 0$, then $\pi_1((\A^1_k)_{\fet})$ can fail to be trivial even when $k$ is algebraically closed. For now, see \cite[Theorem 6.13, Remark 6.23, and Exercises 6.28 and 6.29]{lenstra_1985_galois_theory_for_schemes}.
        \end{example}
        
        Now, let us make sure that the \'etale fundamental group $\pi_1(X_{\fet}, \bar{x})$ as defined in definition \ref{def: etale_fundamental_groups} is meaningful as a formal construction. Namely, we would like to know the behaviours of $\pi_1(X_{\fet}, \bar{x})$ when we change the base point and when we base-change (cf. proposition \ref{prop: etale_fundamental_groups_do_not_depend_on_base_points}), as well as whether or not \'etale fibrations induce homotopy exact sequences of fundamental groups (cf. proposition \ref{prop: etale_homotopy_exact_sequence}). 
        \begin{proposition}[\'Etale fundamental group do not depend on base points] \label{prop: etale_fundamental_groups_do_not_depend_on_base_points}
            Let $f: Y \to X$ be a morphism of connected qcqs\footnote{quasi-compact and quasi-separated} schemes such that the base change functor:
                $$(\Sch_{/X})_{\fet} \to (\Sch_{/Y})_{\fet}$$
                $$X' \mapsto X' \x_X Y$$
            is an equivalence of Galois categories. Then, for any choice of geometric points $\bar{x} \in X$ and $\bar{y} \in Y$, one has the following isomorphism of \'etale fundamental groups $\pi_1(X_{\fet}, \bar{x}) \cong \pi_1(Y_{\fet}, \bar{y})$.
        \end{proposition}
            \begin{proof}
                This is an immediate consequence of the assumption that the base change functor:
                    $$(\Sch_{/X})_{\fet} \to (\Sch_{/Y})_{\fet}$$
                    $$X' \mapsto X' \x_X Y$$
                is an equivalence and from the definition of \'etale fundamental groups (cf. definition \ref{def: etale_fundamental_groups}).
            \end{proof}
        \begin{corollary}[Uniqueness of \'etale fundamental groups] \label{coro: etale_fundamental_group_uniqueness}
            For any connected qcqs scheme $X$ and any pair of possibly distinct geometric points $\bar{x}, \bar{x}' \in X$, one has any isomorphism of \'etale fundamental groups $\pi_1(X_{\fet}, \bar{x}) \cong \pi_1(X_{\fet}, \bar{x}')$, and therefore it makes sense to only speak of \textit{the} fundamental group of $X$, which we shall denote by $\pi_1(X_{\fet})$.
        \end{corollary}
        
        \begin{proposition}[\'Etale fundamental group of products]
            Let $k$ be an algebraically closed field and let $X, Y$ be two connected schemes which are, respectively, proper and locally noetherian over $\Spec k$. In addition, fix two geometric points $x \in X$ and $y \in Y$. Then, there is a canonical isomorphism:
                $$\pi_1((X \x_{\Spec k} Y)_{\fet}, (x, y)) \cong \pi_1(X_{\fet}, x) \x \pi_1(Y_{\fet}, y)$$
        \end{proposition}
            \begin{proof}
                \cite[Corollaire 1.7]{SGA1}
            \end{proof}
        
        \begin{proposition}[The \'etale homotopy exact sequence] \label{prop: etale_homotopy_exact_sequence}
            Let $X$ be a connected scheme. If $f: Y \to X$ be a flat proper morphism of finite presentation whose geometric fibres $Y_{\bar{x}}$ are connected and reduced, then for any geometric point $\bar{x} \in X$, there exists a right-exact sequence of groups as follows:
                $$\pi_1((Y_{\bar{x}})_{\fet}) \to \pi_1(Y_{\fet}) \to \pi_1(X_{\fet}) \to 1$$
        \end{proposition}
            \begin{proof}
                \cite[\href{https://stacks.math.columbia.edu/tag/0C0J}{Tag 0C0J}]{stacks}.
            \end{proof}
    
    \subsection{Grothendieck's Galois Theory}
        We begin by checking whether or not \'etale fundamental group is dual - in some sense - to the construction of an Eilenberg-MacLane space, thereby having concrete connections to the $1^{st}$ \'etale cohomology group and in turn, admitting descriptions in terms of torsors.
        \begin{remark}[Points of the moduli stack of $G$-torsors]
            For proposition \ref{prop: etale_eckmann_hilton_duality}, a basic fact one should keep in mind is that should $G$ be a constant group\footnote{As opposed to say, an algebraic group or more general group schemes}, then the groupoid $\Bun_{\underline{G}}(X) := \Sch_{/\Spec k}(X, \underline{G})$ of \'etale $\underline{G}$-torsors\footnote{Here, $\underline{G}$ denotes the group scheme represented by $\coprod_{g \in G} \Spec k$. Note how it is \'etale over $\Spec k$.} on a scheme $X$ over a field $k$ is equivalent to the groupoid $\Bun_{\underline{G}}(\Spec k)(X) := \Sch_{/\Spec k}(X, \underline{G(k)})$ of $X$-points of the moduli stack of $\underline{G}$-torsors on $\Spec k$.
        \end{remark}
        \begin{proposition}[The \'etale Eckmann-Hilton Duality] \label{prop: etale_eckmann_hilton_duality}
            For any smooth projective connected curve $X$\footnote{Actually, this proposition holds also when $X$ is any irreducible geometrically unibranch scheme (which can be thought of as analogues of path-connected spaces), but we are not interested in such generalities.} over a field $k$ and any constant profinite group $G$, there exists the following adjunction:
                $$
                    \begin{tikzcd}
                    	{\Grp(\Fin)} & {(\Sch_{/X})_{\fet}}
                    	\arrow[""{name=0, anchor=center, inner sep=0}, "{\Sch_{/\Spec k}(X, -)}"', bend right, from=1-1, to=1-2]
                    	\arrow[""{name=1, anchor=center, inner sep=0}, "{\pi_1^{\fet}}"', bend right, from=1-2, to=1-1]
                    	\arrow["\dashv"{anchor=center, rotate=-90}, draw=none, from=1, to=0]
                    \end{tikzcd}
                $$
            and in addition, a canonical equivalence:
                $$\Grp(\Pro\Fin)(\pi_1(X_{\fet}), G) \cong \Bun_{\underline{G}}(X)$$
            between the groupoid of continuous homomorphisms $\pi_1(X_{\fet}) \to G$ of profinite groups and that of $\underline{G}$-torsors on $X$.
        \end{proposition}
            \begin{proof}
                Each continuous homomorphism $\pi_1(X_{\fet}) \to G$ determines a unique $G$-torsor in $\pi_1(X_{\fet})\-\Pro\Fin$. Because there is an equivalence $\pi_1(X_{\fet})\-\Pro\Fin \cong (\Sch_{/X})_{\profet}$ (cf. proposition \ref{prop: categorical_galois_correspondence}) and because schemes are representable \'etale sheaves, each such $G$-torsor in $\pi_1(X_{\fet})\-\Pro\Fin$ corresponds to a unique $\underline{G}$-torsor on $X$. Such a $\underline{G}$-torsor on $X$, in turn, is an $X$-point of the classifying stack $\Bun_{\underline{G}}((\Spec k))$, i.e. a morphism $X \to \underline{G}$ of $k$-schemes, and the proposition follows suite.
            \end{proof}
        \begin{convention}
            From now on, if $E$ is a non-archimedean normed field then its subring of power-bounded elements shall be denoted by $\scrO_E$. 
        \end{convention}
        \begin{corollary}[Continuous representations of the \'etale fundamental group are torsors] \label{coro: continuous_representations_of_the_etale_fundamental_group_are_torsors}
            Let $\ell$ be a prime, let $E$ be an $\ell$-adic number field (i.e. a finite extension of $\Q_{\ell}$). For any smooth projective connected curve over a field $k$ along with any choice of discrete group $G$, there exists a canonical equivalence:
                $$\Rep^n_{\scrO_E}(\pi_1(X_{\fet}))^{\cont} \cong \Bun_{\underline{\GL_n(\scrO_E)}}(X)$$
            between the groupoid of continuous $n$-dimensional $\scrO_E$-linear representations of $\pi_1(X_{\fet})$ and that of $\GL_n(\scrO_E)$-torsors on $X$.
        \end{corollary}
        \begin{remark}[What about higher-dimensional representations ?]
            Corollary \ref{coro: continuous_representations_of_the_etale_fundamental_group_are_torsors} does not hold for $\GL_n(E)$ (for any $n \geq 1$), as these groups are only locally profinite, as opposed to being globally profinite like $\GL_n(\scrO_E)$.
        \end{remark}
    
        \begin{convention}[The Curve] \label{conv: base_curve}
            Henceforth, $X$ shall be a smooth projective \textit{connected} curve over $\Spec k$ (with $k$ some field).
        \end{convention}
        
        \begin{proposition}[$\ell$-adic representations are $\ell$-adic torsors] \label{prop: E_representations_are_E_local_systems}
            Let $\ell$ be a prime and $E$ be an $\ell$-adic number field. Then there exists a canonical equivalence of groupoids as follows, for all $n \geq 1$:
                $$\Rep_E^n(\pi_1(X_{\fet}))^{\cont} \cong \Bun_{\underline{\GL_n(E)}}(X)$$
        \end{proposition}
            \begin{proof}
                Let $\rho: \pi_1(X_{\fet}) \to \GL_n(E)$ be a continuous representation. $\GL_n(\scrO_E)$ is an open subgroup of $\GL_n(E)$, so the preimage $H := \rho^{-1}(\GL_n(\scrO_E))$ must be an open subgroup of $\pi_1(X_{\fet})$; since $\pi_1(X_{\fet})$ is profinite, $H$ is furthermore normal. Consequently, there exists a Galois $X$-scheme $X^H$ such that $\pi_1(X^H_{\fet}) \cong H$; as a result, we obtain the following pullback of topological groups:
                    $$
                        \begin{tikzcd}
                        	{\pi_1(X^H_{\fet})} & {\GL_n(\scrO_E)} \\
                        	{\pi_1(X_{\fet})} & {\GL_n(E)}
                        	\arrow[hook, from=1-2, to=2-2]
                        	\arrow["\rho", from=2-1, to=2-2]
                        	\arrow[from=1-1, to=2-1]
                        	\arrow["{(\rho^H)^{\circ}}", from=1-1, to=1-2]
                        	\arrow["\lrcorner"{anchor=center, pos=0.125}, draw=none, from=1-1, to=2-2]
                        \end{tikzcd}
                    $$
                Now, due to corollary \ref{coro: continuous_representations_of_the_etale_fundamental_group_are_torsors}, each representation $(\rho^H)^{\circ}: \pi_1(X^H_{\fet}) \to \GL_n(\scrO_E)$ corresponds to a unique $\underline{\GL_n(\scrO_E)}$-torsor on $X^H$, and since $\Bun_{\underline{\GL_n(\scrO_E)}}$ satisfies (profinite-)\'etale descent, one thus obtains in addition a unique $\underline{\GL_n(\scrO_E)}$-torsor on $X$, i.e. a representation $\rho^{\circ}: \pi_1(X_{\fet}) \to \GL_n(\scrO_E)$. There is thus a fully faithful embedding of $\Rep_E^n(\pi_1(X_{\fet}))^{\cont}$ into $\Bun_{\underline{\GL_n(E)}}(X)$. Now, to show that this embedding is also essentially surjective, observer that because $(\Sch_{/X})_{\profet} \cong \pi_1(X_{\fet})\-\Pro\Fin$, each $\underline{\GL_n(E)}$-torsor on $X$ corresponds to a unique (continuous) $\GL_n(E)$-torsor in $\pi_1(X_{\fet})\-\Pro\Fin$. But such a torsor is nothing but a continuous representation $\pi_1(X_{\fet}) \to \GL_n(E)$, so we are done.
            \end{proof}
        \begin{corollary}[Representations of the \'etale fundamental group are local systems] \label{coro: representations_of_the_etale_fundamental_group}
            Let $\ell$ be a prime and $E$ be an $\ell$-adic number field. In addition, fix a geometric point $\bar{x} \in X$. Then there exists a canonical equivalence as follows, for all $n \geq 1$:
                $$\Shv^n_{\underline{E}}(X) \cong \Rep_E^n(\pi_1(X_{\fet}))^{\cont}$$
                $$\calL \mapsto \calL_{\bar{x}}$$
        \end{corollary}
            
        The following result crucially exploits the fact that $\GL_1(E)$ is abelian, unlike $\GL_n(E)$ for $n \geq 2$.
        \begin{lemma}[Abelianising continuous characters] \label{lemma: abelianising_continuous_characters}
            Let $G$ be a topological group and $E$ a topological field. Then, there is a group isomorphism:
                $$\Rep^1_E(G)^{\cont} \cong \Rep^1_E(G^{\ab})^{\cont}$$
        \end{lemma}
            \begin{proof}
                There is a natural injective group homomorphism:
                    $$\Rep^1_E(G)^{\cont} \to \Rep^1_E(G^{\ab})^{\cont}$$
                    $$\chi \mapsto \chi^{\ab}$$
                coming from the canonical quotient map $G \to G^{\ab}$, so the only thing to do is to show that this homomorphism is surjective. For this, it shall suffice to show that the group $\Rep^1_E([G, G])^{\cont}$ is trivial: but this is evident from the fact that $\GL_1(E)$ is abelian and from the definition of the commutator subgroup $[G, G]$, namely that $[G, G] := \<ghg^{-1}h^{-1} \mid \forall g, h \in G\>$ (so for all $x \in [G, G]$ and all $\chi \in \Rep^1_E([G, G])$, $\chi(x) = 1$), so we are done.
            \end{proof}
            
        \begin{theorem}[Galois representations are sheaves on $X$] \label{theorem: galois_representations_are_sheaves_on_X}
            Fix a geometric point $\bar{x} \in X$. Then, there is a canonical monoidal equivalence:
                $$\Shv_{\bar{\Q}_{\ell}}^{\ad, 1}(X) \cong \Rep^1_{\bar{\Q}_{\ell}}(\pi_1^{\ab}(X_{\fet}))^{\cont}$$
                $$\calL \mapsto \calL_{\bar{x}}$$
        \end{theorem}
            \begin{proof}
                
            \end{proof}