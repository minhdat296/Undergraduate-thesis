\section{The Galois Side}
    \subsection{The \'etale fundamental group}
        Let us begin with an auxiliary notion, that of pro-representable functors, which is necessary for our first important construction, that of Galois categories.
        \begin{definition}[Pro-representable functors] \label{def: pro_representable_functors}
            \noindent
            \begin{enumerate}
                \item \textbf{(Pro-completions):} Following \cite[Definition 2.1]{isaksen_2001_limits_and_colimits_in_pro_categories}, the \textbf{pro-completion} $\Pro(\C)$ of a small category $\C$ is the category whose objects are cofiltered diagrams in $\C$ and whose hom-sets are given by:
                    $$\Pro(\C)(\{X_i\}_{i \in \calI}, \{Y_j\}_{j \in \calJ}) \cong \underset{j \in \calJ}{\lim} \underset{i \in \calI}{\colim} \C(X_i, Y_j)$$
                The dual notion is that of ind-completions; we denote the ind-completion of $\C$ by $\Ind(\C)$.
                \item \textbf{(Pro-representable functors):} Let $\C$ be a small category that is enriched in some small \href{http://nlab-pages.s3.us-east-2.amazonaws.com/nlab/show/closed+monoidal+category}{\underline{closed monoidal category}} $\V$ (e.g. the category of finite sets or the category of sets where the monoidal structure is given by products). Then, a $\V$-presheaf:
                    $$F: \C \to \V$$
                on $\C^{\op}$ is said to be \textbf{pro-representable} if and only if its \textbf{pro-completion}:
                    $$\Pro(F): \Pro(\C) \to \Pro(\V)$$
                is representable as a $\Pro(\V)$-copresheaf on $\Pro(\C)^{\op}$.
            \end{enumerate}
        \end{definition}
        \begin{remark} \label{remark: pro_representable_functors_are_ind_objects}
            Observe that due to Yoneda's Lemma, for $\C$ any small category and $\V$ any small closed monoidal category, the category of pro-representable $\V$-presheaves on $\C^{\op}$ is equivalent to $\Pro(\C)^{\op}$. 
        \end{remark}
        
        We now officially begin our discussion of Grothendieck's Galois Theory with the notion of Galois categories, axiomatic settings in which one can \say{do Galois theory}, in the sense of classifying subobjects of a given universal object by checking whether or not they remain stable under certain \say{Galois group} actions; the idea is that Galois categories behave similarly to the category of finite sets (which can be thought of as the prototypical Galois category), in the same manner that sheaf topoi resemble the category of sets. Do keep in mind that for the sake of convenience (although without loss of generality, at least for our purposes), definition \ref{def: galois_categories} is a combination of \cite[Expos\'e V, Section 4 and D\'efinition 5.1]{SGA1} and \cite[\href{https://stacks.math.columbia.edu/tag/0BMY}{Tag 0BMY}]{stacks}; namely, we require that the fibre functor is \textit{pro-representable}, which the latter source does not.
        \begin{definition}[Galois categories and their fundamental groups] \label{def: galois_categories}
            \noindent
            \begin{itemize}
                \item \textbf{(Galois categories):} A \textbf{Galois category} is defined via the data contained in a pair $(\calG, F)$ consisting of:
                \begin{itemize}
                    \item a \textit{finitely complete and finitely cocomplete} small category $\calG$, wherein objects can all be written as finite coproducts of \textit{connected} objects\footnote{Objects $X \in \calG$ such that the copresheaf $\calG(X, -)$ preserves all coproducts.}.
                    \item a \textit{pro-representable} $\Fin$-presheaf on $\calG^{\op}$:
                        $$F: \calG \to \Fin$$
                    called the \textbf{fibre functor}, which we shall require to be exact and to reflect isomorphisms (i.e. for all bijections $Fx \cong Fy$ between finite sets, one has an isomorphism $x \cong y$ in $\calG$).
                \end{itemize}
                \item \textbf{(Galois objects):} An object $X$ of a Galois category $\calG$ is a \textbf{Galois object} if and only if it has no non-trivial automorphisms, i.e. if and only if there is a bijection $\calG(\pt, X) \cong \Aut_{\calG}(X)$ (with $\pt$ a terminal object of $\calG$\footnote{Note that Galois categories must have terminal objects, as they are finitely complete and terminal objects are nothing but the limit of the empty diagram (which is finite by virtue of containing no vertices and no edges).})
                \item \textbf{(Galois functors):} A \textbf{Galois functor} is an exact functor $\Phi: \calG \to \calG'$ between Galois categories $(\calG, F), (\calG', F')$ which preserves connected objects and commute with the fibre functors in the following manner:
                    $$
                        \begin{tikzcd}
                        	\calG && {\calG'} \\
                        	& \Fin
                        	\arrow["F"', from=1-1, to=2-2]
                        	\arrow["{F'}", from=1-3, to=2-2]
                        	\arrow["\Phi", from=1-1, to=1-3]
                        \end{tikzcd}
                    $$
            \end{itemize}
        \end{definition}
        \begin{definition}[Fundamental groups of Galois categories] \label{def: fundamental_groups_of_galois_categories}
            The \textbf{fundamental group} of a given Galois category $(\calG, F)$ is the group $\Aut(F)$ of natural automorphisms on thefibre functor $F: \calG \to \Fin$. We shall suggestively denote the fundamental group of a given Galois category by $\pi_1(\calG, F)$.
        \end{definition}
        
        \begin{proposition}[Some basic properties of Galois categories and their fundamental groups] \label{prop: basic_properties_of_galois_categories}
            Fix a Galois category $(\calG, F)$.
                \begin{enumerate}
                    \item There is a natural equivalence of categories $\calG \cong \pi_1(\calG, F)\-\Fin$ given by $X \mapsto F(X)$. Furthermore, if $\pi$ is a profinite group such that $\calG \cong \pi\-\Fin$ then 
                    \item If $F': \calG \to \Fin$ is another fibre functor then $F \cong F'$.
                \end{enumerate}
        \end{proposition}
            \begin{proof}
                \noindent
                \begin{enumerate}
                    \item The group $\prod_{X \in \Ob(\calG)} \Aut(F(X))$\footnote{Note that this product of groups is a group because we have assumed that Galois categories are small (so in particular, this means that $\Ob(\calG)$ is a set and not a proper class).} admits $\Aut(F)$ as a natural subgroup; the natural component maps:
                        $$\alpha_X: \Aut(F) \to \Aut(F(X))$$
                    then endow the sets $F(X)$ with natural $\Aut(F)$-actions (i.e. $\pi_1(\calG, F)$-actions). This implies that $F$ factors through $\pi_1(\calG, F)-\Fin$.
                    
                    Now, because $\calG$ has all finite limits, it is necessarily cofiltered (as every finite diagram must admit a cone), and hence the (co)filtered limit $\underset{X \in \calG}{\lim} \Aut(F(X))$ has a natural profinite group structure (since the groups $\Aut(F(X))$ are finite). Furthermore, the action maps are continuous:
                        $$\alpha_X: \Aut(F) \to \Aut(F(X))$$
                    because the compact-open topology on finite sets (namely the sets $F(X)$) is just the discrete topology and not only that, but one also has an isomorphism of profinite groups:
                        $$\Aut(F) \cong \underset{X \in \calG}{\lim} \Aut(F(X))$$
                    as a consequence of the pro-representablity of $F$:  
                    \item 
                \end{enumerate}
            \end{proof}
        
        Let us now try to adapt definitions \ref{def: galois_categories} and \ref{def: fundamental_groups_of_galois_categories} to a appropriate categories of schemes, namely those spanned by schemes finite-\'etale over a given base.
        \begin{remark}[\'Etale vs. finite-\'etale] \label{remark: etale_vs_finite_etale}
            One crucial tehcnicality that we will need to keep in mind is that finite-\'etale morphisms are \'etale, but the converse need not be true (e.g. the affine line is \'etale but not at all finite). However, \'etale morphisms are indeed finite when the codomain is the spectrum of a field (this is not the only case where \'etale morphisms are finite-\'etale, but it is sufficient for us); a proof can easily derived from \cite[\href{https://stacks.math.columbia.edu/tag/00U3}{Tag 00U3}]{stacks}, which asserts that \'etale (commutative) algebras over a field $k$ are isomorphic to finite direct sums of finite separable extension of $k$. 
        \end{remark}
        \begin{remark}[Finite-\'etale schemes] \label{remark: finite_etale_schemes}
            For any given by scheme $X$, the small category $(\Sch_{/X})_{\fet}$ of finite-\'etale $X$-schemes is a category wherein:
                \begin{itemize}
                    \item all finite limits and all finite colimits exist, and
                    \item all objects can be written as a (possibly empty) finite coproduct of connected objects, which happen to be schemes that are \'etale over $X$.  
                \end{itemize}
            (for a detailed proof, see \cite[\href{https://stacks.math.columbia.edu/tag/0BN9}{Tag 0BN9}]{stacks}) so should we be able to define a fibre functor $(\Sch_{/X})_{\fet} \to \Fin$, we will have succeeded in putting a Galois category structure on $(\Sch_{/X})_{\fet}$. As a matter of fact, such a well-defined fibre functor has good reasons to exist: it is an easy consequence of \cite[\href{https://stacks.math.columbia.edu/tag/00U3}{Tag 00U3}]{stacks} that for any fixed geometric point $\bar{x} \in X$ (corresponding to an algebraic closure $\kappa_x^{\alg}$ of the residue field of $x \in X$\footnote{Certain sources consider geometric points to correspond to separable closures. For us, however, geometric points are algebraically closed fields $K$ so that $\Spec K$ be a Galois object of $(\Sch_{/\Spec K})_{\fet}$ (cf. definition \ref{def: galois_categories}). In practice this choice usually does not matter, since we will mostly work over perfect field, and separable closures of perfect fields are algebraically closed (a notable exception is when we work over perfectoid fields; cf. \cite{scholze2011perfectoid}).}), one has:
                $$(\Spec \kappa_x^{\alg})_{\fet} \cong \Fin$$
            (the forward direct simply involves taking the underlying set, and the inverse functors is given by $I \mapsto \coprod_{i = 1}^{|I|} \Spec \kappa_x^{\alg}$) and so for any $k$-scheme $X$, one has the following canonical defined functor:
                $$(\Sch_{/X})_{\fet} \to (\Sch_{/\Spec \kappa_x^{\alg}})$$
                $$Y \mapsto Y_{\bar{x}}$$
            where $Y_{\bar{x}} \cong Y \x_X \Spec \kappa_x^{\alg}$; one can then take the underlying set of $Y_{\bar{x}}$ to get the following trivially left-exact functor:
                $$F_{\bar{x}}: (\Sch_{/X})_{\fet} \to \Fin$$
                $$Y \mapsto |Y_{\bar{x}}|$$
            We should also verify that the sets $|Y_{\bar{x}}|$ are indeed finite. To this end, let us first apply the fact that pullbacks of \'etale morphisms are \'etale to see that if $Y$ is affine over $X$ then $Y_{\bar{x}}$ will have to be the spectrum of an \'etale $\kappa_x^{\alg}$-algebra; however, according to \cite[\href{https://stacks.math.columbia.edu/tag/00U3}{Tag 00U3}]{stacks}, this means that $Y_{\bar{x}} \cong \Spec (\kappa_x^{\alg})^{\oplus N}$ for some finite $N$. The locality of \'etale-ness and the finiteness of $Y$ as an $X$-scheme then tells us that in general, $Y_{\bar{x}}$ must be a finite disjoint union of affine schemes of the form $\Spec (\kappa_x^{\alg})^{\oplus N}$, meaning that $Y_{\bar{x}} \cong \Spec (\kappa_x^{\alg})^{\oplus N'}$ for some finite $N'$. The set $|Y_{\bar{x}}|$ is therefore always finite. One also sees that an immediate consequence of this proof is that $F_{\bar{x}}$ necessarily \textit{reflects isomorphisms} and is \textit{right-exact}. It thus remains to show that $F_{\bar{x}}$ is \textit{pro-representable}. For this, firstly note that because $F_{\bar{x}}(Y)$ is a finite set for all $Y \in (\Sch_{/X})_{\fet}$, there is a bijection between $F_{\bar{x}}(Y)$ and commutative diagrams of the form:
                $$
                    \begin{tikzcd}
                    	& Y \\
                    	{\bar{x}} & X
                    	\arrow[dashed, from=2-1, to=1-2]
                    	\arrow[from=2-1, to=2-2]
                    	\arrow[from=1-2, to=2-2]
                    \end{tikzcd}
                $$
            which tells us that there is a bijection:
                $$F_{\bar{x}}(Y) \cong \Sch_{/X}(\bar{x}, Y)$$
            It can then be shown using the definition algebraic closures and the Fundamental Theorem of Galois Theory, that $\bar{x} \in \Pro\left((\Sch_{/X})_{\fet}\right)$. According to remark \ref{remark: pro_representable_functors_are_ind_objects} and Yoneda's Lemma, $F_{\bar{x}}$ is therefore pro-representable.
            
            We have thus constructed a well-defined fibre functor:
                $$F_{\bar{x}}: (\Sch_{/X})_{\fet} \to \Fin$$
                $$Y \mapsto |Y_{\bar{x}}|$$
        \end{remark}
        \begin{definition}[\'Etale fundamental group] \label{def: etale_fundamental_groups}
            For any scheme $X$ with a fixed geometric point $\bar{x}$, the pair $((\Sch_{/X})_{\fet}, F_{\bar{x}})$ as in remark \ref{remark: finite_etale_schemes} defines a Galois category. Its fundamental group $\Aut(F_{\bar{x}})$ is commonly denoted by $\pi_1(X_{\fet}, \bar{x})$ and called the \textbf{\'etale fundamental group} of $X$ based at $\bar{x}$.
        \end{definition}
        
        Now, let us make sure that the \'etale fundamental group $\pi_1(X_{\fet}, \bar{x})$ as defined in definition \ref{def: etale_fundamental_groups} actually makes sense topologically and geometrically. Namely, we would like to know the behaviours of $\pi_1(X_{\fet}, \bar{x})$ when we change the base point and when we base-change, as well as whether or not it is \say{insensitive} to (universal) homeomorphisms.
        \begin{lemma} \label{lemma: base_change_of_thickenings}
            If $X \subset X'$ be a \href{https://stacks.math.columbia.edu/tag/04EW}{\underline{thickening}} of schemes. Then, the following base change functor is an equivalence of Galois categories:
                $$(\Sch_{/X'})_{\fet} \to (\Sch_{/X})_{\fet}$$
                $$T \mapsto T \x_{X'} X$$
        \end{lemma}
            \begin{proof}
                
            \end{proof}
        \begin{proposition}[The \'etale fundamental group as a topological invariance] \label{prop: the_etale_fundamental_group_as_a_topological_invariance}
            \noindent
            \begin{enumerate}
                \item Let $f: Y \to X$ be a morphism of connected schemes such that the base change functor:
                    $$(\Sch_{/X})_{\fet} \to (\Sch_{/Y})_{\fet}$$
                    $$X' \mapsto X' \x_X Y$$
                is an equivalence of Galois categories. Then, for any choice of geometric points $\bar{x} \in X$ and $\bar{y} \in Y$, one has the following isomorphism of \'etale fundamental groups $\pi_1(X_{\fet}, \bar{x}) \cong \pi_1(Y_{\fet}, \bar{y})$.
                \item If $f: Y \to X$ is a \href{https://stacks.math.columbia.edu/tag/04DC}{\underline{universal homeomorphism}} between connected schemes and if $\bar{y} \in Y$ is a geometric point lying over a fixed geometric point $\bar{x} \in X$, then not only is the base change functor:
                    $$(\Sch_{/X})_{\fet} \to (\Sch_{/Y})_{\fet}$$
                    $$X' \mapsto X' \x_X Y$$
                an equivalence of Galois categories, but also, one has an isomorphism of \'etale fundamental groups $\pi_1(X_{\fet}, \bar{x}) \cong \pi_1(Y_{\fet}, \bar{y})$. 
            \end{enumerate}
        \end{proposition}
            \begin{proof}
                \noindent
                \begin{enumerate}
                    \item This is an immediate consequence of the assumption that the base change functor:
                        $$(\Sch_{/X})_{\fet} \to (\Sch_{/Y})_{\fet}$$
                        $$X' \mapsto X' \x_X Y$$
                    is an equivalence and from the definition of \'etale fundamental groups (cf. definition \ref{def: etale_fundamental_groups}).
                    \item Because $f: Y \to X$ is a universal homeomorphism, the diagonal $\Delta_{Y/X}: Y \to Y \x_X Y$ is a thickening. By lemma \ref{lemma: base_change_of_thickenings}, this means that the base change functor:
                        $$(\Sch_{Y \x_X Y})_{\fet} \to (\Sch_{/Y})_{\fet}$$
                        $$T \mapsto T \x_{Y \x_X Y} Y$$
                    is an equivalence.
                \end{enumerate}
            \end{proof}
        \begin{corollary}[Uniqueness of \'etale fundamental groups] \label{coro: etale_fundamental_group_uniqueness}
            For any connected scheme $X$ and any pair of possibly distinct geometric points $\bar{x}, \bar{x}' \in X$, one has any isomorphism of \'etale fundamental groups $\pi_1(X_{\fet}, \bar{x}) \cong \pi_1(X_{\fet}, \bar{x}')$, and therefore it makes sense to only speak of \textit{the} fundamental group of $X$, which we shall denote by $\pi_1(X_{\fet})$.
        \end{corollary}
        
        Let us now compute certain instances of the \'etale fundamental group to verify for ourselves its importance within the context of arithemtic algebraic geometry.
        \begin{example}[Examples of the \'etale fundamental group] \label{example: etale_fundamental_groups}
            \noindent
            \begin{itemize}
                \item \textbf{(Fields):} For any field $K$, one has:
                    $$\pi_1((\Spec K)_{\fet}) \cong \Gal(K^{\alg}/K)$$
                \item \textbf{(Projective line):} It is well-known that $\P^1_k$ (for any field $k$) is a smooth connected projective curve with function field $k(t)$. Through proposition \ref{prop: curves_and_function_fields}, one sees that:
                    $$\pi_1((\P^1_k)_{\fet}) \cong \pi_1((\Spec k(t))_{\fet}) \cong \Gal(k(t)^{\alg}/k(t)) \cong \Gal(k^{\alg}(t)/k(t)) \cong \Gal(k^{\alg}/k)$$
                \item \textbf{(Discrete valuation rings):} Let us start with the prototypical case of $\Z_p$. Because objects of $(\Sch_{/\Spec \Z_p})_{\fet}$ are \textit{a priori} finite unramified integral extensions of $\Z_p$ whose fraction fields are $p$-adic fields of the same degree over $\Q_p$ (the fraction field of $\Z_p$), one obtains a canonical equivalence of Galois categories:
                    $$(\Sch_{/\Spec \Q_p})_{\fet} \cong (\Sch_{/\Spec \Z_p})_{\fet}$$
                As a consequence:
                    $$\pi_1((\Spec \Z_p)_{\fet}) \cong \pi_1((\Spec \Q_p)_{\fet}) \cong \Gal(\Q_p^{\unr}/\Q_p) \cong \hat{\Z}$$
                where $\Q_p^{\unr}/\Q_p$ is the maximal unramified extension of $\Q_p$, which one obtains by adjoining all $n^{th}$ roots of unity to $\Q_p$ (for all $n$ coprime to $p$).
                
                By arguing similarly, one will also see that $\pi_1((\Spec \F_p(\!(t)\!))_{\fet}) \cong \hat{\Z}$ as well. In fact, if $\scrV$ is any discrete valuation ring with residue field $k$ then $\pi_1((\Spec \scrV)_{\fet}) \cong \Gal(k^{\alg}/k)$ (one can start with $\scrV \cong \Z_p$, and notice that the residue field of any finite unramified extension of $\Q_p$ is $\F_{p^n}$, with $n$ being the degree of that extension).
                \item \textbf{(Algebraic integers):} Minkowski's Theorem tells us that every non-trivial finite extension of $\Q$ ramifies at some prime $(p) \in \Spec \Z$, so:
                    $$\pi_1((\Spec \Z)_{\fet}) \cong 1$$
                \item \textbf{(Affine line):} 
            \end{itemize}
        \end{example}
    
    \subsection{\texorpdfstring{$\ell$}{}-adic sheaves and Grothendieck's Galois Theory}
        \begin{convention}
            \noindent
            \begin{itemize}
                \item From this point on, $(\Sch_{/X})_{\fet}$ will also be used to denote the small \'etale site of finite \'etale $X$-schemes. Note that this is a full subsite of the small \'etale site $(\Sch_{/X})_{\et}$ of \'etale $X$-schemes. 
                \item Additionally, for any base scheme $X$ and any field $F$, $\LocSys^n_F(X)$ shall denote the groupoid of locally constant sheaves of $n$-dimensional $F$-vector spaces and isomorphisms between them. Likewise, for any topological group $G$, we shall write $\Rep_F^n(G)^{\cont}$ for the groupoid of continuous $n$-dimensional $F$-linear representations of $G$ and isomorphisms between them. 
            \end{itemize}
        \end{convention}
        \begin{convention}[The Curve] \label{conv: base_curve}
            Henceforth, $X$ shall be a smooth projective \textit{connected} curve over $\Spec k$ (with $k$ some field).
        \end{convention}
        
        \begin{remark}
            For how stalks of \'etale sheaves are computed, see \cite[\href{https://stacks.math.columbia.edu/tag/03PN}{Tag 03PN}]{stacks}.
        \end{remark}
        \begin{lemma}[Representations of the \'etale fundamental group are local systems] \label{lemma: representations_of_the_etale_fundamental_group}
            Let $F$ be a topological field and fix a geometric point $\bar{x} \in X$. Then, for each $n \geq 1$, there exists a canonical equivalence:
                $$\LocSys_F^n(X_{\et}) \cong \Rep_F^n(\pi_1(X_{\fet}))^{\cont}$$
                $$\calL \mapsto \calL_{\bar{x}}$$
        \end{lemma}
            \begin{proof}
                
            \end{proof}
        \begin{remark}[Groupoid structure on character groups] \label{remark: groupoids_of_characters}
            Let $G$ be a topological group and $F$ an algebraically closed topological field. Observe that because characters - by being identically $1$-dimensional - are necessarily irreducible as linear representations, one gets via Schur's Lemma (cf. \cite[Lemma 3.6, pp. 35]{lam_first_course_in_noncommutative_rings}) that all discrete $F$-characters of $G$ are isomorphic. Additionally, since $F$ is an algebraically closed field, we get through another application of Schur's Lemma (or rather, the fact that invertible matrices over an algebraically closed field are diagonal) that the isomorphism between any pair of discrete $\ell$-adic characters $\chi_1, \chi_2 \in \Rep^1_F(G)$ are group homomorphisms $\varphi_{\lambda}: \GL_1(F) \to \GL_1(F)$ (for some $\lambda \in F^{\x}$) given by:
                $$\forall g \in G: \forall v \in \GL_1(F): \chi_2(g)(v) = (\chi_1 \circ \varphi_{\lambda})(g)(v) = \lambda \cdot \chi_1(g)(v)$$
            Evidently, if $\chi_1(g)$ is continuous for all $g \in G$ then so is $\lambda \cdot \chi_1(g)$. Thus, all the continuous $F$-characters of $G$ are also isomorphic or in other words, $\Rep^1_F(G)^{\cont}$ is a groupoid. Furthermore, $\Rep^1_F(G)^{\cont}$ has an underlying group structure, namely that of the group whose elements are continuous $F$-characters $\chi: G \to \GL_1(F)$, on which the group structure is pointwise multiplication (note that this is compatible with the previous interpretation of $\Rep^1_F(G)^{\cont}$ as a groupoid, since $\chi(g) \in F^{\x}$ for all $g \in G$ and all $\chi \in \Rep^1_F(G)^{\cont}$, and hence the characters do in fact differ by non-zero scalar multiples).
        \end{remark}
        \begin{lemma}[Abelianising continuous characters] \label{lemma: abelianising_continuous_characters}
            Let $G$ be a topological group and $F$ a topological field. Then, there is an equivalence of groupoids:
                $$\Rep^1_F(G)^{\cont} \cong \Rep^1_F(G^{\ab})^{\cont}$$
        \end{lemma}
            \begin{proof}
                Thanks to remark \ref{remark: groupoids_of_characters}, it shall suffice to show that the underlying discrete groups of the two groupoids are isomorphic. There is a natural injective group homomorphism:
                    $$\Rep^1_F(G)^{\cont} \to \Rep^1_F(G^{\ab})^{\cont}$$
                    $$\chi \mapsto \chi^{\ab}$$
                coming from the canonical quotient map $G \to G^{\ab}$, so the only thing to do is to show that this homomorphism is surjective. For this, it shall suffice to show that the group $\Rep^1_F([G, G])^{\cont}$ is trivial: but this is evident from the fact that $\GL_1(F)$ is abelian and from the definition of the commutator subgroup $[G, G]$, namely that $[G, G] := \<ghg^{-1}h^{-1} \mid \forall g, h \in G\>$ (so for all $x \in [G, G]$ and all $\chi \in \Rep^1_F([G, G])$, $\chi(x) = 1$), so we are done.
            \end{proof}
        \begin{theorem}[Unramified representations are sheaves on $X$] \label{theorem: unramified_representations_are_sheaves_on_X}
            Fix a geometric point $\bar{x} \in X$. Then, there is a canonical equivalence:
                $$\LocSys^1_{\overline{\Q_{\ell}}}(X) \cong \Rep^1_{\overline{\Q_{\ell}}}(\pi_1^{\ab}(X_{\fet}))^{\cont}$$
                $$\calL \mapsto \calL_{\bar{x}}$$
        \end{theorem}
            \begin{proof}
                Simply combine lemma \ref{lemma: representations_of_the_etale_fundamental_group} and lemma \ref{lemma: abelianising_continuous_characters}.
            \end{proof}