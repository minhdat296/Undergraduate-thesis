\section{The Automorphic Side}
    \subsection{The Hecke action and spectral decomposition}
        \begin{convention}[The setting of the main theorem]
            From this point on, $k$ shall be a separably closed field and $X$ shall be a smooth projective \textit{connected} curve over $\Spec k$. Additionally, $\Bun_{\GL_1}(X)$ shall denote the moduli space of line bundles on $X$.
        \end{convention}
        
        \begin{definition}[The Hecke correspondence] \label{def: hecke_correspondence}
            The Hecke correspondence is a span, i.e. a diagram of the form:
                $$
                    \begin{tikzcd}
                    	& {\Hecke_{\GL_1}(X)} \\
                    	{\Bun_{\GL_1}(X)} && {X \x \Bun_{\GL_1}(X)}
                    	\arrow["{\cev{h}}"', from=1-2, to=2-1]
                    	\arrow["{\supp_X \x \vec{h}}", from=1-2, to=2-3]
                    \end{tikzcd}
                $$
            wherein $\Hecke_{\GL_1}(X)$ is the moduli space of quadruples:
                $$(\E_1, \E_2, x, \beta_x)$$
            consisting of $\ell$-adic sheaves $\E_1, \E_2 \in \Shv_{\overline{\Q_{\ell}}}^1(\Bun_{\GL_1}(X))$, points $x \in X$, and for each such point $x$, a monomorphism $\beta_x: \E_1 \hookrightarrow \E_2$ whose cokernel is isomorphic to $k_x^{\oplus d}$ (where $k_x$ denotes the skyscraper sheaf supported at $x \in X$). It is naturally equipped with two projection functors $\cev{h}$ and $\supp_X \x \vec{h}$, which are defined via:
                $$\cev{h}(\E_1, \E_2, x, \beta_x) \cong \E_1$$
                $$\supp_X(\E_1, \E_2, x, \beta_x) \cong (x, \E_2)$$
            and hence one obtains the Hecke correspondence as a span.
        \end{definition}
        
        \begin{definition}[Hecke operators] \label{def: hecke_operators}
            
        \end{definition}
    
        \begin{definition}[Hecke eigensheaves] \label{def: hecke_eigensheaves}
            A \textit{non-zero} $\ell$-adic sheaf $\E \in \Shv_{\overline{\Q_{\ell}}}^1(\Bun_{\GL_1}(X))$ is called a \textbf{Hecke eigensheaf} (of rank $1$) if and only if there exists an $\ell$-adic sheaf $\calL \in \Shv_{\overline{\Q_{\ell}}}^1(X)$ (called the \textbf{eigenvalue} of $\E$) such that:
                $$\H_X(\E) \cong \calL \boxtimes \E$$
            It is easy to see that Hecke eigensheaves form a full symmetric monoidal subcategory of $\Shv_{\overline{\Q_{\ell}}}^1(\Bun_{\GL_1}(X))$, which we shall denote by $\Eig^1_{\overline{\Q_{\ell}}}(\Bun_{\GL_1}(X))$.
        \end{definition}
    
    \subsection{The Abel-Jacobi map and Deligne-Artin Reciprocity for global function fields}
        \begin{theorem}[Unramified abelian geometric class field theory] \label{theorem: unramified_abelian_geometric_class_field_theory}
            There exists a canonical equivalence between the category of rank-$1$ $\ell$-adic local systems on $X$ and the category of ($\ell$-adic) Hecke eigensheaves of rank $1$ on $\Bun_{\GL_1(X)}$:
                $$\LocSys_{\overline{\Q_{\ell}}}^1(X) \cong \Eig^1_{\overline{\Q_{\ell}}}(\Bun_{\GL_1}(X))$$
            which maps each local system $\calL \in \LocSys_{\overline{\Q_{\ell}}}^1(X)$ to a Hecke eigensheaf $\Aut_{\calL} \in \Eig^1_{\overline{\Q_{\ell}}}(\Bun_{\GL_1}(X))$ with eigenvalue $\calL$.
        \end{theorem}
            \begin{proof}
                \noindent
                \begin{enumerate}
                    \item \textbf{():}
                    \item
                    \item 
                \end{enumerate}
            \end{proof}
        \begin{remark}[How should we interpret theorem \ref{theorem: unramified_abelian_geometric_class_field_theory} ?] \label{remark: unramified_abelian_geometric_class_field_theory_explanation}
            By putting theorem \ref{theorem: unramified_representations_are_sheaves_on_X} and theorem \ref{theorem: unramified_abelian_geometric_class_field_theory} together, one gets a canonical equivalence of categories as follows:
                $$\Rep_{\overline{\Q_{\ell}}}^1(\pi_1^{\ab}(X_{\fet}))^{\cont} \cong \Eig^1_{\overline{\Q_{\ell}}}(\Bun_{\GL_1}(X))$$
            Modulo technicalities, what this essentially tells us is that $1$-dimensional continuous $\ell$-adic Galois representations are the same as automorphic forms associated to $\GL_1$, and as such, the combination of theorem \ref{theorem: unramified_representations_are_sheaves_on_X} and theorem \ref{theorem: unramified_abelian_geometric_class_field_theory} can be understood as the Categorical Global Unramified Geometric Langlands Correspondence in its simplest non-trivial form, that being for the (connnected reductive group $G \cong \GL_1$)\footnote{Incidentally, this is why it is commonly asserted that the Langlands Correspondence for $\GL_1$ \say{is just class field theory}.}. 
        \end{remark}
        \begin{example}[Some examples of geometric reciprocity] \label{example: geometric_reciprocity}
            \noindent
            \begin{itemize}
                \item \textbf{($X \cong \P^1$):}
                \item \textbf{(Elliptic curves):}
                \item \textbf{(Counter-example: $X \cong \A^1$):}
            \end{itemize}
        \end{example}
        
        \begin{remark}[What about local class field theory ?]
            
        \end{remark}