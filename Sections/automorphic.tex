\section{The Automorphic Side}
    \subsection{Divisors and the Abel-Jacobi map}
        \subsubsection{Divisors}
            \begin{convention}[The Picard group] \label{conv: picard_group}
                For any base scheme $S$ and any $S$-scheme $Y$, we shall write $|\Pic_{Y/S}|$ for the group of isomorphism classes of line bundles on $Y$, whose group structure is given by tensor products of invertible quasi-coherent $\calO_{Y/S}$-modules.
            \end{convention}
            \begin{definition}[Divisors] \label{def: divisors}
                Let $Y$ be a scheme. An \textbf{effective (Cartier) divisor}\footnote{Historically referred to as a \say{modulus}.} on $Y$ is then a closed subscheme $D \subset Y$ whose ideal sheaf $\calI_D$ is a line bundle on $Y$. Its \textbf{degree} is the \href{https://stacks.math.columbia.edu/tag/0AYQ}{\underline{degree}} of the line bundle $\calI_D$.  
                    
                The set of effective divisors (respectively, effective divisors of degree $d$) shall be denoted by $|\Div_Y^{\eff}|$ (respectively, $|\Div_Y^{\eff, (d)}|$), and as a straightforward consequence of the definition of effective divisors, it is precisely the set of invertible\footnote{A quasi-coherent ideal sheaf $\calJ \subset \calO_Y$ is invertible if and only if its local sections $\calJ(V)$ are principal ideals of $\calO_Y(V)$.} quasi-coherent $\calO_Y$-ideals (respectively, quasi-coherent $\calO_Y$-ideals of degree $d$).
            \end{definition}
            \begin{remark}[Why do we care about divisors ?] 
                Because the function field of the curve $X$ over $\Spec k$ from convention \ref{conv: automorphic_side_conventions} is some (global) field of the form $K \cong k'(t)$, where $k'/k$ is an algebraic extension (cf. proposition \ref{prop: curves_and_function_fields}), and since its ring of integers $\bbO_X \cong k'[t]$ is a Dedekind domain (this is due to Hilbert's Basis Theorem, which tells us that $\dim k'[t] = \dim k + 1 = 0 + 1 = 1$), every of the stalks $\calO_{X, x}$ at points $x \in |X|$ must be a discrete valuation ring. This tells us that there is a bijective correspondence between points $x \in |X|$ and places of $K$ that are trivial on $k'$. Now, also because $\bbO_X$ is a Dedekind domain, every ideal $\a$ therein factors into a (formal) product of primes, say $\a = \prod_{i = 1}^n \p_i^{e_i}$. Furthemore, $\bbO_X$ is actually a PID, as it is isomorphic to $k'[t]$, which is a UFD thanks to $k'$ itself being a UFD, by virtue of being a field. Thus, points $x \in |X|$ are not only in bijection with places of $K$ that are trivial on $k'$, but also effective divisors on $X$.
            \end{remark}
            
            \begin{convention}
                Given any effective divisor $D \subset Y$, any integer $n$, and any $\E \in \QCoh_X$, let us write $\E(nD) \cong \E \tensor_{\calO_Y} \calI_D^{\tensor (-n)}$ (wherein $\calI_D^{\tensor (-n)} \cong (\calI_D^{\tensor (-1)})^{\tensor n}$). In particular, note that $\calO_Y(-D) \cong \calI_D$.
            \end{convention}
            \begin{remark}[Addition of effective divisors] \label{remark: adding_effective_divisors}
                Let $Y$ be a scheme. Because effective divisors are line bundles, they are trivially flat\footnote{Given any line bundle $\calL \in |\Pic_Y|$, the functor $- \tensor_{\calO_Y} \calL$ is an auto-equivalence of $\Coh_Y$, which is an abelian category, so $- \tensor_{\calO_Y} \calL$ is automatically (left-)exact.}, so given any pair of effective divisors $D, D' \subset Y$, one can define a new effective divisor $D + D'$ corresponding to the $\calO_Y$-ideal $\calI_D \calI_{D'}$, which is isomorphic to $\calI_D \tensor_{\calO_{X/k}} \calI_{D'}$ due to flatness, and hence $\deg(D + D') = \deg D + \deg D'$.
            \end{remark}
            
            \begin{convention}[The setting for geometric class field theory] \label{conv: automorphic_side_conventions}
                \noindent
                \begin{itemize}
                    \item First of all, let us fix once and for all a smooth, projective, and geometrically connected curve over an algebraically closed field $k$ (we will actually relax the algebraic closure constraint on $k$ later, as this is simply a matter of convenience).
                    \item For us, $\Bun_{\GL_1}(X)$ shall denote the moduli space of line bundles on $X$. Traditionally, this is usually referred to as the \say{Picard stack} (in this case, it is actually a scheme, not just a stack; see remark \ref{remark: geometry_of_the_picard_stack}) and denoted by $\Pic_{X/k}$ (cf. \cite[\href{https://stacks.math.columbia.edu/tag/0372}{Tag 0372}]{stacks}), but we opt for the notation $\Bun_{\GL_1}(X)$ because in the wider context of the Geometric Langlands Programme, one works with $\Bun_G(X)$ for $G$ a general connected reductive group (of which $\GL_1$ is a special case), and to not confuse the moduli space $\Bun_{\GL_1}(X)$ with the group $|\Pic_{X/k}|$ (cf. convention \ref{conv: picard_group}). 
                \end{itemize}
            \end{convention}
            \begin{remark}[The geometry of $\Bun_{\GL_1}$] \label{remark: geometry_of_the_picard_stack}
                Since we are working with a smooth projective curve $X$ over an algebraically closed field (cf. convention \ref{conv: automorphic_side_conventions}) and hence over a separably closed field, the prestack $\Bun_{\GL_1}$ is \textit{a priori} represented by a scheme (cf. \cite[\href{https://stacks.math.columbia.edu/tag/0B9Z}{Tag 0B9Z}]{stacks}) as an fppf sheaf on $X$ (and hence as an \'etale and as a Zariski sheaves, since the fppf toppology is finer than both these topologies). As a result, when considering sheaves on $\Bun_{\GL_1}(X)$, we will only need to know about sheaves on schemes instead of the entire fully general theory of sheaves on prestacks. Furthermore, if the genus of $X$ is $g \geq 0$, then one will have a decomposition $\Bun_{\GL_1}(X) \cong \coprod_{d \geq 0} \Bun_{\GL_1}^{(d)}(X)$, wherein each $\Bun_{\GL_1}^{(d)}(X)$ is the moduli scheme of line bundles of degree $d$  on $X$, which is a proper smooth variety of dimension $g$ over $\Spec k$ (cf. \cite[\href{https://stacks.math.columbia.edu/tag/0BA0}{Tag 0BA0}]{stacks}).
            \end{remark}
            
            \begin{remark}[Moduli space of effective divisors] \label{remark: moduli_space_of_effective_divisors}
                For any base scheme $S$ and any $S$-scheme $Y$, the \textbf{Hilbert functor} of closed subschemes of degree $d \geq 0$ is the presheaf:
                    $$\Hilb_{Y/S}^{(d)}: \Sch_{/S}^{\op} \to \Sets$$
                    $$T \mapsto \{\text{Finite locally free closed subschemes $D \subset Y_T$ of degree $d$}\}$$
                Should $Y$ be a geometrically irreducible smooth proper (respectively projective) curve over a field $C$ then interestingly, not only are finite locally free closed subschemes $D \subset Y_{C'}$ of degree $d$ precisely the effective divisors of degree $d$ on $Y_{C'}$ for any field extension $C'/C$ (cf. \cite[\href{https://stacks.math.columbia.edu/tag/0B9D}{Tag 0B9D}]{stacks}), but also, one has a bijection:
                    $$|\Div_{Y_{C'}/C'}^{\eff, (d)}| \cong \Hilb_{Y/C}^{(d)}(C')$$
                between the set of $C'$-rational points of $\Hilb_{Y/C}^{(d)}$ and that of degree-$d$ effective divisors on $Y_{C'}$ (cf. \cite[\href{https://stacks.math.columbia.edu/tag/0B9I}{Tag 0B9I}]{stacks}). From this, one see that should $Y$ be proper (respectively Zariski-locally projective) and flat over some arbitrary base scheme $S$, and if its fibres $Y_s$ over points $s \in |S|$ are geometrically irreducible smooth proper (respectively projective) curves, then $\Hilb_{Y/S}^{(d)}$ would be the moduli space of degree-$d$ effective divisors on $Y$; thus, for proper (respectively Zariski-locally projective) and flat morphisms $Y \to S$, let us suggestively write $\Div_{Y/S}^{\eff, (d)}$ instead of $\Hilb_{Y/S}^{(d)}$. It is known moreover that $\Hilb_{Y/C}^{(d)}$ is represented by a smooth proper variety of dimension $d$ over $\Spec C$ (cf. \cite[\href{https://stacks.math.columbia.edu/tag/0B9I}{Tag 0B9I}]{stacks}). By putting everything together, one obtains a moduli space $\Div_{Y/C}^{\eff, (d)} \in (Y/C)_{\fppf}$ parametrising degree-$d$ effective divisors on $Y$, represented by a smooth proper variety of dimension $d$ over $\Spec C$ and naturally isomorphic to $\Hilb_{Y/C}^{(d)}$.
            \end{remark}
            
            It should also be noted that what we have just discussed is not the only way to show that $\Div_{X/k}^{\eff, (d)}$ is represented by a smooth proper variety of dimension $d$. Remark \ref{remark: moduli_space_of_effective_divisors} serves more as a demonstration that there \textit{should} be a moduli space of effective divisors (of a given degree $d$), rather than that there \textit{is} one. In fact, by combining remark \ref{remark: quotients_of_schemes_by_finite_group_schemes}, lemma \ref{lemma: smoothness_of_symmetric_powers}, and proposition \ref{prop: symmetric_powers_of_curves_parametrise_divisors}, we shall see that the functor $\Div_{X/k}^{\eff, (d)}$ is represented by a smooth proper variety of (pure) dimension $d$ by virtue of being naturally isomorphic to the functor of points of $X^{(k)}$ (which, of course, is smooth, proper, and of dimension $d$). It is, however, important to know that $\Div_{X/k}^{\eff, (d)}$ indeed satisfies fppf descent (hence \'etale descent) to prove proposition \ref{prop: symmetric_powers_of_curves_parametrise_divisors}, and since this comes from the general fact that the Hilbert functor satisfies fppf descent, remark \ref{remark: moduli_space_of_effective_divisors} remains necessary. 
            
            We now know that effective divisors of a given degree $d$ are parametrised by some smooth proper variety $\Div_{Y/S}^{\eff, (d)}$ of (relative) dimension $d$, but this is not entirely satisfactory: we would also like to know the identity of this smooth proper variety, and we shall after proposition \ref{prop: symmetric_powers_of_curves_parametrise_divisors}.
            \begin{remark}[Quotient of schemes by finite group schemes] \label{remark: quotients_of_schemes_by_finite_group_schemes}
                For details on quotients of schemes by finite groups, we refer the reader to \cite[Expos\'e V]{SGA1}. For our purposes, we shall only need to keep in mind the following facts: 
                    \begin{enumerate}
                        \item Let $S$ be a base scheme and $Y$ be an $S$-scheme that is either \textit{(quasi-)projective or (quasi-)affine}, and if additionally. If $G$ be a \textit{finite, flat, and locally of finite presentation} group $S$-scheme acting \textit{freely}\footnote{This is to ensure that the $G$-action induces an fppf equivalence relation on $Y$, since the $G$-action on $Y$ is free if and only if the corresponding homomorphism of sheaves of groups $G \to \Aut_{S_{\fppf}}(Y)$ is injective.} on $Y$, then the \href{https://stacks.math.columbia.edu/tag/025X}{\underline{algebraic space}} $Y/G$ is a scheme (cf. \cite[\href{https://stacks.math.columbia.edu/tag/07S7}{Tag 07S7}]{stacks}). 
                        \item If $Y$ is a (quasi-)affine scheme $\Spec A$ then the quotient $Y/G$ will also be affine and will be isomorphic to $\Spec A^G$, thanks to the group-cohomological fact that $H^0(A, G) \cong A^G$. 
                        
                        If $Y$ is (quasi-)projective then $Y/G$ will also be (quasi-)projective.
                    \end{enumerate}
            \end{remark}
            
        \subsubsection{Symmetric powers of curves; the Abel-Jacobi map}
            \begin{definition}[Symmetric powers of schemes] \label{def: symmetric_powers_of_schemes}
                Let $S$ be a base scheme and let $Y$ be an $S$-scheme that is either (quasi-)projective or (quasi-)affine\footnote{In particular, the curve $X$ from convention \ref{conv: automorphic_side_conventions} is projective.}. Then for any $d \geq 1$, the \textbf{$d^{th}$ symmetric power} of $Y$ is the quotient scheme $\Sym^d_S(Y) := Y^d/\underline{\Sigma_d}_{/S}$ of $Y$ by the constant symmetric $S$-group scheme $\underline{\Sigma_d}_{/S}$ on $d$ elements (which is finite, flat, and locally of finite presentation over $S$, since it is represented by $\coprod_{\sigma \in \Sigma_d} S$); here, $\underline{\Sigma_d}_{/S}$ acts via permutations, which is well-known to be a free action.
            \end{definition}
            \begin{convention}
                Because the base field $k$ of our curve from convention \ref{conv: automorphic_side_conventions} is fixed, let us write $X^{(d)}$ instead of $\Sym^d_k(X)$ for simplicity.
            \end{convention}
            \begin{lemma}[Smoothness of symmetric powers] \label{lemma: smoothness_of_symmetric_powers}
                If $Y$ is a smooth variety over some field $C$ of dimension $\leq 1$ then so is $\Sym_C^d(Y)$.
            \end{lemma}
                \begin{proof}
                    Smoothness is preserved by base change, so we might as well assume that $C$ is algebraically closed. The case $\dim Y = 0$ is trivial, so let us assume that $\dim Y = 1$ (i.e. that $Y$ is a curve); in this case, the formal completion of the stalk $\calO_{Y, y}$ at any point $y \in |Y|$ is necessarily isomorphic to $C[\![y]\!]$\footnote{We are intentionally confusing the point $y \in |Y|$ and the formal variable $y \in C[\![y]\!]$ because $(y)$ is the unique maximal ideal of $C[\![y]\!]$.}, since $\calO_{Y, y}$ shall be a finite-type commutative algebra over an algebraically closed field; as a result, the formal completion of the stalk $\calO_{\Sym^d_C(Y), \vec{y}}$ at any point $\vec{y} \in \Sym^d_C(Y)$ is isomorphic to symmetric formal power series ring $k[\![y_1, ..., y_d]\!]^{\Sigma_d}$, which itself is isomorphic to $k[\![y_1, ..., y_d]\!]$ \textit{a priori}. A Noetherian local ring $(A, \m)$ is regular if and only if its $\m$-adic completion is regular, and if $A, B$ are finite-type regular commutative algebras over an algebraically closed field $C$ then $A \tensor_C B$ will also be regular as a $C$-algebra, so $\Sym_C^d(Y)$ will be smooth if $C[\![y]\!]$ is regular, which is definitely the case since $\dim C[\![y]\!] = \dim_C (y)/(y)^2 = 1$.
                \end{proof}
            \begin{proposition}[Symmetric powers of curves parametrise divisors] \label{prop: symmetric_powers_of_curves_parametrise_divisors}
                For each $d$, the moduli space $\Div_{X/k}^{\eff, (d)}$ is represented by the smooth variety $X^{(d)}$.
            \end{proposition}
                \begin{proof}
                    Let us denote \textit{unordered} $d$-tuples by $\<x_1, ..., x_d\>$ and also, write $[x]$ for the divisor cut out by any closed point $x \in |X|$. Now, to begin, consider the following $\Sigma_d$-invariant function, which if shown to be bijective will demonstrate that $\Div_{X/k}^{\eff, (d)}$ is represented by the smooth variety $X^{(d)}$ via \'etale descent (cf. \cite[\href{https://stacks.math.columbia.edu/tag/024V}{Tag 024V}]{stacks}):
                        $$X^{(d)} \to |\Div_{X/k}^{\eff, (d)}|$$
                        $$\<x_1, ..., x_d\> \mapsto [x_1] + ... + [x_d]$$
                    Now, $X$ is a geometrically connected smooth curve, so it is geometrically normal (cf. \cite[\href{https://stacks.math.columbia.edu/tag/056T}{Tag 056T}]{stacks}), which in turn implies that the stalk of its structure sheaf over its unique generic point is a normal local domain of Krull dimension $1$, hence a Dedekind domain (cf. \cite[\href{https://stacks.math.columbia.edu/tag/034X}{Tag 034X}]{stacks}). This implies that every divisor on $X$ splits into prime divisors (which correspond to closed points of $X$), and thus the function $\<x_1, ..., x_d\> \mapsto [x_1] + ... + [x_d]$ is surjective. Dedekind domains are special cases of UFDs, so we have also demonstrated that the function $\<x_1, ..., x_d\> \mapsto [x_1] + ... + [x_d]$ is injective.
                \end{proof}
            \begin{corollary}[$X \cong \Div_{X/k}^{\eff, (1)}$]
                $k$-rational points of $X$ are precisely the degree-$1$ effective divisors.
            \end{corollary}
            \begin{convention}
                From this point on, we shall write $\Div_{X/k}^{\eff, (d)}$ instead of $X^{(d)}$ whenever we would like to put emphasis on the fact that points of $X^{(d)}$ are effective divisors of degree $d$ on $X$ (such as in \ref{prop: the_unramified_abel_jacobi_map_is_a_projective_bundle}), and \textit{vice versa}, we shall write $X^{(d)}$ when symmetry is of importance, like in theorem \ref{theorem: unramified_abelian_geometric_class_field_theory}.
            \end{convention}
            
            \begin{definition}[The Abel-Jacobi map] \label{def: the_abel_jacobi_map}
                Let $Y$ be a geometrically connected smooth projective curve over some field $C$. Then, the \textbf{$d^{th}$ Abel-Jacobi map} associated to $Y/C$ is the morphism of smooth proper varieties:
                    $$\AJ_{Y/C}^{(d)}: \Div_{Y/C}^{\eff, (d)} \to \Bun_{\GL_1}^{(d)}(Y)$$
                which section-wise (i.e. at each field extension $C'/C$) associates to each degree-$d$ effective divisor $D \in |\Div_{Y_{C'}/C'}^{\eff, (d)}|$ to its corresponding invertible quasi-coherent $\calO_{Y_{C'}}$-ideal $\calI_D \in |\Pic_{Y_{C'}/C'}^{(d)}|$.
            \end{definition}
            \begin{remark}[What does the Abel-Jacobi map do ?] \label{remark: abel_jacobi_map}
                
            \end{remark}
            \begin{convention}[Genus of the curve] \label{conv: genus_of_the_curve}
                From now on, denote the \href{https://stacks.math.columbia.edu/tag/0BY6}{\underline{genus}} of our curve $X$ by $g$.
            \end{convention}
            \begin{proposition}[The Abel-Jacobi map is a projective bundle] \label{prop: the_unramified_abel_jacobi_map_is_a_projective_bundle}
                If $d \geq 2g - 1$ then every Abel-Jacobi map $\AJ_{X/k}^{(d)}: \Div_{X/k}^{\eff, (d)} \to \Bun_{\GL_1}^{(d)}(X)$ will be a surjective smooth projective morphism with fibres\footnote{Note that these are precisely the geometric fibres, since $k$ is algebraically closed.} over $k$-rational points isomorphic to $\P^{d - g}_k$.
            \end{proposition}
                \begin{proof}
                    Let us note, first of all, that thanks due to a descent-theoretic feature of the \'etale topology, namely \cite[\href{https://stacks.math.columbia.edu/tag/024V}{Tag 024V}]{stacks}, it shall suffice to demonstrate that the \textit{set-theoretic} fibres of the Abel-Jacobi maps are in bijection with the set of $k$-rational points of $\P_k^{d - g}$; for the same reason, it also suffices to only show that the Abel-Jacobi map is surjective at each point $k$-rational point $\calL \in |\Bun_{\GL_1}^{(d)}(X)|$. We shall proceed in steps, for the sake of clarity.
                        \begin{enumerate}
                            \item \textbf{(Projectivity):} Let us first apply the Riemann-Roch Theorem (cf. \cite[\href{https://stacks.math.columbia.edu/tag/0BS6}{Tag 0BS6}]{stacks}; note that the theorem is applicable to our situation because $X$ is a geometrically connected smooth projective curve and therefore a Gorenstein\footnote{Smooth schemes are Gorenstein because their stalks are regular local rings.} scheme of equidimension $1$ over a field), which tells us that should $\E \in \Vect_{X/k}^n$ be a be a locally free quasi-coherent $\calO_{X/k}$-module of constant rank $n$, then:
                                $$\chi(X, \E) = \deg(\E) - \frac12\rank(\E) \deg(\omega_{X/k})$$
                            where $\omega_{X/k}$ denotes the dualising sheaf (which is a line bundle due also to the Riemann-Roch Theorem) and $\chi(X, \E)$ denotes the Euler characteristic of $\E$ as a coherent sheaf on the proper $k$-scheme $X$. Because $\deg(\omega_{X/k}) = 2g - 2$ (thanks to \cite[\href{https://stacks.math.columbia.edu/tag/0C19}{Tag 0C19}]{stacks}, which is applicable in this situation because $X$ is a proper Gorenstein $k$-scheme such that $H^0_{\Zar}(X, \calO_{X/k}) \cong k$), the above tells us that for any degree-$d$ line bundle $\calL \in |\Bun_{\GL_1}^{(d)}(X)|$, we have:
                                $$\chi(X, \calL) = \deg(\calL) - \frac12 \rank(\calL) \deg(\omega_{X/k}) = d - \frac12 \cdot 1 \cdot (2g - 2) = d - g + 1$$
                            Line bundles are particular cases of coherent sheaves with support dimension $\leq 0$, and since $X$ is proper over a field, we can apply \cite[\href{https://stacks.math.columbia.edu/tag/0AYT}{Tag 0AYT}]{stacks} to get that:
                                $$\dim_k H^0_{\Zar}(X, \calL) = \chi(X, \calL) = d - g + 1$$
                            This implies that the (set-theoretic) fibres of $\AJ_{X/k}^{(d)}$ over $k$-rational points $\calL \in |\Bun_{\GL_1}^{(d)}(X)|$ are isomorphic to $\P^{d - g}_k$ (empty if $d < g$).
                            \item \textbf{(Smoothness):} $\Bun_{\GL_1}^{(d)}(X)$ is smooth (cf. remark \ref{remark: geometry_of_the_picard_stack}), hence it is regular, and $X^{(d)}$ is also smooth (cf. lemma \ref{lemma: smoothness_of_symmetric_powers}) and hence it is Cohen-Macaulay (because regular local rings are Cohen-Macaulay \textit{a priori}; cf. \cite[\href{https://stacks.math.columbia.edu/tag/00NQ}{Tag 00NQ}]{stacks}). Moreover, we have shown above that given any $\calL \in \Bun_{\GL_1}^{(d)}(X)$, the corresponding fibre of the Abel-Jacobi map is isomorphic to $\P^{d - g}_k$, and since $\dim \Bun_{\GL_1}^{(d)}(X) = g$ (cf. remark \ref{remark: geometry_of_the_picard_stack}) while $\dim X^{(d)} = \dim \Div_{X/k}^{\eff, (d)} = d$ (cf. remark \ref{remark: moduli_space_of_effective_divisors}), we have:
                                $$\dim \P^{d - g}_k = \dim (\AJ_{X/k}^{(d)})^{-1}(\calL) = \dim \Div_{X/k}^{\eff, (d)} - \dim \Bun_{\GL_1}^{(d)}(X) = d - g$$
                            The Miracle Flatness Theorem (cf. \cite[\href{https://stacks.math.columbia.edu/tag/00R4}{Tag 00R4}]{stacks}) can then be applied, which tells us that the Abel-Jacobi map is flat everywhere, and because the fibres are isomorphic to $\P^{d - g}_k$, which is smooth over $\Spec k$, this means that the Abel-Jacobi map is also smooth everywhere.
                            \item \textbf{(Surjectivity):} Finally, let us demonstrate that the canonical map $H^0_{\Zar}(X, \calL) \to H^0(X, \calL|_D)$ is surjective
                        \end{enumerate}
                \end{proof}
            \begin{corollary}[Unramified Galois representations induced by the Abel-Jacobi map] \label{coro: unramified_galois_representations_induced_by_the_abel_jacobi_map}
                Because $\AJ_{X/k}^{(d)}$ is proper and smooth, it is proper, flat, and of finite presentation, so by proposition \ref{prop: etale_homotopy_exact_sequence} there is an induced \'etale homotopy sequence as follows:
                    $$\pi_1((\P^{d - g}_k)_{\fet}) \to \pi_1((\Div_{X/k}^{\eff, (d)})_{\fet}) \to \pi_1((\Bun_{\GL_1}^{(d)}(X))_{\fet}) \to 1$$
                Since $\P^{d - g}_k$ is \'etale-simply connected (this is a consequence of $k$ being algebraically closed; cf. example \ref{example: etale_fundamental_group_of_a_curve}), one thus obtains an equivalence between the categories of continuous $\ell$-adic characters of $\pi_1((\Div_{X/k}^{\eff, (d)})_{\fet})$ and of $\pi_1((\Bun_{\GL_1}^{(d)}(X))_{\fet})$ as below, wherein $(\AJ_{X/k}^{(d)})^*$ is the pullback of $\ell$-adic sheaves along the Abel-Jacobi map $\AJ_{X/k}^{(d)}: \Div_{X/k}^{\eff, (d)} \to \Bun_{\GL_1}^{(d)}(X)$:
                    $$\Rep^{\cont, 1}_{\bar{\Q}_{\ell}}(\pi_1((\Div_{X/k}^{\eff, (d)})_{\fet})) \cong \Rep^{\cont, 1}_{\bar{\Q}_{\ell}}(\pi_1((\Bun_{\GL_1}^{(d)}(X))_{\fet}))$$
                    $$\chi \mapsto \chi \circ (\AJ_{X/k}^{(d)})^*$$
            \end{corollary}
    
     \subsection{Hecke eigensheaves and geometric class field theory}
        \begin{convention}[Symmetric powers of line bundles] \label{conv: symmetric_powers_of_line_bundles}
            \noindent
            \begin{itemize}
                \item Write $\sigma^{(d)}: X^d \to X^{(d)}$ for the canonical quotient map and set $\Delta_X^{(d)} := \sigma^{(d)} \circ \Delta_X^d$. Next, for each $\ell$-adic local system $\calL \in \Shv_{\underline{\bar{\Q}_{\ell}}}(X)$, we can construct an $\ell$-adic local system $\calL^{(d)} \in \Shv_{\underline{\bar{\Q}_{\ell}}}(X^{(d)})$ given by $\calL^{(d)} \cong ((\Delta_X^{(d)})_*\calL)^{\Sigma_d}$. 
                \item In addition, write $\tilde{h}_X^{(d)}: X \x X^{(d)} \to X^{(d + 1)}$ for the map given by $(-[x], D) \mapsto [x] + D$, and likewise, write $\cev{h}_X^{(d)}: X \x \Bun_{\GL_1}^{(d)}(X) \to \Bun_{\GL_1}^{(d + 1)}(X)$ for the map given by $\cev{h}_X^{(d)}(-[x], \E) \cong \E(-[x])$. 
            \end{itemize}
        \end{convention}
        \begin{definition}[Hecke eigensheaves] \label{def: hecke_eigensheaves}
            A \textit{non-zero} $\ell$-adic local system $\E \in \Shv_{\underline{\bar{\Q}_{\ell}}}^{\ad, 1}(\Bun_{\GL_1}(X))$ is called a \textbf{Hecke eigensheaf} (of rank $1$) if and only if there exists an $\ell$-adic sheaf $\calL \in \Shv_{\underline{\bar{\Q}_{\ell}}}^{\ad, 1}(X)$ (called the \textbf{eigenvalue} of $\E$) such that:
                $$\cev{h}_X^*(\E) \cong \calL \boxtimes \E$$
        \end{definition}
        \begin{remark}
            It is easy to see that Hecke eigensheaves form a full symmetric monoidal subcategory of $\Shv_{\underline{\bar{\Q}_{\ell}}}^{\ad, 1}(\Bun_{\GL_1}(X))$, which we shall denote by $\Eig\Shv_{\underline{\bar{\Q}_{\ell}}}^1(\Bun_{\GL_1}(X))$. In fact, the set of isomorphism classes of Hecke eigensheaves form a subgroup of $\Shv_{\underline{\bar{\Q}_{\ell}}}^{\ad, 1}(\Bun_{\GL_1}(X))$ with respect to tensor products. 
        \end{remark}
        
        Lemma \ref{lemma: hecke_eigensheaves_extend_to_lower_degrees}, which is regarding the \say{globality} of Hecke eigensheaves is merely a technicality in service of theorem \ref{theorem: unramified_abelian_geometric_class_field_theory}. The reader can safely skip ahead and refer back to it later.
        \begin{lemma}[Hecke eigensheaves extend to lower degrees] \label{lemma: hecke_eigensheaves_extend_to_lower_degrees}
            Let $\E$ be a Hecke eigensheaf on $\bigcup_{d \geq d_0 + 1} \Bun_{\GL_1}^{(d)}(X)$ (for any $d_0 \geq 0$) with eigenvalue $\calL \in \Shv_{\underline{\bar{\Q}_{\ell}}}^{\ad, 1}(X)$. Then, $\E$ can be extended uniquely to a Hecke eigensheaf on $\bigcup_{d \geq d_0} \Bun_{\GL_1}^{(d)}(X)$, also with eigenvalue $\calL$.
        \end{lemma}
            \begin{proof}
                Consider the following commutative diagram, where $\cev{h}_x^{(d)}$ is given by $\calL \mapsto \calL(-[x])$:
                    $$
                        \begin{tikzcd}
                        	{\Spec k \x \Bun_{\GL_1}^{(d)}(X)} & {\Bun_{\GL_1}^{(d)}(X)} \\
                        	{X \x \Bun_{\GL_1}^{(d)}(X)} & {\Bun_{\GL_1}^{(d)}(X)}
                        	\arrow["{\cev{h}_X^{(d)}}", from=2-1, to=2-2]
                        	\arrow["{\cev{h}_x^{(d)}}", from=1-1, to=1-2]
                        	\arrow["{x \x \id}"', from=1-1, to=2-1]
                        	\arrow[Rightarrow, no head, from=1-2, to=2-2]
                        \end{tikzcd}
                    $$
                By definition, we have an isomorphism $(\cev{h}_x^{(d)})^*\E \cong \calL \boxtimes \E$ for any Hecke eigensheaf $\E$ on $\Bun_{\GL_1}^{(d)}(X)$ with eigenvalue $\calL$, which induces an isomorphism $(\cev{h}_x^{(d)})^*\E \cong x^*\calL \boxtimes \E$ at each geometric point $x \in X$; but since $x \in X$ is a geometric point, $x^*\calL$ is - by definition - nothing but the stalk $\calL_x$, which is isomorphic to $\bar{\Q}_{\ell}$, since $\calL$ is an $\ell$-adic local system of rank $1$ on $X$. Next, fix an arbitrary finite set of geometric points $x_1, x_2, ..., x_n \in X$ and consider the following for any Hecke eigensheaf $\E_{\calL^{(d + n)}}$ on $\bigcup_{d \geq 2g - 1} \Bun_{\GL_1}^{(d)}(X)$ that corresponds to $\calL^{(d + n)} \in \Shv_{\underline{\bar{\Q}_{\ell}}}^{\ad, 1}(X^{(d + n)})$:
                    $$(\cev{h}_{x_1}^* \circ ... \circ \cev{h}_{x_n}^*)(\E_{\calL^{(d + n)}}) \cong \bigotimes_{i = 1}^n (\calL_{x_i} \boxtimes \E_{\calL^{(d)}})$$
                which implies that:
                    $$\bigotimes_{i = 1}^n \calL_{x_i}^{\tensor (-1)} \boxtimes (\cev{h}_{x_1}^* \circ ... \circ \cev{h}_{x_n}^*)(\E_{\calL^{(d + n)}}) \cong \E_{\calL^{(d)}}$$
                If we now take $d := 2g - 1 - n$, we will get the following equation on $\Spec k \x \Bun_{\GL_1}^{2g - 1 - n}(X)$, which yields us a \textit{unique} Hecke eigensheaf $\E_{\calL^{(2g - 1 - n)}}$ on $\bigcup_{d \geq 2g - 1 - n} \Bun_{\GL_1}^{(d)}(X)$ from $\E_{\calL^{(2g - 1)}}$ (which we have already):
                    $$\E_{\calL^{(2g - 1 - n)}} \cong \bigotimes_{i = 1}^n \calL_{x_i}^{\tensor (-1)} \boxtimes (\cev{h}_{x_1}^* \circ ... \circ \cev{h}_{x_n}^*)(\E_{\calL^{(2g - 1)}})$$
            \end{proof}
            
        \begin{theorem}[Unramified abelian geometric class field theory] \label{theorem: unramified_abelian_geometric_class_field_theory}
            There exists a canonical monoidal equivalence between the groupoid of rank-$1$ $\ell$-adic local systems on $X$ and the groupoid of ($\ell$-adic) Hecke eigensheaves of rank $1$ on $\Bun_{\GL_1}(X)$:
                $$\Autom: \Shv_{\underline{\bar{\Q}_{\ell}}}^{\ad, 1}(X) \to \Eig\Shv_{\underline{\bar{\Q}_{\ell}}}^1(\Bun_{\GL_1}(X))$$
            which maps each $\ell$-adic local system $\calL \in \Shv_{\underline{\bar{\Q}_{\ell}}}^{\ad, 1}(X)$ to a Hecke eigensheaf $\Autom(\calL) \in \Eig\Shv_{\underline{\bar{\Q}_{\ell}}}^1(\Bun_{\GL_1}(X))$ with eigenvalue $\calL$.
        \end{theorem}
            \begin{proof}
                Our strategy for this proof is to explicitly construct - for each $\calL \in \Shv_{\underline{\bar{\Q}_{\ell}}}^{\ad, 1}(X)$ - the corresponding Hecke eigensheaf $\Autom(\calL)$, and this will involve three steps:
                    \begin{enumerate}
                        \item \textbf{(A \say{Hecke eigensheaf} on $X^{(d)}$):} In this step we shall construct an $\ell$-local system on $X^{(d)}$ (hence on $\bigcup_{d \geq 2g - 1} X^{(d)}$); cf. proposition \ref{prop: the_unramified_abel_jacobi_map_is_a_projective_bundle} satisfying an analogue of the Hecke eigensheaf property (cf. definition \ref{def: hecke_eigensheaves}) with the purpose in mind being that by pushing this local system forward using the Abel-Jacobi map, one shall obtain a legitimate Hecke eigensheaf on $\Bun_{\GL_1}^{(d)}(X)$ (hence on $\bigcup_{d \geq 2g - 1} \Bun_{\GL_1}^{(d)}(X)$).
                        
                        The first observation that one can make is that there is an isomorphism $(\sigma^{(d)})^*\calL^{(d)} \cong \calL^{\boxtimes d}$ of $\ell$-adic local systems on $X^d$. Next, notice how there are commutative diagrams of the following form, wherein the maps $\tilde{h}_X^{(d)}$ are given by $(-[x], D) \mapsto [x] + D$ (cf. remark \ref{remark: adding_effective_divisors} and convention \ref{conv: symmetric_powers_of_line_bundles}):
                            $$
                                \begin{tikzcd}
                                	{X \x X^d} & {X^{(d + 1)}} \\
                                	{X \x X^{(d)}}
                                	\arrow["{\id_X \x \sigma^{(d)}}"', from=1-1, to=2-1]
                                	\arrow["{\tilde{h}_X^{(d)}}"', dashed, from=2-1, to=1-2]
                                	\arrow["{\sigma^{(d + 1)}}", from=1-1, to=1-2]
                                \end{tikzcd}
                            $$
                        We thus have, as follows, a $\Sigma_d$-equivariant analogue on $X \x X^{(d)}$ of the Hecke eigensheaf property for all $\calL \in \Shv_{\underline{\bar{\Q}_{\ell}}}^{\ad, 1}(X)$:
                            $$(\tilde{h}_X^{(d)})^* \calL^{(d + 1)} \cong \calL \boxtimes \calL^{(d)}$$
                        \item \textbf{(A Hecke eigensheaf on $\Bun_{\GL_1}^{(d)}(X)$):} Now, recall from corollary \ref{coro: unramified_galois_representations_induced_by_the_abel_jacobi_map} that for every $d \geq 2g - 1$, there exists an adjoint equivalence:
                            $$
                                \begin{tikzcd}
                                	{\Shv_{\underline{\bar{\Q}_{\ell}}}^{\ad, 1}(X^{(d)})} & {\Shv_{\underline{\bar{\Q}_{\ell}}}^{\ad, 1}(\Bun_{\GL_1}^{(d)}(X))}
                                	\arrow[""{name=0, anchor=center, inner sep=0}, "{(\AJ_{X/k}^{(d)})_*}"', bend right, from=1-1, to=1-2]
                                	\arrow[""{name=1, anchor=center, inner sep=0}, "{(\AJ_{X/k}^{(d)})^*}"', bend right, from=1-2, to=1-1]
                                	\arrow["\dashv"{anchor=center, rotate=-90}, draw=none, from=1, to=0]
                                \end{tikzcd}
                            $$ 
                        From this, one infers that every $\ell$-adic local system $\calF \in \Shv_{\underline{\bar{\Q}_{\ell}}}^{\ad, 1}(X^{(d)})$ has the form $(\AJ_{X/k}^{(d)})^* \E$ for some \textit{unique} $\ell$-adic local system $\E \in \Shv_{\underline{\bar{\Q}_{\ell}}}^{\ad, 1}(\Bun_{\GL_1}^{(d)}(X))$, meaning that there exists $\ell$-adic local systems $\E_{\calL^{(d)}}, \E_{\calL^{(d + 1)}} \in \Shv_{\underline{\bar{\Q}_{\ell}}}^{\ad, 1}(\Bun_{\GL_1}^{(d)}(X))$ corresponding to $\calL^{(d)}, \calL^{(d + 1)} \in \Shv_{\underline{\bar{\Q}_{\ell}}}^{\ad, 1}(X^{(d)})$ respectively (and ultimately, to $\calL \in \Shv_{\underline{\bar{\Q}_{\ell}}}^{\ad, 1}(X)$) satisfying the following equation in $\Shv_{\underline{\bar{\Q}_{\ell}}}^{\ad, 1}(X \x X^{(d)})$:
                            $$(\tilde{h}_X^{(d)})^* (\AJ_{X/k}^{(d + 1)})^* \E_{\calL^{(d + 1)}} \cong \calL \boxtimes (\AJ_{X/k}^{(d)})^*\E_{\calL^{(d)}}$$
                        Next, consider the following commutative diagram:
                            $$
                                \begin{tikzcd}
                                	{X \x X^{(d)}} & {X^{(d + 1)}} \\
                                	{X \x \Bun_{\GL_1}^{(d)}(X)} & {\Bun_{\GL_1}^{(d + 1)}(X)}
                                	\arrow["{\cev{h}_X^{(d)}}", from=2-1, to=2-2]
                                	\arrow["{\id_X \x \AJ_{X/k}^{(d)}}"', from=1-1, to=2-1]
                                	\arrow["{\AJ^{(d + 1)}}", from=1-2, to=2-2]
                                	\arrow["{\tilde{h}_X^{(d)}}", from=1-1, to=1-2]
                                \end{tikzcd}
                            $$
                        which induces the following equations in $\Shv_{\underline{\bar{\Q}_{\ell}}}^{\ad, 1}(X \x X^{(d)})$ for all $\calL \in \Shv_{\underline{\bar{\Q}_{\ell}}}^{\ad, 1}(X)$:
                            $$
                                \begin{aligned}
                                    (\id_X \x \AJ_{X/k}^{(d)})^* (\cev{h}_X^{(d)})^* \E_{\calL^{(d + 1)}} & \cong (\tilde{h}_X^{(d)})^* (\AJ^{(d + 1)})^* \E_{\calL^{(d + 1)}}
                                    \\
                                    & \cong \calL \boxtimes (\AJ_{X/k}^{(d)})^*\E_{\calL^{(d)}}
                                    \\
                                    & \cong (\id_X \x \AJ_{X/k}^{(d)})^*(\calL \boxtimes \E_{\calL^{(d)}})
                                \end{aligned}
                            $$
                        and since $(\id_X \x \AJ_{X/k}^{(d)})^*$ is an invertible functor (cf. corollary \ref{coro: unramified_galois_representations_induced_by_the_abel_jacobi_map}), we have, furthermore, the following equation in $\Shv_{\underline{\bar{\Q}_{\ell}}}^{\ad, 1}(X \x \Bun_{\GL_1}^{(d)}(X))$, which is precisely the Hecke eigensheaf property from definition \ref{def: hecke_eigensheaves}:
                            $$(\cev{h}_X^{(d)})^* \E_{\calL^{(d + 1)}} \cong \calL \boxtimes \E_{\calL^{(d)}}$$
                        We have thus obtained a \textit{unique} Hecke eigensheaf on $\bigcup_{d \geq 2g - 1} \Bun_{\GL_1}^{(d)}(X)$ from an arbitrary $\ell$-adic local system $\calL \in \Shv_{\underline{\bar{\Q}_{\ell}}}^{\ad, 1}(X)$.
                        \item \textbf{(A Hecke eigensheaf on $\Bun_{\GL_1}(X)$):} Finally, in order extend the Hecke eigensheaf that we have constructed on $\bigcup_{d \geq 2g - 1} \Bun_{\GL_1}^{(d)}(X)$ to $\Bun_{\GL_1}(X)$ (i.e. to degrees $d < 2g - 1$), simply apply lemma \ref{lemma: hecke_eigensheaves_extend_to_lower_degrees}.
                    \end{enumerate}
                The monoidality of the functor $\Autom$ is a trivial consequence of the definition of Hecke eigensheaves, so we leave this to our readers.
            \end{proof}
        
        \begin{corollary}[Geometric Langlands for $\GL_1$] \label{coro: geometric_langlands_for_GL1}
            By putting theorem \ref{theorem: galois_representations_are_local_systems} and theorem \ref{theorem: unramified_abelian_geometric_class_field_theory} together, one gets a canonical monoidal equivalence of categories as follows:
                $$\Rep^{\cont, 1}_{\bar{\Q}_{\ell}}(\pi_1^{\ab}(X_{\fet})) \cong \Eig\Shv_{\underline{\bar{\Q}_{\ell}}}^1(\Bun_{\GL_1}(X))$$
                $$\chi \mapsto \Autom(\chi)$$
            What this essentially tells us is that $1$-dimensional continuous $\ell$-adic Galois representations are the same as automorphic forms associated to $\GL_1$, and as such, the combination of theorem \ref{theorem: galois_representations_are_local_systems} and theorem \ref{theorem: unramified_abelian_geometric_class_field_theory} can be understood as a geometrisation of the Global Langlands Correspondence for the (connnected reductive) group $\GL_1$\footnote{Incidentally, this is why it is commonly asserted that the Langlands Correspondence for $\GL_1$ \say{is just class field theory}.}. 
        \end{corollary}
        
    \subsection{Artin Reciprocity via Grothendieck's Sheaf-Function Correspondence}
        As a final step, let us decategorify the left-hand side of the monoidal equivalence $\Rep^{\cont, 1}_{\bar{\Q}_{\ell}}(\pi_1^{\ab}(X_{\fet})) \cong \Eig\Shv_{\underline{\bar{\Q}_{\ell}}}^1(\Bun_{\GL_1}(X))$ from corollary \ref{coro: geometric_langlands_for_GL1} to obtain a correspondence between continuous $\ell$-adic characters of $\pi_1^{\ab}(X_{\fet})$ and so-called Hecke characters\footnote{Auf Deutsch: \say{\textit{Die Gr\"o{\ss}encharaktere}}.}, using Grothendieck's Sheaf-Function Correspondence. Doing so will yield us the usual version of Artin Reciprocity for global function fields over perfect fields of positive characteristics, in terms of towers of finite abelian extensions and so-called Hecke characters (see theorems \ref{theorem: hecke_characters_from_hecke_eigensheaves} and \ref{theorem: artin_reciprocity_for_global_function_fields}).
        
        \subsubsection{Traces of Frobenii and Grothendieck's Sheaf-Function Correspondence via character sheaves}
            We begin by introducing Grothendieck's Sheaf-Function Corresondence\footnote{En Français: \say{\textit{La Correspondance Faisceaux-Fonctions de Grothendieck}}.}, which allows us to formally realise the notion of Hecke eigensheaves from definition \ref{def: hecke_eigensheaves} as a categorification of the classical notion of Hecke characters (cf. theorem \ref{theorem: hecke_characters_from_hecke_eigensheaves}). For this, let us first discuss traces of Frobenius endomorphisms on $\ell$-adic sheaves, which unfortunately only makes sense over positive characteristics.
            \begin{convention} \label{conv: frobenii}
                From now on, the base field $k$ from convention \ref{conv: automorphic_side_conventions} shall be some finite field $\F_q$ (which we note to be perfect). This gives us access to the (absolute) Frobenius on $X_S$, with $S$ being some perfect scheme over $\Spec \F_q$, which we denote by $\Frob_{X_S}$.
            \end{convention}
            
            \begin{definition}[Traces of Frobenii] \label{def: traces_of_frobenii}
                The Frobenius on any Noetherian $\F_q$-scheme induces a functor $\Frob_Z^*: \Shv_{\underline{\bar{\Q}_{\ell}}}^{\ad, 1}(Z) \to \Shv_{\underline{\bar{\Q}_{\ell}}}^{\ad, 1}(Z)$, and at the level of stalks at geometric points $\bar{z} \in Z(\bar{\F}_q)$, said functor gives rise to endomorphisms $\Frob_{\bar{z}}: \calF_{\bar{z}} \to \calF_{\bar{z}}$. The trace of any such endomorphism is called the \textbf{trace of Frobenius} on $\calF$ at $\bar{z}$.
            \end{definition}
            \begin{remark}[Basic properties of traces of Frobenii] \label{remark: basic_properties_of_traces_of_frobenii}
                The following properties are trivial consequences of definition \ref{def: traces_of_frobenii}:
                    \begin{itemize}
                        \item The first thing that one should note is that in taking traces of Frobenii with respect to a fixed $\ell$-adic local system $\calF \in \Shv_{\underline{\bar{\Q}_{\ell}}}^{\ad, 1}(Z)$, one obtains a function $\frob^{\calF}: Z(\bar{\F}_q) \to \bar{\Q}_{\ell}$.
                        \item For any pair of rank-$1$ $\ell$-adic local systems $\calF, \calF' \in \Shv_{\underline{\bar{\Q}_{\ell}}}^{\ad, 1}(Z)$ and any fixed geometric point $\bar{z} \in Z(\bar{\F}_q)$, one has $\frob^{\calF \tensor \calF'} = \frob^{\calF} \frob^{\calF'}$.
                        \item Let $\theta: Y \to Z$ be a morphism between $\F_q$-schemes that are locally of finite type and let $\calF$ be an $\ell$-adic local system of rank $1$ on $Z$. Then $f^{\theta^*\calF} = f^{\calF} \circ \theta_{\bar{\F}_q}$\footnote{To prove this, simply recall also the basic sheaf-theoretic fact that for any geometric point $\bar{y} \in Y(\bar{\F}_q)$ over a fixed geometric point $\bar{z} \in Z(\bar{F}_q)$ (i.e. such that $\theta_{\bar{\F}_q}(\bar{y}) = \bar{z}$), one has an isomorphism $(\theta^*\calF)_{\theta_{\bar{\F}_q}(\bar{y})} \cong \calF_{\bar{z}}$ of stalks.}.
                    \end{itemize}
            \end{remark}
            
            In order to obtain continuous $\ell$-adic characters via taking traces of Frobenii on $\ell$-adic local systems $\calF$, it is crucial that we demonstrate how for a suitable group scheme $G$, the function $\frob^{\calF}$ as in definition \ref{def: traces_of_frobenii} gives rise to a continuous group homomorphism $\xi^{\calF}: G(\F_q) \to \bar{\Q}_{\ell}^{\x}$. For this, it will be convenient to have the notion of \textbf{character sheaves} at our disposal, for which a good reference is \cite{cunningham_roe_function_sheaf_dictionary_quasi_characters_p_adic_tori} (whose presentation we shall also follow closely). 
            
            We begin this discussion by introducing the so-called \say{Weil sheaf condition}, which among other things, allow us to descend functions obtained via taking traces of Frobenii from sets of geometric points to sets of rational points (see remark \ref{remark: why_are_character_sheaves_weil_sheaves}).
            \begin{definition}[Weil sheaves] \label{def: weil_sheaves}
                Let $k$ be a perfect field of characteristic $p > 0$. The category of $\ell$-adic \textbf{Weil sheaves} of rank $1$ on a Noetherian $k$-schemes is the full subcategory of $\Shv_{\underline{\bar{\Q}_{\ell}}}^{\ad, 1}(Z)$ on which the pullback functor $\Frob_{Z}^*: \Shv_{\underline{\bar{\Q}_{\ell}}}^{\ad, 1}(Z) \to \Shv_{\underline{\bar{\Q}_{\ell}}}^{\ad, 1}(Z)$ is an equivalence\footnote{In other words, we can think of Weil sheaves as Frobenius-fixed $\ell$-adic local systems.}. We suggestively denote this category by $\Shv_{\underline{\bar{\Q}_{\ell}}}^{\ad, 1}(Z)^{\Frob}$.
            \end{definition}
            \begin{remark}[Basic properties of Weil sheaves] \label{remark: properties_of_weil_sheaves}
                Let $k$ be a perfect field of characteristic $p > 0$ and let $Z$ be a Noetherian $k$-scheme. 
                \begin{itemize}
                    \item First of all, one sees - via the fact that $\Frob_Z^*$ commutes with tensor products of $\ell$-adic sheaves on $Z$ - that the category $\Shv_{\underline{\bar{\Q}_{\ell}}}^{\ad, 1}(Z)^{\Frob}$ of $\ell$-adic Weil sheaves over $Z$ is symmetric monoidal (cf. \cite[Definition 8.1.12]{EGNO}) with respect to the usual tensor product of $\ell$-adic sheaves.
                    \item \cite[Proposition 5.20]{tendler_2015_geometric_class_field_theory} Let $\theta: Y \to Z$ be a morphism between $\F_q$-schemes that are locally of finite type and let $\calF$ be a Weil sheaf on $Z$. Then $\theta^*\calF$ shall, in turn, be a Weil sheaf on $Y$. Since $\Shv_{\underline{\bar{\Q}_{\ell}}}^{\ad, 1}(Z)^{\Frob}$ is a full subcategory of $\Shv_{\underline{\bar{\Q}_{\ell}}}^{\ad, 1}(Z)$ - which is compatible with $*$-pullbacks - one then sees that $\Shv_{\underline{\bar{\Q}_{\ell}}}^{\ad, 1}(Z)^{\Frob}$ is also compatible with $*$-pullbacks.
                \end{itemize}
            \end{remark}
            
            \begin{definition}[Character sheaves] \label{def: character_sheaves}
                Let $k$ be a perfect field of some prime characteristic $p$ and $G$ be a Noetherian\footnote{In \cite{cunningham_roe_function_sheaf_dictionary_quasi_characters_p_adic_tori}, the more specialised case of smooth group schemes over $\Spec \F_q$ is considered.} commutative group scheme over $\Spec k$, whose group structure is given by $\mu_G: G \x G \to G$. In such a situation, an $\ell$-adic \textbf{character sheaf} of $G$ shall be an Weil sheaf $\E$ of rank $1$ over $G$, such that $\mu_G^*\E \cong \E \boxtimes \E$.
            \end{definition}
            \begin{remark}[Rigid symmetric monoidal categories of character sheaves] \label{remark: rigid_monoidal_categories_of_character_sheaves}
                \cite[Subsection 1.2]{cunningham_roe_function_sheaf_dictionary_quasi_characters_p_adic_tori} It can be easily checked, through verifying the relevant axioms (see \cite[Definition 2.10.11]{EGNO}), that $\ell$-adic character sheaves on a given Noetherian commutative group scheme $G$ over $\Spec k$ (for some field $k$) form a rigid symmetric monoidal subcategory of $\Shv_{\underline{\bar{\Q}_{\ell}}}^{\ad, 1}(G)$, which we denote by $\Char\Shv_{\underline{\bar{\Q}_{\ell}}}(G)$. Interestingly, if $G$ is a \textit{smooth} commutative group scheme over $\Spec \F_q$ then $\Char\Shv_{\underline{\bar{\Q}_{\ell}}}(G)$ will actually be a groupoid, a phenomenon that is in perfect analogy with Schur's Lemma (cf. \cite[Lemma 3.6]{lam_first_course_in_noncommutative_rings}) and coincidentally, can be shown via a straightforward application of Schur's Lemma to stalks of character sheaves (cf. \cite[Lemma 1.3]{cunningham_roe_function_sheaf_dictionary_quasi_characters_p_adic_tori}); as a consequence, the isomorphism classes of character sheaves of a smooth commutative group scheme over $\Spec \F_q$ form an abelian group. 
            \end{remark}
            \begin{remark}[The necessity of the Weil sheaf condition] \label{remark: why_are_character_sheaves_weil_sheaves}
                Let us first remark that it is a matter of necessity, not convenience, that we define functions associated to $\ell$-adic local systems $\calF \in \Shv_{\underline{\bar{\Q}_{\ell}}}^{\ad, 1}(Z)$ over the set $Z(\bar{\F}_q)$ instead of $Z(\F_q)$, and this is because stalks of \'etale sheaves can only be computed over geometric points (cf. \cite[\href{https://stacks.math.columbia.edu/tag/03PN}{Tag 03PN}]{stacks}). But we do want functions over $Z(\F_q)$ as well, and for this, recall that $Z(\F_q) \cong Z(\bar{\F}_q)^{\Gal(\bar{\F}_q/\F_q)}$; from this, one infers that because Weil sheaves correspond to $\bar{\Q}_{\ell}$-valued functions on $Z(\bar{\F}_q)^{\rmW_{\F_q}} \subset Z(\F_q)$ (with $\rmW_{\F_q}$ being the subgroup of $\Gal(\bar{\F}_q/\F_q)$ generated by $\Frob_{\F_q}$), Weil sheaves in particular correspond to functions on $Z(\F_q)$. Because we eventually will need to consider $\F_q$-points instead of $\bar{\F}_q$-points\footnote{The keyword here is Lang's Theorem, which holds only for $\F_q$-points of suitable algebraic groups.}, this is one reason to consider character sheaves that are Frobenius-equivariant.
            \end{remark}
            
            It turns out that Hecke eigensheaves on $\Bun_{\GL_1}(X)$, which is rather fortunate for us, as this directly implies that Hecke eigensheaves correspond to certain continuous $\ell$-adic characters of $\Bun_{\GL_1}(X)(\F_q)$ via lemma \ref{lemma: sheaf_function_correspondence_for_connected_algebraic_groups}. Nevertheless, there remains a technical difficulty that we need to overcome, that being the fact that $\Bun_{\GL_1}(X)$ is not connected (cf. remark \ref{remark: geometry_of_the_picard_stack}).
            \begin{remark}[The connected-smooth-\'etale short exact sequence] \label{remark: the_connected_smooth_etale_short_exact_sequence}
                For theorem \ref{theorem: sheaf_function_correspondence_for_smooth_groups}, recall that for any scheme $Y$ that is locally of finite type over a field $k$, there exists a maximal $k$-subalgebra $O(Y)$ of $\Gamma(Y, \calO_Y)$ which is \'etale over $k$ (cf. \cite[Proposition 5.44]{milne_algebraic_groups}). We can then define the scheme $\pi_0(Y)$ of \say{connected components} of $Y$ to be $\Spm O(Y)$ and in doing so, we will have obtained a morphism $Y \to \pi_0(Y)$ that is universal among all \'etale $\F_q$-schemes via the fundamental adjunction:
                    $$
                        \begin{tikzcd}
                        	{({}^{k/}\Comm\Alg^{\ft, \red})^{\op}} & {\Sch_{/\Spec k}}
                        	\arrow[""{name=0, anchor=center, inner sep=0}, "{\Spm }"', bend right, from=1-1, to=1-2]
                        	\arrow[""{name=1, anchor=center, inner sep=0}, "\Gamma"', bend right, from=1-2, to=1-1]
                        	\arrow["\dashv"{anchor=center, rotate=-90}, draw=none, from=1, to=0]
                        \end{tikzcd}
                    $$
                Let $G$ be a smooth commutative group scheme over $\Spec k$, let $G^0$ be the connected component of the identity thereof (known to be a connected commutative algebraic group over $\Spec k$; cf. \cite[Lemma 3.2]{cunningham_roe_function_sheaf_dictionary_quasi_characters_p_adic_tori}). It is known that the universal morphism $G \to \pi_0(G)$ induces the following short exact sequence of commutative algebraic groups (cf. \cite[Proposition 5.48]{milne_algebraic_groups}), known as the \textbf{connected-smooth-\'etale short exact sequence}:
                    $$
                        \begin{tikzcd}
                        	0 & {G^0} & G & {\pi_0(G)} & 0
                        	\arrow[from=1-1, to=1-2]
                        	\arrow["{\iota_{G^0}}", from=1-2, to=1-3]
                        	\arrow["{\pi_{G^0}}", from=1-3, to=1-4]
                        	\arrow[from=1-4, to=1-5]
                        \end{tikzcd}
                    $$
                This short exact sequence can be shown (see \cite[Proposition 3.3, Lemma 3.4, and Proposition 3.5]{cunningham_roe_function_sheaf_dictionary_quasi_characters_p_adic_tori}) to induce another short exact sequence of abelian groups as follows:
                    $$
                        \begin{tikzcd}
                        	0 & {\Char\Shv_{\underline{\bar{\Q}_{\ell}}}(\pi_0(G))} & {\Char\Shv_{\underline{\bar{\Q}_{\ell}}}(G)} & {\Char\Shv_{\underline{\bar{\Q}_{\ell}}}(G^0)} & 0
                        	\arrow[from=1-1, to=1-2]
                        	\arrow["{\pi_{G^0}^*}", from=1-2, to=1-3]
                        	\arrow["{\iota_{G^0}^*}", from=1-3, to=1-4]
                        	\arrow[from=1-4, to=1-5]
                        \end{tikzcd}
                    $$
                One thus infers that any investigation of character sheaves on smooth commutative group schemes can be broken down into analyses of character sheaves on connected commutative group schemes of finite type and of those on \'etale commutative group schemes (see lemmas \ref{lemma: sheaf_function_correspondence_for_connected_algebraic_groups} and \ref{lemma: sheaf_function_correspondence_for_etale_commutative_group_scheme} respectively).
            \end{remark}
            \begin{remark}[The category of compact Hausdorff abelian groups] \label{remark: the_category_of_compact_hausdorff_abelian_groups}
                A fine technicality that one shall need to keep in mind for theorem \ref{theorem: sheaf_function_correspondence_for_smooth_groups} is that the category $\Comp\Ab$ of (small) compact Hausdorff abelian groups is equivalent to the opposite category $\Ab^{\op}$ of abelian groups (cf. \cite[Proposition IV.5.2]{maclane}) and therefore is abelian, as the opposite of any abelian category is also abelian category (this is a straightforward consequence of the definition of abelian categories; cf. \cite[Section VIII.3]{maclane})\footnote{For all $n \geq 2$, the groups $\GL_n(\bar{\Q}_{\ell})$ are - in addition to being non-abelian - actually only strictly locally compact instead of being compact. This makes higher-dimensional representations of \'etale fundamental groups much harder to analyse.}.
            \end{remark}
            \begin{lemma}[Grothendieck's Sheaf-Function Correspondence for connected algebraic groups] \label{lemma: sheaf_function_correspondence_for_connected_algebraic_groups}
                \cite[Proposition 1.14]{cunningham_roe_function_sheaf_dictionary_quasi_characters_p_adic_tori} If $H$ is a connected commutative algebraic group over $\Spec \F_q$ then taking traces of Frobenii (in the sense of definition \ref{def: traces_of_frobenii}) will yield us a group isomorphism:
                    $$\trace(\Frob_H^* \mid -): \Char\Shv_{\underline{\bar{\Q}_{\ell}}}(H) \to \Rep_{\bar{\Q}_{\ell}}^1(H(\F_q))$$
            \end{lemma}
            \begin{lemma}[Grothendieck's Sheaf-Function Correspondence for \'etale commutative group schemes] \label{lemma: sheaf_function_correspondence_for_etale_commutative_group_schemes}
                \cite[Proposition 2.7]{cunningham_roe_function_sheaf_dictionary_quasi_characters_p_adic_tori} 
            \end{lemma}
                \begin{proof}
                    
                \end{proof}
            \begin{theorem}[Grothendieck's Sheaf-Function Correspondence for smooth groups] \label{theorem: sheaf_function_correspondence_for_smooth_groups}
                \cite[Theorem 3.6]{cunningham_roe_function_sheaf_dictionary_quasi_characters_p_adic_tori} If $G$ is a smooth commutative group scheme over $\Spec \F_q$ then $\trace(\Frob_G^* \mid -)$ as in lemma \ref{lemma: sheaf_function_correspondence_for_connected_algebraic_groups} will be a surjective group homomorphism with kernel $H^2(\pi_0(G_{\bar{\F}_q}), \bar{\Q}_{\ell}^{\x})^{\rmW_{\F_q}}$, with $\rmW_{\F_q}$ denoting the subgroup of $\Gal(\bar{\F}_q/\F_q)$ that is generated by $\Frob_{\F_q}$.
            \end{theorem}
                \begin{proof}
                    Because the compact abelian group $\bar{\Q}_{\ell}^{\x}$ is divisible (which means that for any $n \in \Z$, the multiplication-by-$n$ endomorphism $n \cdot: \bar{\Q}_{\ell}^{\x} \to \bar{\Q}_{\ell}^{\x}$ is surjective) by virtue of being the group of units of a field, the functor $\Comp\Ab(-, \bar{\Q}_{\ell}^{\x}): \Comp\Ab^{\op} \to \Ab$\footnote{Which is tautologically equal to $\Rep^{\cont, 1}_{\bar{\Q}_{\ell}}(-)$.} is exact \textit{a priori}\footnote{One can show this by applying Baer's Cirterion for Injectivity: this tells us that all divisible compact Hausdorff abelian groups $H$ are injective objects of $\Comp\Ab$, and the functors $\Comp\Ab(-, H)$ are therefore exact by the definition of injective objects.}. Through taking traces of Frobenii, we then obtain the following commutative diagram in $\Ab$, wherein the rows are short exact sequences:
                        $$
                            \begin{tikzcd}
                            	0 & {\Char\Shv_{\underline{\bar{\Q}_{\ell}}}(\pi_0(G))} & {\Char\Shv_{\underline{\bar{\Q}_{\ell}}}(G)} & {\Char\Shv_{\underline{\bar{\Q}_{\ell}}}(G^0)} & 0 \\
                            	0 & {\Rep^{\cont, 1}_{\bar{\Q}_{\ell}}(\pi_0(G)(\F_q))} & {\Rep^{\cont, 1}_{\bar{\Q}_{\ell}}(G(\F_q))} & {\Rep^{\cont, 1}_{\bar{\Q}_{\ell}}(G^0(\F_q))} & 0
                            	\arrow[from=1-1, to=1-2]
                            	\arrow["{\pi_{G^0}^*}", from=1-2, to=1-3]
                            	\arrow["{\iota_{G^0}^*}", from=1-3, to=1-4]
                            	\arrow[from=1-4, to=1-5]
                            	\arrow[from=2-1, to=2-2]
                            	\arrow[from=2-2, to=2-3]
                            	\arrow[from=2-3, to=2-4]
                            	\arrow[from=2-4, to=2-5]
                            	\arrow["{\trace(\Frob_{\pi_0(G)}^* \mid -)}", from=1-2, to=2-2]
                            	\arrow["{\trace(\Frob_G^* \mid -)}", from=1-3, to=2-3]
                            	\arrow["{\trace(\Frob_{G^0}^* \mid -)}", from=1-4, to=2-4]
                            \end{tikzcd}
                        $$
                    Now, because $G^0$ is a connected commutative algebraic group, we get via lemma \ref{lemma: sheaf_function_correspondence_for_connected_algebraic_groups} to see that $\trace(\Frob_{G^0}^* \mid -)$ must be an isomorphism; at the same time, $\trace(\Frob_{\pi_0(G)^* \mid -})$ is surjective due to $\pi_0(G)$ being \'etale as a commutative algebraic group over $\Spec \F_q$ (cf. \cite[Proposition 2.6]{cunningham_roe_function_sheaf_dictionary_quasi_characters_p_adic_tori}). An application of the Snake Lemma then yields us the following red long exact sequence in $\Ab$, from which one clearly sees that the cokernel of $\trace(\Frob_G^* \mid -)$ is trivial and the map is therefore surjective as claimed:
                        $$
                            \begin{tikzcd}
                            	{\textcolor{red}{0}} & {\textcolor{red}{\ker \trace(\Frob_{\pi_0(G)}^* \mid -)}} & {\textcolor{red}{\ker \trace(\Frob_G^* \mid -)}} & {\textcolor{red}{0}} \\
                            	0 & {\Char\Shv_{\underline{\bar{\Q}_{\ell}}}(\pi_0(G))} & {\Char\Shv_{\underline{\bar{\Q}_{\ell}}}(G)} & {\Char\Shv_{\underline{\bar{\Q}_{\ell}}}(G_0)} & 0 \\
                            	0 & {\Rep^1_{\bar{\Q}_{\ell}}(\pi_0(G)(\F_q))} & {\Rep^1_{\bar{\Q}_{\ell}}(G(\F_q))} & {\Rep^1_{\bar{\Q}_{\ell}}(G_0(\F_q))} & 0 \\
                            	& {\textcolor{red}{0}} & {\textcolor{red}{\coker \trace(\Frob_G^* \mid -)}} & {\textcolor{red}{0}} & {\textcolor{red}{0}}
                            	\arrow[from=2-1, to=2-2]
                            	\arrow["{\pi_{G_0}^*}", from=2-2, to=2-3]
                            	\arrow["{\iota_{G_0}^*}", from=2-3, to=2-4]
                            	\arrow[from=2-4, to=2-5]
                            	\arrow[from=3-1, to=3-2]
                            	\arrow[from=3-2, to=3-3]
                            	\arrow[from=3-3, to=3-4]
                            	\arrow[from=3-4, to=3-5]
                            	\arrow["{\trace(\Frob_{\pi_0(G)}^* \mid -)}", two heads, from=2-2, to=3-2]
                            	\arrow["{\trace(\Frob_G^* \mid -)}", from=2-3, to=3-3]
                            	\arrow["{\trace(\Frob_{G_0}^* \mid -)}", tail, two heads, from=2-4, to=3-4]
                            	\arrow[from=1-2, to=2-2]
                            	\arrow[from=1-3, to=2-3]
                            	\arrow[color={red}, from=1-1, to=1-2]
                            	\arrow[color={red}, from=1-2, to=1-3]
                            	\arrow[color={red}, from=1-3, to=1-4]
                            	\arrow[from=3-3, to=4-3]
                            	\arrow[from=3-2, to=4-2]
                            	\arrow[from=3-4, to=4-4]
                            	\arrow["\delta"{description}, color={red}, dashed, from=1-4, to=4-2]
                            	\arrow[color={red}, from=4-2, to=4-3]
                            	\arrow[color={red}, from=4-3, to=4-4]
                            	\arrow[color={red}, from=4-4, to=4-5]
                            \end{tikzcd}
                        $$
                \end{proof}
        
        \subsubsection{Artin Reciprocity for global function fields over finite fields}
            \begin{convention}[Global function field of $X$] \label{conv: global_function_field}
                Let us write $K_X$ for the global function field of $X$ (i.e the stalk of the structure sheaf of $X$ at the unique generic point), $\bbO_X$ for its ring of integers, and $\A_X$ for the corresponding ring of ad\`eles. For more details on these constructions, we refer the reader to \cite[Section VI.1]{neukirch_2010_algebraic_number_theory}. 
            \end{convention}
            
            \begin{lemma}[Hecke eigensheaves are Weil sheaves] \label{lemma: hecke_eigensheaves_are_weil_sheaves}
                Any Hecke eigensheaf on $\Bun_{\GL_1}(X)$ is a Weil sheaf.
            \end{lemma}
                \begin{proof}
                    Fix an arbitrary Hecke eigensheaf $\E_{\calL}$ with eigenvalue $\calL$ and consider the following commutative diagram:
                        $$
                            \begin{tikzcd}
                            	{X \x \Bun_{\GL_1}(X)} & {\Bun_{\GL_1}(X)} \\
                            	{X \x \Bun_{\GL_1}(X)} & {\Bun_{\GL_1}(X)}
                            	\arrow["{\id_X \x \Frob_{\Bun_{\GL_1}(X)}}"', from=1-1, to=2-1]
                            	\arrow["{\cev{h}_X}", from=2-1, to=2-2]
                            	\arrow["{\cev{h}_X}", from=1-1, to=1-2]
                            	\arrow["{\Frob_{\Bun_{\GL_1}(X)}}", from=1-2, to=2-2]
                            \end{tikzcd}
                        $$
                    from which one infers that $\cev{h}_X^* \Frob_{\Bun_{\GL_1}(X)}^* \E_{\calL} \cong (\id_X \x \Frob_{\Bun_{\GL_1}(X)})^* \cev{h}_X^* \E_{\calL} \cong \calL \boxtimes \Frob_{\Bun_{\GL_1}(X)}^* \E_{\calL}$, which in turn tells us that $\Frob^*_{\Bun_{\GL_1}(X)} \E_{\calL}$ is a Hecke eigensheaf with eigenvalue $\calL$. This can be interpreted as $\cev{h}_X^*$ acting on $\E_{\calL}$ and $\Frob^*_{\Bun_{\GL_1}(X)} \E_{\calL}$ as $\calL \boxtimes -$, so we shall now attempt to show that $\calL \boxtimes -$ is a faithful functor. For this, recall that faithful flatness is a local property and so we can simply check whether $\calL_{\bar{x}} \tensor -$ is a faithful functor for all geometric points $\bar{x} \in X$. However, $\calL_{\bar{x}}$ is a finite-dimensional vector space, so it is faithfully flat \textit{a priori}, so the functor $\calL_{\bar{x}} \tensor -$ is necessarily faithful. As stated, this implies that $\E_{\calL} \cong \Frob_{\Bun_{\GL_1}(X)}^* \E_{\calL}$, i.e. that $\E_{\calL}$ is a Weil sheaf on $\Bun_{\GL_1}(X)$.
                \end{proof}
            \begin{lemma}[Hecke eigensheaves are character sheaves] \label{lemma: hecke_eigensheaves_are_character_sheaves}
                There exists a monoidal equivalence:
                    $$\Eig\Shv_{\underline{\bar{\Q}_{\ell}}}^1(\Bun_{\GL_1}(X)) \cong \Char\Shv_{\underline{\bar{\Q}_{\ell}}}(\Bun_{\GL_1}(X))$$
            \end{lemma}
                \begin{proof}
                    
                \end{proof}
            \begin{convention}[Weil Uniformisation] \label{conv: weil_uniformisation}
                For theorem \ref{theorem: hecke_characters_from_hecke_eigensheaves}, suppose that we are willing to take the special case of the Weil Uniformisation Theorem for $\Bun_{\GL_1}(X)$ (cf. \cite[Proposition 3.8]{tendler_2015_geometric_class_field_theory}\footnote{For details on Weil Uniformisation for $\Bun_G(X)$, with $G$ being a general connected reductive group, see \cite{sorger_BunG_uniformisation}}) for granted, which states that there is a group isomorphism $\Bun_{\GL_1}(X)(\F_q) \cong \GL_1(K_X)\backslash\GL_1(\A_X)/\GL_1(\bbO_{X})$.
            \end{convention}
            \begin{definition}[Hecke characters] \label{def: hecke_characters}
                An $\ell$-adic \textbf{Hecke character} of $X$ (or rather, of $K_X$) is a continuous $\ell$-adic character of $\GL_1(K_X)\backslash\GL_1(\A_X)/\GL_1(\bbO_{X})$. They form a group, denoted by $\scrA_{\GL_1}(X)$.
            \end{definition}
            \begin{theorem}[Hecke characters from Hecke eigensheaves] \label{theorem: hecke_characters_from_hecke_eigensheaves}
                There is a group isomorphism via traces of Frobenii as follows, through which Hecke characters are obtained from Hecke eigensheaves of rank $1$ on $\Bun_{\GL_1}(X)$:
                    $$\trace(\Frob_{\Bun_{\GL_1}(X)}^* \mid -): \Eig\Shv_{\underline{\bar{\Q}_{\ell}}}^1(\Bun_{\GL_1}(X)) \to \scrA_{\GL_1}(X)$$
            \end{theorem}
                \begin{proof}
                    
                \end{proof}
            
            \begin{theorem}[Artin Reciprocity for global function fields] \label{theorem: artin_reciprocity_for_global_function_fields}
                There exists a group isomorphism:
                    $$\Rep^{\cont, 1}_{\bar{\Q}_{\ell}}(\Gal(K^{\ab}/K)) \cong \scrA_{\GL_1}(X)$$
                sending continuous $\ell$-adic Galois characters $\chi$ to Hecke characters $\xi$, such that for all places $v$ of $K$, one has $\chi(\Frob_v) = \xi(\alpha_v)$.
            \end{theorem}
                \begin{proof}
                    
                \end{proof}