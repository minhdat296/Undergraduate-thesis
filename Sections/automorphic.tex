\section{The Automorphic Side}
    \subsection{The Hecke action and spectral decomposition}
        \begin{convention}[The setting of the main theorem] \label{conv: automorphic_side_conventions}
            For us, $\Bun_{\GL_1}(X)$ shall denote the moduli space of line bundles on $X$. Traditionally, this is usually referred to as the \textbf{Picard stack} and denoted by $\Pic_X$ (cf. \cite[\href{https://stacks.math.columbia.edu/tag/0372}{Tag 0372}]{stacks}), but we opt for the notation $\Bun_{\GL_1}(X)$ because in the wider context of the Geometric Langlands Programme, one works with $\Bun_G(X)$ for $G$ a general connected reductive group (of which $\GL_1$ is a special case). 
            
            In addition, let us now suppose that the base field $k$ from convention \ref{conv: base_curve} is algebraically closed (hence separably closed).
        \end{convention}
        \begin{remark}[Some remarks on the geometry of $\Bun_{\GL_1}$] \label{remark: geometry_of_the_picard_stack}
            Since we are working with a smooth projective curve $X$ over an algebraically closed field (cf. convention \ref{conv: automorphic_side_conventions}) and hence over a separably closed field, the prestack $\Bun_{\GL_1}$ is \textit{a priori} representable by a scheme (cf. \cite[\href{https://stacks.math.columbia.edu/tag/0B9Z}{Tag 0B9Z}]{stacks}) as a sheaf on $(\Sch_{/X})_{\fppf}$ (and by representablity, also as a sheaf on $(\Sch_{/X})_{\et}$ and $(\Sch_{/X})_{\Zar}$). As a result, when considering sheaves on $\Bun_{\GL_1}(X)$, we will only need to know about sheaves on schemes instead of the entire fully general theory of sheaves on prestacks. Furthermore, should $X$ be of \href{https://stacks.math.columbia.edu/tag/0BY6}{\underline{genus}} $g \geq 0$, then one has the following decomposision:
                $$\Bun_{\GL_1}(X) \cong \coprod_{d \in \Z} \Bun_{\GL_1}^d(X)$$
            wherein each $\Bun_{\GL_1}^d(X)$ is the moduli scheme of line bundles of \href{https://stacks.math.columbia.edu/tag/0AYQ}{\underline{degree}} $d$ on $X$, which is a smooth and proper variety (cf. \cite[\href{https://stacks.math.columbia.edu/tag/0BA0}{Tag 0BA0}]{stacks}).
        \end{remark}
        
        \begin{definition}[Hecke correspondences] \label{def: hecke_correspondences}
            \noindent
            \begin{enumerate}
                \item \textbf{(The global Hecke correspondence):} The \textbf{global Hecke correspondence} is a span, i.e. a diagram of the form:
                    $$
                        \begin{tikzcd}
                        	& {\Hecke_{\GL_1}(X)} \\
                        	{\Bun_{\GL_1}(X)} && {X \x \Bun_{\GL_1}(X)}
                        	\arrow["{\cev{h}_X}"', from=1-2, to=2-1]
                        	\arrow["{\supp_X \x \vec{h}_X}", from=1-2, to=2-3]
                        \end{tikzcd}
                    $$
                wherein $\Hecke_{\GL_1}(X)$ is the moduli space of quadruples:
                    $$(\E_1, \E_2, x, \beta_x)$$
                consisting of line bundles $\E_1, \E_2 \in \Bun_{\GL_1}(X)$, points $x \in X$, and for each such point $x$, a monomorphism $\beta_x: \E_1 \hookrightarrow \E_2$ whose cokernel is the skyscraper sheaf $k_x$ supported at $x \in X$ with value $k$. It is naturally equipped with two projection functors $\cev{h}_X$ and $\supp_X \x \vec{h}_X$, which are defined via:
                    $$\cev{h}_X(\E_1, \E_2, x, \beta_x) \cong \E_1$$
                    $$(\supp_X \x \vec{h}_X)(\E_1, \E_2, x, \beta_x) \cong (x, \E_2)$$
                and hence one obtains the global Hecke correspondence as a span.
                \item \textbf{(Local Hecke correspondences):} Since $(\supp_X \x \vec{h}_X)(\E_1, \E_2, x, \beta_x) \cong (x, \E_2)$, one obtains (via taking fibres) a canonical \textbf{local Hecke correspondence} at each point $x \in X$ as follows:
                    $$
                        \begin{tikzcd}
                        	& {\Hecke_{\GL_1}(x)} \\
                        	{\Bun_{\GL_1}(X)} && {\Bun_{\GL_1}(X)}
                        	\arrow["{\cev{h}_x}"', from=1-2, to=2-1]
                        	\arrow["{\vec{h}_x}", from=1-2, to=2-3]
                        \end{tikzcd}
                    $$
                wherein $\Hecke_{\GL_1}(x)$ is the moduli space of triples $(\E_1, \E_2, \beta_x)$ consisting of line bundles $\E_1, \E_2 \in \Bun_{\GL_1}(X)$ along with a monomorphism $\beta_x: \E_1 \hookrightarrow \E_2$ whose cokernel is the skyscraper sheaf $k_x$ supported at $x \in X$ with value $k$, and $\cev{h}_x$ and $\vec{h}_x$ are defined in the obvious manner:
                    $$\cev{h}_x(\E_1, \E_2, \beta_x) \cong \E_1$$
                    $$\vec{h}_x(\E_1, \E_2, \beta_x) \cong (x, \E_2)$$
            \end{enumerate}
        \end{definition}
        
        \begin{definition}[Hecke operators] \label{def: hecke_operators}
            \noindent
            \begin{enumerate}
                \item \textbf{(The global Hecke operators):} The global Hecke correspondence induces the following sheaf pull-push diagram:
                    $$
                        \begin{tikzcd}
                        	& {\Shv_{\overline{\Q_{\ell}}}^1(\Hecke_{\GL_1}(X))} \\
                        	{\Shv_{\overline{\Q_{\ell}}}^1(\Bun_{\GL_1}(X))} && {\Shv_{\overline{\Q_{\ell}}}^1(X \x \Bun_{\GL_1}(X))}
                        	\arrow["{(\cev{h}_X)^*}", from=2-1, to=1-2]
                        	\arrow["{(\supp_X \x \vec{h}_X)_*}", from=1-2, to=2-3]
                        \end{tikzcd}
                    $$
                and by composing the two functors in the obvious manner, one gets a new functor:
                    $$\scrH_X := (\vec{h}_X)_* (\supp_X \x \cev{h}_X)^*$$
                which we shall call the \textbf{global Hecke operator}. 
                \item \textbf{(Local Hecke operators):} Similarly, we define the \textbf{local Hecke operator} at each point $x \in X$ 
                    $$
                        \begin{tikzcd}
                        	& {\Shv_{\overline{\Q_{\ell}}}^1(\Hecke_{\GL_1}(x))} \\
                        	{\Shv_{\overline{\Q_{\ell}}}^1(\Bun_{\GL_1}(X))} && {\Shv_{\overline{\Q_{\ell}}}^1(\Bun_{\GL_1}(X))}
                        	\arrow["{(\cev{h}_x)^*}", from=2-1, to=1-2]
                        	\arrow["{(\vec{h}_x)_*}", from=1-2, to=2-3]
                        \end{tikzcd}
                    $$
                to be the following composition:
                    $$\scrH_x := (\vec{h}_x)_* (\cev{h}_x)^*$$
                with the functors $\cev{h}_x, \vec{h}_x$ as in definition \ref{def: hecke_correspondences}.
            \end{enumerate}
        \end{definition}
    
        \begin{definition}[Hecke eigensheaves] \label{def: hecke_eigensheaves}
            A \textit{non-zero} $\ell$-adic sheaf $\E \in \Shv_{\overline{\Q_{\ell}}}^1(\Bun_{\GL_1}(X))$ is called a \textbf{Hecke eigensheaf} (of rank $1$) if and only if there exists an $\ell$-adic sheaf $\calL \in \Shv_{\overline{\Q_{\ell}}}^1(X)$ (called the \textbf{eigenvalue} of $\E$) such that:
                $$\scrH_X(\E) \cong \calL \boxtimes \E$$
        \end{definition}
        \begin{remark}
            It is easy to see that Hecke eigensheaves form a full symmetric monoidal subcategory of $\Shv_{\overline{\Q_{\ell}}}^1(\Bun_{\GL_1}(X))$, which we shall denote by $\Eig^1_{\overline{\Q_{\ell}}}(\Bun_{\GL_1}(X))$. Furthermore, each Hecke eigensheaf $\E$ with eigenvalue $\calL$ is an \say{eigenvector} of any of the local Hecke operators $\scrH_x$ in the following manner\footnote{Note that the category of $\ell$-adic local systesm of rank $1$ on each point $x \in X$ is nothing but $\Vect^1(\overline{\Q_{\ell}})$, the category of $1$-dimensional $\overline{\Q_{\ell}}$-vector spaces.}:
                $$\scrH_x(\E) \cong (\overline{\Q_{\ell}})_x \boxtimes \E$$
            thanks to the fact that the stalks $\calL_x$ are all isomorphic to $\overline{\Q_{\ell}}$, since $\calL$ is an $\ell$-adic local system of rank $1$.
        \end{remark}
    
    \subsection{The Abel-Jacobi map and Deligne-Artin Reciprocity for global function fields}
        \begin{theorem}[Unramified abelian geometric class field theory] \label{theorem: unramified_abelian_geometric_class_field_theory}
            There exists a canonical equivalence between the category of rank-$1$ $\ell$-adic local systems on $X$ and the category of ($\ell$-adic) Hecke eigensheaves of rank $1$ on $\Bun_{\GL_1(X)}$:
                $$\LocSys_{\overline{\Q_{\ell}}}^1(X) \cong \Eig^1_{\overline{\Q_{\ell}}}(\Bun_{\GL_1}(X))$$
            which maps each local system $\calL \in \LocSys_{\overline{\Q_{\ell}}}^1(X)$ to a Hecke eigensheaf $\Aut_{\calL} \in \Eig^1_{\overline{\Q_{\ell}}}(\Bun_{\GL_1}(X))$ with eigenvalue $\calL$.
        \end{theorem}
            \begin{proof}
                \noindent
                \begin{enumerate}
                    \item 
                    \item
                    \item 
                \end{enumerate}
            \end{proof}
        \begin{remark}[How should we interpret theorem \ref{theorem: unramified_abelian_geometric_class_field_theory} ?] \label{remark: unramified_abelian_geometric_class_field_theory_explanation}
            By putting theorem \ref{theorem: unramified_representations_are_sheaves_on_X} and theorem \ref{theorem: unramified_abelian_geometric_class_field_theory} together, one gets a canonical equivalence of categories as follows:
                $$\Rep_{\overline{\Q_{\ell}}}^1(\pi_1^{\ab}(X_{\fet}))^{\cont} \cong \Eig^1_{\overline{\Q_{\ell}}}(\Bun_{\GL_1}(X))$$
            Modulo technicalities, what this essentially tells us is that $1$-dimensional continuous $\ell$-adic Galois representations are the same as automorphic forms associated to $\GL_1$, and as such, the combination of theorem \ref{theorem: unramified_representations_are_sheaves_on_X} and theorem \ref{theorem: unramified_abelian_geometric_class_field_theory} can be understood as the Categorical Global Unramified Geometric Langlands Correspondence in its simplest non-trivial form, that being for the (connnected reductive group $G \cong \GL_1$)\footnote{Incidentally, this is why it is commonly asserted that the Langlands Correspondence for $\GL_1$ \say{is just class field theory}.}. 
        \end{remark}
        \begin{example}[Some examples of geometric reciprocity] \label{example: geometric_reciprocity}
            \noindent
            \begin{itemize}
                \item \textbf{($X \cong \P^1_k$):}
                \item \textbf{(Elliptic curves over char. $0$):}
                \item \textbf{(Counter-example: $X \cong \A^1_k, \chara k = 0$):}
            \end{itemize}
        \end{example}
        
    \subsection{Grothendieck's Sheaf-Function Dictionary and \textit{die Gr\"o{\ss}encharaktere}}
        As a final step, let us decategorify the left-hand side of the equivalence:
            $$\Rep_{\overline{\Q_{\ell}}}^1(\pi_1^{\ab}(X_{\fet}))^{\cont} \cong \Eig^1_{\overline{\Q_{\ell}}}(\Bun_{\GL_1}(X))$$
        from remark \ref{remark: unramified_abelian_geometric_class_field_theory_explanation} to obtain a correspondence between continuous $\ell$-adic characters of $\pi_1^{\ab}(X_{\fet})$ and \say{\textit{die Gr\"o{\ss}encharaktere}}, using Grothendieck's Sheaf-Function Dictionary.