\section{The Automorphic Side}
    \subsection{Symmetric powers of curves, Jacobians, and the Abel-Jacobi map}
        \begin{remark}[Quotient of schemes by finite group schemes] \label{remark: quotients_of_schemes_by_finite_group_schemes}
            For details on quotients of schemes by finite groups, we refer the reader to \cite[Expos\'e V]{SGA1}. For our purposes, we shall only need to keep in mind the following facts: 
                \begin{enumerate}
                    \item Let $S$ be a base scheme and $Y$ be an $S$-scheme that is either \textit{(quasi-)projective or (quasi-)affine}, and if additionally. If $G$ be a \textit{finite, flat, and locally of finite presentation} group $S$-scheme acting \textit{freely}\footnote{This is to ensure that the $G$-action induces an fppf equivalence relation on $Y$, since the $G$-action on $Y$ is free if and only if the corresponding homomorphism of sheaves of groups $G \to \Aut_{S_{\fppf}}(Y)$ is injective.} on $Y$, then the algebraic space $Y/G$ is a scheme (cf. \cite[\href{https://stacks.math.columbia.edu/tag/07S7}{Tag 07S7}]{stacks}). 
                    \item If $Y$ is a (quasi-)affine scheme $\Spec A$ then the quotient $Y/G$ will also be affine and will be isomorphic to $\Spec A^G$, thanks to the group-cohomological fact that $H^0(A, G) \cong A^G$. 
                    
                    If $Y$ is (quasi-)projective then $Y/G$ will also be (quasi-)projective.
                \end{enumerate}
        \end{remark}
        \begin{definition}[Symmetric powers of schemes] \label{def: symmetric_powers_of_schemes}
            Let $S$ be a base scheme and let $Y$ be an $S$-scheme that is either (quasi-)projective or (quasi-)affine\footnote{In particular, the curve $X$ from convention \ref{conv: base_curve} is projective.}. Then for any $d \geq 1$, the \textbf{$d^{th}$ symmetric power} of $Y$ is the quotient scheme $\Sym^d_S(Y) := Y^d/\underline{\Sigma_d}_{/S}$ of $Y$ by the constant symmetric $S$-group scheme $\underline{\Sigma_d}_{/S}$ on $d$ elements (which is finite, flat, and locally of finite presentation over $S$, since it is represented by $\coprod_{\sigma \in \Sigma_d} S$); here, $\underline{\Sigma_d}_{/S}$ acts via permutations, which is well-known to be a free action.
        \end{definition}
        \begin{convention}
            Because the base field $k$ of our curve from convention \ref{conv: base_curve} is fixed, let us write $X^{(d)}$ instead of $\Sym^d_k(X)$ for simplicity.
        \end{convention}
        \begin{proposition}[Smoothness of symmetric powers] \label{prop: smoothness_of_symmetric_powers}
            If $Y$ is a smooth variety over some field $C$ of dimension $\leq 1$ then so is $\Sym_F^d(Y)$.
        \end{proposition}
            \begin{proof}
                Smoothness is preserved by base change, so we might as well assume that $C$ is algebraically closed. The case $\dim Y = 0$ is trivial, so let us assume that $\dim Y = 1$ (i.e. that $Y$ is a curve); in this case, the formal completion of the stalk $\calO_{Y, y}$ at any point $y \in |Y|$ is necessarily isomorphic to $C[\![y]\!]$\footnote{We are intentionally confusing the point $y \in |Y|$ and the formal variable $y \in C[\![y]\!]$ because $(y)$ is the unique maximal ideal of $C[\![y]\!]$.}, since $\calO_{Y, y}$ shall be a finite-type commutative algebra over an algebraically closed field; as a result, the formal completion of the stalk $\calO_{\Sym^d_C(Y), \vec{y}}$ at any point $\vec{y} \in \Sym^d_C(Y)$ is isomorphic to symmetric formal power series ring $k[\![y_1, ..., y_d]\!]^{\Sigma_d}$, which itself is isomorphic to $k[\![y_1, ..., y_d]\!]$ \textit{a priori}. A Noetherian local ring $(A, \m)$ is regular if and only if its $\m$-adic completion is regular, and if $A, B$ are finite-type regular commutative algebras over an algebraically closed field $C$ then $A \tensor_C B$ will also be regular as a $C$-algebra, so $\Sym_C^d(Y)$ will be smooth if $C[\![y]\!]$ is regular, which is rather self-evident.
            \end{proof}
        
        \begin{definition}[Divisors] \label{def: divisors}
            Let $Y$ be a scheme. An \textbf{effective (Cartier) divisor} on $Y$ is then a closed subscheme $D \subset Y$ whose ideal sheaf $\calI_D$ is a line bundle on $Y$.
        \end{definition}
        \begin{convention}
            Given any effective divisor $D \subset Y$, any integer $n$, and any $\E \in \QCoh(X_{\Zar})$, let us write $\E(nD) \cong \E \tensor_{\calO_Y} \calI_D^{\tensor (-n)}$ (wherein $\calI_D^{\tensor (-n)} \cong (\calI_D^{\vee})^{\tensor n}$). In particular, note that $\calO_Y(-D) \cong \calI_D$.
        \end{convention}
        \begin{definition}[The Abel-Jacobi map] \label{def: the_abel_jacobi_map}
            
        \end{definition}
        
        \begin{convention}[Fibrations of schemes] \label{conv: fibrations}
            Following \cite[\href{https://stacks.math.columbia.edu/tag/01OA}{Tag 01OA}]{stacks}, if $S$ is a scheme and $\E$ is a quasi-coherent $\calO_S$-module, then its associated \textbf{projective bundle} shall be the $S$-scheme $\P_S(\E) \cong \Proj_S(\Sym_{\calO_S}(\E))$ (here, $\Proj_S: \Sch_{/S}^{\op} \to \Sets$ denotes the $\Proj$-construction; for details, see \cite[\href{https://stacks.math.columbia.edu/tag/01NM}{Tag 01NM} and \href{https://stacks.math.columbia.edu/tag/01NS}{Tag 01NS}]{stacks}).
        \end{convention}
        \begin{lemma}[The Abel-Jacobi map is a smooth fibration] \label{lemma: the_abel_jacobi_map_is_a_smooth_fibration}
            
        \end{lemma}
            \begin{proof}
                
            \end{proof}
        \begin{convention}
            Denote the \href{https://stacks.math.columbia.edu/tag/0BY6}{\underline{genus}} of our curve $X$ by $g$.  
        \end{convention}
        \begin{remark}
            If we were to take the Riemann-Roch Theorem for granted (a proof along with related discussions can be found at \cite[\href{https://stacks.math.columbia.edu/tag/0B5B}{Tag 0B5B}]{stacks}), then a direct consequence will be that if $d \geq 2g - 1$ then
        \end{remark}
    
    \subsection{Hecke eigensheaves and Categorical-Geometric Global Langlands for \texorpdfstring{$\GL_1$}{} (the function field case)}
        \begin{convention}[The setting of the main theorem] \label{conv: automorphic_side_conventions}
            For us, $\Bun_{\GL_1}(X)$ shall denote the moduli stack of line bundles on $X$. Traditionally, this is usually referred to as the \textbf{Picard stack} and denoted by $\Pic_X$ (cf. \cite[\href{https://stacks.math.columbia.edu/tag/0372}{Tag 0372}]{stacks}), but we opt for the notation $\Bun_{\GL_1}(X)$ because in the wider context of the Geometric Langlands Programme, one works with $\Bun_G(X)$ for $G$ a general connected reductive group (of which $\GL_1$ is a special case). 
            
            In addition, let us now suppose that the base field $k$ from convention \ref{conv: base_curve} is perfect, and that our curve $X$ from convention \ref{conv: base_curve} is, in addition, geometrically connected.
        \end{convention}
        \begin{remark}[The geometry of $\Bun_{\GL_1}$] \label{remark: geometry_of_the_picard_stack}
            Since we are working with a smooth projective curve $X$ over an algebraically closed field (cf. convention \ref{conv: automorphic_side_conventions}) and hence over a separably closed field, the prestack $\Bun_{\GL_1}$ is \textit{a priori} represented by a scheme (cf. \cite[\href{https://stacks.math.columbia.edu/tag/0B9Z}{Tag 0B9Z}]{stacks}) as an fppf-sheaf on $X$ (and hence as an \'etale and as a Zariski sheaves, since the fppf toppology is finer than both these topologies). As a result, when considering sheaves on $\Bun_{\GL_1}(X)$, we will only need to know about sheaves on schemes instead of the entire fully general theory of sheaves on prestacks. Furthermore, should $X$ be of \href{https://stacks.math.columbia.edu/tag/0BY6}{\underline{genus}} $g \geq 0$, then one has the following decomposision:
                $$\Bun_{\GL_1}(X) \cong \coprod_{d \in \Z} \Bun_{\GL_1}^{(d)}(X)$$
            wherein each $\Bun_{\GL_1}^{(d)}(X)$ is the moduli scheme of line bundles of \href{https://stacks.math.columbia.edu/tag/0AYQ}{\underline{degree}} $d$ on $X$, which is a proper smooth variety (cf. \cite[\href{https://stacks.math.columbia.edu/tag/0BA0}{Tag 0BA0}]{stacks}).
        \end{remark}
        
        \begin{definition}[Hecke correspondences] \label{def: hecke_correspondences}
            \noindent
            \begin{enumerate}
                \item \textbf{(The global Hecke correspondence):} The \textbf{global Hecke correspondence} is a span, i.e. a diagram of the form:
                    $$
                        \begin{tikzcd}
                        	& {\Hecke_{\GL_1}(X)} \\
                        	{\Bun_{\GL_1}(X)} && {X \x \Bun_{\GL_1}(X)}
                        	\arrow["{\cev{h}_X}"', from=1-2, to=2-1]
                        	\arrow["{\supp_X \x \vec{h}_X}", from=1-2, to=2-3]
                        \end{tikzcd}
                    $$
                wherein $\Hecke_{\GL_1}(X)$ is the moduli stack of quadruples:
                    $$(\E_1, \E_2, x, \beta_x)$$
                consisting of line bundles $\E_1, \E_2 \in \Bun_{\GL_1}(X)$, points $x \in X$, and for each such point $x$, a monomorphism $\beta_x: \E_1 \hookrightarrow \E_2$ whose cokernel is the skyscraper sheaf $k_x$ supported at $x \in X$ with value $k$. It is naturally equipped with two projection functors $\cev{h}_X$ and $\supp_X \x \vec{h}_X$, which are defined via:
                    $$\cev{h}_X(\E_1, \E_2, x, \beta_x) \cong \E_1$$
                    $$(\supp_X \x \vec{h}_X)(\E_1, \E_2, x, \beta_x) \cong (x, \E_2)$$
                and hence one obtains the global Hecke correspondence as a span.
                \item \textbf{(Local Hecke correspondences):} Since $(\supp_X \x \vec{h}_X)(\E_1, \E_2, x, \beta_x) \cong (x, \E_2)$, one obtains (via taking fibres) a canonical \textbf{local Hecke correspondence} at each point $x \in X$ as follows:
                    $$
                        \begin{tikzcd}
                        	& {\Hecke_{\GL_1}(x)} \\
                        	{\Bun_{\GL_1}(X)} && {\Bun_{\GL_1}(X)}
                        	\arrow["{\cev{h}_x}"', from=1-2, to=2-1]
                        	\arrow["{\vec{h}_x}", from=1-2, to=2-3]
                        \end{tikzcd}
                    $$
                wherein $\Hecke_{\GL_1}(x)$ is the moduli stack of triples $(\E_1, \E_2, \beta_x)$ consisting of line bundles $\E_1, \E_2 \in \Bun_{\GL_1}(X)$ along with a monomorphism $\beta_x: \E_1 \hookrightarrow \E_2$ whose cokernel is the skyscraper sheaf $k_x$ supported at $x \in X$ with value $k$, and $\cev{h}_x$ and $\vec{h}_x$ are defined in the obvious manner:
                    $$\cev{h}_x(\E_1, \E_2, \beta_x) \cong \E_1$$
                    $$\vec{h}_x(\E_1, \E_2, \beta_x) \cong (x, \E_2)$$
            \end{enumerate}
        \end{definition}
        
        \begin{definition}[Hecke operators] \label{def: hecke_operators}
            The Hecke operators are integral transforms induced by the Hecke Correspondence in the following manners:
            \begin{enumerate}
                \item \textbf{(The global Hecke operators):} The global Hecke correspondence induces the following sheaf pull-push diagram:
                    $$
                        \begin{tikzcd}
                        	& {\Shv_{\overline{\Q_{\ell}}}^1(\Hecke_{\GL_1}(X))} \\
                        	{\Shv_{\overline{\Q_{\ell}}}^1(\Bun_{\GL_1}(X))} && {\Shv_{\overline{\Q_{\ell}}}^1(X \x \Bun_{\GL_1}(X))}
                        	\arrow["{(\cev{h}_X)^*}", from=2-1, to=1-2]
                        	\arrow["{(\supp_X \x \vec{h}_X)_*}", from=1-2, to=2-3]
                        \end{tikzcd}
                    $$
                and by composing the two functors in the obvious manner, one gets a new functor:
                    $$\scrH_X := (\vec{h}_X)_* (\supp_X \x \cev{h}_X)^*$$
                which we shall call the \textbf{global Hecke operator}. 
                \item \textbf{(Local Hecke operators):} Similarly, we define the \textbf{local Hecke operator} at each point $x \in X$ 
                    $$
                        \begin{tikzcd}
                        	& {\Shv_{\overline{\Q_{\ell}}}^1(\Hecke_{\GL_1}(x))} \\
                        	{\Shv_{\overline{\Q_{\ell}}}^1(\Bun_{\GL_1}(X))} && {\Shv_{\overline{\Q_{\ell}}}^1(\Bun_{\GL_1}(X))}
                        	\arrow["{(\cev{h}_x)^*}", from=2-1, to=1-2]
                        	\arrow["{(\vec{h}_x)_*}", from=1-2, to=2-3]
                        \end{tikzcd}
                    $$
                to be the following composition:
                    $$\scrH_x := (\vec{h}_x)_* (\cev{h}_x)^*$$
                with the functors $\cev{h}_x, \vec{h}_x$ as in definition \ref{def: hecke_correspondences}.
            \end{enumerate}
        \end{definition}
    
        \begin{definition}[Hecke eigensheaves] \label{def: hecke_eigensheaves}
            A \textit{non-zero} $\ell$-adic sheaf $\E \in \Shv_{\overline{\Q_{\ell}}}^1(\Bun_{\GL_1}(X))$ is called a \textbf{Hecke eigensheaf} (of rank $1$) if and only if there exists an $\ell$-adic sheaf $\calL \in \Shv_{\overline{\Q_{\ell}}}^1(X)$ (called the \textbf{eigenvalue} of $\E$) such that:
                $$\scrH_X(\E) \cong \calL \boxtimes \E$$
        \end{definition}
        \begin{remark}
            It is easy to see that Hecke eigensheaves form a full symmetric monoidal subcategory of $\Shv_{\overline{\Q_{\ell}}}^1(\Bun_{\GL_1}(X))$, which we shall denote by $\Eig^1_{\overline{\Q_{\ell}}}(\Bun_{\GL_1}(X))$. Furthermore, each Hecke eigensheaf $\E$ with eigenvalue $\calL$ is an \say{eigenvector} of any of the local Hecke operators $\scrH_x$ in the following manner\footnote{Note that the category of $\ell$-adic local systesm of rank $1$ on each point $x \in X$ is nothing but $\Vect^1(\overline{\Q_{\ell}})$, the category of $1$-dimensional $\overline{\Q_{\ell}}$-vector spaces.}:
                $$\scrH_x(\E) \cong (\overline{\Q_{\ell}})_x \boxtimes \E$$
            thanks to the fact that the stalks $\calL_x$ are all isomorphic to $\overline{\Q_{\ell}}$, since $\calL$ is an $\ell$-adic local system of rank $1$.
        \end{remark}
    
        \begin{theorem}[Unramified abelian geometric class field theory] \label{theorem: unramified_abelian_geometric_class_field_theory}
            There exists a canonical equivalence between the groupoid of rank-$1$ $\ell$-adic local systems on $X$ and the groupoid of ($\ell$-adic) Hecke eigensheaves of rank $1$ on $\Bun_{\GL_1(X)}$:
                $$\LocSys^1_{\overline{\Q_{\ell}}}(X) \cong \Eig^1_{\overline{\Q_{\ell}}}(\Bun_{\GL_1}(X))$$
                $$\calL \to \Autom_X(\calL)$$
            which maps each local system $\calL \in \LocSys^1_{\overline{\Q_{\ell}}}(X)$ to a Hecke eigensheaf $\Autom_X(\calL) \in \Eig^1_{\overline{\Q_{\ell}}}(\Bun_{\GL_1}(X))$ with eigenvalue $\calL$.
        \end{theorem}
            \begin{proof}
                Our strategy for this proof is to explicitly construct - for each $\calL \in \LocSys^1_{\overline{\Q_{\ell}}}$ - the corresponding Hecke eigensheaf $\Autom_X(\calL)$, and this will involve three steps:
                    \begin{enumerate}
                        \item \textbf{(A local system on $X^{(d)}$):} Denote the quotient map defining $X^{(d)}$ by $q^{(d)}: X^d \to X^{(d)}$ and define $\Delta_X^{(d)} := q^{(d)} \circ \Delta_X^d$. Next, consider the following right-exact functor:
                            $$\LocSys^1_{\overline{\Q_{\ell}}}(X) \to \LocSys^1_{\overline{\Q_{\ell}}}(X^{(d)})$$
                            $$\calL \mapsto \calL^{(d)} := ((\Delta_X^{(d)})_*\calL)^{\Sigma_d}$$
                        The first observation that one can make is that $(q^{(d)})^*\calL^{(d)} \cong \calL^{\boxtimes d} \cong (\Delta_X^d)_*\calL$.
                        \item \textbf{(Pushing the local system forward):}
                        \item \textbf{(Obtaining a Hecke eigensheaf):}
                    \end{enumerate}
            \end{proof}
        \begin{remark}[How should we interpret theorem \ref{theorem: unramified_abelian_geometric_class_field_theory} ?] \label{remark: unramified_abelian_geometric_class_field_theory_explanation}
            By putting theorem \ref{theorem: unramified_representations_are_sheaves_on_X} and theorem \ref{theorem: unramified_abelian_geometric_class_field_theory} together, one gets a canonical equivalence of categories as follows:
                $$\Rep_{\overline{\Q_{\ell}}}^1(\pi_1^{\ab}(X_{\fet}))^{\cont} \cong \Eig^1_{\overline{\Q_{\ell}}}(\Bun_{\GL_1}(X))$$
                $$\chi \mapsto \Autom_X(\chi)$$
            Modulo technicalities, what this essentially tells us is that $1$-dimensional continuous $\ell$-adic Galois representations are the same as automorphic forms associated to $\GL_1$, and as such, the combination of theorem \ref{theorem: unramified_representations_are_sheaves_on_X} and theorem \ref{theorem: unramified_abelian_geometric_class_field_theory} can be understood as the Categorical Global Unramified Geometric Langlands Correspondence in its simplest non-trivial form, that being for the (connnected reductive group $G \cong \GL_1$)\footnote{Incidentally, this is why it is commonly asserted that the Langlands Correspondence for $\GL_1$ \say{is just class field theory}.}. 
        \end{remark}
        
    \subsection{Grothendieck's Sheaf-Function Dictionary and \textit{die Gr\"o{\ss}encharaktere}}
        As a final step, let us decategorify the left-hand side of the equivalence:
            $$\Rep_{\overline{\Q_{\ell}}}^1(\pi_1^{\ab}(X_{\fet}))^{\cont} \cong \Eig^1_{\overline{\Q_{\ell}}}(\Bun_{\GL_1}(X))$$
        from remark \ref{remark: unramified_abelian_geometric_class_field_theory_explanation} to obtain a correspondence between continuous $\ell$-adic characters of $\pi_1^{\ab}(X_{\fet})$ and \say{\textit{die Gr\"o{\ss}encharaktere}}, using Grothendieck's Sheaf-Function Dictionary.