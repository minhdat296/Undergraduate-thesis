\section{The Automorphic Side}
    \subsection{Symmetric powers of curves, Jacobians, and the Abel-Jacobi map}
        \begin{convention}[The Picard group] \label{conv: picard_group}
            For any base scheme $S$ and any $S$-scheme $Y$, we shall write $|\Pic_{Y/S}|$ for the group of isomorphism classes of line bundles on $Y$, whose group structure is given by tensor products of invertible quasi-coherent $\calO_{Y/S}$-modules.
        \end{convention}
        \begin{definition}[Divisors] \label{def: divisors}
            Let $Y$ be a scheme. An \textbf{effective (Cartier) divisor}\footnote{Historically referred to as a \say{modulus}.} on $Y$ is then a closed subscheme $D \subset Y$ whose ideal sheaf $\calI_D$ is a line bundle on $Y$. Its \textbf{degree} is the \href{https://stacks.math.columbia.edu/tag/0AYQ}{\underline{degree}} of the line bundle $\calI_D$.  
                
            The set of effective divisors (respectively, effective divisors of degree $d$) shall be denoted by $|\Div_Y^{\eff}|$ (respectively, $|\Div_Y^{\eff, (d)}|$), and as a straightforward consequence of the definition of effective divisors, it is precisely the set of invertible\footnote{A quasi-coherent ideal sheaf $\calJ \subset \calO_Y$ is invertible if and only if its local sections $\calJ(V)$ are principal ideals of $\calO_Y(V)$.} quasi-coherent $\calO_Y$-ideals (respectively, quasi-coherent $\calO_Y$-ideals of degree $d$).
        \end{definition}
        \begin{remark}[Why do we care about divisors ?] 
            Because the function field of the curve $X$ over $\Spec k$ from convention \ref{conv: base_curve} is some (global) field of the form $K \cong k'(t)$, where $k'/k$ is an algebraic extension (cf. proposition \ref{prop: curves_and_function_fields}), and since its ring of integers $\scrO_K \cong k'[t]$ is a Dedekind domain (this is due to Hilbert's Basis Theorem, which tells us that $\dim k'[t] = \dim k + 1 = 0 + 1 = 1$), every of the stalks $\calO_{X, x}$ at points $x \in |X|$ must be a discrete valuation ring. This tells us that there is a bijective correspondence between points $x \in |X|$ and places of $K$ that are trivial on $k'$. Now, also because $\scrO_K$ is a Dedekind domain, every ideal $\a$ therein factors into a (formal) product of primes, say $\a = \prod_{i = 1}^n \p_i^{e_i}$. Furthemore, $\scrO_K$ is actually a PID, as it is isomorphic to $k'[t]$, which is a UFD thanks to $k'$ itself being a UFD, by virtue of being a field. Thus, points $x \in |X|$ are not only in bijection with places of $K$ that are trivial on $k'$, but also effective divisors on $X$.
        \end{remark}
        
        \begin{convention}
            Given any effective divisor $D \subset Y$, any integer $n$, and any $\E \in \QCoh_X$, let us write $\E(nD) \cong \E \tensor_{\calO_Y} \calI_D^{\tensor (-n)}$ (wherein $\calI_D^{\tensor (-n)} \cong (\calI_D^{\vee})^{\tensor n}$). In particular, note that $\calO_Y(-D) \cong \calI_D$.
        \end{convention}
        \begin{remark}[Addition of effective divisors] \label{remark: adding_effective_divisors}
            Let $Y$ be a scheme. Because effective divisors are line bundles, they are trivially flat\footnote{Given any line bundle $\calL \in |\Pic_Y|$, the functor $- \tensor_{\calO_Y} \calL$ is an auto-equivalence of $\Coh_Y$, which is an abelian category, so $- \tensor_{\calO_Y} \calL$ is automatically (left-)exact.}, so given any pair of effective divisors $D, D' \subset Y$, one can define a new effective divisor $D + D'$ corresponding to the $\calO_Y$-ideal $\calI_D \calI_{D'}$, which is isomorphic to $\calI_D \tensor_{\calO_{X/k}} \calI_{D'}$ due to flatness, and hence $\deg(D + D') = \deg D + \deg D'$.
        \end{remark}
        
        \begin{convention}[The setting for geometric class field theory] \label{conv: automorphic_side_conventions}
            \noindent
            \begin{itemize}
                \item For us, $\Bun_{\GL_1}(X)$ shall denote the moduli space of line bundles on $X$. Traditionally, this is usually referred to as the \textbf{Picard stack} and denoted by $\Pic_{X/k}$ (cf. \cite[\href{https://stacks.math.columbia.edu/tag/0372}{Tag 0372}]{stacks}), but we opt for the notation $\Bun_{\GL_1}(X)$ because in the wider context of the Geometric Langlands Programme, one works with $\Bun_G(X)$ for $G$ a general connected reductive group (of which $\GL_1$ is a special case), and to not confuse the moduli space $\Bun_{\GL_1}(X)$ with the group $|\Pic_{X/k}|$ (cf. convention \ref{conv: picard_group}). 
            
                An exception to this convention is the subscheme $\Bun_{\GL_1}^{(0)}(X)$, which happens to be a(n) (abelian) group scheme that is commonly denote by $\Jac_{X/k}$. This is typically referred to as the \textbf{Jacobian variety}, and it is an abelian variety by virtue of being a geometrically connected smooth projective (hence proper) group scheme over $\Spec k$ (cf. remark \ref{remark: geometry_of_the_picard_stack}).
                \item In addition, let us now suppose that the base field $k$ from convention \ref{conv: base_curve} is algebraically closed, and that our curve $X$ from convention \ref{conv: base_curve} is, in addition, geometrically connected. 
            \end{itemize}
        \end{convention}
        \begin{remark}[The geometry of $\Bun_{\GL_1}$] \label{remark: geometry_of_the_picard_stack}
            Since we are working with a smooth projective curve $X$ over an algebraically closed field (cf. convention \ref{conv: automorphic_side_conventions}) and hence over a separably closed field, the prestack $\Bun_{\GL_1}$ is \textit{a priori} represented by a scheme (cf. \cite[\href{https://stacks.math.columbia.edu/tag/0B9Z}{Tag 0B9Z}]{stacks}) as an fppf sheaf on $X$ (and hence as an \'etale and as a Zariski sheaves, since the fppf toppology is finer than both these topologies). As a result, when considering sheaves on $\Bun_{\GL_1}(X)$, we will only need to know about sheaves on schemes instead of the entire fully general theory of sheaves on prestacks. Furthermore, if the genus of $X$ is $g \geq 0$, then one will have the following decomposision:
                $$\Bun_{\GL_1}(X) \cong \coprod_{d \geq 0} \Bun_{\GL_1}^{(d)}(X)$$
            wherein each $\Bun_{\GL_1}^{(d)}(X)$ is the moduli scheme of line bundles of degree $d$ on $X$, which is a proper smooth variety of dimension $g$ over $\Spec k$ (cf. \cite[\href{https://stacks.math.columbia.edu/tag/0BA0}{Tag 0BA0}]{stacks}).
        \end{remark}
        
        \begin{remark}[Moduli space of effective divisors] \label{remark: moduli_space_of_effective_divisors}
            For any base scheme $S$ and any $S$-scheme $Y$, the \textbf{Hilbert functor} of closed subschemes of degree $d \geq 0$ is the presheaf:
                $$\Hilb_{Y/S}^{(d)}: \Sch_{/S}^{\op} \to \Sets$$
                $$T \mapsto \{\text{Finite locally free closed subschemes $D \subset Y_T$ of degree $d$}\}$$
            Should $Y$ be a geometrically irreducible smooth proper (respectively projective) curve over a field $C$ then interestingly, not only are finite locally free closed subschemes $D \subset Y_{C'}$ of degree $d$ precisely the effective divisors of degree $d$ on $Y_{C'}$ for any field extension $C'/C$ (cf. \cite[\href{https://stacks.math.columbia.edu/tag/0B9D}{Tag 0B9D}]{stacks}), but also, one has a bijection:
                $$|\Div_{Y_{C'}/C'}^{\eff, (d)}| \cong \Hilb_{Y/C}^{(d)}(C')$$
            between the set of $C'$-rational points of $\Hilb_{Y/C}^{(d)}$ and that of degree-$d$ effective divisors on $Y_{C'}$ (cf. \cite[\href{https://stacks.math.columbia.edu/tag/0B9I}{Tag 0B9I}]{stacks}). From this, one see that should $Y$ be proper (respectively Zariski-locally projective) and flat over some arbitrary base scheme $S$, and if its fibres $Y_s$ over points $s \in |S|$ are geometrically irreducible smooth proper (respectively projective) curves, then $\Hilb_{Y/S}^{(d)}$ would be the moduli space of degree-$d$ effective divisors on $Y$; thus, for proper (respectively Zariski-locally projective) and flat morphisms $Y \to S$, let us suggestively write $\Div_{Y/S}^{\eff, (d)}$ instead of $\Hilb_{Y/S}^{(d)}$. It is known moreover that $\Hilb_{Y/C}^{(d)}$ is represented by a smooth proper variety of dimension $d$ over $\Spec C$ (cf. \cite[\href{https://stacks.math.columbia.edu/tag/0B9I}{Tag 0B9I}]{stacks}). By putting everything together, one obtains a moduli space $\Div_{Y/C}^{\eff, (d)} \in (Y/C)_{\fppf}$ parametrising degree-$d$ effective divisors on $Y$, represented by a smooth proper variety of dimension $d$ over $\Spec C$ and naturally isomorphic to $\Hilb_{Y/C}^{(d)}$.
        \end{remark}
        
        It should also be noted that what we have just discussed is not the only way to show that $\Div_{X/k}^{\eff, (d)}$ is represented by a smooth proper variety of dimension $d$. Remark \ref{remark: moduli_space_of_effective_divisors} serves more as a demonstration that there \textit{should} be a moduli space of effective divisors (of a given degree $d$), rather than that there \textit{is} one. In fact, by combining remark \ref{remark: quotients_of_schemes_by_finite_group_schemes}, lemma \ref{lemma: smoothness_of_symmetric_powers}, and proposition \ref{prop: symmetric_powers_of_curves_parametrise_divisors}, we shall see that the functor $\Div_{X/k}^{\eff, (d)}$ is represented by a smooth proper variety of (pure) dimension $d$ by virtue of being naturally isomorphic to the functor of points of $X^{(k)}$ (which, of course, is smooth, proper, and of dimension $d$). It is, however, important to know that $\Div_{X/k}^{\eff, (d)}$ indeed satisfies fppf descent (hence \'etale descent) to prove proposition \ref{prop: symmetric_powers_of_curves_parametrise_divisors}, and since this comes from the general fact that the Hilbert functor satisfies fppf descent, remark \ref{remark: moduli_space_of_effective_divisors} remains necessary. 
        
        We now know that effective divisors of a given degree $d$ are parametrised by some smooth proper variety $\Div_{Y/S}^{\eff, (d)}$ of (relative) dimension $d$, but this is not entirely satisfactory: we would also like to know the identity of this smooth proper variety, and we shall after proposition \ref{prop: symmetric_powers_of_curves_parametrise_divisors}.
        \begin{remark}[Quotient of schemes by finite group schemes] \label{remark: quotients_of_schemes_by_finite_group_schemes}
            For details on quotients of schemes by finite groups, we refer the reader to \cite[Expos\'e V]{SGA1}. For our purposes, we shall only need to keep in mind the following facts: 
                \begin{enumerate}
                    \item Let $S$ be a base scheme and $Y$ be an $S$-scheme that is either \textit{(quasi-)projective or (quasi-)affine}, and if additionally. If $G$ be a \textit{finite, flat, and locally of finite presentation} group $S$-scheme acting \textit{freely}\footnote{This is to ensure that the $G$-action induces an fppf equivalence relation on $Y$, since the $G$-action on $Y$ is free if and only if the corresponding homomorphism of sheaves of groups $G \to \Aut_{S_{\fppf}}(Y)$ is injective.} on $Y$, then the \href{https://stacks.math.columbia.edu/tag/025X}{\underline{algebraic space}} $Y/G$ is a scheme (cf. \cite[\href{https://stacks.math.columbia.edu/tag/07S7}{Tag 07S7}]{stacks}). 
                    \item If $Y$ is a (quasi-)affine scheme $\Spec A$ then the quotient $Y/G$ will also be affine and will be isomorphic to $\Spec A^G$, thanks to the group-cohomological fact that $H^0(A, G) \cong A^G$. 
                    
                    If $Y$ is (quasi-)projective then $Y/G$ will also be (quasi-)projective.
                \end{enumerate}
        \end{remark}
        \begin{definition}[Symmetric powers of schemes] \label{def: symmetric_powers_of_schemes}
            Let $S$ be a base scheme and let $Y$ be an $S$-scheme that is either (quasi-)projective or (quasi-)affine\footnote{In particular, the curve $X$ from convention \ref{conv: base_curve} is projective.}. Then for any $d \geq 1$, the \textbf{$d^{th}$ symmetric power} of $Y$ is the quotient scheme $\Sym^d_S(Y) := Y^d/\underline{\Sigma_d}_{/S}$ of $Y$ by the constant symmetric $S$-group scheme $\underline{\Sigma_d}_{/S}$ on $d$ elements (which is finite, flat, and locally of finite presentation over $S$, since it is represented by $\coprod_{\sigma \in \Sigma_d} S$); here, $\underline{\Sigma_d}_{/S}$ acts via permutations, which is well-known to be a free action.
        \end{definition}
        \begin{convention}
            Because the base field $k$ of our curve from convention \ref{conv: base_curve} is fixed, let us write $X^{(d)}$ instead of $\Sym^d_k(X)$ for simplicity.
        \end{convention}
        \begin{lemma}[Smoothness of symmetric powers] \label{lemma: smoothness_of_symmetric_powers}
            If $Y$ is a smooth variety over some field $C$ of dimension $\leq 1$ then so is $\Sym_C^d(Y)$.
        \end{lemma}
            \begin{proof}
                Smoothness is preserved by base change, so we might as well assume that $C$ is algebraically closed. The case $\dim Y = 0$ is trivial, so let us assume that $\dim Y = 1$ (i.e. that $Y$ is a curve); in this case, the formal completion of the stalk $\calO_{Y, y}$ at any point $y \in |Y|$ is necessarily isomorphic to $C[\![y]\!]$\footnote{We are intentionally confusing the point $y \in |Y|$ and the formal variable $y \in C[\![y]\!]$ because $(y)$ is the unique maximal ideal of $C[\![y]\!]$.}, since $\calO_{Y, y}$ shall be a finite-type commutative algebra over an algebraically closed field; as a result, the formal completion of the stalk $\calO_{\Sym^d_C(Y), \vec{y}}$ at any point $\vec{y} \in \Sym^d_C(Y)$ is isomorphic to symmetric formal power series ring $k[\![y_1, ..., y_d]\!]^{\Sigma_d}$, which itself is isomorphic to $k[\![y_1, ..., y_d]\!]$ \textit{a priori}. A Noetherian local ring $(A, \m)$ is regular if and only if its $\m$-adic completion is regular, and if $A, B$ are finite-type regular commutative algebras over an algebraically closed field $C$ then $A \tensor_C B$ will also be regular as a $C$-algebra, so $\Sym_C^d(Y)$ will be smooth if $C[\![y]\!]$ is regular, which is definitely the case since $\dim C[\![y]\!] = \dim_C (y)/(y)^2 = 1$.
            \end{proof}
        \begin{proposition}[Symmetric powers of curves parametrise divisors] \label{prop: symmetric_powers_of_curves_parametrise_divisors}
            For each $d$, the moduli space $\Div_{X/k}^{\eff, (d)}$ is represented by the smooth variety $X^{(d)}$.
        \end{proposition}
            \begin{proof}
                Let us denote \textit{unordered} $d$-tuples by $\<x_1, ..., x_d\>$ and also, write $[x]$ for the divisor cut out by any closed point $x \in |X|$. Now, to begin, consider the following $\Sigma_d$-invariant function, which if shown to be bijective will demonstrate that $\Div_{X/k}^{\eff, (d)}$ is represented by the smooth variety $X^{(d)}$ via \'etale descent (cf. \cite[\href{https://stacks.math.columbia.edu/tag/024V}{Tag 024V}]{stacks}):
                    $$X^{(d)} \to |\Div_{X/k}^{\eff, (d)}|$$
                    $$\<x_1, ..., x_d\> \mapsto [x_1] + ... + [x_d]$$
                Now, $X$ is a geometrically connected smooth curve, so it is geometrically normal (cf. \cite[\href{https://stacks.math.columbia.edu/tag/056T}{Tag 056T}]{stacks}), which in turn implies that the stalk of its structure sheaf over its unique generic point is a normal local domain of Krull dimension $1$, hence a Dedekind domain (cf. \cite[\href{https://stacks.math.columbia.edu/tag/034X}{Tag 034X}]{stacks}). This implies that every divisor on $X$ splits into prime divisors (which correspond to closed points of $X$), and thus the function $\<x_1, ..., x_d\> \mapsto [x_1] + ... + [x_d]$ is surjective. Dedekind domains are special cases of UFDs, so we have also demonstrated that the function $\<x_1, ..., x_d\> \mapsto [x_1] + ... + [x_d]$ is injective.
            \end{proof}
        \begin{corollary}[$X \cong \Div_{X/k}^{\eff, (1)}$]
            $k$-rational points of $X$ are precisely the degree-$1$ effective divisors.
        \end{corollary}
        \begin{convention}
            From this point on, we shall write $\Div_{X/k}^{\eff, (d)}$ instead of $X^{(d)}$ whenever we would like to put emphasis on the fact that points of $X^{(d)}$ are effective divisors of degree $d$ on $X$ (such as in \ref{prop: the_abel_jacobi_map_is_a_smooth_projective_fibration}), and \textit{vice versa}, we shall write $X^{(d)}$ when symmetry is of importance, like in theorem \ref{theorem: unramified_abelian_geometric_class_field_theory}.
        \end{convention}
        
        \begin{definition}[The Abel-Jacobi map] \label{def: the_abel_jacobi_map}
            Let $Y$ be a geometrically connected smooth projective curve over some field $C$. Then, the \textbf{$d^{th}$ Abel-Jacobi map} associated to $Y/C$ is the morphism of smooth proper varieties:
                $$\alpha_{Y/C}^{(d)}: \Div_{Y/C}^{\eff, (d)} \to \Bun_{\GL_1}^{(d)}(Y)$$
            which section-wise (i.e. at each field extension $C'/C$) associates to each degree-$d$ effective divisor $D \in |\Div_{Y_{C'}/C'}^{\eff, (d)}|$ to its corresponding invertible quasi-coherent $\calO_{Y_{C'}}$-ideal $\calI_D \in |\Pic_{Y_{C'}/C'}^{(d)}|$.
        \end{definition}
        \begin{remark}[What does the Abel-Jacobi map do ?] \label{remark: abel_jacobi_map}
            
        \end{remark}
        \begin{convention}[Genus of the curve] \label{conv: genus_of_the_curve}
            From now on, denote the \href{https://stacks.math.columbia.edu/tag/0BY6}{\underline{genus}} of our curve $X$ by $g$.
        \end{convention}
        \begin{proposition}[The Abel-Jacobi map is a smooth projective fibration] \label{prop: the_abel_jacobi_map_is_a_smooth_projective_fibration}
            If $d \geq 2g - 1$ then every Abel-Jacobi map $\alpha_{X/k}^{(d)}: \Div_{X/k}^{\eff, (d)} \to \Bun_{\GL_1}^{(d)}(X)$ will be a surjective smooth projective morphism with fibres\footnote{Note that these are precisely the geometric fibres, since $k$ is algebraically closed.} over $k$-rational points isomorphic to $\P^{d - g}_k$.
        \end{proposition}
            \begin{proof}
                Let us note, first of all, that thanks due to a descent-theoretic feature of the \'etale topology, namely \cite[\href{https://stacks.math.columbia.edu/tag/024V}{Tag 024V}]{stacks}, it shall suffice to demonstrate that the \textit{set-theoretic} fibres of the Abel-Jacobi maps are in bijection with the set of $k$-rational points of $\P_k^{d - g}$; for the same reason, it also suffices to only show that the Abel-Jacobi map is surjective at each point $k$-rational point $\calL \in |\Bun_{\GL_1}^{(d)}(X)|$. We shall proceed in steps, for the sake of clarity.
                    \begin{enumerate}
                        \item \textbf{(Projectivity):} Let us first apply the Riemann-Roch Theorem (cf. \cite[\href{https://stacks.math.columbia.edu/tag/0BS6}{Tag 0BS6}]{stacks}; note that the theorem is applicable to our situation because $X$ is a geometrically connected smooth projective curve and therefore a Gorenstein\footnote{Smooth schemes are Gorenstein because their stalks are regular local rings.} scheme of equidimension $1$ over a field), which tells us that should $\E \in \Vect_{X/k}^n$ be a be a locally free quasi-coherent $\calO_{X/k}$-module of constant rank $n$, then:
                            $$\chi(X, \E) = \deg(\E) - \frac12\rank(\E) \deg(\omega_{X/k})$$
                        where $\omega_{X/k}$ denotes the dualising sheaf (which is a line bundle due also to the Riemann-Roch Theorem) and $\chi(X, \E)$ denotes the Euler characteristic of $\E$ as a coherent sheaf on the proper $k$-scheme $X$. Because $\deg(\omega_{X/k}) = 2g - 2$ (thanks to \cite[\href{https://stacks.math.columbia.edu/tag/0C19}{Tag 0C19}]{stacks}, which is applicable in this situation because $X$ is a proper Gorenstein $k$-scheme such that $H^0_{\Zar}(X, \calO_{X/k}) \cong k$), the above tells us that for any degree-$d$ line bundle $\calL \in |\Bun_{\GL_1}^{(d)}(X)|$, we have:
                            $$\chi(X, \calL) = \deg(\calL) - \frac12 \rank(\calL) \deg(\omega_{X/k}) = d - \frac12 \cdot 1 \cdot (2g - 2) = d - g + 1$$
                        Line bundles are particular cases of coherent sheaves with support dimension $\leq 0$, and since $X$ is proper over a field, we can apply \cite[\href{https://stacks.math.columbia.edu/tag/0AYT}{Tag 0AYT}]{stacks} to get that:
                            $$\dim_k H^0_{\Zar}(X, \calL) = \chi(X, \calL) = d - g + 1$$
                        This implies that the (set-theoretic) fibres of $\alpha_{X/k}^{(d)}$ over $k$-rational points $\calL \in |\Bun_{\GL_1}^{(d)}(X)|$ are isomorphic to $\P^{d - g}_k$ (empty if $d < g$).
                        \item \textbf{(Smoothness):} $\Bun_{\GL_1}^{(d)}(X)$ is smooth (cf. remark \ref{remark: geometry_of_the_picard_stack}), hence it is regular, and $X^{(d)}$ is also smooth (cf. lemma \ref{lemma: smoothness_of_symmetric_powers}) and hence it is Cohen-Macaulay (because regular local rings are Cohen-Macaulay \textit{a priori}; cf. \cite[\href{https://stacks.math.columbia.edu/tag/00NQ}{Tag 00NQ}]{stacks}). Moreover, we have shown above that given any $\calL \in \Bun_{\GL_1}^{(d)}(X)$, the corresponding fibre of the Abel-Jacobi map is isomorphic to $\P^{d - g}_k$, and since $\dim \Bun_{\GL_1}^{(d)}(X) = g$ (cf. remark \ref{remark: geometry_of_the_picard_stack}) while $\dim X^{(d)} = \dim \Div_{X/k}^{\eff, (d)} = d$ (cf. remark \ref{remark: moduli_space_of_effective_divisors}), we have:
                            $$\dim \P^{d - g}_k = \dim (\alpha_{X/k}^{(d)})^{-1}(\calL) = \dim \Div_{X/k}^{\eff, (d)} - \dim \Bun_{\GL_1}^{(d)}(X) = d - g$$
                        The Miracle Flatness Theorem (cf. \cite[\href{https://stacks.math.columbia.edu/tag/00R4}{Tag 00R4}]{stacks}) can then be applied, which tells us that the Abel-Jacobi map is flat everywhere, and because the fibres are isomorphic to $\P^{d - g}_k$, which is smooth over $\Spec k$, this means that the Abel-Jacobi map is also smooth everywhere.
                        \item \textbf{(Surjectivity):}
                    \end{enumerate}
            \end{proof}
        \begin{corollary}[Galois representations induced by the Abel-Jacobi map] \label{coro: galois_representations_induced_by_the_abel_jacobi_map}
            Because $\alpha_{X/k}^{(d)}$ is proper and smooth, it is proper, flat, and of finite presentation, so by proposition \ref{prop: etale_homotopy_exact_sequence} there is an induced \'etale homotopy sequence as follows:
                $$\pi_1((\P^{d - g}_k)_{\fet}) \to \pi_1((\Div_{X/k}^{\eff, (d)})_{\fet}) \to \pi_1((\Bun_{\GL_1}^{(d)}(X))_{\fet}) \to 1$$
            Since $\P^{d - g}_k$ is \'etale-simply connected (this is a consequence of $k$ being algebraically closed; cf. example \ref{example: etale_fundamental_group_of_a_curve}), one thus obtains an equivalence between the categories of continuous $\ell$-adic characters of $\pi_1((\Div_{X/k}^{\eff, (d)})_{\fet})$ and of $\pi_1((\Bun_{\GL_1}^{(d)}(X))_{\fet})$ as below, wherein $(\alpha_{X/k}^{(d)})^*$ is the pullback of $\ell$-adic sheaves along the Abel-Jacobi map $\alpha_{X/k}^{(d)}: \Div_{X/k}^{\eff, (d)} \to \Bun_{\GL_1}^{(d)}(X)$:
                $$\Rep^1_{\bar{\Q}_{\ell}}(\pi_1((\Div_{X/k}^{\eff, (d)})_{\fet})) \cong \Rep^1_{\bar{\Q}_{\ell}}(\pi_1((\Bun_{\GL_1}^{(d)}(X))_{\fet}))$$
                $$\chi \mapsto \chi \circ (\alpha_{X/k}^{(d)})^*$$
        \end{corollary}
    
     \subsection{Hecke eigensheaves and geometric class field theory}
        \begin{definition}[Hecke eigensheaves] \label{def: hecke_eigensheaves}
            A \textit{non-zero} $\ell$-adic local system $\E \in \Shv_{\underline{\bar{\Q}_{\ell}}}^{\ad, 1}(\Bun_{\GL_1}(X))$ is called a \textbf{Hecke eigensheaf} (of rank $1$) if and only if there exists an $\ell$-adic sheaf $\calL \in \Shv_{\underline{\bar{\Q}_{\ell}}}^{\ad, 1}(X)$ (called the \textbf{eigenvalue} of $\E$) such that:
                $$\scrH(\E) \cong \calL \boxtimes \E$$
            where $\scrH := \cev{h}^*$, with $\cev{h}$ being the canonical action of $\Bun_{\GL_1}(X)$ on $X$, namely:
                $$\cev{h}: X \x \Bun_{\GL_1}(X) \to \Bun_{\GL_1}(X)$$
                $$([x], \calL) \mapsto \calL([x])$$
        \end{definition}
        \begin{remark}
            It is easy to see that Hecke eigensheaves form a full symmetric monoidal subcategory of $\Shv_{\underline{\bar{\Q}_{\ell}}}^{\ad, 1}(\Bun_{\GL_1}(X))$, which we shall denote by $\Eig_{\underline{\bar{\Q}_{\ell}}}^1(\Bun_{\GL_1}(X))$.
        \end{remark}
        The following lemma regarding the \say{globality} of Hecke eigensheaves is merely a technicality in service of theorem \ref{theorem: unramified_abelian_geometric_class_field_theory}. The reader can safely skip ahead and refer back to it later.
        \begin{lemma}[Hecke eigensheaves extend to lower degrees] \label{lemma: hecke_eigensheaves_extend_to_lower_degrees}
            Let $\E$ be a Hecke eigensheaf on $\bigcup_{d \geq d_0 + 1} \Bun_{\GL_1}^{(d)}(X)$ (for any $d_0 \geq 0$). Then, $\E$ can be extended uniquely to a Hecke eigensheaf on $\bigcup_{d \geq d_0} \Bun_{\GL_1}^{(d)}(X)$.
        \end{lemma}
            \begin{proof}
                
            \end{proof}
            
        Let us now begin our discussion of geometric class field theory with the unramified case. The proof of the ramified case (cf. theorem \ref{theorem: ramified_abelian_geometric_class_field_theory}), as it turns out, will only include a slight modification of the proof of the unramified case (cf. theorem \ref{theorem: unramified_abelian_geometric_class_field_theory}).
        \begin{theorem}[Unramified abelian geometric class field theory] \label{theorem: unramified_abelian_geometric_class_field_theory}
            There exists a canonical equivalence between the groupoid of rank-$1$ $\ell$-adic local systems on $X$ and the groupoid of ($\ell$-adic) Hecke eigensheaves of rank $1$ on $\Bun_{\GL_1(X)}$:
                $$\Autom: \Shv_{\underline{\bar{\Q}_{\ell}}}^{\ad, 1}(X) \to \Eig_{\underline{\bar{\Q}_{\ell}}}^1(\Bun_{\GL_1}(X))$$
            which maps each $\ell$-adic local system $\calL \in \Shv_{\underline{\bar{\Q}_{\ell}}}^{\ad, 1}(X)$ to a Hecke eigensheaf $\Autom(\calL) \in \Eig_{\underline{\bar{\Q}_{\ell}}}^1(\Bun_{\GL_1}(X))$ with eigenvalue $\calL$.
        \end{theorem}
            \begin{proof}
                Our strategy for this proof is to explicitly construct - for each $\calL \in \Shv_{\underline{\bar{\Q}_{\ell}}}^{\ad, 1}(X)$ - the corresponding Hecke eigensheaf $\Autom(\calL)$, and this will involve three steps:
                    \begin{enumerate}
                        \item \textbf{(A \say{Hecke eigensheaf} on $X^{(d)}$):} In this step we shall construct an $\ell$-local system on $X^{(d)}$ (hence on $\bigcup_{d \geq 2g - 1} X^{(d)}$); cf. proposition \ref{prop: the_abel_jacobi_map_is_a_smooth_projective_fibration} satisfying an analogue of the Hecke eigensheaf property (cf. definition \ref{def: hecke_eigensheaves}) with the purpose in mind being that by pushing this local system forward using the Abel-Jacobi map, one shall obtain a legitimate Hecke eigensheaf on $\Bun_{\GL_1}^{(d)}(X)$ (hence on $\bigcup_{d \geq 2g - 1} \Bun_{\GL_1}^{(d)}(X)$).
                        
                        Write $\sigma^{(d)}: X^d \to X^{(d)}$ for the canonical quotient map and set $\Delta_X^{(d)} := \sigma^{(d)} \circ \Delta_X^d$. Next, for each $\ell$-adic local system $\calL \in \Shv_{\underline{\bar{\Q}_{\ell}}}(X)$, we can construct an $\ell$-adic local system $\calL^{(d)} \in \Shv_{\underline{\bar{\Q}_{\ell}}}(X^{(d)})$ given by:
                            $$\calL^{(d)} \cong ((\Delta_X^{(d)})_*\calL)^{\Sigma_d}$$
                        The first observation that one can make is that there is an isomorphism $(\sigma^{(d)})^*\calL^{(d)} \cong \calL^{\boxtimes d}$ of $\ell$-adic local systems on $X^d$. Next, notice how there are commutative diagrams of the following form, wherein the maps $\tilde{h}^{(d)}$ are given by $([x], D) \mapsto [x] + D$ (cf. remark \ref{remark: adding_effective_divisors}):
                            $$
                                \begin{tikzcd}
                                	{X \x X^d} & {X^{(d + 1)}} \\
                                	{X \x X^{(d)}}
                                	\arrow["{\id_X \x \sigma^{(d)}}"', from=1-1, to=2-1]
                                	\arrow["{\tilde{h}^{(d)}}"', dashed, from=2-1, to=1-2]
                                	\arrow["{\sigma^{(d + 1)}}", from=1-1, to=1-2]
                                \end{tikzcd}
                            $$
                        We thus have, as follows, a $\Sigma_d$-equivariant analogue on $X \x X^{(d)}$ of the Hecke eigensheaf property for all $\calL \in \Shv_{\underline{\bar{\Q}_{\ell}}}^{\ad, 1}(X)$:
                            $$(\tilde{h}^{(d)})^* \calL^{(d + 1)} \cong \calL \boxtimes \calL^{(d)}$$
                        \item \textbf{(A Hecke eigensheaf on $\Bun_{\GL_1}^{(d)}(X)$):} Now, recall from corollary \ref{coro: galois_representations_induced_by_the_abel_jacobi_map} that for every $d \geq 2g - 1$ and example \ref{example: gluing_constructible_sheaves}, there exists an adjoint equivalence:
                            $$
                                \begin{tikzcd}
                                	{\Shv_{\underline{\bar{\Q}_{\ell}}}^{\ad, 1}(X^{(d)})} & {\Shv_{\underline{\bar{\Q}_{\ell}}}^{\ad, 1}(\Bun_{\GL_1}^{(d)}(X))}
                                	\arrow[""{name=0, anchor=center, inner sep=0}, "{(\alpha_{X/k}^{(d)})_*}"', bend right, from=1-1, to=1-2]
                                	\arrow[""{name=1, anchor=center, inner sep=0}, "{(\alpha_{X/k}^{(d)})^*}"', bend right, from=1-2, to=1-1]
                                	\arrow["\dashv"{anchor=center, rotate=-90}, draw=none, from=1, to=0]
                                \end{tikzcd}
                            $$ 
                        From this, one infers that every $\ell$-adic local system $\calF \in \Shv_{\underline{\bar{\Q}_{\ell}}}^{\ad, 1}(X^{(d)})$ has the form $(\alpha_{X/k}^{(d)})^* \E$ for some other $\ell$-adic local system $\E \in \Shv_{\underline{\bar{\Q}_{\ell}}}^{\ad, 1}(\Bun_{\GL_1}^{(d)}(X))$, meaning that there exists $\ell$-adic local systems $\E_{\calL^{(d)}}, \E_{\calL^{(d + 1)}} \in \Shv_{\underline{\bar{\Q}_{\ell}}}^{\ad, 1}(\Bun_{\GL_1}^{(d)}(X))$ corresponding to $\calL^{(d)}, \calL^{(d + 1)} \in \Shv_{\underline{\bar{\Q}_{\ell}}}^{\ad, 1}(X^{(d)})$ respectively (and ultimately, to $\calL \in \Shv_{\underline{\bar{\Q}_{\ell}}}^{\ad, 1}(X)$) satisfying the following equation in $\Shv_{\underline{\bar{\Q}_{\ell}}}^{\ad, 1}(X \x X^{(d)})$:
                            $$(\tilde{h}^{(d)})^* (\alpha_{X/k}^{(d + 1)})^* \E_{\calL^{(d + 1)}} \cong \calL \boxtimes (\alpha_{X/k}^{(d)})^*\E_{\calL^{(d)}}$$
                        Now, for each $d \geq 2g - 1$, let us write $\cev{h}t^{(d)}: X \x \Bun_{\GL_1}^{(d)}(X) \to \Bun_{\GL_1}^{(d + 1)}(X)$ for the map given by $\cev{h}^{(d)}([x], \E) \cong \calO_X([x]) \tensor_{\calO_X} \E$. Next, consider the following commutative diagram:
                            $$
                                \begin{tikzcd}
                                	{X \x X^{(d)}} & {X^{(d + 1)}} \\
                                	{X \x \Bun_{\GL_1}^{(d)}(X)} & {\Bun_{\GL_1}^{(d + 1)}(X)}
                                	\arrow["{\cev{h}^{(d)}}", from=2-1, to=2-2]
                                	\arrow["{\id_X \x \alpha_{X/k}^{(d)}}"', from=1-1, to=2-1]
                                	\arrow["{\alpha^{(d + 1)}}", from=1-2, to=2-2]
                                	\arrow["{\tilde{h}^{(d)}}", from=1-1, to=1-2]
                                \end{tikzcd}
                            $$
                        which induces the following equations in $\Shv_{\underline{\bar{\Q}_{\ell}}}^{\ad, 1}(X \x X^{(d)})$ for all $\calL \in \Shv_{\underline{\bar{\Q}_{\ell}}}^{\ad, 1}(X)$:
                            $$
                                \begin{aligned}
                                    (\id_X \x \alpha_{X/k}^{(d)})^* (\cev{h}^{(d)})^* \E_{\calL^{(d + 1)}} & \cong (\tilde{h}^{(d)})^* (\alpha^{(d + 1)})^* \E_{\calL^{(d + 1)}}
                                    \\
                                    & \cong \calL \boxtimes (\alpha_{X/k}^{(d)})^*\E_{\calL^{(d)}}
                                    \\
                                    & \cong (\id_X \x \alpha_{X/k}^{(d)})^*(\calL \boxtimes \E_{\calL^{(d)}})
                                \end{aligned}
                            $$
                        and since $(\id_X \x \alpha_{X/k}^{(d)})^*$ is an invertible functor (cf. corollary \ref{coro: galois_representations_induced_by_the_abel_jacobi_map}), we have, furthermore, the following equation in $\Shv_{\underline{\bar{\Q}_{\ell}}}^{\ad, 1}(X \x \Bun_{\GL_1}^{(d)}(X))$, which is precisely the Hecke eigensheaf property from definition \ref{def: hecke_eigensheaves}:
                            $$(\cev{h}^{(d)})^* \E_{\calL^{(d + 1)}} \cong \calL \boxtimes \E_{\calL^{(d)}}$$
                        We have thus obtained a unique Hecke eigensheaf on $\bigcup_{d \geq 2g - 1} \Bun_{\GL_1}^{(d)}(X)$ from an arbitrary $\ell$-adic local system $\calL \in \Shv_{\underline{\bar{\Q}_{\ell}}}^{\ad, 1}(X)$.
                        \item \textbf{(A Hecke eigensheaf on $\Bun_{\GL_1}(X)$):} Finally, in order extend the Hecke eigensheaf that we have constructed on $\bigcup_{d \geq 2g - 1} \Bun_{\GL_1}^{(d)}(X)$ to $\Bun_{\GL_1}(X)$ (i.e. to degrees $d < 2g - 1$), simply apply lemma \ref{lemma: hecke_eigensheaves_extend_to_lower_degrees}.
                    \end{enumerate}
            \end{proof}
        
        The keen reader might have noticed that the proof of theorem \ref{theorem: unramified_abelian_geometric_class_field_theory} did not make use of the fact that the Abel-Jacobi map is surjective (cf. proposition \ref{prop: the_abel_jacobi_map_is_a_smooth_projective_fibration}). This, however, was to be expected, since the crucial technicality that was utilised in that proof was corollary \ref{coro: galois_representations_induced_by_the_abel_jacobi_map}, which is ultimately a direct consequence of proposition \ref{prop: etale_homotopy_exact_sequence}, in which one considers \'etale fibrations (which, by definition, need not be surjective). However, in order to establish a version of geometric class field theory that is also capable of describing ramifications, one will need to rely crucially on the fact that the Abel-Jacobi map is surjective\footnote{One might heuristically think of how primes that ramify split into many primes in the extension field by definition.}.
        
        \begin{convention}[Line bundles with prescribed trivialisations] \label{conv: line_bundles_with_prescribed_trivialisations}
            If $D$ is an effective Weil divisor on $X$ then we shall denote by $\Bun_{\GL_1}(X \setminus D)$ the moduli space of line bundles $\calL \in \Bun_{\GL_1}(X)$ with a trivialisation $\calL|_D \cong \calO_X|_D$.
        \end{convention}
        An interesting feature of ramified geometric class field theory is that the zero-characteristic and the positive-characteristic cases behave somewhat differently.
        \begin{theorem}[Ramified abelian geometric class field theory] \label{theorem: ramified_abelian_geometric_class_field_theory}
            Throughout, we work with an effective Weil divisors $D \in \Div_{X/k}^{\eff}$.
                \begin{enumerate}
                    \item \textbf{(Characteristic $0$):} 
                    \item \textbf{(Characteristics $p > 0$):}
                \end{enumerate}
        \end{theorem}
            \begin{proof}
                \noindent
                \begin{enumerate}
                    \item \textbf{(Characteristic $0$):} 
                    \item \textbf{(Characteristics $p > 0$):}
                \end{enumerate}
            \end{proof}
        
        \begin{corollary}[Geometric Langlands for $\GL_1$] \label{coro: geometric_langlands_for_GL1}
            By putting theorem \ref{theorem: unramified_representations_are_sheaves_on_X} (respectively theorem \ref{theorem: ramified_abelian_geometric_class_field_theory}) and theorem \ref{theorem: unramified_abelian_geometric_class_field_theory} together, one gets a canonical equivalence of categories as follows:
                $$\Rep_{\bar{\Q}_{\ell}}^1(\pi_1^{\ab}((X \setminus D)_{\fet}))^{\cont} \cong \Eig_{\underline{\bar{\Q}_{\ell}}}^1(\Bun_{\GL_1}(X \setminus D))$$
                $$\chi \mapsto \Autom(\chi)$$
            wherein $D$ is a Weil divisor on $X$ (the unramified version corresponds to taking $D \cong \varnothing$). What this essentially tells us is that $1$-dimensional continuous $\ell$-adic Galois representations are the same as automorphic forms associated to $\GL_1$, and as such, the combination of theorem \ref{theorem: unramified_representations_are_sheaves_on_X} and theorem \ref{theorem: unramified_abelian_geometric_class_field_theory} can be understood as a geometrisation of the Global Langlands Correspondence for the (connnected reductive) group $\GL_1$\footnote{Incidentally, this is why it is commonly asserted that the Langlands Correspondence for $\GL_1$ \say{is just class field theory}.}. 
        \end{corollary}
        \begin{remark}[Comparison with the classical formulation] \label{remark: geometric_vs_classical_class_field_theory}
            One interesting feature of the ramified version of geometric class field theory is that due to $D$ being a finite formal linear combination of closed points of $X$ (with multiplicities), the resulting equivalence of categories can be interpreted to hold only \textit{almost everywhere} on $X$. This is, however, to be expected. To understand why, let us first recall 
        \end{remark}
        
    \subsection{Grothendieck's Sheaf-Function Dictionary and \textit{die Gr\"o{\ss}encharaktere}}
        As a final step, let us decategorify the left-hand side of the equivalence:
            $$\Rep_{\bar{\Q}_{\ell}}^1(\pi_1^{\ab}((X \setminus D)_{\fet}))^{\cont} \cong \Eig_{\underline{\bar{\Q}_{\ell}}}^1(\Bun_{\GL_1}(X \setminus D))$$
        from corollary \ref{coro: geometric_langlands_for_GL1} to obtain a correspondence between continuous $\ell$-adic characters of $\pi_1^{\ab}(X_{\fet})$ and \say{\textit{die Gr\"o{\ss}encharaktere}}, using Grothendieck's Sheaf-Function Dictionary.
        
        We begin with the notion of character sheaves.
        \begin{definition}[Character sheaves] \label{def: character_sheaves}
            Let $G$ be a connected and separated abelian algebraic group a perfect field $k$ of characteristic $p > 0$, whose group structure is given by $m: G \x G \to G$. Then, an \textbf{$\ell$-adic character sheaf} of $G$ shall be an $\ell$-adic local system $\calL \in \Shv_{\underline{\bar{\Q}_{\ell}}}^{\ad, 1}(G)$ such that $m^*\calL \cong \calL \boxtimes \calL$. 
        \end{definition}
        \begin{lemma}[Character sheaves of group schemes and group characters] \label{lemma: character_sheav es_of_group_schemes_and_group_characters}
            Let $G$ be a connected and separated abelian algebraic group a perfect field $k$ of characteristic $p > 0$. Then, there is an equivalence of categories between the category of character sheaves of $G$ and that of $\ell$-adic characters of $G(k)$:
                $$\Char\Shv_{\underline{\bar{\Q}_{\ell}}}(G) \cong \Rep^1_{\bar{\Q}_{\ell}}(G(k))$$
        \end{lemma}
            \begin{proof}
                
            \end{proof}