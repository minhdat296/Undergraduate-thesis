\documentclass[a4paper, 11pt]{article}

\usepackage{amsfonts, amssymb, amsmath, amsthm}

\usepackage{pdfrender, xcolor}
\pdfrender{StrokeColor=black,LineWidth=.4pt,TextRenderingMode=2}

%\usepackage{minitoc}
%\setcounter{section}{-1}
%\setcounter{tocdepth}{}
%\setcounter{minitocdepth}{}
%\setcounter{secnumdepth}{}

\usepackage{graphicx}

\usepackage[english]{babel}
\usepackage[utf8]{inputenc}
%\usepackage{mathpazo}
%\usepackage{euler}
\usepackage{eucal}
\usepackage{bbm}
\usepackage{bm}
\usepackage{csquotes}
\usepackage[nottoc]{tocbibind}
\usepackage{appendix}
\usepackage{float}
\usepackage[T1]{fontenc}
\usepackage[
    left = \flqq{},% 
    right = \frqq{},% 
    leftsub = \flq{},% 
    rightsub = \frq{} %
]{dirtytalk}

\usepackage{imakeidx}
\makeindex

%\usepackage[dvipsnames]{xcolor}
\usepackage{hyperref}
    \hypersetup{
        colorlinks=true,
        linkcolor=teal,
        filecolor=pink,      
        urlcolor=teal,
        citecolor=magenta
    }
\usepackage{comment}

% You would set the PDF title, author, etc. with package options or
% \hypersetup.

\usepackage[backend=biber, style=alphabetic, sorting=nty]{biblatex}
    \addbibresource{bibliography.bib}

\raggedbottom

\usepackage{mathrsfs}
\usepackage{mathtools} 
\mathtoolsset{showonlyrefs} 
%\usepackage{amssymb}
%\usepackage{amsthm}
\renewcommand\qedsymbol{$\blacksquare$}
\usepackage{tikz-cd}
\usepackage{tikz}
\usepackage{setspace}
\usepackage[version=3]{mhchem}
\parskip=0.1in
\usepackage[margin=17.5mm]{geometry}

\usepackage{listings, lstautogobble}
\lstset{
	language=matlab,
	basicstyle=\scriptsize\ttfamily,
	commentstyle=\ttfamily\itshape\color{gray},
	stringstyle=\ttfamily,
	showstringspaces=false,
	breaklines=true,
	frameround=ffff,
	frame=single,
	rulecolor=\color{black},
	autogobble=true
}

\usepackage{todonotes,tocloft,xpatch,hyperref}

% This is based on classicthesis chapter definition
\let\oldsec=\section
\renewcommand*{\section}{\secdef{\Sec}{\SecS}}
\newcommand\SecS[1]{\oldsec*{#1}}%
\newcommand\Sec[2][]{\oldsec[\texorpdfstring{#1}{#1}]{#2}}%

\newcounter{istodo}[section]

% http://tex.stackexchange.com/a/61267/11984
\makeatletter
%\xapptocmd{\Sec}{\addtocontents{tdo}{\protect\todoline{\thesection}{#1}{}}}{}{}
\newcommand{\todoline}[1]{\@ifnextchar\Endoftdo{}{\@todoline{#1}}}
\newcommand{\@todoline}[3]{%
	\@ifnextchar\todoline{}
	{\contentsline{section}{\numberline{#1}#2}{#3}{}{}}%
}
\let\l@todo\l@subsection
\newcommand{\Endoftdo}{}

\AtEndDocument{\addtocontents{tdo}{\string\Endoftdo}}
\makeatother

\usepackage{lipsum}

%   Reduce the margin of the summary:
\def\changemargin#1#2{\list{}{\rightmargin#2\leftmargin#1}\item[]}
\let\endchangemargin=\endlist 

%   Generate the environment for the abstract:
%\newcommand\summaryname{Abstract}
%\newenvironment{abstract}%
    %{\small\begin{center}%
    %\bfseries{\summaryname} \end{center}}

\newtheorem{theorem}{Theorem}[section]
    \numberwithin{theorem}{subsection}
\newtheorem{proposition}{Proposition}[section]
    \numberwithin{proposition}{subsection}
\newtheorem{lemma}{Lemma}[section]
    \numberwithin{lemma}{subsection}
\newtheorem{claim}{Claim}[section]
    \numberwithin{claim}{subsection}

\theoremstyle{definition}
    \newtheorem{definition}{Definition}[section]
        \numberwithin{definition}{subsection}

\theoremstyle{remark}
    \newtheorem{remark}{Remark}[section]
        \numberwithin{remark}{subsection}
    \newtheorem{example}{Example}[section]
        \numberwithin{example}{subsection}    
    \newtheorem{convention}{Convention}[section]
        \numberwithin{convention}{subsection}
    \newtheorem{corollary}{Corollary}[section]
        \numberwithin{corollary}{subsection}
\setcounter{section}{-1}

\renewcommand{\cong}{\simeq}
\newcommand{\ladjoint}{\dashv}
\newcommand{\radjoint}{\vdash}
\newcommand{\<}{\langle}
\renewcommand{\>}{\rangle}
\newcommand{\ndiv}{\hspace{-2pt}\not|\hspace{5pt}}
\newcommand{\cond}{\blacktriangle}
\newcommand{\solid}{\blacksquare}
\newcommand{\ot}{\leftarrow}
\renewcommand{\-}{\text{-}}
\renewcommand{\mapsto}{\leadsto}
\renewcommand{\leq}{\leqslant}
\renewcommand{\geq}{\geqslant}
\renewcommand{\setminus}{\smallsetminus}
\makeatletter
\DeclareRobustCommand{\cev}[1]{%
  {\mathpalette\do@cev{#1}}%
}
\newcommand{\do@cev}[2]{%
  \vbox{\offinterlineskip
    \sbox\z@{$\m@th#1 x$}%
    \ialign{##\cr
      \hidewidth\reflectbox{$\m@th#1\vec{}\mkern4mu$}\hidewidth\cr
      \noalign{\kern-\ht\z@}
      $\m@th#1#2$\cr
    }%
  }%
}
\makeatother

\newcommand{\N}{\mathbb{N}}
\newcommand{\Z}{\mathbb{Z}}
\newcommand{\Q}{\mathbb{Q}}
\newcommand{\R}{\mathbb{R}}
\newcommand{\bbC}{\mathbb{C}}
\NewDocumentCommand{\x}{e{_^}}{%
  \mathbin{\mathop{\times}\displaylimits
    \IfValueT{#1}{_{#1}}
    \IfValueT{#2}{^{#2}}
  }%
}
\NewDocumentCommand{\pushout}{e{_^}}{%
  \mathbin{\mathop{\sqcup}\displaylimits
    \IfValueT{#1}{_{#1}}
    \IfValueT{#2}{^{#2}}
  }%
}
\newcommand{\supp}{\operatorname{supp}}
\newcommand{\im}{\operatorname{im}}
\newcommand{\coker}{\operatorname{coker}}
\newcommand{\id}{\mathrm{id}}
\newcommand{\chara}{\operatorname{char}}
\newcommand{\trdeg}{\operatorname{trdeg}}
\newcommand{\rank}{\operatorname{rank}}
\newcommand{\trace}{\operatorname{tr}}
\newcommand{\length}{\operatorname{length}}
\newcommand{\height}{\operatorname{height}}
\renewcommand{\span}{\operatorname{span}}
\newcommand{\e}{\epsilon}
\newcommand{\p}{\mathfrak{p}}
\newcommand{\q}{\mathfrak{q}}
\newcommand{\m}{\mathfrak{m}}
\newcommand{\n}{\mathfrak{n}}
\newcommand{\calF}{\mathcal{F}}
\newcommand{\calG}{\mathcal{G}}
\newcommand{\calO}{\mathcal{O}}
\newcommand{\F}{\mathbb{F}}
\DeclareMathOperator{\lcm}{lcm}
\newcommand{\gr}{\operatorname{gr}}
\newcommand{\vol}{\mathrm{vol}}

\newcommand{\GL}{\operatorname{GL}}
\newcommand{\SL}{\operatorname{SL}}
\newcommand{\Sp}{\operatorname{Sp}}
\newcommand{\GSp}{\operatorname{GSp}}
\newcommand{\GSpin}{\operatorname{GSpin}}
\newcommand{\opO}{\operatorname{O}}
\newcommand{\SO}{\operatorname{SO}}
\newcommand{\SU}{\operatorname{SU}}
\newcommand{\opU}{\operatorname{U}}
\newcommand{\Spec}{\mathrm{Spec}}
\newcommand{\Spf}{\mathrm{Spf}}
\newcommand{\Spm}{\mathrm{Spm}}
\newcommand{\Spv}{\mathrm{Spv}}
\newcommand{\Spa}{\mathrm{Spa}}
\newcommand{\Spd}{\mathrm{Spd}}
\newcommand{\Proj}{\mathrm{Proj}}
\newcommand{\Gr}{\mathrm{Gr}}
\newcommand{\Hecke}{\mathrm{Hecke}}
\newcommand{\Sht}{\mathrm{Sht}}
\newcommand{\Quot}{\mathrm{Quot}}
\newcommand{\Hilb}{\mathrm{Hilb}}
\newcommand{\Pic}{\mathrm{Pic}}
\newcommand{\Div}{\mathrm{Div}}
\newcommand{\Jac}{\mathrm{Jac}}
\newcommand{\Alb}{\mathrm{Alb}} %albanese variety
\newcommand{\Bun}{\mathrm{Bun}}
\newcommand{\loopspace}{\mathbf{\Omega}}
\newcommand{\suspension}{\mathbf{\Sigma}}
\newcommand{\tangent}{\mathrm{T}} %tangent space
\newcommand{\Eig}{\mathrm{Eig}}

\newcommand{\Ring}{\mathrm{Ring}}
\newcommand{\Cring}{\mathrm{CRing}}
\newcommand{\Alg}{\mathrm{Alg}}
\newcommand{\Leib}{\mathrm{Leib}} %leibniz algebras
\newcommand{\Fld}{\mathrm{Fld}}
\newcommand{\Sets}{\mathrm{Sets}}
\newcommand{\Cat}{\mathrm{Cat}}
\newcommand{\Grp}{\mathrm{Grp}}
\newcommand{\Ab}{\mathrm{Ab}}
\newcommand{\Sch}{\mathrm{Sch}}
\newcommand{\Coh}{\mathrm{Coh}}
\newcommand{\QCoh}{\mathrm{QCoh}}
\newcommand{\Desc}{\mathrm{Desc}}
\newcommand{\Sh}{\mathrm{Sh}}
\newcommand{\Psh}{\mathrm{PSh}}
\newcommand{\Fib}{\mathrm{Fib}}
\renewcommand{\mod}{\-\mathrm{mod}}
\newcommand{\bimod}{\-\mathrm{bimod}}
\newcommand{\Vect}{\mathrm{Vect}}
\newcommand{\Rep}{\mathrm{Rep}}
\newcommand{\Grpd}{\mathrm{Grpd}}
\newcommand{\Arr}{\mathrm{Arr}}
\newcommand{\Esp}{\mathrm{Esp}}
\newcommand{\Ob}{\mathrm{Ob}}
\newcommand{\Mor}{\mathrm{Mor}}
\newcommand{\Mfd}{\mathrm{Mfd}}
%\newcommand{\LR}{\mathrm{LR}}
%\newcommand{\RSpc}{\mathrm{RSpc}}
\newcommand{\Spc}{\mathrm{Spc}}
\newcommand{\Top}{\mathrm{Top}}
\newcommand{\Topos}{\mathrm{Topos}}
\newcommand{\Nil}{\mathfrak{Nil}}
\newcommand{\J}{\mathfrak{J}}
\newcommand{\Stk}{\mathrm{Stk}}
\newcommand{\Pre}{\mathrm{Pre}}
\newcommand{\simp}{\mathbf{\Delta}}
\newcommand{\Ind}{\mathrm{Ind}}
\newcommand{\Pro}{\mathrm{Pro}}
\newcommand{\Mon}{\mathrm{Mon}}
\newcommand{\Comm}{\mathrm{Comm}}
\newcommand{\Fin}{\mathrm{Fin}}
\newcommand{\Assoc}{\mathrm{Assoc}}
\newcommand{\Co}{\mathrm{Co}}
\newcommand{\Comp}{\mathrm{Comp}} %compact hausdorff spaces
\newcommand{\Stone}{\mathrm{Stone}} %stone spaces
\newcommand{\sfExt}{\mathrm{Ext}} %extremely disconnected spaces
\newcommand{\Ouv}{\mathrm{Ouv}}
\newcommand{\Str}{\mathrm{Str}}
\newcommand{\Func}{\mathrm{Func}}
\newcommand{\Crys}{\mathrm{Crys}}
\newcommand{\LocSys}{\mathrm{LocSys}}
\newcommand{\Sieves}{\mathrm{Sieves}}
\newcommand{\pt}{\mathrm{pt}}
\newcommand{\Graphs}{\mathrm{Graphs}}
\newcommand{\Lie}{\mathrm{Lie}}
\newcommand{\Env}{\mathrm{Env}}
\newcommand{\Ho}{\mathrm{Ho}}
\newcommand{\rmD}{\mathrm{D}}
\newcommand{\Cov}{\mathrm{Cov}}
\newcommand{\Frames}{\mathrm{Frames}}
\newcommand{\Locales}{\mathrm{Locales}}
\newcommand{\Span}{\mathrm{Span}}
\newcommand{\Corr}{\mathrm{Corr}}
\newcommand{\Monad}{\mathrm{Monad}}
\newcommand{\Var}{\mathrm{Var}}
\newcommand{\sfN}{\mathrm{N}} %nerve
\newcommand{\Dia}{\mathrm{Dia}}
\newcommand{\co}{\mathrm{co}}
\newcommand{\ev}{\mathrm{ev}}
\newcommand{\bi}{\mathrm{bi}}
\newcommand{\Nat}{\mathrm{Nat}}
\newcommand{\Hopf}{\mathrm{Hopf}}
\newcommand{\Dmod}{\mathrm{D}\mod}
\newcommand{\Perv}{\mathrm{Perv}}
\newcommand{\Sph}{\mathrm{Sph}}
\newcommand{\Moduli}{\mathrm{Moduli}}
\newcommand{\Pseudo}{\mathrm{Pseudo}}
\newcommand{\Lax}{\mathrm{Lax}}
\newcommand{\Strict}{\mathrm{Strict}}
\newcommand{\Opd}{\mathrm{Opd}} %operads
\newcommand{\Shv}{\mathrm{Shv}}
\newcommand{\Char}{\mathrm{Char}} %CharShv = character sheaves
\newcommand{\Huber}{\mathrm{Huber}}
\newcommand{\Tate}{\mathrm{Tate}}
\newcommand{\Ad}{\mathrm{Ad}} %adic spaces
\newcommand{\Perfd}{\mathrm{Perfd}} %perfectoid spaces
\newcommand{\Sub}{\mathrm{Sub}} %subobjects
\newcommand{\Ideals}{\mathrm{Ideals}}
\newcommand{\Isoc}{\mathrm{Isoc}}
\newcommand{\Ban}{\-\mathrm{Ban}} %Banach spaces
\newcommand{\Fre}{\-\mathrm{Fre}} %Frechet spaces
\newcommand{\Ch}{\mathrm{Ch}} %chain complexes
\newcommand{\Mot}{\mathrm{Mot}} %motives
\newcommand{\KL}{\mathrm{KL}} %category of Kazhdan-Lusztig modules
\newcommand{\Pres}{\mathrm{Pres}} %presentable categories

\newcommand{\Aut}{\mathrm{Aut}}
\newcommand{\Inn}{\mathrm{Inn}}
\newcommand{\Out}{\mathrm{Out}}
\newcommand{\frakgl}{\mathfrak{gl}}
\newcommand{\der}{\mathfrak{der}} %derivations on Lie algebras
\newcommand{\inn}{\mathfrak{inn}} %inner derivations
\newcommand{\out}{\mathfrak{out}} %outer derivations
\newcommand{\Stab}{\mathrm{Stab}}
\newcommand{\Cent}{\mathrm{Cent}}
\newcommand{\Conj}{\mathrm{Conj}}
\newcommand{\Gal}{\mathrm{Gal}}
\newcommand{\bfG}{\mathbf{G}} %absolute Galois group
\newcommand{\Frac}{\mathrm{Frac}}
\newcommand{\Ann}{\mathrm{Ann}}
\newcommand{\Val}{\mathrm{Val}}
\newcommand{\Chow}{\mathrm{Chow}}
\newcommand{\Sym}{\mathrm{Sym}}
\newcommand{\End}{\mathrm{End}}
\newcommand{\Mat}{\mathrm{Mat}}
\newcommand{\Diff}{\mathrm{Diff}}
\newcommand{\Autom}{\mathrm{Autom}}

\newcommand{\colim}{\operatorname{colim} \:}
\renewcommand{\lim}{\operatorname{lim} \:}
\newcommand{\toto}{\rightrightarrows}
%\newcommand{\tensor}{\otimes}
\NewDocumentCommand{\tensor}{e{_^}}{%
  \mathbin{\mathop{\otimes}\displaylimits
    \IfValueT{#1}{_{#1}}
    \IfValueT{#2}{^{#2}}
  }%
}
\newcommand{\eq}{\operatorname{eq}}
\newcommand{\coeq}{\operatorname{coeq}}
\newcommand{\Hom}{\mathrm{Hom}}
\newcommand{\Maps}{\mathrm{Maps}}
\newcommand{\Tor}{\mathrm{Tor}}
\newcommand{\Ext}{\mathrm{Ext}}
\newcommand{\Isom}{\mathrm{Isom}}
\newcommand{\stalk}{\mathbf{stalk}}
\newcommand{\RKE}{\operatorname{RKE}}
\newcommand{\LKE}{\operatorname{LKE}}
\newcommand{\oblv}{\mathbf{oblv}}
\newcommand{\const}{\mathbf{const}}
%\newcommand{\forget}{\mathbf{forget}}
\newcommand{\adrep}{\mathbf{ad}} %adjoint representation
\newcommand{\NL}{\mathbf{NL}} %naive cotangent complex
\newcommand{\bfL}{\mathbf{L}} %cotangent complex
\newcommand{\pr}{\operatorname{pr}}
\newcommand{\Der}{\mathbf{Der}}
\newcommand{\Frob}{\mathrm{Frob}} %Frobenius
\newcommand{\frob}{\mathrm{frob}} %trace of Frobenius
\newcommand{\bfpt}{\mathbf{pt}}
\newcommand{\bfloc}{\mathbf{loc}}
\newcommand{\1}{\mathbbm{1}}
\newcommand{\2}{\mathbbm{2}}
\newcommand{\Jet}{\mathbf{Jet}}
\newcommand{\Split}{\mathbf{Split}}
\newcommand{\Sq}{\mathbf{Sq}}
\newcommand{\Zero}{\mathbf{Z}}
\newcommand{\SqZ}{\Sq\Zero}
\newcommand{\frakLie}{\mathfrak{Lie}}
\newcommand{\y}{\mathbf{y}} %yoneda
\newcommand{\Sm}{\mathrm{Sm}}
\newcommand{\AJ}{\mathrm{AJ}} %abel-jacobi map
\newcommand{\act}{\mathrm{act}}
\newcommand{\ram}{\mathrm{ram}} %ramification index
\newcommand{\inv}{\mathrm{inv}}

\newcommand{\bbU}{\mathbb{U}}
\newcommand{\V}{\mathbb{V}}
\newcommand{\U}{\mathrm{U}}
\newcommand{\rmI}{\mathrm{I}} %augmentation ideal
\newcommand{\bfV}{\mathbf{V}}
\newcommand{\C}{\mathcal{C}}
\newcommand{\D}{\mathcal{D}}
\newcommand{\T}{\mathbf{T}} %Tate modules
\newcommand{\calM}{\mathcal{M}}
\newcommand{\calN}{\mathcal{N}}
\newcommand{\calP}{\mathcal{P}}
\newcommand{\calQ}{\mathcal{Q}}
\newcommand{\A}{\mathbb{A}}
\renewcommand{\P}{\mathbb{P}}
\newcommand{\calL}{\mathcal{L}}
\newcommand{\E}{\mathcal{E}}
\renewcommand{\H}{\mathbf{H}}
\newcommand{\calX}{\mathcal{X}}
\newcommand{\calY}{\mathcal{Y}}
\newcommand{\calZ}{\mathcal{Z}}
\newcommand{\scrX}{\mathscr{X}}
\newcommand{\scrY}{\mathscr{Y}}
\newcommand{\scrZ}{\mathscr{Z}}
\newcommand{\calA}{\mathcal{A}}
\newcommand{\calB}{\mathcal{B}}
\newcommand{\sfT}{\mathrm{T}}
\renewcommand{\S}{\mathcal{S}}
\newcommand{\B}{\mathbb{B}}
\newcommand{\bbD}{\mathbb{D}}
\newcommand{\G}{\mathbb{G}}
\newcommand{\horn}{\mathbf{\Lambda}}
\renewcommand{\L}{\mathbb{L}}
\renewcommand{\a}{\mathfrak{a}}
\renewcommand{\b}{\mathfrak{b}}
\renewcommand{\t}{\mathfrak{t}}
\renewcommand{\r}{\mathfrak{r}}
\newcommand{\bbX}{\mathbb{X}}
\newcommand{\g}{\mathfrak{g}}
\newcommand{\h}{\mathfrak{h}}
\renewcommand{\k}{\mathfrak{k}}
\newcommand{\del}{\partial}
\newcommand{\bbE}{\mathbb{E}}
\newcommand{\scrO}{\mathscr{O}}
\newcommand{\bbO}{\mathbb{O}}
\newcommand{\scrA}{\mathscr{A}}
\newcommand{\scrB}{\mathscr{B}}
\newcommand{\scrF}{\mathscr{F}}
\newcommand{\scrG}{\mathscr{G}}
\newcommand{\scrM}{\mathscr{M}}
\newcommand{\scrN}{\mathscr{N}}
\newcommand{\scrP}{\mathscr{P}}
\newcommand{\frakS}{\mathfrak{S}}
\newcommand{\calI}{\mathcal{I}}
\newcommand{\calJ}{\mathcal{J}}
\newcommand{\calK}{\mathcal{K}}
\newcommand{\scrV}{\mathscr{V}}
\newcommand{\bbS}{\mathbb{S}}
\newcommand{\scrH}{\mathscr{H}}
\newcommand{\bfB}{\mathbf{B}}
\newcommand{\W}{\mathbf{W}}
%\newcommand{\bfA}{\mathbf{A}}
\renewcommand{\O}{\mathbb{O}}
\newcommand{\calV}{\mathcal{V}}
\newcommand{\scrR}{\mathscr{R}} %radical
\newcommand{\rmZ}{\mathrm{Z}} %centre of algebra
\newcommand{\bfGamma}{\mathbf{\Gamma}}
\newcommand{\scrU}{\mathscr{U}}
\newcommand{\rmW}{\mathrm{W}} %Weil group

\newcommand{\aff}{\mathrm{aff}}
\newcommand{\ft}{\mathrm{ft}} %finite type
\newcommand{\fp}{\mathrm{fp}} %finite presentation
\newcommand{\aft}{\mathrm{aft}}
\newcommand{\lft}{\mathrm{lft}}
\newcommand{\laft}{\mathrm{laft}}
\newcommand{\cmpt}{\mathrm{cmpt}}
\newcommand{\qc}{\mathrm{qc}}
\newcommand{\qs}{\mathrm{qs}}
\newcommand{\lcmpt}{\mathrm{lcmpt}}
%\newcommand{\conv}{\mathrm{conv}}
\newcommand{\red}{\mathrm{red}}
\newcommand{\fin}{\mathrm{fin}}
\newcommand{\gen}{\mathrm{gen}}
\newcommand{\petit}{\mathrm{petit}}
\newcommand{\gros}{\mathrm{gros}}
\newcommand{\loc}{\mathrm{loc}}
\newcommand{\glob}{\mathrm{glob}}
\newcommand{\ringed}{\mathrm{ringed}}
\newcommand{\qcoh}{\mathrm{qcoh}}
\newcommand{\cl}{\mathrm{cl}}
\newcommand{\et}{\mathrm{\acute{e}t}}
\newcommand{\fet}{\mathrm{f\acute{e}t}}
\newcommand{\profet}{\mathrm{prof\acute{e}t}}
\newcommand{\proet}{\mathrm{pro\acute{e}t}}
\newcommand{\Zar}{\mathrm{Zar}}
\newcommand{\fppf}{\mathrm{fppf}}
\newcommand{\fpqc}{\mathrm{fpqc}}
\newcommand{\smooth}{\mathrm{sm}}
\newcommand{\sh}{\mathrm{sh}}
\newcommand{\op}{\mathrm{op}}
\newcommand{\open}{\mathrm{open}}
\newcommand{\closed}{\mathrm{closed}}
\newcommand{\geom}{\mathrm{geom}}
\newcommand{\alg}{\mathrm{alg}}
\newcommand{\sober}{\mathrm{sober}}
\newcommand{\dR}{\mathrm{dR}}
\newcommand{\rad}{\mathrm{rad}}
\newcommand{\discrete}{\mathrm{discrete}}
%\newcommand{\add}{\mathrm{add}}
%\newcommand{\lin}{\mathrm{lin}}
\newcommand{\Krull}{\mathrm{Krull}}
\newcommand{\qis}{\mathrm{qis}} %quasi-isomorphism
\newcommand{\ho}{\mathrm{ho}} %homotopy equivalence
\newcommand{\sep}{\mathrm{sep}}
\newcommand{\unr}{\mathrm{unr}}
\newcommand{\tame}{\mathrm{tame}}
\newcommand{\wild}{\mathrm{wild}}
\newcommand{\nil}{\mathrm{nil}}
\newcommand{\defm}{\mathrm{defm}}
\newcommand{\Art}{\mathrm{Art}}
\newcommand{\Noeth}{\mathrm{Noeth}}
\newcommand{\affd}{\mathrm{affd}}
%\newcommand{\adic}{\mathrm{adic}}
\newcommand{\pre}{\mathrm{pre}}
\newcommand{\perf}{\mathrm{perf}}
\newcommand{\perfd}{\mathrm{perfd}}
\newcommand{\rat}{\mathrm{rat}}
\newcommand{\cont}{\mathrm{cont}}
\newcommand{\dg}{\mathrm{dg}}
\newcommand{\almost}{\mathrm{a}}
%\newcommand{\stab}{\mathrm{stab}}
\newcommand{\heart}{\heartsuit}
\newcommand{\proj}{\mathrm{proj}}
\newcommand{\qproj}{\mathrm{qproj}}
\newcommand{\pd}{\mathrm{pd}}
\newcommand{\crys}{\mathrm{crys}}
\newcommand{\prisma}{\mathrm{prisma}}
\newcommand{\FF}{\mathrm{FF}}
\newcommand{\sph}{\mathrm{sph}}
\newcommand{\lax}{\mathrm{lax}}
\newcommand{\weak}{\mathrm{weak}}
\newcommand{\strict}{\mathrm{strict}}
\newcommand{\mon}{\mathrm{mon}}
\newcommand{\sym}{\mathrm{sym}}
\newcommand{\lisse}{\mathrm{lisse}}
\newcommand{\an}{\mathrm{an}}
\newcommand{\ad}{\mathrm{ad}}
\newcommand{\sch}{\mathrm{sch}}
\newcommand{\rig}{\mathrm{rig}}
\newcommand{\pol}{\mathrm{pol}}
\newcommand{\plat}{\mathrm{flat}}
\newcommand{\proper}{\mathrm{proper}}
\newcommand{\compl}{\mathrm{compl}}
\newcommand{\non}{\mathrm{non}}
\newcommand{\access}{\mathrm{access}}
\newcommand{\comp}{\mathrm{comp}}
\newcommand{\tstructure}{\mathrm{t}} %t-structures
\newcommand{\pure}{\mathrm{pure}} %pure motives
\newcommand{\mixed}{\mathrm{mixed}} %mixed motives
\newcommand{\num}{\mathrm{num}} %numerical motives
\newcommand{\ess}{\mathrm{ess}}
\newcommand{\topological}{\mathrm{top}}
\newcommand{\convex}{\mathrm{cv}}
\newcommand{\ab}{\mathrm{ab}} %abelian extensions
\newcommand{\surj}{\mathrm{surj}} %coverage on sets generated by surjections
\newcommand{\eff}{\mathrm{eff}} %effective Cartier divisors
\newcommand{\Weil}{\mathrm{Weil}} %weil divisors
\newcommand{\lex}{\mathrm{lex}}
\newcommand{\rex}{\mathrm{rex}}
\newcommand{\AR}{\mathrm{A\-R}}

%prism custom command
\usepackage{relsize}
\usepackage[bbgreekl]{mathbbol}
\usepackage{amsfonts}
\DeclareSymbolFontAlphabet{\mathbb}{AMSb} %to ensure that the meaning of \mathbb does not change
\DeclareSymbolFontAlphabet{\mathbbl}{bbold}
\newcommand{\prism}{{\mathlarger{\mathbbl{\Delta}}}}

\begin{document}

	\title{\textbf{MATH518: Thesis
	\\
	Geometric unramified abelian class field theory}}
	
	\author{Dat Minh Ha (UCID: 30067407)\\Supervisor: Jerrod Smith}
	\maketitle
	
	\begin{abstract}
	    The Langlands Programme is a network of many deep conjectures (and recently, some theorem!) with far-reaching consequences, notably in the realms of algebraic number theory and representation theory. At its core, it is about \textbf{reciprocity}, the idea that Galois groups should admit descriptions in terms of canonical constructions. As a matter of fact, the starting point of the Langlands Programme is what we nowadays call \textbf{class field theory}, the very topic of this thesis. More specifically, we are interested in what is known as \textbf{unramified abelian class field theory}, the simplest version of class field theory, which we shall approach via algebraic geometry. This is not the traditional approach to class field theory, but it will help us understand why the study of Galois groups naturally requires representation theory, which as a consequence, helps us makes sense of class field theory being the same as the simplest case of the Langlands Correspondence, that being for the group $\GL_1$.
	\end{abstract}
	
	{
      \hypersetup{} 
      %\dominitoc
      \tableofcontents %sort sections alphabetically
    }
    
    \section{Introduction}
    The paper will be organised into two main sections, detailing what we shall call the \textbf{Galois Side} and the \textbf{Automorphic Side} of geometric class field theory. This introductory section will be dedicated to the outlining of our approaches to these sections, as well as for laying down some conventions that we will be following until the end of the paper.

    \subsection{The Galois Side}
        The goal of this section is to define the \'etale fundamental group, as well as investigate its relationship with representations of absolute Galois groups. Our starting point is the following result, which explains why one might even suspect any sort of involvement of algebraic geometry in the first place:
        \begin{lemma}[Varieties and field extensions] \label{lemma: varieties_and_field_extensions}
            \cite[\href{https://stacks.math.columbia.edu/tag/0BXN}{Tag 0BXN}]{stacks} Let $k$ be a field. Then, $\trdeg K_X = \dim X$ for all varieties $X/k$, and there exists a canonical equivalence of categories as follows:
                $$\{\text{Finite-type field extensions $K/k$ and $k$-algebra homomorphisms}\}^{\op}$$
                $$\cong$$
                $$\{\text{Varieties $X/k$ and \href{https://stacks.math.columbia.edu/tag/01RI}{\underline{dominant}} \href{https://stacks.math.columbia.edu/tag/01RR}{\underline{rational}} maps}\}$$
        \end{lemma}
            \begin{proof}
                If $f: X \to Y$ is a dominant map of varieties and if $\eta_X$ and $\eta_Y$ are the unique generic points of $X$ and $Y$ (unique because varieties are integral by definition, and every integral scheme \textit{a priori} has a unique generic point), then $f(\eta_X) = \eta_Y$ per the definition of dominant morphisms. The residue field at generic points are precisely the function fields (consider the stalk of the structure sheaves at the generic points to see why this is the case), so we have obtained a map of function fields $K_Y \to K_X$. We leave the proof of finiteness up to the reader.
            \end{proof}
        Through lemma \ref{lemma: varieties_and_field_extensions}, one obtains the following regarding the relationship between curves (i.e. algebraic varieties of dimension $1$) and their function fields (which \textit{a priori} are of transcedence degree $1$ over the ground field) with little difficulty:
        \begin{proposition}[Curves and function fields] \label{prop: curves_and_function_fields}
            \cite[\href{https://stacks.math.columbia.edu/tag/0BY1}{Tag 0BY1}]{stacks} For any field $k$, one has the following canonical equivalences of categories:
                $$\{\text{Field extensions $K/k$ of transcendence degree $1$ and $k$-algebra homomorphisms}\}^{\op}$$
                $$\cong$$
                $$\{\text{Curves $X/k$ and dominant rational maps}\}$$
                $$\cong$$
                $$\{\text{Non-singular projective curves $X/k$ and dominant rational maps}\}$$
        \end{proposition}
            \begin{proof}
                The first equivalence is an obvious consequence of lemma \ref{lemma: varieties_and_field_extensions}. To show that the second equivalence holds, note firstly that there is an evident fully faithful functor from the third category to the second; we shall need to show that this functor is also essentially surjective. For this, simply recall that for each curve $X/k$, there exists a non-singular projective curve $\tilde{X}/k$ that is birational to $X/k$, namely $X^{\nu} \cup \{\infty_1, ..., \infty_n\}$, the normalisation $X^{\nu}/k$ of $X/k$ with finitely many extra points (recall also that any normal Noetherian scheme of dimension $\leq 1$ is \textit{a priori} non-singular; cf. \cite[\href{https://stacks.math.columbia.edu/tag/0BX2}{Tag 0BX2}]{stacks}).
            \end{proof}
            
        Through proposition \ref{prop: curves_and_function_fields}, we obtain the first crucial tool for the geometrisation of class field theory.
        \begin{corollary}[Galois covers of curves and Galois extensions] \label{coro: galois_covers_of_curves_and_galois_extensions}
            Let $k$ be a field. If $X$ is a connected non-singular projective curve over $\Spec k$ with function field $K$, then there is a canonical equivalence:
                $$({}^{K/}\Fld^{\fin, \Gal})^{\op} \cong (\Sch_{/X})_{\fet}^{\Gal}$$
            between the category of finite Galois extensions of $K$ and finite \'etale-Galois covers of $X$ (cf. remark \ref{remark: geometric_galois_correspondence}). 
        \end{corollary}
        The importance of corollary \ref{coro: galois_covers_of_curves_and_galois_extensions} can not be understated: what it tells us is essentially that the Galois group $\Gal(K^{\ab}/K)$ is nothing but the \'etale fundamental group of $(\Sch_{/X})_{\fet}^{\Gal}$ (more on this later, after we have introduced the \'etale fundamental group; cf. definition \ref{def: etale_fundamental_groups}). This, already, is one side of the Artin's Reciprocity Law, which we shall refer to as \say{\textbf{The Galois Side}} per popular conventions. Actually, this is a bit of a lie: instead of formulating geometric class field theory directly in terms of the \'etale fundamental group $\pi_1^{\ab}(X_{\fet})$ (or rather, its continuous $\ell$-adic characters), we will be phrasing things in terms of $\ell$-adic local systems of rank $1$ on $X$; the main result on the Galois Side shall be theorem \ref{theorem: unramified_representations_are_sheaves_on_X}, which rigorously and precisely establishes the categorification of continuous representations of $\pi_1^{\ab}(X_{\fet})$ to $\ell$-adic local systems on $X$ via a canonical equivalence of categories:
            $$\Rep^1_{\overline{\Q_{\ell}}}(\pi_1^{\ab}(X_{\fet}))^{\cont} \cong \Shv^1_{\overline{\Q_{\ell}}}(X)$$
    
    \subsection{The Automorphic Side}
        \begin{convention}[The setting of the main theorem]
            In what follows, $k$ shall be algebraically closed field and $X$ shall be a smooth projective \textit{connected} curve over $\Spec k$. Additionally, $\Bun_{\GL_1}(X)$ shall denote the moduli stack of line bundles on $X$.
        \end{convention}
    
        Let us now move on to what is known as the \say{\textbf{Automorphic Side}}, and we shall begin with the notion of \textbf{Hecke eigensheaves}. To introduce these gadgets, however, we will first need to discuss the \textbf{Hecke correspondence}, the categorification of the action of the Hecke algebra on the space of functions satisfying certain smoothness and growth conditions on the double ad\`elic quotient $\GL_1(K) \backslash \GL_1(\A_K) / \GL_1(\scrO_K)$ (i.e. automorphic forms). The details will be spelled out in definition \ref{def: hecke_correspondences}, but for now, let us think of the Hecke correspondence as a span, i.e. a diagram of the form:
            $$
                \begin{tikzcd}
                	& {\Hecke_{\GL_1}(X)} \\
                	{\Bun_{\GL_1}(X)} && {X \x \Bun_{\GL_1}(X)}
                	\arrow["{\cev{h}_X}"', from=1-2, to=2-1]
                	\arrow["{\supp_X \x \vec{h}_X}", from=1-2, to=2-3]
                \end{tikzcd}
            $$
        If we were to generically denote a given category of \say{good sheaf theory} by $\Shv(-)$\footnote{Eventually, we will be interested particularly in $\ell$-adic sheaves of rank $1$, which shall be denoted by $\Shv_{\overline{\Q_{\ell}}}^1(-)$, but more on this later. In the wider context of the Geometric Langlands Programme, $\Shv(-)$ might mean perverse sheaves, or when we are working over $\bbC$, D-modules; we shall not touch on these sheaf theories.}, then the Hecke correspondence induces the following sheaf pull-push diagram:
            $$
                \begin{tikzcd}
                	& {\Shv(\Hecke_{\GL_1}(X))} \\
                	{\Shv(\Bun_{\GL_1}(X))} && {\Shv(X \x \Bun_{\GL_1}(X))}
                	\arrow["{(\cev{h}_X)^*}", from=2-1, to=1-2]
                	\arrow["{(\supp_X \x \vec{h}_X)_*}", from=1-2, to=2-3]
                \end{tikzcd}
            $$
        and by composing the two functors in the obvious manner, one gets a new functor:
            $$\scrH_X: \Shv(\Bun_{\GL_1}(X)) \to \Shv(X \x \Bun_{\GL_1}(X))$$
        This is commonly known as the \textbf{Hecke functor} or the \textbf{Hecke operator} (should we want to put emphasis on the spectral nature of $\scrH_X$), and it is of central importance to us. However, before we can explain why this is the case, observe that the Hecke operator $\scrH_X$ is actually \say{global} in a sense: the fibre of $\supp_X \x \vec{h}$ over any given point $x \in X$ is the \say{local} Hecke correspondence:
            $$
                \begin{tikzcd}
                	& {\Hecke_{\GL_1}(x)} \\
                	{\Bun_{\GL_1}(X)} && {\Bun_{\GL_1}(X)}
                	\arrow["{\cev{h}_x}"', from=1-2, to=2-1]
                	\arrow["{\vec{h}_x}", from=1-2, to=2-3]
                \end{tikzcd}
            $$
        and one can thus define the local Hecke operator at $x \in X$ as:
            $$\scrH_x := (\vec{h}_x)_* (\cev{h}_x)^*$$
        Arguably, this is more akin to the classical Hecke operator, as it is an endofunctor on $\Shv(\Bun_{\GL_1}(X))$ as opposed to the global operator $\scrH_X$, which has differing domain and codomain; henceforth, we will be thinking of the global Hecke operator $\scrH_X$ as a family $\{\scrH_x\}_{x \in X}$ of local Hecke operators parametrised by points $x \in X$. 
        
        It is now an essential technicality that we work with $\ell$-adic local systems of rank $1$ (for which we shall write $\Shv_{\overline{\Q_{\ell}}}^1(-)$) instead of simply with a generic sheaf theory $\Shv(-)$ as we have until this moment. Via the Hecke operators, one can define the Hecke eigensheaves that we eluded to earlier: as the name suggests, these are nothing but sheaves $\E \in \Shv_{\overline{\Q_{\ell}}}^1(\Bun_{\GL_1}(X))$ that are \say{eigenvectors} of the \textit{global} Hecke operators $\scrH_X$, i.e. for each such $\E$, there exists $\calL \in \Shv_{\overline{\Q_{\ell}}}^1(X)$ such that\footnote{Here, $\boxtimes$ denotes the Deligne tensor product in the $1$-category of finite abelian categories and right-exact functors; in this particular case, we shall be content with $\calL \boxtimes \E \cong \pr_1^*\calL \tensor \pr_2^*\E$.}:
            $$\scrH_X(\E) \cong \calL \boxtimes \E$$
        Now, because $\calL$ is an $\ell$-adic local systems of rank $1$ on $X$, its stalk $\calL_x$ at any point $x \in X$ is nothing but $\overline{\Q_{\ell}}$. Consequently, the corresponding local Hecke operators $\scrH_x$ admit $(\overline{\Q_{\ell}})_x \in \Shv(X)$ - the skyscraper sheaf with value $\overline{\Q_{\ell}}$ and supported at $x \in X$ - as an \say{eigenvalue}:
            $$\scrH_x(\E) \cong (\overline{\Q_{\ell}})_x \boxtimes \E$$
        (and thus one may think of $\calL$ as a family of eigenvalues of $\E$ parametrised by points $x \in X$). It is easy to see that Hecke eigensheaves form a full symmetric monoidal subcategory of $\Shv_{\overline{\Q_{\ell}}}^1(\Bun_{\GL_1}(X))$, which we shall denote by $\Eig^1_{\overline{\Q_{\ell}}}(\Bun_{\GL_1}(X))$. 
        
        At this point, we can state and prove the main theorem of this section, which establishes a canonical equivalence between the category of rank-$1$ $\ell$-adic local systems on $X$ and the category of ($\ell$-adic) Hecke eigensheaves of rank $1$ on $\Bun_{\GL_1(X)}$:
            $$\Shv_{\overline{\Q_{\ell}}}^1(X) \cong \Eig^1_{\overline{\Q_{\ell}}}(\Bun_{\GL_1}(X))$$
        which maps each local system $\calL \in \Shv_{\overline{\Q_{\ell}}}^1(X)$ to a Hecke eigensheaf $\Autom_X(\calL) \in \Eig^1_{\overline{\Q_{\ell}}}(\Bun_{\GL_1}(X))$ with eigenvalue $\calL$. Finally, by putting theorem \ref{theorem: unramified_representations_are_sheaves_on_X} and theorem \ref{theorem: unramified_abelian_geometric_class_field_theory} together, one obtains a canonical equivalence of categories as follows:
            $$\Rep_{\overline{\Q_{\ell}}}^1(\pi_1^{\ab}(X_{\fet}))^{\cont} \cong \Eig^1_{\overline{\Q_{\ell}}}(\Bun_{\GL_1}(X))$$
        This is the version of global class field theory that we seek, and it tells us that $1$-dimensional continuous $\ell$-adic Galois representations are the same as automorphic forms associated to $\GL_1$.
        
    \subsection{Conventions}
        \begin{convention}[Category theory] \label{conv: category_theory}
            \noindent
            \begin{itemize}
                \item \textbf{(Fundamentals):} We assume familiarity with the notion of categories along with fundamental categorical concepts such as universal properties, (co)slice categories, (co)limits, and adjunctions, etc. \cite{maclane} and \cite[\href{https://stacks.math.columbia.edu/tag/0011}{Tag 0011}]{stacks} shall be our main references regarding these notions.
                \item \textbf{(Sheaf theory):} We shall assume to have theory of sheaves of sets (i.e. the theory of sheaf topoi) and of sheaves taking values in tensor categories such as module categories (cf. \cite{EGNO}) at our disposal. Readers who are not too familiar with the former are encouraged to consult \cite{sga4} and \cite[\href{https://stacks.math.columbia.edu/tag/00UZ}{Tag 00UZ}]{stacks}; for the latter, we recommend \cite[\href{https://stacks.math.columbia.edu/tag/006A}{Tag 006A}, \href{https://stacks.math.columbia.edu/tag/01AC}{Tag 01AC}, and \href{https://stacks.math.columbia.edu/tag/03A4}{Tag 03A4}]{stacks}. Also related is descent theory, for which we shall refer to \cite{vistoli_descent}, \cite[section C2.1]{elephant1}, \cite[Chapter III]{sheaves_in_geometry_and_logic}, and \cite[\href{https://stacks.math.columbia.edu/tag/0266}{Tag 0266} and \href{https://stacks.math.columbia.edu/tag/0238}{Tag 0238}]{stacks}.
                \item \textbf{(Monoidal categories):} For information on monoidal categories, we refer the readers to \cite{EGNO}.
            \end{itemize}
        \end{convention}
        
        \begin{convention}[Algebraic geometry] \label{conv: algebraic_geometry}
            \noindent
            \begin{itemize}
                \item \textbf{(Schemes):} For generalities on schemes, we refer the reader to \cite[Chapters II and III]{hartshorne}, as well as \cite[\href{https://stacks.math.columbia.edu/tag/01H8}{Tag 01H8}, \href{https://stacks.math.columbia.edu/tag/01QL}{Tag 01QL}, and \href{https://stacks.math.columbia.edu/tag/0209}{Tag 0209}]{stacks}.
                
                One notion that is indispensible for us is that of base change, for which we introduce the following short-hand: for any base scheme $S$, and any pair of $S$-schemes $X, T$, the base change $X \x_S T$ shall be denoted by $X_T$.
                \item \textbf{(Topologies on schemes):} For information on topologies on the category of schemes, we refer the reader to \cite[\href{https://stacks.math.columbia.edu/tag/0214}{Tag 0214}]{stacks}. For a general categorical treatment of descent theory, we refer the reader to \cite{vistoli_descent}, \cite[section C2.1]{elephant1}, \cite[Chapter III]{sheaves_in_geometry_and_logic}, and \cite[\href{https://stacks.math.columbia.edu/tag/0266}{Tag 0266} and \href{https://stacks.math.columbia.edu/tag/0238}{Tag 0238}]{stacks}.
            \end{itemize}
        \end{convention}
    
    \section{The Galois Side}
    \subsection{The \'etale fundamental group}
        Let us begin with an auxiliary notion, that of pro-representable functors, which is necessary for our first important construction, that of Galois categories.
        \begin{definition}[Pro-representable functors] \label{def: pro_representable_functors}
            \noindent
            \begin{enumerate}
                \item \textbf{(Pro-completions):} Following \cite[Definition 2.1]{isaksen_2001_limits_and_colimits_in_pro_categories}, the \textbf{pro-completion} $\Pro(\C)$ of a small category $\C$ is the category whose objects are cofiltered diagrams in $\C$ and whose hom-sets are given by:
                    $$\Pro(\C)(\{X_i\}_{i \in \calI}, \{Y_j\}_{j \in \calJ}) \cong \underset{j \in \calJ}{\lim} \underset{i \in \calI}{\colim} \C(X_i, Y_j)$$
                The dual notion is that of ind-completions; we denote the ind-completion of $\C$ by $\Ind(\C)$.
                \item \textbf{(Pro-representable functors):} Let $\C$ be a small category, and suppose that $\C$ is enriched in some small \href{http://nlab-pages.s3.us-east-2.amazonaws.com/nlab/show/closed+monoidal+category}{\underline{closed monoidal category}} $\V$ (e.g. the category of finite sets or the category of sets where the monoidal structure is given by products). Then, a $\V$-presheaf:
                    $$F: \C \to \V$$
                on $\C^{\op}$ is said to be \textbf{pro-representable} if and only if its \textbf{pro-completion}:
                    $$\Pro(F): \Pro(\C) \to \Pro(\V)$$
                is representable as a $\Pro(\V)$-copresheaf on $\Pro(\C)^{\op}$.
            \end{enumerate}
        \end{definition}
        \begin{remark} \label{remark: pro_representable_functors_are_ind_objects}
            Observe that due to Yoneda's Lemma, for $\C$ any small category and $\V$ any small closed monoidal category, the category of pro-representable $\V$-presheaves on $\C^{\op}$ is equivalent to $\Pro(\C)^{\op}$.
            
            Additionally, note that any pro-completion of a finite complete small category is necessarily cofiltered, since every finite cone must therefore admit a cone. Furthermore, pro-completions are their own maximal cofinal cofiltered subdiagram.
        \end{remark}
        
        We now officially begin our discussion of Grothendieck's Galois Theory with the notion of Galois categories, axiomatic settings in which one can \say{do Galois theory}, in the sense of classifying subobjects of a given universal object by checking whether or not they remain stable under certain \say{Galois group} actions; the idea is that Galois categories behave similarly to the category of finite sets (which can be thought of as the prototypical Galois category), in the same manner that sheaf topoi resemble the category of sets. Do keep in mind that for the sake of convenience (although without loss of generality, at least for our purposes), definition \ref{def: galois_categories} is a combination of \cite[D\'efinition V.4.5.1]{SGA1} and \cite[\href{https://stacks.math.columbia.edu/tag/0BMY}{Tag 0BMY}]{stacks}; namely, we require that the fibre functor is \textit{pro-representable}, which the latter source does not.
        \begin{definition}[Galois categories and their fundamental groups] \label{def: galois_categories}
            \noindent
            \begin{itemize}
                \item \textbf{(Galois categories):} A \textbf{Galois category} is defined via the data contained in a pair $(\calG, F)$ consisting of:
                \begin{itemize}
                    \item a \textit{finitely complete and finitely cocomplete} small category $\calG$, wherein objects can all be written as finite coproducts of \textit{connected} objects\footnote{Objects $X \in \calG$ such that the copresheaf $\calG(X, -)$ preserves all coproducts.}.
                    \item a \textit{pro-representable} $\Fin$-presheaf on $\calG^{\op}$:
                        $$F: \calG \to \Fin$$
                    called the \textbf{fibre functor}, which we shall require to be exact and to reflect isomorphisms (i.e. for all bijections $Fx \cong Fy$ between finite sets, one has an isomorphism $x \cong y$ in $\calG$).
                \end{itemize}
                \item \textbf{(Galois objects):} An object $X$ of a Galois category $\calG$ is a \textbf{Galois object} if and only if it has no non-trivial automorphisms, i.e. if and only if $X/\Aut_{\calG}(X) \cong \pt$, with $\pt$ a terminal object of $\calG$.\footnote{Note that Galois categories must have terminal objects, as they are finitely complete and terminal objects are nothing but the limit of the empty diagram (which is finite by virtue of containing no vertices and no edges).}
                \item \textbf{(Galois functors):} A \textbf{Galois functor} is an exact functor $\Phi: \calG \to \calG'$ between Galois categories $(\calG, F), (\calG', F')$ which preserves connected objects and commute with the fibre functors in the following manner:
                    $$
                        \begin{tikzcd}
                        	\calG && {\calG'} \\
                        	& \Fin
                        	\arrow["F"', from=1-1, to=2-2]
                        	\arrow["{F'}", from=1-3, to=2-2]
                        	\arrow["\Phi", from=1-1, to=1-3]
                        \end{tikzcd}
                    $$
            \end{itemize}
        \end{definition}
        \begin{definition}[Fundamental groups of Galois categories] \label{def: fundamental_groups_of_galois_categories}
            The \textbf{fundamental group} of a given Galois category $(\calG, F)$, denoted by $\pi_1(\calG, F)$, is defined to be the automorphism group $\Aut(\Pro(F))$.
        \end{definition}
        
        \begin{proposition}[The Categorical Galois Correspondence] \label{prop: categorical_galois_correspondence}
            Fix a Galois category $(\calG, F)$ with terminal objects $\pt$. Then there are the following equivalences of categories:
                $$\calG \cong \pi_1(\calG, F)\-\Fin\Sets$$
                $$Y \mapsto F(Y)$$
                $$\{\text{Finite-index subgroups of $\pi_1(\calG, F)$}\} \cong \pi_1(\calG, F)\-\Fin\Sets$$
                $$H \mapsto \pi_1(\calG, F)/H$$
            Furthermore, there is the following restricted equivalence:
                $$\calG^{\Gal} \cong \{\text{Finite-index normal subgroups of $\pi_1(\calG, F)$}\}$$
        \end{proposition}
            \begin{proof}
                \todo{Cite a proof}
            \end{proof}
            
        \begin{definition}[Universal covers] \label{def: universal_covers}
            Let $(\calG, F)$ be a Galois category. A pro-object $\tilde{X} \in \Pro(\calG)$ is called a \textbf{universal cover} if and only if its fundamental group $\pi_1(\tilde{X}) \cong \Aut(\Pro(F)(\tilde{X}))$ is trivial (i.e. if and only if it is simply-connected).
        \end{definition}
        \begin{remark}[Fundamental groups are automorphism groups of universal covers] \label{remark: fundamental_groups_are_automorphism_groups_of_universal_covers}
            
        \end{remark}
        
        Let us now try to adapt definitions \ref{def: galois_categories} and \ref{def: fundamental_groups_of_galois_categories} to a appropriate categories of schemes, namely those spanned by schemes finite-\'etale over a given base.
        \begin{remark}[\'Etale vs. finite-\'etale] \label{remark: etale_vs_finite_etale}
            One crucial tehcnicality that we will need to keep in mind is that finite-\'etale morphisms are \'etale, but the converse need not be true (e.g. the affine line is \'etale but not at all finite). However, \'etale morphisms are indeed finite when the codomain is the spectrum of a field (this is not the only case where \'etale morphisms are finite-\'etale, but it is sufficient for us); a proof can easily derived from \cite[\href{https://stacks.math.columbia.edu/tag/00U3}{Tag 00U3}]{stacks}, which asserts that \'etale (commutative) algebras over a field $k$ are isomorphic to finite direct sums of finite separable extension of $k$. 
        \end{remark}
        \begin{remark}[Finite-\'etale schemes] \label{remark: finite_etale_schemes}
            For any given by scheme $X$, the small category $(\Sch_{/X})_{\fet}$ of finite-\'etale $X$-schemes is a category wherein:
                \begin{itemize}
                    \item all finite limits and all finite colimits exist, and
                    \item all objects can be written as a (possibly empty) finite coproduct of connected objects, which happen to be schemes that are \'etale over $X$.  
                \end{itemize}
            (for a detailed proof, see \cite[\href{https://stacks.math.columbia.edu/tag/0BN9}{Tag 0BN9}]{stacks}) so should we be able to define a fibre functor $(\Sch_{/X})_{\fet} \to \Fin$, we will have succeeded in putting a Galois category structure on $(\Sch_{/X})_{\fet}$. As a matter of fact, such a well-defined fibre functor has good reasons to exist: it is an easy consequence of \cite[\href{https://stacks.math.columbia.edu/tag/00U3}{Tag 00U3}]{stacks} that for any fixed geometric point $\bar{x} \in X$ (corresponding to an algebraic closure $\bar{\kappa}_x$ of the residue field of $x \in X$\footnote{Certain sources consider geometric points to correspond to separable closures. For us, however, geometric points are algebraically closed fields $K$ so that $\Spec K$ be a Galois object of $(\Sch_{/\Spec K})_{\fet}$ (cf. definition \ref{def: galois_categories}). In practice this choice usually does not matter, since we will mostly work over perfect field, and separable closures of perfect fields are algebraically closed (a notable exception is when we work over perfectoid fields; cf. \cite{scholze2011perfectoid}).}), one has:
                $$(\Spec \bar{\kappa}_x)_{\fet} \cong \Fin$$
            (the forward direct simply involves taking the underlying set, and the inverse functors is given by $I \mapsto \coprod_{i = 1}^{|I|} \Spec \bar{\kappa}_x$) and so for any $k$-scheme $X$, one has the following canonical defined functor:
                $$(\Sch_{/X})_{\fet} \to (\Sch_{/\Spec \bar{\kappa}_x})$$
                $$Y \mapsto Y_{\bar{x}}$$
            where $Y_{\bar{x}} \cong Y \x_X \Spec \bar{\kappa}_x$; one can then take the underlying set of $Y_{\bar{x}}$ to get the following trivially left-exact functor:
                $$F_{\bar{x}}: (\Sch_{/X})_{\fet} \to \Fin$$
                $$Y \mapsto |Y_{\bar{x}}|$$
            We should also verify that the sets $|Y_{\bar{x}}|$ are indeed finite. To this end, let us first apply the fact that pullbacks of \'etale morphisms are \'etale to see that if $Y$ is affine over $X$ then $Y_{\bar{x}}$ will have to be the spectrum of an \'etale $\bar{\kappa}_x$-algebra; however, according to \cite[\href{https://stacks.math.columbia.edu/tag/00U3}{Tag 00U3}]{stacks}, this means that $Y_{\bar{x}} \cong \Spec (\bar{\kappa}_x)^{\oplus N}$ for some finite $N$. The locality of \'etale-ness and the finiteness of $Y$ as an $X$-scheme then tells us that in general, $Y_{\bar{x}}$ must be a finite disjoint union of affine schemes of the form $\Spec (\bar{\kappa}_x)^{\oplus N}$, meaning that $Y_{\bar{x}} \cong \Spec (\bar{\kappa}_x)^{\oplus N'}$ for some finite $N'$. The set $|Y_{\bar{x}}|$ is therefore always finite. One also sees that an immediate consequence of this proof is that $F_{\bar{x}}$ necessarily \textit{reflects isomorphisms} and is \textit{right-exact}. 
            
            It thus remains to show that $F_{\bar{x}}$ is \textit{pro-representable}. For this, observe first of all that as a functor on $\Sch_{/X}$ (as opposed to a functor on $(\Sch_{/X})_{\fet}$), $F_{\bar{x}}$ is naturally isomorphic to $\Sch_{/X}(\bar{x}, -)$. Since $\bar{x}$ is a pro-object of $(\Sch_{/\Spec \kappa_x})_{\fet}$, 
            
            We have thus constructed a well-defined fibre functor, in the sense of definition \ref{def: galois_categories}:
                $$F_{\bar{x}}: (\Sch_{/X})_{\fet} \to \Fin$$
                $$Y \mapsto |Y_{\bar{x}}|$$
        \end{remark}
        \begin{remark}[Finite \'etale Galois schemes] \label{remark: galois_schemes}
            Fix a base scheme $X$, and thanks to the fact that objects of Galois categories ($(\Sch_{/X})_{\fet}$ in this instance) can be written as finite coproducts of connected objects, we can assume without loss of generality that $X$ is connected. By definition \ref{def: galois_categories}, a Galois object in $(\Sch_{/X})_{\fet}$ is a finite-\'etale $X$-scheme $Y$ such that $Y/\Aut_X(Y) \cong X$. 
        \end{remark}
        \begin{definition}[\'Etale fundamental groups] \label{def: etale_fundamental_groups}
            For any scheme $X$ with a fixed geometric point $\bar{x}$, the pair $((\Sch_{/X})_{\fet}, F_{\bar{x}})$ as in remark \ref{remark: finite_etale_schemes} defines a Galois category. Its fundamental group is commonly denoted by $\pi_1(X_{\fet}, \bar{x})$ and called the \textbf{\'etale fundamental group} of $X$ based at $\bar{x}$.
        \end{definition}
        \begin{remark}[\'Etale fundamental groups as automorphism groups of universal covers] \label{remark: etale_fundamental_groups_as_automorphism_groups_of_universal_covers}
            In remark \ref{remark: finite_etale_schemes}, we have implicitly shown that the full subcategory $(\Sch_{/X})_{\fet}^{\Gal}$ is a diagram in $(\Sch_{/X})_{\fet}$ such that  
        \end{remark}
        \begin{remark}[The Geometric Galois Correspondence] \label{remark: geometric_galois_correspondence}
            Let $(X, \bar{x})$ be a pointed connected scheme. Then by proposition \ref{prop: categorical_galois_correspondence}, there is an equivalence of categories:
                $$\{\text{Finite-index subgroups of $\pi_1(X_{\fet}, \bar{x})$}\}$$
                $$\cong$$
                $$\{\text{Finite \'etale $X$-schemes $Y$ with base points $\bar{y}$ lying over $\bar{x}$}\}$$
            Furthermore (and also thanks to proposition \ref{prop: categorical_galois_correspondence}), this equivalence restricts down to:
                $$\{\text{Finite-index normal subgroups of $\pi_1(X_{\fet}, \bar{x})$}\}$$
                $$\cong$$
                $$\{\text{Finite \'etale Galois $X$-schemes $Y$ with base points $\bar{y}$ lying over $\bar{x}$}\}$$
            As a consequence, should $H$ be a finite-index normal subgroup of $\pi_1(X_{\fet}, \bar{x})$ and $(X^H, \bar{x}^H)$ be the corresponding Galois $X$-scheme with a choice of base point $\bar{x}^H$ lying over $\bar{x}$, then $\pi_1(X^H_{\fet}, \bar{x}^H) \cong H$. 
        \end{remark}
        \begin{example}[The \'etale fundamental group of a field] \label{example: etale_fundamental_group_of_a_field}
            As a sanity check, note that if $K$ is a field then finite-\'etale Galois schemes over $\Spec K$ shall be of the form $\Spec L \to \Spec K$, where $L/K$ is a finite Galois extension, and as a consequence, there are there are the following equivalences of lattices, which demonstrate that remark \ref{remark: geometric_galois_correspondence} directly generalises the classical Galois Correspondence:
                $$\{\text{Finite-index normal subgroups of $\pi_1((\Spec K)_{\fet})$}\}$$
                $$\cong$$
                $$\{\text{Finite \'etale Galois schemes over $\Spec K$}\}$$
                $$\cong$$
                $$\{\text{Finite Galois extensions of $K$}\}^{\op}$$
                $$\cong$$
                $$\{\text{Finite-index normal subgroups of $\Gal(\bar{K}/K)$}\}$$
        \end{example}
        \begin{example}[The \'etale fundamental group of a curve] \label{example: etale_fundamental_group_of_a_curve}
            Let $k$ be a field. If $X$ is a connected non-singular projective curve over $\Spec k$ with function field $K$, then there is a canonical equivalence $({}^{K/}\Fld^{\fin, \Gal})^{\op} \cong (\Sch_{/X})_{\fet}^{\Gal}$ between the lattice of finite Galois extensions of $K$ and Galois $X$-schemes (which are precisely dominant rational maps whose associated function field extensions are Galois). Through this, it is easy to see that:
                $$\pi_1(X_{\fet}) \cong \Gal(\bar{K}/K)$$
            For instance, we have:
                $$\pi_1((\P^1_k)_{\fet}) \cong \Gal(\bar{k}/k)$$
            (since the function field of $\P^1_k$ is $k(t)$), which tells us that $\P^1_k$ is simply \'etale-connected if and only if $k$ is algebraically closed (since $\Gal(\bar{k}/k)$ is \textit{a fortiori} trivial in that case). 
            
            Another interesting case that one might wish to consider is that of elliptic curves; for the sake of simplicity, let us work with an elliptic curve $E$ over an algebraically closed field $k$ of characteristic $0$. First of all, because $k$ is algebraically closed, every $k$-rational point $x \in E(k)$ is automatically geometric; therefore, we might as well work with $x = 0$. Now, it can be shown without too much difficulty (cf. \cite[Proposition 5.11]{kundu_etale_fundamental_group_of_elliptic_curves}) that every scheme finite \'etale over an elliptic curve over any field is automatically Galois. Together with the definition of $\pi_1(E_{\fet})$, this implies that for any cofinal diagram $\{Y_i\}_{i \in \calI}$ in $(\Sch_{/E})_{\fet}$, one has:
                $$\pi_1(E_{\fet}) \cong \underset{i \in \calI}{\lim} \Aut(F_0(Y_i)) \cong \underset{i \in \calI}{\lim} \Aut(|(Y_i)_0|)$$
            Luckily, there is a canonical choice of such a cofinal diagram, namely $\{E[n]\}_{n \in \N}$ (cf. \cite[Proposition 3.8]{kundu_etale_fundamental_group_of_elliptic_curves}), which are nothing but the fibres over $0 \in E(k)$ (i.e. kernels) of the $n$-torsion maps $[n]: E \to E$ (this is why we chose the base point $x = 0$). It is well-known that:
                $$\Aut(|E[n]|) \cong (\Z/n\Z)^{\oplus 2}$$
            so by taking the limit, one obtains:
                $$\pi_1(E_{\fet}) \cong \hat{\Z}^{\oplus 2}$$
            Elliptic curves over (algebraically closed) fields of positive characteristics $p_0$ behave somewhat differently, but it is also difficult to compute their \'etale fundamental groups (provided). If $E$ is supersingular (i.e. if the $p_0$-torsion map $[p_0]: E \to E$ has trivial kernel) then one has the following description of the \'etale fundamental group of $E$ (cf. \cite[Proposition 5.13]{kundu_etale_fundamental_group_of_elliptic_curves}):
                $$\pi_1(E_{\fet}) \cong \bigoplus_{(p) \in |\Spec \Z| \setminus \{(0), (p_0)\}} \Z_p^{\oplus 2}$$
            and otherwise, if $E$ is an ordinary elliptic curve, one has the following (cf. \cite[Proposition 5.14]{kundu_etale_fundamental_group_of_elliptic_curves}):
                $$\pi_1(E_{\fet}) \cong \Z_{p_0} \oplus \bigoplus_{(p) \in |\Spec \Z| \setminus \{(0), (p_0)\}} \Z_p^{\oplus 2}$$
        \end{example}
        \begin{example}[\'Etale fundamental group of the affine line] \label{example: etale_fundamental_group_of_the_affine_line}
            If $k$ is a field then $\pi_1((\A^1_k)_{\fet}) \cong \Gal(\bar{k}/k)$ when $\chara k = 0$, and hence $\pi_1((\A^1_k)_{\fet}) \cong 1$ if $k$ is furthermore algebraically closed. If however $\chara k > 0$, then $\pi_1((\A^1_k)_{\fet})$ can fail to be trivial even when $k$ is algebraically closed. For now, see \cite[Theorem 6.13, Remark 6.23, and Exercises 6.28 and 6.29]{lenstra_1985_galois_theory_for_schemes}.
        \end{example}
        
        Now, let us make sure that the \'etale fundamental group $\pi_1(X_{\fet}, \bar{x})$ as defined in definition \ref{def: etale_fundamental_groups} is meaningful as a formal construction. Namely, we would like to know the behaviours of $\pi_1(X_{\fet}, \bar{x})$ when we change the base point and when we base-change (cf. proposition \ref{prop: etale_fundamental_groups_do_not_depend_on_base_points}), as well as whether or not \'etale fibrations induce homotopy exact sequences of fundamental groups (cf. proposition \ref{prop: etale_homotopy_exact_sequence}). 
        \begin{proposition}[\'Etale fundamental group do not depend on base points] \label{prop: etale_fundamental_groups_do_not_depend_on_base_points}
            Let $f: Y \to X$ be a morphism of connected qcqs\footnote{quasi-compact and quasi-separated} schemes such that the base change functor:
                $$(\Sch_{/X})_{\fet} \to (\Sch_{/Y})_{\fet}$$
                $$X' \mapsto X' \x_X Y$$
            is an equivalence of Galois categories. Then, for any choice of geometric points $\bar{x} \in X$ and $\bar{y} \in Y$, one has the following isomorphism of \'etale fundamental groups $\pi_1(X_{\fet}, \bar{x}) \cong \pi_1(Y_{\fet}, \bar{y})$.
        \end{proposition}
            \begin{proof}
                This is an immediate consequence of the assumption that the base change functor:
                    $$(\Sch_{/X})_{\fet} \to (\Sch_{/Y})_{\fet}$$
                    $$X' \mapsto X' \x_X Y$$
                is an equivalence and from the definition of \'etale fundamental groups (cf. definition \ref{def: etale_fundamental_groups}).
            \end{proof}
        \begin{corollary}[Uniqueness of \'etale fundamental groups] \label{coro: etale_fundamental_group_uniqueness}
            For any connected qcqs scheme $X$ and any pair of possibly distinct geometric points $\bar{x}, \bar{x}' \in X$, one has any isomorphism of \'etale fundamental groups $\pi_1(X_{\fet}, \bar{x}) \cong \pi_1(X_{\fet}, \bar{x}')$, and therefore it makes sense to only speak of \textit{the} fundamental group of $X$, which we shall denote by $\pi_1(X_{\fet})$.
        \end{corollary}
        
        \begin{proposition}[\'Etale fundamental group of products]
            Let $k$ be an algebraically closed field and let $X, Y$ be two connected schemes which are, respectively, proper and locally noetherian over $\Spec k$. In addition, fix two geometric points $x \in X$ and $y \in Y$. Then, there is a canonical isomorphism:
                $$\pi_1((X \x_{\Spec k} Y)_{\fet}, (x, y)) \cong \pi_1(X_{\fet}, x) \x \pi_1(Y_{\fet}, y)$$
        \end{proposition}
            \begin{proof}
                \cite[Corollaire 1.7]{SGA1}
            \end{proof}
        
        \begin{proposition}[The \'etale homotopy exact sequence] \label{prop: etale_homotopy_exact_sequence}
            Let $X$ be a connected scheme. If $f: Y \to X$ be a flat proper morphism of finite presentation whose geometric fibres $Y_{\bar{x}}$ are connected and reduced, then for any geometric point $\bar{x} \in X$, there exists a right-exact sequence of groups as follows:
                $$\pi_1((Y_{\bar{x}})_{\fet}) \to \pi_1(Y_{\fet}) \to \pi_1(X_{\fet}) \to 1$$
        \end{proposition}
            \begin{proof}
                \cite[\href{https://stacks.math.columbia.edu/tag/0C0J}{Tag 0C0J}]{stacks}.
            \end{proof}
    
    \subsection{Grothendieck's Galois Theory}
        We begin by checking whether or not \'etale fundamental group is dual - in some sense - to the construction of an Eilenberg-MacLane space, thereby having concrete connections to the $1^{st}$ \'etale cohomology group and in turn, admitting descriptions in terms of torsors.
        \begin{remark}[Points of the moduli stack of $G$-torsors]
            For proposition \ref{prop: etale_eckmann_hilton_duality}, a basic fact one should keep in mind is that should $G$ be a constant group\footnote{As opposed to say, an algebraic group or more general group schemes}, then the groupoid $\Bun_{\underline{G}}(X) := \Sch_{/\Spec k}(X, \underline{G})$ of \'etale $\underline{G}$-torsors\footnote{Here, $\underline{G}$ denotes the group scheme represented by $\coprod_{g \in G} \Spec k$. Note how it is \'etale over $\Spec k$.} on a scheme $X$ over a field $k$ is equivalent to the groupoid $\Bun_{\underline{G}}(\Spec k)(X) := \Sch_{/\Spec k}(X, \underline{G(k)})$ of $X$-points of the moduli stack of $\underline{G}$-torsors on $\Spec k$.
        \end{remark}
        \begin{proposition}[The \'etale Eckmann-Hilton Duality] \label{prop: etale_eckmann_hilton_duality}
            For any smooth projective connected curve $X$\footnote{Actually, this proposition holds also when $X$ is any irreducible geometrically unibranch scheme (which can be thought of as analogues of path-connected spaces), but we are not interested in such generalities.} over a field $k$ and any constant profinite group $G$, there exists the following adjunction:
                $$
                    \begin{tikzcd}
                    	{\Grp(\Fin)} & {(\Sch_{/X})_{\fet}}
                    	\arrow[""{name=0, anchor=center, inner sep=0}, "{\Sch_{/\Spec k}(X, -)}"', bend right, from=1-1, to=1-2]
                    	\arrow[""{name=1, anchor=center, inner sep=0}, "{\pi_1^{\fet}}"', bend right, from=1-2, to=1-1]
                    	\arrow["\dashv"{anchor=center, rotate=-90}, draw=none, from=1, to=0]
                    \end{tikzcd}
                $$
            and in addition, a canonical equivalence:
                $$\Grp(\Pro\Fin)(\pi_1(X_{\fet}), G) \cong \Bun_{\underline{G}}(X)$$
            between the groupoid of continuous homomorphisms $\pi_1(X_{\fet}) \to G$ of profinite groups and that of $\underline{G}$-torsors on $X$.
        \end{proposition}
            \begin{proof}
                Each continuous homomorphism $\pi_1(X_{\fet}) \to G$ determines a unique $G$-torsor in $\pi_1(X_{\fet})\-\Pro\Fin$. Because there is an equivalence $\pi_1(X_{\fet})\-\Pro\Fin \cong (\Sch_{/X})_{\profet}$ (cf. proposition \ref{prop: categorical_galois_correspondence}) and because schemes are representable \'etale sheaves, each such $G$-torsor in $\pi_1(X_{\fet})\-\Pro\Fin$ corresponds to a unique $\underline{G}$-torsor on $X$. Such a $\underline{G}$-torsor on $X$, in turn, is an $X$-point of the classifying stack $\Bun_{\underline{G}}((\Spec k))$, i.e. a morphism $X \to \underline{G}$ of $k$-schemes, and the proposition follows suite.
            \end{proof}
        \begin{convention}
            From now on, if $E$ is a non-archimedean normed field then its subring of power-bounded elements shall be denoted by $\scrO_E$. 
        \end{convention}
        \begin{corollary}[Continuous representations of the \'etale fundamental group are torsors] \label{coro: continuous_representations_of_the_etale_fundamental_group_are_torsors}
            Let $\ell$ be a prime, let $E$ be an $\ell$-adic number field (i.e. a finite extension of $\Q_{\ell}$). For any smooth projective connected curve over a field $k$ along with any choice of discrete group $G$, there exists a canonical equivalence:
                $$\Rep^n_{\scrO_E}(\pi_1(X_{\fet}))^{\cont} \cong \Bun_{\underline{\GL_n(\scrO_E)}}(X)$$
            between the groupoid of continuous $n$-dimensional $\scrO_E$-linear representations of $\pi_1(X_{\fet})$ and that of $\GL_n(\scrO_E)$-torsors on $X$.
        \end{corollary}
        \begin{remark}[What about higher-dimensional representations ?]
            Corollary \ref{coro: continuous_representations_of_the_etale_fundamental_group_are_torsors} does not hold for $\GL_n(E)$ (for any $n \geq 1$), as these groups are only locally profinite, as opposed to being globally profinite like $\GL_n(\scrO_E)$.
        \end{remark}
    
        \begin{convention}[The Curve] \label{conv: base_curve}
            Henceforth, $X$ shall be a smooth projective \textit{connected} curve over $\Spec k$ (with $k$ some field).
        \end{convention}
        
        \begin{proposition}[$\ell$-adic representations are $\ell$-adic torsors] \label{prop: E_representations_are_E_local_systems}
            Let $\ell$ be a prime and $E$ be an $\ell$-adic number field. Then there exists a canonical equivalence of groupoids as follows, for all $n \geq 1$:
                $$\Rep_E^n(\pi_1(X_{\fet}))^{\cont} \cong \Bun_{\underline{\GL_n(E)}}(X)$$
        \end{proposition}
            \begin{proof}
                Let $\rho: \pi_1(X_{\fet}) \to \GL_n(E)$ be a continuous representation. $\GL_n(\scrO_E)$ is an open subgroup of $\GL_n(E)$, so the preimage $H := \rho^{-1}(\GL_n(\scrO_E))$ must be an open subgroup of $\pi_1(X_{\fet})$; since $\pi_1(X_{\fet})$ is profinite, $H$ is furthermore normal. Consequently, there exists a Galois $X$-scheme $X^H$ such that $\pi_1(X^H_{\fet}) \cong H$; as a result, we obtain the following pullback of topological groups:
                    $$
                        \begin{tikzcd}
                        	{\pi_1(X^H_{\fet})} & {\GL_n(\scrO_E)} \\
                        	{\pi_1(X_{\fet})} & {\GL_n(E)}
                        	\arrow[hook, from=1-2, to=2-2]
                        	\arrow["\rho", from=2-1, to=2-2]
                        	\arrow[from=1-1, to=2-1]
                        	\arrow["{(\rho^H)^{\circ}}", from=1-1, to=1-2]
                        	\arrow["\lrcorner"{anchor=center, pos=0.125}, draw=none, from=1-1, to=2-2]
                        \end{tikzcd}
                    $$
                Now, due to corollary \ref{coro: continuous_representations_of_the_etale_fundamental_group_are_torsors}, each representation $(\rho^H)^{\circ}: \pi_1(X^H_{\fet}) \to \GL_n(\scrO_E)$ corresponds to a unique $\underline{\GL_n(\scrO_E)}$-torsor on $X^H$, and since $\Bun_{\underline{\GL_n(\scrO_E)}}$ satisfies (profinite-)\'etale descent, one thus obtains in addition a unique $\underline{\GL_n(\scrO_E)}$-torsor on $X$, i.e. a representation $\rho^{\circ}: \pi_1(X_{\fet}) \to \GL_n(\scrO_E)$. There is thus a fully faithful embedding of $\Rep_E^n(\pi_1(X_{\fet}))^{\cont}$ into $\Bun_{\underline{\GL_n(E)}}(X)$. Now, to show that this embedding is also essentially surjective, observer that because $(\Sch_{/X})_{\profet} \cong \pi_1(X_{\fet})\-\Pro\Fin$, each $\underline{\GL_n(E)}$-torsor on $X$ corresponds to a unique (continuous) $\GL_n(E)$-torsor in $\pi_1(X_{\fet})\-\Pro\Fin$. But such a torsor is nothing but a continuous representation $\pi_1(X_{\fet}) \to \GL_n(E)$, so we are done.
            \end{proof}
        \begin{corollary}[Representations of the \'etale fundamental group are local systems] \label{coro: representations_of_the_etale_fundamental_group}
            Let $\ell$ be a prime and $E$ be an $\ell$-adic number field. In addition, fix a geometric point $\bar{x} \in X$. Then there exists a canonical equivalence as follows, for all $n \geq 1$:
                $$\Shv^n_{\underline{E}}(X) \cong \Rep_E^n(\pi_1(X_{\fet}))^{\cont}$$
                $$\calL \mapsto \calL_{\bar{x}}$$
        \end{corollary}
            
        The following result crucially exploits the fact that $\GL_1(E)$ is abelian, unlike $\GL_n(E)$ for $n \geq 2$.
        \begin{lemma}[Abelianising continuous characters] \label{lemma: abelianising_continuous_characters}
            Let $G$ be a topological group and $E$ a topological field. Then, there is a group isomorphism:
                $$\Rep^1_E(G)^{\cont} \cong \Rep^1_E(G^{\ab})^{\cont}$$
        \end{lemma}
            \begin{proof}
                There is a natural injective group homomorphism:
                    $$\Rep^1_E(G)^{\cont} \to \Rep^1_E(G^{\ab})^{\cont}$$
                    $$\chi \mapsto \chi^{\ab}$$
                coming from the canonical quotient map $G \to G^{\ab}$, so the only thing to do is to show that this homomorphism is surjective. For this, it shall suffice to show that the group $\Rep^1_E([G, G])^{\cont}$ is trivial: but this is evident from the fact that $\GL_1(E)$ is abelian and from the definition of the commutator subgroup $[G, G]$, namely that $[G, G] := \<ghg^{-1}h^{-1} \mid \forall g, h \in G\>$ (so for all $x \in [G, G]$ and all $\chi \in \Rep^1_E([G, G])$, $\chi(x) = 1$), so we are done.
            \end{proof}
            
        \begin{theorem}[Galois representations are sheaves on $X$] \label{theorem: galois_representations_are_sheaves_on_X}
            Fix a geometric point $\bar{x} \in X$. Then, there is a canonical monoidal equivalence:
                $$\Shv_{\bar{\Q}_{\ell}}^{\ad, 1}(X) \cong \Rep^1_{\bar{\Q}_{\ell}}(\pi_1^{\ab}(X_{\fet}))^{\cont}$$
                $$\calL \mapsto \calL_{\bar{x}}$$
        \end{theorem}
            \begin{proof}
                
            \end{proof}
    
    \section{The Automorphic Side}
    \subsection{The Hecke action and spectral decomposition}
        \begin{convention}[The setting of the main theorem]
            From this point on, $k$ shall be a separably closed field and $X$ shall be a smooth projective \textit{connected} curve over $\Spec k$. Additionally, $\Bun_{\GL_1}(X)$ shall denote the moduli space of line bundles on $X$.
        \end{convention}
        
        \begin{definition}[The Hecke correspondence] \label{def: hecke_correspondence}
            The Hecke correspondence is a span, i.e. a diagram of the form:
                $$
                    \begin{tikzcd}
                    	& {\Hecke_{\GL_1}(X)} \\
                    	{\Bun_{\GL_1}(X)} && {X \x \Bun_{\GL_1}(X)}
                    	\arrow["{\cev{h}}"', from=1-2, to=2-1]
                    	\arrow["{\supp_X \x \vec{h}}", from=1-2, to=2-3]
                    \end{tikzcd}
                $$
            wherein $\Hecke_{\GL_1}(X)$ is the moduli space of quadruples:
                $$(\E_1, \E_2, x, \beta_x)$$
            consisting of $\ell$-adic sheaves $\E_1, \E_2 \in \Shv_{\overline{\Q_{\ell}}}^1(\Bun_{\GL_1}(X))$, points $x \in X$, and for each such point $x$, a monomorphism $\beta_x: \E_1 \hookrightarrow \E_2$ whose cokernel is isomorphic to $k_x^{\oplus d}$ (where $k_x$ denotes the skyscraper sheaf supported at $x \in X$). It is naturally equipped with two projection functors $\cev{h}$ and $\supp_X \x \vec{h}$, which are defined via:
                $$\cev{h}(\E_1, \E_2, x, \beta_x) \cong \E_1$$
                $$\supp_X(\E_1, \E_2, x, \beta_x) \cong (x, \E_2)$$
            and hence one obtains the Hecke correspondence as a span.
        \end{definition}
        
        \begin{definition}[Hecke operators] \label{def: hecke_operators}
            
        \end{definition}
    
        \begin{definition}[Hecke eigensheaves] \label{def: hecke_eigensheaves}
            A \textit{non-zero} $\ell$-adic sheaf $\E \in \Shv_{\overline{\Q_{\ell}}}^1(\Bun_{\GL_1}(X))$ is called a \textbf{Hecke eigensheaf} (of rank $1$) if and only if there exists an $\ell$-adic sheaf $\calL \in \Shv_{\overline{\Q_{\ell}}}^1(X)$ (called the \textbf{eigenvalue} of $\E$) such that:
                $$\H_X(\E) \cong \calL \boxtimes \E$$
            It is easy to see that Hecke eigensheaves form a full symmetric monoidal subcategory of $\Shv_{\overline{\Q_{\ell}}}^1(\Bun_{\GL_1}(X))$, which we shall denote by $\Eig^1_{\overline{\Q_{\ell}}}(\Bun_{\GL_1}(X))$.
        \end{definition}
    
    \subsection{The Abel-Jacobi map and Deligne-Artin Reciprocity for global function fields}
        \begin{theorem}[Unramified abelian geometric class field theory] \label{theorem: unramified_abelian_geometric_class_field_theory}
            There exists a canonical equivalence between the category of rank-$1$ $\ell$-adic local systems on $X$ and the category of ($\ell$-adic) Hecke eigensheaves of rank $1$ on $\Bun_{\GL_1(X)}$:
                $$\LocSys_{\overline{\Q_{\ell}}}^1(X) \cong \Eig^1_{\overline{\Q_{\ell}}}(\Bun_{\GL_1}(X))$$
            which maps each local system $\calL \in \LocSys_{\overline{\Q_{\ell}}}^1(X)$ to a Hecke eigensheaf $\Aut_{\calL} \in \Eig^1_{\overline{\Q_{\ell}}}(\Bun_{\GL_1}(X))$ with eigenvalue $\calL$.
        \end{theorem}
            \begin{proof}
                \noindent
                \begin{enumerate}
                    \item \textbf{():}
                    \item
                    \item 
                \end{enumerate}
            \end{proof}
        \begin{remark}[How should we interpret theorem \ref{theorem: unramified_abelian_geometric_class_field_theory} ?] \label{remark: unramified_abelian_geometric_class_field_theory_explanation}
            By putting theorem \ref{theorem: unramified_representations_are_sheaves_on_X} and theorem \ref{theorem: unramified_abelian_geometric_class_field_theory} together, one gets a canonical equivalence of categories as follows:
                $$\Rep_{\overline{\Q_{\ell}}}^1(\pi_1^{\ab}(X_{\fet}))^{\cont} \cong \Eig^1_{\overline{\Q_{\ell}}}(\Bun_{\GL_1}(X))$$
            Modulo technicalities, what this essentially tells us is that $1$-dimensional continuous $\ell$-adic Galois representations are the same as automorphic forms associated to $\GL_1$, and as such, the combination of theorem \ref{theorem: unramified_representations_are_sheaves_on_X} and theorem \ref{theorem: unramified_abelian_geometric_class_field_theory} can be understood as the Categorical Global Unramified Geometric Langlands Correspondence in its simplest non-trivial form, that being for the (connnected reductive group $G \cong \GL_1$)\footnote{Incidentally, this is why it is commonly asserted that the Langlands Correspondence for $\GL_1$ \say{is just class field theory}.}. 
        \end{remark}
        \begin{example}[Some examples of geometric reciprocity] \label{example: geometric_reciprocity}
            \noindent
            \begin{itemize}
                \item \textbf{($X \cong \P^1$):}
                \item \textbf{(Elliptic curves):}
                \item \textbf{(Counter-example: $X \cong \A^1$):}
            \end{itemize}
        \end{example}
        
        \begin{remark}[What about local class field theory ?]
            
        \end{remark}
    
    \begin{appendices}
    \section{What about local class field theory ?}
    \end{appendices}
	
	\printbibliography

\end{document}