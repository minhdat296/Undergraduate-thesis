\input{article preambles}
\setcounter{section}{-1}

\input{commands}

\begin{document}

	\title{\textbf{MATH518: Thesis
	\\
	Geometric abelian class field theory}}
	
	\author{Dat Minh Ha (UCID: 30067407)\\Supervisor: Jerrod Smith}
	\maketitle
	
	\begin{abstract}
	    The Langlands Programme is a network of many deep conjectures (and recently, some outstanding theorems!) with far-reaching consequences, notably in the realms of algebraic number theory and representation theory. Its starting point is what we nowadays call \textbf{class field theory}, the very topic of this thesis. More specifically, we are interested in what is known as \textbf{unramified abelian class field theory}, the simplest version of class field theory, which we shall approach via algebraic geometry. This is a non-traditional approach, but it will help us understand why the study of Galois groups naturally requires representation theory, which as a consequence, helps us makes sense of class field theory being the same as the Langlands Correspondence for the group $\GL_1$.
	\end{abstract}
	
	{
      \hypersetup{} 
      %\dominitoc
      \tableofcontents %sort sections alphabetically
    }
    
    \section{Introduction}
    The paper will be organised into two main sections, detailing what we shall call the \textbf{Galois Side} and the \textbf{Automorphic Side} of geometric class field theory. This introductory section will be dedicated to the outlining of our approach to these sections, as well as for laying down some conventions that we will be following until the end of the paper.

    \subsection{The Galois Side}
        Geometric class field theory is the idea that the study of function fields $K$ over $\F_q$ (i.e. finite extensions of $\F_q(t)$) can be formulated purely in terms of the machineries of algebraic geometry. More specifically, it is the idea that given Galois extensions $L/K$, one can decribe the Galois groups $\Gal(L/K)$ in purely geometric terms, via a gadget called the \textbf{\'etale fundamental group}. This, however, is only a meaningful endeavour should we know the underlying geometric space whose \'etale fundamental group would end up being the Galois groups that we are interested in.
        
        Our starting point is the following result, which explains why one might even suspect any sort of involvement of algebraic geometry in the first place:
        \begin{lemma}[Varieties and field extensions] \label{lemma: varieties_and_field_extensions}
            \cite[\href{https://stacks.math.columbia.edu/tag/0BXN}{Tag 0BXN}]{stacks} Let $k$ be a field. Then, $\trdeg K_X = \dim X$ for all varieties $X/k$, and there exists a canonical equivalence of categories as follows:
                $$\{\text{Finite-type field extensions $K/k$ and $k$-algebra homomorphisms}\}^{\op}$$
                $$\cong$$
                $$\{\text{Varieties $X/k$ and \href{https://stacks.math.columbia.edu/tag/01RI}{\underline{dominant}} \href{https://stacks.math.columbia.edu/tag/01RR}{\underline{rational}} maps}\}$$
        \end{lemma}
            \begin{proof}
                If $f: X \to Y$ is a dominant map of varieties and if $\eta_X$ and $\eta_Y$ are the unique generic points of $X$ and $Y$ (unique because varieties are integral by definition, and every integral scheme \textit{a priori} has a unique generic point), then $f(\eta_X) = \eta_Y$ per the definition of dominant morphisms. The residue field at generic points are precisely the function fields (consider the stalk of the structure sheaves at the generic points to see why this is the case), so we have obtained a map of function fields $K_Y \to K_X$. We leave the proof of finiteness up to the reader.
            \end{proof}
        Through lemma \ref{lemma: varieties_and_field_extensions}, one obtains the following regarding the relationship between curves (i.e. algebraic varieties of dimension $1$) and their function fields (which \textit{a priori} are of transcedence degree $1$ over the ground field) with little difficulty:
        \begin{proposition}[Curves and function fields] \label{prop: curves_and_function_fields}
            \cite[\href{https://stacks.math.columbia.edu/tag/0BY1}{Tag 0BY1}]{stacks} For any field $k$, one has the following canonical equivalences of categories:
                $$\{\text{Field extensions $K/k$ of transcendence degree $1$ and $k$-algebra homomorphisms}\}^{\op}$$
                $$\cong$$
                $$\{\text{Curves $X/k$ and dominant rational maps}\}$$
                $$\cong$$
                $$\{\text{Non-singular projective curves $X/k$ and dominant rational maps}\}$$
        \end{proposition}
            \begin{proof}
                The first equivalence is an obvious consequence of lemma \ref{lemma: varieties_and_field_extensions}. To show that the second equivalence holds, note firstly that there is an evident fully faithful functor from the third category to the second; we shall need to show that this functor is also essentially surjective. For this, simply recall that for each curve $X/k$, there exists a non-singular projective curve $\tilde{X}/k$ that is birational to $X/k$, namely $X^{\nu} \cup \{\infty_1, ..., \infty_n\}$, the normalisation $X^{\nu}/k$ of $X/k$ with finitely many extra points (recall also that any normal Noetherian scheme of dimension $\leq 1$ is \textit{a priori} non-singular; cf. \cite[\href{https://stacks.math.columbia.edu/tag/0BX2}{Tag 0BX2}]{stacks}).
            \end{proof}
            
        Through proposition \ref{prop: curves_and_function_fields}, we obtain the first crucial tool for the geometrisation of class field theory.
        \begin{corollary}[Galois covers of curves and Galois extensions] \label{coro: galois_covers_of_curves_and_galois_extensions}
            Let $k$ be a field. If $X$ is a connected non-singular projective curve over $\Spec k$ with function field $K$, then there is a canonical equivalence:
                $$({}^{K/}\Fld^{\fin, \Gal})^{\op} \cong \Sch_{/X}^{\fet, \Gal}$$
            between the category of finite Galois extensions of $K$ and finite \'etale-Galois covers of $X$ (i.e. finite \'etale covers generated by Galois morphisms, which are necessarily dominant rational maps such that the associated function field extensions are Galois). 
        \end{corollary}
        The importance of corollary \ref{coro: galois_covers_of_curves_and_galois_extensions} can not be understated: what it tells us is essentially that the Galois group $\Gal(K^{\ab}/K)$ is nothing but the \'etale fundamental group of $\Sch_{/X}^{\fet, \Gal}$ (more on this later, after we have introduced the \'etale fundamental group; cf. definition \ref{def: etale_fundamental_groups}). This, already, is one side of the Artin's Reciprocity Law, which we shall refer to as \say{\textbf{The Galois Side}} per popular conventions. Actually, this is a bit of a lie: instead of formulating geometric class field theory directly in terms of the \'etale fundamental group $\pi_1^{\ab}(X_{\fet})$ (or rather, its continuous $\ell$-adic characters), we will be phrasing things in terms of $\ell$-adic local systems of rank $1$ on $X$; the main result on the Galois Side shall be theorem \ref{theorem: unramified_representations_are_sheaves_on_X}, which rigorously and precisely establishes the categorification of continuous representations of $\pi_1^{\ab}(X_{\fet})$ to $\ell$-adic local systems on $X$ via a canonical equivalence of categories:
            $$\Rep^1_{\overline{\Q_{\ell}}}(\pi_1^{\ab}(X_{\fet}))^{\cont} \cong \LocSys^1_{\overline{\Q_{\ell}}}(X)$$
        
        One last remark pertaining to the Galois Side is that because we are working in the \'etale topology, everything is necessarily unramified, since \'etale morphisms are unramified smooth morphisms by definition.
    
    \subsection{The Automorphic Side}
        Let us now move on to what is known as the \say{\textbf{Automorphic Side}}, and we shall begin with the notion of \textbf{Hecke eigensheaves}. To introduce these gadgets, however, we will first need to discuss the \textbf{Hecke correspondence}, the categorification of the action of the Hecke algebra on the space of functions satisfying certain smoothness and growth conditions on the double ad\`elic quotient $\GL_1(K) \backslash \GL_1(\A_K) / \GL_1(\scrO_K)$ (i.e. automorphic forms). The details will be spelled out in definition \ref{def: hecke_correspondence}, but for now, let us think of the Hecke correspondence as a span, i.e. a diagram of the form:
            $$
                \begin{tikzcd}
                	& {\Hecke_{\GL_1}(X)} \\
                	{\Bun_{\GL_1}(X)} && {X \x \Bun_{\GL_1}(X)}
                	\arrow["{\cev{h}}"', from=1-2, to=2-1]
                	\arrow["{\supp_X \x \vec{h}}", from=1-2, to=2-3]
                \end{tikzcd}
            $$
        If we were to generically denote a given category of \say{good sheaf theory} by $\Shv(-)$\footnote{Eventually, we will be interested particularly in $\ell$-adic sheaves of rank $1$, which shall be denoted by $\Shv_{\overline{\Q_{\ell}}}^1(-)$, but more on this later. In the wider context of the Geometric Langlands Programme, $\Shv(-)$ might mean perverse sheaves, or when we are working over $\bbC$, D-modules; we shall not touch on these sheaf theories.}, then the Hecke correspondence induces the following sheaf pull-push diagram:
            $$
                \begin{tikzcd}
                	& {\Shv(\Hecke_{\GL_1}(X))} \\
                	{\Shv(\Bun_{\GL_1}(X))} && {\Shv(X \x \Bun_{\GL_1}(X))}
                	\arrow["{\cev{h}^*}", from=2-1, to=1-2]
                	\arrow["{(\supp_X \x \vec{h})_*}", from=1-2, to=2-3]
                \end{tikzcd}
            $$
        and by composing the two functors in the obvious manner, one gets a new functor:
            $$\scrH_X: \Shv(\Bun_{\GL_1}(X)) \to \Shv(X \x \Bun_{\GL_1}(X))$$
        This is commonly known as the \textbf{Hecke functor} or the \textbf{Hecke operator} (should we want to put emphasis on the spectral nature of $\scrH_X$), and it is of central importance to us. However, before we can explain why this is the case, observe that the Hecke operator $\scrH_X$ is actually \say{global} in a sense: the fibre of $\supp_X \x \vec{h}$ over any given point $x \in X$ is the \say{local} Hecke correspondence:
            $$
                \begin{tikzcd}
                	& {\Hecke_{\GL_1}(x)} \\
                	{\Bun_{\GL_1}(X)} && {\Bun_{\GL_1}(X)}
                	\arrow["{\cev{h}}"', from=1-2, to=2-1]
                	\arrow["{\vec{h}}", from=1-2, to=2-3]
                \end{tikzcd}
            $$
        and one can thus define the local Hecke operator at $x \in X$ as:
            $$\scrH_x := \vec{h}_* \cev{h}^*$$
        Arguably, this is more akin to the classical Hecke operator, as it is an endofunctor on $\Shv(\Bun_{\GL_1}(X))$ as opposed to a functor $\scrH_X: \Shv(\Bun_{\GL_1}(X)) \to \Shv(X \x \Bun_{\GL_1}(X))$; henceforth, we will be thinking of the global Hecke operator $\scrH_X$ as a family $\{\scrH_x\}_{x \in X}$ of local Hecke operators parametrised by points $x \in X$. 
        
        Now, via the local Hecke operators, one can define the Hecke eigensheaves that we eluded to earlier: as the name suggests, these are nothing but sheaves $\E \in \Shv(\Bun_{\GL_1}(X))$ that are \say{\textbf{eigenvectors}} of the local Hecke operators $\scrH_x$, i.e. for each such $\E$, there exists an \say{\textbf{eigenvalue}} $L \in \Vect(\overline{\Q_{\ell}})$ such that:
            $$\scrH_x(\E) \cong L_x \boxtimes \E$$
        (where $L_x \in \Shv(X)$ denotes the skyscraper sheaf with value $L$ and supported at $x \in X$). It is easy to see that Hecke eigensheaves form a full symmetric monoidal subcategory of $\Shv_{\overline{\Q_{\ell}}}^n(\Bun_{\GL_1}(X))$, which we shall denote by $\Eig^1_{\overline{\Q_{\ell}}}(\Bun_{\GL_1}(X))$. At this point, we will be able to state and prove the main theorem of this section, which establishes a canonical equivalence between the category of rank-$1$ $\ell$-adic local systems on $X$ and the category of ($\ell$-adic) Hecke eigensheaves of rank $1$ on $\Bun_{\GL_1(X)}$:
            $$\LocSys_{\overline{\Q_{\ell}}}^1(X) \cong \Eig^1_{\overline{\Q_{\ell}}}(\Bun_{\GL_1}(X))$$
        which maps each local system $\calL \in \LocSys_{\overline{\Q_{\ell}}}^1(X)$ to a Hecke eigensheaf $\Aut_{\calL} \in \Eig^1_{\overline{\Q_{\ell}}}(\Bun_{\GL_1}(X))$ with eigenvalue $\calL$.
        
        By putting theorem \ref{theorem: unramified_representations_are_sheaves_on_X} and theorem \ref{theorem: unramified_abelian_geometric_class_field_theory} together, one gets a canonical equivalence of categories:
            $$\Rep_{\overline{\Q_{\ell}}}^1(\pi_1^{\ab}(X_{\fet}))^{\cont} \cong \Eig^1_{\overline{\Q_{\ell}}}(\Bun_{\GL_1}(X))$$
        This is the version of global class field theory that we seek, and it tells us that $1$-dimensional continuous $\ell$-adic Galois representations are the same as automorphic forms associated to $\GL_1$.
        
    
    \section{The Galois Side}
    \subsection{\'Etale fundamental groups}
        \subsubsection{Construction of \'etale fundamental groups}
            Let us begin with an auxiliary notion, that of pro-representable functors, which is necessary for our first important construction, that of Galois categories.
            \begin{definition}[Pro-representable functors] \label{def: pro_representable_functors}
                \noindent
                \begin{enumerate}
                    \item \textbf{(Pro-completions):} Following \cite[Definition 2.1]{isaksen_2001_limits_and_colimits_in_pro_categories}, the \textbf{pro-completion} $\Pro(\C)$ of a small category $\C$ is the category whose objects are cofiltered diagrams in $\C$ and whose hom-sets are given by $\Pro(\C)(\{X_i\}_{i \in \calI}, \{Y_j\}_{j \in \calJ}) \cong \underset{j \in \calJ}{\lim} \underset{i \in \calI}{\colim} \C(X_i, Y_j)$.
                    \item \textbf{(Pro-representable functors):} Let $\C$ be a small category, and suppose that $\C$ is enriched in some small \href{http://nlab-pages.s3.us-east-2.amazonaws.com/nlab/show/closed+monoidal+category}{\underline{closed monoidal category}} $\V$ (e.g. the category of finite sets or the category of sets where the monoidal structure is given by products). Then, a $\V$-presheaf on $\C^{\op}$ is said to be \textbf{pro-representable} if and only if it is naturally isomorphic to a filtered colimit of representable presheaves on $\C^{\op}$.
                \end{enumerate}
            \end{definition}
            \begin{remark} \label{remark: pro_representable_functors_are_ind_objects}
                Observe that due to Yoneda's Lemma, for $\C$ any small category and $\V$ any small closed monoidal category, the category of pro-representable $\V$-presheaves on $\C^{\op}$ is equivalent to $\Pro(\C)^{\op}$.
                
                Additionally, note that any pro-completion of a finite complete small category is necessarily cofiltered, since every finite cone must therefore admit a cone. Furthermore, pro-completions are their own maximal cofinal cofiltered subdiagram.
            \end{remark}
            
            We now officially begin our discussion of Grothendieck's Galois Theory with the notion of Galois categories, axiomatic settings in which one can \say{do Galois theory}, in the sense of classifying subobjects of a given universal object by checking whether or not they remain stable under certain \say{Galois group} actions; the idea is that Galois categories behave similarly to the category of finite sets (which can be thought of as the prototypical Galois category), in the same manner that sheaf topoi resemble the category of sets. Do keep in mind that for the sake of convenience (although without loss of generality, at least for our purposes), definition \ref{def: galois_categories} is a combination of \cite[D\'efinition V.4.5.1]{SGA1} and \cite[\href{https://stacks.math.columbia.edu/tag/0BMY}{Tag 0BMY}]{stacks}; namely, we require that the fibre functor is \textit{pro-representable}, which the latter source does not.
            \begin{definition}[Galois categories and their fundamental groups] \label{def: galois_categories}
                \noindent
                \begin{itemize}
                    \item \textbf{(Galois categories):} A \textbf{Galois category} is defined via the data contained in a pair $(\calG, F)$ consisting of:
                    \begin{itemize}
                        \item a \textit{finitely complete and finitely cocomplete} small category $\calG$, wherein objects can all be written as finite coproducts of \textit{connected} objects\footnote{Objects $X \in \calG$ such that the copresheaf $\calG(X, -)$ preserves all coproducts.}.
                        \item a \textit{pro-representable} functor $F: \calG \to \Fin\Sets$ - called the \textbf{fibre functor} - which we shall require to be exact and to reflect isomorphisms (i.e. for all bijections $Fx \cong Fy$ between finite sets, one has an isomorphism $x \cong y$ in $\calG$).
                    \end{itemize}
                    \item \textbf{(Galois objects):} An object $X$ of a Galois category $\calG$ is a \textbf{Galois object} if and only if it has no non-trivial automorphisms, i.e. if and only if $X/\Aut_{\calG}(X) \cong \pt$, with $\pt$ a terminal object of $\calG$.\footnote{Note that Galois categories must have terminal objects, as they are finitely complete and terminal objects are nothing but the limit of the empty diagram (which is finite by virtue of containing no vertices and no edges).}
                    \item \textbf{(Galois functors):} A \textbf{Galois functor} is an exact functor $\Phi: \calG \to \calG'$ between Galois categories $(\calG, F), (\calG', F')$ which preserves connected objects and commute with the fibre functors in the following manner:
                        $$
                            \begin{tikzcd}
                            	\calG && {\calG'} \\
                            	& {\Fin\Sets}
                            	\arrow["F"', from=1-1, to=2-2]
                            	\arrow["{F'}", from=1-3, to=2-2]
                            	\arrow["\Phi", from=1-1, to=1-3]
                            \end{tikzcd}
                        $$
                \end{itemize}
            \end{definition}
            \begin{definition}[Fundamental groups of Galois categories] \label{def: fundamental_groups_of_galois_categories}
                The \textbf{fundamental group} of a given Galois category $(\calG, F)$, denoted by $\pi_1(\calG, F)$, is defined to be the automorphism group $\Aut(\Pro(F))$.
            \end{definition}
            
            \begin{proposition}[The Categorical Galois Correspondence] \label{prop: categorical_galois_correspondence}
                For every Galois category $(\calG, F)$, there an equivalence of categories (cf. \cite[Propositions 5.2]{SGA1}):
                    $$\calG \cong \pi_1(\calG, F)\-\Fin\Sets$$
                    $$Y \mapsto F(Y)$$
                Furthermore, one has the following equivalences characterising the relationship between subgroups of the fundamental group $\pi_1(\calG, F)$ and permutations of Galois covers in $\calG$ (cf. \cite[Propositions 5.5]{SGA1}):
                    $$\{\text{Finite-index subgroups of $\pi_1(\calG, F)$}\}^{\op} \cong \pi_1(\calG, F)\-\Fin\Sets$$
                    $$H \mapsto \pi_1(\calG, F)/H$$
                    $$\calG^{\Gal} \cong \{\text{Finite-index normal subgroups of $\pi_1(\calG, F)$}\}$$
            \end{proposition}
                
            \begin{definition}[Universal covers] \label{def: universal_covers}
                Let $(\calG, F)$ be a Galois category. A pro-object $\tilde{X} \in \Pro(\calG)$ is called a \textbf{universal cover} if and only if its fundamental group $\pi_1(\tilde{X}) \cong \Aut(\Pro(F)(\tilde{X}))$ is trivial (i.e. if and only if it is simply-connected).
            \end{definition}
            \begin{remark}[Fundamental groups are automorphism groups of universal covers] \label{remark: fundamental_groups_are_automorphism_groups_of_universal_covers}
                
            \end{remark}
            
            Let us now try to adapt definitions \ref{def: galois_categories} and \ref{def: fundamental_groups_of_galois_categories} to a appropriate categories of schemes, namely those spanned by schemes finite-\'etale over a given base.
            \begin{remark}[\'Etale vs. finite-\'etale] \label{remark: etale_vs_finite_etale}
                One crucial tehcnicality that we will need to keep in mind is that finite-\'etale morphisms are \'etale, but the converse need not be true (e.g. the affine line is \'etale but not at all finite). However, \'etale morphisms are indeed finite when the codomain is the spectrum of a field (this is not the only case where \'etale morphisms are finite-\'etale, but it is sufficient for us); a proof can easily derived from \cite[\href{https://stacks.math.columbia.edu/tag/00U3}{Tag 00U3}]{stacks}, which asserts that \'etale (commutative) algebras over a field $k$ are isomorphic to finite direct sums of finite separable extension of $k$. 
            \end{remark}
            \begin{remark}[Finite-\'etale schemes] \label{remark: finite_etale_schemes}
                For any given by scheme $X$, the small category $(\Sch_{/X})_{\fet}$ of finite-\'etale $X$-schemes is a category wherein:
                    \begin{itemize}
                        \item all finite limits and all finite colimits exist, and
                        \item all objects can be written as a (possibly empty) finite coproduct of connected objects, which happen to be schemes that are \'etale over $X$.  
                    \end{itemize}
                (for a detailed proof, see \cite[\href{https://stacks.math.columbia.edu/tag/0BN9}{Tag 0BN9}]{stacks}) so should we be able to define a fibre functor $(\Sch_{/X})_{\fet} \to \Fin\Sets$, we will have succeeded in putting a Galois category structure on $(\Sch_{/X})_{\fet}$. As a matter of fact, such a well-defined fibre functor has good reasons to exist: it is an easy consequence of \cite[\href{https://stacks.math.columbia.edu/tag/00U3}{Tag 00U3}]{stacks} that for any fixed geometric point $\bar{x} \in X$ (corresponding to an algebraic closure $\bar{\kappa}_x$ of the residue field of $x \in X$\footnote{Certain sources consider geometric points to correspond to separable closures. For us, however, geometric points are algebraically closed fields $K$ so that $\Spec K$ be a Galois object of $(\Sch_{/\Spec K})_{\fet}$ (cf. definition \ref{def: galois_categories}). In practice this choice usually does not matter, since we will mostly work over perfect field, and separable closures of perfect fields are algebraically closed.}), one has:
                    $$(\Spec \bar{\kappa}_x)_{\fet} \cong \Fin\Sets$$
                (the forward direct simply involves taking the underlying set, and the inverse functors is given by $I \mapsto \coprod_{i = 1}^{|I|} \Spec \bar{\kappa}_x$) and so for any $k$-scheme $X$, one has the following canonical defined functor:
                    $$(\Sch_{/X})_{\fet} \to (\Sch_{/\Spec \bar{\kappa}_x})$$
                    $$Y \mapsto Y_{\bar{x}}$$
                where $Y_{\bar{x}} \cong Y \x_X \Spec \bar{\kappa}_x$; one can then take the underlying set of $Y_{\bar{x}}$ to get the following trivially left-exact functor:
                    $$F_{\bar{x}}: (\Sch_{/X})_{\fet} \to \Fin\Sets$$
                    $$Y \mapsto |Y_{\bar{x}}|$$
                We should also verify that the sets $|Y_{\bar{x}}|$ are indeed finite. To this end, let us first apply the fact that pullbacks of \'etale morphisms are \'etale to see that if $Y$ is affine over $X$ then $Y_{\bar{x}}$ will have to be the spectrum of an \'etale $\bar{\kappa}_x$-algebra; however, according to \cite[\href{https://stacks.math.columbia.edu/tag/00U3}{Tag 00U3}]{stacks}, this means that $Y_{\bar{x}} \cong \Spec (\bar{\kappa}_x)^{\oplus N}$ for some finite $N$. The locality of \'etale-ness and the finiteness of $Y$ as an $X$-scheme then tells us that in general, $Y_{\bar{x}}$ must be a finite disjoint union of affine schemes of the form $\Spec (\bar{\kappa}_x)^{\oplus N}$, meaning that $Y_{\bar{x}} \cong \Spec (\bar{\kappa}_x)^{\oplus N'}$ for some finite $N'$. The set $|Y_{\bar{x}}|$ is therefore always finite. One also sees that an immediate consequence of this proof is that $F_{\bar{x}}$ necessarily \textit{reflects isomorphisms} and is \textit{right-exact}. 
                
                It thus remains to show that $F_{\bar{x}}$ is \textit{pro-representable}. For this, observe first of all that as a functor on $\Sch_{/X}$ (as opposed to a functor on $(\Sch_{/X})_{\fet}$), $F_{\bar{x}}$ is naturally isomorphic to $\Sch_{/X}(\bar{x}, -)$. Then, we get the pro-representability of $F_{\bar{x}}$ from the fact that $\bar{x}$ is a pro-object of $(\Sch_{/\Spec \kappa_x})_{\fet}$.
                
                We have thus constructed a well-defined fibre functor, in the sense of definition \ref{def: galois_categories}:
                    $$F_{\bar{x}}: (\Sch_{/X})_{\fet} \to \Fin\Sets$$
                    $$Y \mapsto |Y_{\bar{x}}|$$
            \end{remark}
            \begin{remark}[Finite \'etale Galois schemes] \label{remark: galois_schemes}
                Fix a base scheme $X$, and thanks to the fact that objects of Galois categories ($(\Sch_{/X})_{\fet}$ in this instance) can be written as finite coproducts of connected objects, we can assume without loss of generality that $X$ is connected. By definition \ref{def: galois_categories}, a Galois object in $(\Sch_{/X})_{\fet}$ is a finite-\'etale $X$-scheme $Y$ such that $Y/\Aut_X(Y) \cong X$. 
            \end{remark}
            \begin{definition}[\'Etale fundamental groups] \label{def: etale_fundamental_groups}
                For any scheme $X$ with a fixed geometric point $\bar{x}$, the pair $((\Sch_{/X})_{\fet}, F_{\bar{x}})$ as in remark \ref{remark: finite_etale_schemes} defines a Galois category. Its fundamental group is commonly denoted by $\pi_1(X_{\fet}, \bar{x})$ and called the \textbf{\'etale fundamental group} of $X$ based at $\bar{x}$.
            \end{definition}
            \begin{remark}[\'Etale fundamental groups as automorphism groups of universal covers] \label{remark: etale_fundamental_groups_as_automorphism_groups_of_universal_covers}
                In remark \ref{remark: finite_etale_schemes}, we have implicitly shown that the full subcategory $(\Sch_{/X})_{\fet}^{\Gal}$ is a diagram in $(\Sch_{/X})_{\fet}$ such that  
            \end{remark}
            \begin{remark}[The Geometric Galois Correspondence] \label{remark: geometric_galois_correspondence}
                Let $(X, \bar{x})$ be a pointed connected scheme. Then by proposition \ref{prop: categorical_galois_correspondence}, there is an equivalence of categories:
                    $$\{\text{Finite-index subgroups of $\pi_1(X_{\fet}, \bar{x})$}\}$$
                    $$\cong$$
                    $$\{\text{Finite \'etale $X$-schemes $Y$ with base points $\bar{y}$ lying over $\bar{x}$}\}$$
                Furthermore (and also thanks to proposition \ref{prop: categorical_galois_correspondence}), this equivalence restricts down to:
                    $$\{\text{Finite-index normal subgroups of $\pi_1(X_{\fet}, \bar{x})$}\}$$
                    $$\cong$$
                    $$\{\text{Finite \'etale Galois $X$-schemes $Y$ with base points $\bar{y}$ lying over $\bar{x}$}\}$$
                As a consequence, should $H$ be a finite-index normal subgroup of $\pi_1(X_{\fet}, \bar{x})$ and $(X^H, \bar{x}^H)$ be the corresponding Galois $X$-scheme with a choice of base point $\bar{x}^H$ lying over $\bar{x}$, then $\pi_1(X^H_{\fet}, \bar{x}^H) \cong H$. 
            \end{remark}
            \begin{example}[The \'etale fundamental group of a field] \label{example: etale_fundamental_group_of_a_field}
                As a sanity check, note that if $K$ is a field then finite-\'etale Galois schemes over $\Spec K$ shall be of the form $\Spec L \to \Spec K$, where $L/K$ is a finite Galois extension, and as a consequence, there are there are the following equivalences of lattices, which demonstrate that remark \ref{remark: geometric_galois_correspondence} directly generalises the classical Galois Correspondence:
                    $$\{\text{Finite-index normal subgroups of $\pi_1((\Spec K)_{\fet})$}\}$$
                    $$\cong$$
                    $$\{\text{Finite \'etale Galois schemes over $\Spec K$}\}$$
                    $$\cong$$
                    $$\{\text{Finite Galois extensions of $K$}\}^{\op}$$
                    $$\cong$$
                    $$\{\text{Finite-index normal subgroups of $\Gal(\bar{K}/K)$}\}$$
            \end{example}
            \begin{example}[The \'etale fundamental group of a curve] \label{example: etale_fundamental_group_of_a_curve}
                Let $k$ be a field. If $X$ is a connected non-singular projective curve over $\Spec k$ with function field $K$, then there is a canonical equivalence $({}^{K/}\Fld^{\fin, \Gal})^{\op} \cong (\Sch_{/X})_{\fet}^{\Gal}$ between the lattice of finite Galois extensions of $K$ and Galois $X$-schemes (which are precisely dominant rational maps whose associated function field extensions are Galois). Through this, it is easy to see that:
                    $$\pi_1(X_{\fet}) \cong \Gal(\bar{K}/K)$$
                For instance, we have:
                    $$\pi_1((\P^1_k)_{\fet}) \cong \Gal(\bar{k}/k)$$
                (since the function field of $\P^1_k$ is $k(t)$), which tells us that $\P^1_k$ is simply \'etale-connected if and only if $k$ is algebraically closed (since $\Gal(\bar{k}/k)$ is \textit{a fortiori} trivial in that case). 
            \end{example}
        
        \subsubsection{Properties of \'etale fundamental groups}
            Now, let us make sure that the \'etale fundamental group $\pi_1(X_{\fet}, \bar{x})$ as defined in definition \ref{def: etale_fundamental_groups} is meaningful as a formal construction. Namely, we would like to know the behaviours of $\pi_1(X_{\fet}, \bar{x})$ when we change the base point and when we base-change (cf. proposition \ref{prop: etale_fundamental_groups_do_not_depend_on_base_points}), as well as whether or not \'etale fibrations induce homotopy exact sequences of fundamental groups (cf. proposition \ref{prop: etale_homotopy_exact_sequence}). 
            \begin{proposition}[\'Etale fundamental group do not depend on base points] \label{prop: etale_fundamental_groups_do_not_depend_on_base_points}
                \cite[\href{https://stacks.math.columbia.edu/tag/0BQA}{Tag 0BQA}]{stacks} Let $f: Y \to X$ be a morphism of connected qcqs\footnote{quasi-compact and quasi-separated} schemes such that the base change functor $X' \mapsto X' \x_X Y$ is an equivalence of Galois categories between $(\Sch_{/X})_{\fet}$ and $(\Sch_{/Y})_{\fet}$. Then, for any choice of geometric points $\bar{x} \in X$ and $\bar{y} \in Y$, one has the following isomorphism of \'etale fundamental groups $\pi_1(X_{\fet}, \bar{x}) \cong \pi_1(Y_{\fet}, \bar{y})$.
            \end{proposition}
            \begin{corollary}[Uniqueness of \'etale fundamental groups] \label{coro: etale_fundamental_group_uniqueness}
                For any connected qcqs scheme $X$ and any pair of possibly distinct geometric points $\bar{x}, \bar{x}' \in X$, one has any isomorphism of \'etale fundamental groups $\pi_1(X_{\fet}, \bar{x}) \cong \pi_1(X_{\fet}, \bar{x}')$, and therefore it makes sense to only speak of \textit{the} fundamental group of $X$, which we shall denote by $\pi_1(X_{\fet})$.
            \end{corollary}
            
            \begin{proposition}[The \'etale homotopy exact sequence] \label{prop: etale_homotopy_exact_sequence}
                \cite[\href{https://stacks.math.columbia.edu/tag/0C0J}{Tag 0C0J}]{stacks} Let $X$ be a connected scheme. If $f: Y \to X$ be a flat proper morphism of finite presentation whose geometric fibres $Y_{\bar{x}}$ are connected and reduced, then for any geometric point $\bar{x} \in X$, there exists a right-exact sequence of groups as follows:
                    $$\pi_1((Y_{\bar{x}})_{\fet}) \to \pi_1(Y_{\fet}) \to \pi_1(X_{\fet}) \to 1$$
            \end{proposition}
    
    \subsection{\texorpdfstring{$\ell$}{}-adic sheaves and Grothendieck's Galois Theory}
        \subsubsection{Artin-Rees categories and adic sheaves}
            \begin{definition}[Artin-Rees categories] \label{def: artin_rees_categories}
                The \textbf{Artin-Rees category} associated to an abelian category $\calA$ is the full subcategory of $\Pro(\calA)$ spanned by cofiltered diagrams $\{M_n\}_{n \in \Z}$; we denote it by $\calA_{\bullet}^{\AR})$. Of particular interest are the so-called \textbf{null systems}, which are objects $\{M_n\}_{n \in \Z} \in \calA_{\bullet}^{\AR})$ such that there exists $\nu \in \N$ so that for all $n \in \Z$ the morphism $M_n \to M_{n + \nu}$ is zero.
            \end{definition}
            
            \begin{proposition}[Artin-Rees categories are abelian] \label{prop: artin_rees_categories_are_abelian}
                For any abelian category $\calA$, the associated Artin-Rees category $\calA_{\bullet}^{\AR})$ is also abelian, with zero objects being the null systems.
            \end{proposition}
            \begin{corollary}[AR-isomorphisms] \label{coro: AR_isomorphisms}
                For any abelian category $\calA$, an isomorphism in $\calA_{\bullet}^{\AR})$ (henceforth referred to as an \textbf{AR-isomorphism}) is a morphism in $\Pro(\calA)$ whose kernel and cokernel are null.
            \end{corollary}
            \begin{proposition}[Artin-Rees categories are tensor categories] \label{prop: artin_rees_categories_are_linear}
                For all rings $\Lambda$, if a given abelian category $\calA$ is (locally finite) $\Lambda$-linear\footnote{I.e. if hom-sets of $\calA$ are (locally finite) $\Lambda$-modules.} for some commutative ring $\Lambda$ (e.g. $\calA \cong \Lambda\mod$) then the associated Artin-Rees category $\calA_{\bullet}^{\AR})$ will also be a (locally finite) $\Lambda$-linear abelian category. Furthermore, if $\calA$ is (closed) symmetric monoidal then $\calA_{\bullet}^{\AR})$ will also be (closed) symmetric monoidal (with respect to term-wise tensor products). 
            \end{proposition}
                \begin{proof}
                    The first assertion is \cite[Example 1.4.1.6]{conrad_etale_cohomology}.
                \end{proof}
            
            \begin{convention}[The setting for adic sheaves] \label{conv: l_adic_sheaves_conventions}
                For our purposes, $\calX$ shall be a scheme that is locally of finite type\footnote{Althought $\calX$ might actually be an algebraic stack of finite type over $S$; for details, see \cite{laszlo_olsson_adic_sheaves_on_artin_stacks_1} and \cite{laszlo_olsson_adic_sheaves_on_artin_stacks_2}. It should also be noted that in \cite[Subsection 1.4]{conrad_etale_cohomology}, it was only required that $\calX$ would be Noetherian, which is not sufficient for us, as $\Bun_{\GL_1}(X)$ is merely locally of finite type, and hence only locally Noetherian \textit{a priori}.} over a base scheme $S$, which we take to be is affine, regular, Noetherian\footnote{Note that this implies that $\calX$ is locally Noetherian (cf. \cite[\href{https://stacks.math.columbia.edu/tag/01T6}{Tag 01T6}]{stacks}).} and of dimension $\leq 1$, and of characteristic $p \geq 0$; moreoever, we would like to work under the assumption that every finite-type $S$-scheme $T$ is also of finite cohomological dimension. In addition, $\Lambda$ shall be a discrete valuation ring with maximal ideal $\m$, residue characteristic $\ell \not = p$ (e.g. $\Lambda \cong \Z_{\ell}$), and fraction field $K$.
            \end{convention}
            
            \begin{definition}[Torsion objects in tensor categories] \label{def: torsion_objects_in_tensor_categories}
                Let $A$ be a commutative ring, $I$ be an ideal of $A$, and $(\calA, \tensor, \1)$ be an $A$-linear clsoed monoidal category. Then, the subcategory of $\calA$ spanned by $I$-torsion objects is the one wherein the hom-sets are $\Hom_{\calA/I}(M, N) \cong A/I \tensor_A \Hom_\calA(M, N)$.
            \end{definition}
            \begin{definition}[Adic objects and lisse objects of Artin-Rees categories] \label{def: adic_objects_and_lisse_objects_of_artin_rees_categories}
                Consider the Artin-Rees category $\calA_{\bullet}^{\AR}$ associated to a locally finite $\Lambda$-linear closed monoidal category $(\calA, \tensor, \1)$. 
                    \begin{enumerate}
                        \item \textbf{(Adic objects):} $\calA_{\bullet}^{\AR}$ admits a full subcategory, denoted by $\calA_{\bullet}^{\ad}$, whose objects $\{M_n\}_{n \in \Z}$ are such that:
                            \begin{itemize}
                                \item $M_n \cong 0$ for all $n < 0$,
                                \item $M_n$ is $\m^{n + 1}$-torsion for all $n \geq 0$, and
                                \item the canonical maps $\Lambda/\m^{n + 2} \tensor_{\Lambda} \Hom_{\calA_{\bullet}^{\AR}}(M_{\bullet}, N_{\bullet}) \to \Hom_{\calA_{\bullet}^{\AR}}(M_{\bullet}, N_{\bullet})/\m^n$ are isomorphisms of $\Lambda/\m^{n + 1}$-modules for all $n \geq 0$.
                            \end{itemize}
                        Objects of this full subcategory are said to be \textbf{$\m$-adic}.
                        \item \textbf{(Lisse objects):} Let $\calA_{\bullet}^{\fin}$ denote the category of Artin-Rees projective systems of finite (i.e. Artinian and Noetherian) objects of $\calA$. Objects of the category $\calA_{\bullet}^{\lisse} := \calA_{\bullet}^{\ad} \cap \calA_{\bullet}^{\fin}$ are then said to be \textbf{lisse}. 
                    \end{enumerate}
            \end{definition}
            \begin{proposition}[Adic categories are tensor categories] \label{prop: adic_categories_are_tensor_categories}
                Let $\calA$ be a locally finite $\Lambda$-linear closed monoidal category. Then, we shall have a tower $\calA_{\bullet}^{\lisse} \subset \calA_{\bullet}^{\ad} \subset \calA_{\bullet}^{\AR}$ of locally finite $\Lambda$-linear closed monoidal categories, wherein the inclusions are fully faithful monoidal functors.
            \end{proposition}
            \begin{example}[Adic and lisse $\Lambda$-sheaves] \label{example: adic_sheaves}
                Recall first of all that the category of constructible \'etale sheaves of $\Lambda$-modules on a Noetherian scheme (of which $\Lambda\mod^{\cons}(\calX_{\et})$ is a special case) is a locally finite $\Lambda$-linear closed monoidal category (to see why, combine \cite[Propositions 3.20 and 3.22]{behrend_l_adic_sheaves_for_algebraic_stacks}). Then, the category of adic constructible $\m$-adic sheaves on $\calX$ (also called constructible $\Lambda$-sheaves), commonly denoted by $\Shv_{\Lambda}^{\ad}(\calX)$, is nothing but $\Lambda\mod^{\cons}(\calX_{\et})_{\bullet}^{\ad}$, and the category of lisse $\m$-adic sheaves on $\calX$, denoted by $\Shv_{\Lambda}^{\lisse}(\calX)$ is simply $\Lambda\mod^{\cons}(\calX_{\et})_{\bullet}^{\lisse}$.
            \end{example}
            
            \begin{theorem}[Galois representations are lisse $\bar{\Q}_{\ell}$-sheaves] \label{theorem: galois_representations_are_lisse_sheaves}
                \cite[Theorem 1.4.5.7]{conrad_etale_cohomology} Let $\calX$ be a connected Noetherian scheme and fix a geometric point $\bar{x} \in \calX$. Then, there exists a monoidal equivalence given by $\calF \mapsto \calF_{\bar{x}}$, from the symmetric monoidal category $\Shv_{\bar{\Q}_{\ell}}^{\lisse}(\calX)$ of lisse $\bar{\Q}_{\ell}$-sheaves to the symmetric monoidal category $\Rep_{\bar{\Q}_{\ell}}^{\cont, \fin}(\pi_1(\calX_{\fet}))$ of finite-dimensional continuous $\bar{\Q}_{\ell}$-representations of $\pi_1(X_{\fet})$.
            \end{theorem}
            
        \subsubsection{Pullbacks and pushforwards}
            
    
    \section{The Automorphic Side}
    \subsection{Symmetric powers of curves, Jacobians, and the Abel-Jacobi map}
        \begin{convention}[The Picard group] \label{conv: picard_group}
            For any base scheme $S$ and any $S$-scheme $Y$, we shall write $|\Pic_{Y/S}|$ for the group of isomorphism classes of line bundles on $Y$, whose group structure is given by tensor products of invertible quasi-coherent $\calO_{Y/S}$-modules.
        \end{convention}
        \begin{definition}[Divisors] \label{def: divisors}
            Let $Y$ be a scheme. An \textbf{effective (Cartier) divisor}\footnote{Historically referred to as a \say{modulus}.} on $Y$ is then a closed subscheme $D \subset Y$ whose ideal sheaf $\calI_D$ is a line bundle on $Y$. Its \textbf{degree} is the \href{https://stacks.math.columbia.edu/tag/0AYQ}{\underline{degree}} of the line bundle $\calI_D$.  
                
            The set of effective divisors (respectively, effective divisors of degree $d$) shall be denoted by $|\Div_Y^{\eff}|$ (respectively, $|\Div_Y^{\eff, (d)}|$), and as a straightforward consequence of the definition of effective divisors, it is precisely the set of invertible\footnote{A quasi-coherent ideal sheaf $\calJ \subset \calO_Y$ is invertible if and only if its local sections $\calJ(V)$ are principal ideals of $\calO_Y(V)$.} quasi-coherent $\calO_Y$-ideals (respectively, quasi-coherent $\calO_Y$-ideals of degree $d$).
        \end{definition}
        \begin{remark}[Why do we care about divisors ?] 
            Because the function field of the curve $X$ over $\Spec k$ from convention \ref{conv: base_curve} is some (global) field of the form $K \cong k'(t)$, where $k'/k$ is an algebraic extension (cf. proposition \ref{prop: curves_and_function_fields}), and since its ring of integers $\scrO_K \cong k'[t]$ is a Dedekind domain (this is due to Hilbert's Basis Theorem, which tells us that $\dim k'[t] = \dim k + 1 = 0 + 1 = 1$), every of the stalks $\calO_{X, x}$ at points $x \in |X|$ must be a discrete valuation ring. This tells us that there is a bijective correspondence between points $x \in |X|$ and places of $K$ that are trivial on $k'$. Now, also because $\scrO_K$ is a Dedekind domain, every ideal $\a$ therein factors into a (formal) product of primes, say $\a = \prod_{i = 1}^n \p_i^{e_i}$. Furthemore, $\scrO_K$ is actually a PID, as it is isomorphic to $k'[t]$, which is a UFD thanks to $k'$ itself being a UFD, by virtue of being a field. Thus, points $x \in |X|$ are not only in bijection with places of $K$ that are trivial on $k'$, but also effective divisors on $X$.
        \end{remark}
        
        \begin{convention}
            Given any effective divisor $D \subset Y$, any integer $n$, and any $\E \in \QCoh_X$, let us write $\E(nD) \cong \E \tensor_{\calO_Y} \calI_D^{\tensor (-n)}$ (wherein $\calI_D^{\tensor (-n)} \cong (\calI_D^{\tensor (-1)})^{\tensor n}$). In particular, note that $\calO_Y(-D) \cong \calI_D$.
        \end{convention}
        \begin{remark}[Addition of effective divisors] \label{remark: adding_effective_divisors}
            Let $Y$ be a scheme. Because effective divisors are line bundles, they are trivially flat\footnote{Given any line bundle $\calL \in |\Pic_Y|$, the functor $- \tensor_{\calO_Y} \calL$ is an auto-equivalence of $\Coh_Y$, which is an abelian category, so $- \tensor_{\calO_Y} \calL$ is automatically (left-)exact.}, so given any pair of effective divisors $D, D' \subset Y$, one can define a new effective divisor $D + D'$ corresponding to the $\calO_Y$-ideal $\calI_D \calI_{D'}$, which is isomorphic to $\calI_D \tensor_{\calO_{X/k}} \calI_{D'}$ due to flatness, and hence $\deg(D + D') = \deg D + \deg D'$.
        \end{remark}
        
        \begin{convention}[The setting for geometric class field theory] \label{conv: automorphic_side_conventions}
            \noindent
            \begin{itemize}
                \item For us, $\Bun_{\GL_1}(X)$ shall denote the moduli space of line bundles on $X$. Traditionally, this is usually referred to as the \textbf{Picard stack} and denoted by $\Pic_{X/k}$ (cf. \cite[\href{https://stacks.math.columbia.edu/tag/0372}{Tag 0372}]{stacks}), but we opt for the notation $\Bun_{\GL_1}(X)$ because in the wider context of the Geometric Langlands Programme, one works with $\Bun_G(X)$ for $G$ a general connected reductive group (of which $\GL_1$ is a special case), and to not confuse the moduli space $\Bun_{\GL_1}(X)$ with the group $|\Pic_{X/k}|$ (cf. convention \ref{conv: picard_group}). 
                \item In addition, let us now suppose that the base field $k$ from convention \ref{conv: base_curve} is algebraically closed, and that our curve $X$ from convention \ref{conv: base_curve} is, in addition, geometrically connected. 
            \end{itemize}
        \end{convention}
        \begin{remark}[The geometry of $\Bun_{\GL_1}$] \label{remark: geometry_of_the_picard_stack}
            Since we are working with a smooth projective curve $X$ over an algebraically closed field (cf. convention \ref{conv: automorphic_side_conventions}) and hence over a separably closed field, the prestack $\Bun_{\GL_1}$ is \textit{a priori} represented by a scheme (cf. \cite[\href{https://stacks.math.columbia.edu/tag/0B9Z}{Tag 0B9Z}]{stacks}) as an fppf sheaf on $X$ (and hence as an \'etale and as a Zariski sheaves, since the fppf toppology is finer than both these topologies). As a result, when considering sheaves on $\Bun_{\GL_1}(X)$, we will only need to know about sheaves on schemes instead of the entire fully general theory of sheaves on prestacks. Furthermore, if the genus of $X$ is $g \geq 0$, then one will have the following decomposision:
                $$\Bun_{\GL_1}(X) \cong \coprod_{d \geq 0} \Bun_{\GL_1}^{(d)}(X)$$
            wherein each $\Bun_{\GL_1}^{(d)}(X)$ is the moduli scheme of line bundles of degree $d$ on $X$, which is a proper smooth variety of dimension $g$ over $\Spec k$ (cf. \cite[\href{https://stacks.math.columbia.edu/tag/0BA0}{Tag 0BA0}]{stacks}).
        \end{remark}
        
        \begin{remark}[Moduli space of effective divisors] \label{remark: moduli_space_of_effective_divisors}
            For any base scheme $S$ and any $S$-scheme $Y$, the \textbf{Hilbert functor} of closed subschemes of degree $d \geq 0$ is the presheaf:
                $$\Hilb_{Y/S}^{(d)}: \Sch_{/S}^{\op} \to \Sets$$
                $$T \mapsto \{\text{Finite locally free closed subschemes $D \subset Y_T$ of degree $d$}\}$$
            Should $Y$ be a geometrically irreducible smooth proper (respectively projective) curve over a field $C$ then interestingly, not only are finite locally free closed subschemes $D \subset Y_{C'}$ of degree $d$ precisely the effective divisors of degree $d$ on $Y_{C'}$ for any field extension $C'/C$ (cf. \cite[\href{https://stacks.math.columbia.edu/tag/0B9D}{Tag 0B9D}]{stacks}), but also, one has a bijection:
                $$|\Div_{Y_{C'}/C'}^{\eff, (d)}| \cong \Hilb_{Y/C}^{(d)}(C')$$
            between the set of $C'$-rational points of $\Hilb_{Y/C}^{(d)}$ and that of degree-$d$ effective divisors on $Y_{C'}$ (cf. \cite[\href{https://stacks.math.columbia.edu/tag/0B9I}{Tag 0B9I}]{stacks}). From this, one see that should $Y$ be proper (respectively Zariski-locally projective) and flat over some arbitrary base scheme $S$, and if its fibres $Y_s$ over points $s \in |S|$ are geometrically irreducible smooth proper (respectively projective) curves, then $\Hilb_{Y/S}^{(d)}$ would be the moduli space of degree-$d$ effective divisors on $Y$; thus, for proper (respectively Zariski-locally projective) and flat morphisms $Y \to S$, let us suggestively write $\Div_{Y/S}^{\eff, (d)}$ instead of $\Hilb_{Y/S}^{(d)}$. It is known moreover that $\Hilb_{Y/C}^{(d)}$ is represented by a smooth proper variety of dimension $d$ over $\Spec C$ (cf. \cite[\href{https://stacks.math.columbia.edu/tag/0B9I}{Tag 0B9I}]{stacks}). By putting everything together, one obtains a moduli space $\Div_{Y/C}^{\eff, (d)} \in (Y/C)_{\fppf}$ parametrising degree-$d$ effective divisors on $Y$, represented by a smooth proper variety of dimension $d$ over $\Spec C$ and naturally isomorphic to $\Hilb_{Y/C}^{(d)}$.
        \end{remark}
        
        It should also be noted that what we have just discussed is not the only way to show that $\Div_{X/k}^{\eff, (d)}$ is represented by a smooth proper variety of dimension $d$. Remark \ref{remark: moduli_space_of_effective_divisors} serves more as a demonstration that there \textit{should} be a moduli space of effective divisors (of a given degree $d$), rather than that there \textit{is} one. In fact, by combining remark \ref{remark: quotients_of_schemes_by_finite_group_schemes}, lemma \ref{lemma: smoothness_of_symmetric_powers}, and proposition \ref{prop: symmetric_powers_of_curves_parametrise_divisors}, we shall see that the functor $\Div_{X/k}^{\eff, (d)}$ is represented by a smooth proper variety of (pure) dimension $d$ by virtue of being naturally isomorphic to the functor of points of $X^{(k)}$ (which, of course, is smooth, proper, and of dimension $d$). It is, however, important to know that $\Div_{X/k}^{\eff, (d)}$ indeed satisfies fppf descent (hence \'etale descent) to prove proposition \ref{prop: symmetric_powers_of_curves_parametrise_divisors}, and since this comes from the general fact that the Hilbert functor satisfies fppf descent, remark \ref{remark: moduli_space_of_effective_divisors} remains necessary. 
        
        We now know that effective divisors of a given degree $d$ are parametrised by some smooth proper variety $\Div_{Y/S}^{\eff, (d)}$ of (relative) dimension $d$, but this is not entirely satisfactory: we would also like to know the identity of this smooth proper variety, and we shall after proposition \ref{prop: symmetric_powers_of_curves_parametrise_divisors}.
        \begin{remark}[Quotient of schemes by finite group schemes] \label{remark: quotients_of_schemes_by_finite_group_schemes}
            For details on quotients of schemes by finite groups, we refer the reader to \cite[Expos\'e V]{SGA1}. For our purposes, we shall only need to keep in mind the following facts: 
                \begin{enumerate}
                    \item Let $S$ be a base scheme and $Y$ be an $S$-scheme that is either \textit{(quasi-)projective or (quasi-)affine}, and if additionally. If $G$ be a \textit{finite, flat, and locally of finite presentation} group $S$-scheme acting \textit{freely}\footnote{This is to ensure that the $G$-action induces an fppf equivalence relation on $Y$, since the $G$-action on $Y$ is free if and only if the corresponding homomorphism of sheaves of groups $G \to \Aut_{S_{\fppf}}(Y)$ is injective.} on $Y$, then the \href{https://stacks.math.columbia.edu/tag/025X}{\underline{algebraic space}} $Y/G$ is a scheme (cf. \cite[\href{https://stacks.math.columbia.edu/tag/07S7}{Tag 07S7}]{stacks}). 
                    \item If $Y$ is a (quasi-)affine scheme $\Spec A$ then the quotient $Y/G$ will also be affine and will be isomorphic to $\Spec A^G$, thanks to the group-cohomological fact that $H^0(A, G) \cong A^G$. 
                    
                    If $Y$ is (quasi-)projective then $Y/G$ will also be (quasi-)projective.
                \end{enumerate}
        \end{remark}
        \begin{definition}[Symmetric powers of schemes] \label{def: symmetric_powers_of_schemes}
            Let $S$ be a base scheme and let $Y$ be an $S$-scheme that is either (quasi-)projective or (quasi-)affine\footnote{In particular, the curve $X$ from convention \ref{conv: base_curve} is projective.}. Then for any $d \geq 1$, the \textbf{$d^{th}$ symmetric power} of $Y$ is the quotient scheme $\Sym^d_S(Y) := Y^d/\underline{\Sigma_d}_{/S}$ of $Y$ by the constant symmetric $S$-group scheme $\underline{\Sigma_d}_{/S}$ on $d$ elements (which is finite, flat, and locally of finite presentation over $S$, since it is represented by $\coprod_{\sigma \in \Sigma_d} S$); here, $\underline{\Sigma_d}_{/S}$ acts via permutations, which is well-known to be a free action.
        \end{definition}
        \begin{convention}
            Because the base field $k$ of our curve from convention \ref{conv: base_curve} is fixed, let us write $X^{(d)}$ instead of $\Sym^d_k(X)$ for simplicity.
        \end{convention}
        \begin{lemma}[Smoothness of symmetric powers] \label{lemma: smoothness_of_symmetric_powers}
            If $Y$ is a smooth variety over some field $C$ of dimension $\leq 1$ then so is $\Sym_C^d(Y)$.
        \end{lemma}
            \begin{proof}
                Smoothness is preserved by base change, so we might as well assume that $C$ is algebraically closed. The case $\dim Y = 0$ is trivial, so let us assume that $\dim Y = 1$ (i.e. that $Y$ is a curve); in this case, the formal completion of the stalk $\calO_{Y, y}$ at any point $y \in |Y|$ is necessarily isomorphic to $C[\![y]\!]$\footnote{We are intentionally confusing the point $y \in |Y|$ and the formal variable $y \in C[\![y]\!]$ because $(y)$ is the unique maximal ideal of $C[\![y]\!]$.}, since $\calO_{Y, y}$ shall be a finite-type commutative algebra over an algebraically closed field; as a result, the formal completion of the stalk $\calO_{\Sym^d_C(Y), \vec{y}}$ at any point $\vec{y} \in \Sym^d_C(Y)$ is isomorphic to symmetric formal power series ring $k[\![y_1, ..., y_d]\!]^{\Sigma_d}$, which itself is isomorphic to $k[\![y_1, ..., y_d]\!]$ \textit{a priori}. A Noetherian local ring $(A, \m)$ is regular if and only if its $\m$-adic completion is regular, and if $A, B$ are finite-type regular commutative algebras over an algebraically closed field $C$ then $A \tensor_C B$ will also be regular as a $C$-algebra, so $\Sym_C^d(Y)$ will be smooth if $C[\![y]\!]$ is regular, which is definitely the case since $\dim C[\![y]\!] = \dim_C (y)/(y)^2 = 1$.
            \end{proof}
        \begin{proposition}[Symmetric powers of curves parametrise divisors] \label{prop: symmetric_powers_of_curves_parametrise_divisors}
            For each $d$, the moduli space $\Div_{X/k}^{\eff, (d)}$ is represented by the smooth variety $X^{(d)}$.
        \end{proposition}
            \begin{proof}
                Let us denote \textit{unordered} $d$-tuples by $\<x_1, ..., x_d\>$ and also, write $[x]$ for the divisor cut out by any closed point $x \in |X|$. Now, to begin, consider the following $\Sigma_d$-invariant function, which if shown to be bijective will demonstrate that $\Div_{X/k}^{\eff, (d)}$ is represented by the smooth variety $X^{(d)}$ via \'etale descent (cf. \cite[\href{https://stacks.math.columbia.edu/tag/024V}{Tag 024V}]{stacks}):
                    $$X^{(d)} \to |\Div_{X/k}^{\eff, (d)}|$$
                    $$\<x_1, ..., x_d\> \mapsto [x_1] + ... + [x_d]$$
                Now, $X$ is a geometrically connected smooth curve, so it is geometrically normal (cf. \cite[\href{https://stacks.math.columbia.edu/tag/056T}{Tag 056T}]{stacks}), which in turn implies that the stalk of its structure sheaf over its unique generic point is a normal local domain of Krull dimension $1$, hence a Dedekind domain (cf. \cite[\href{https://stacks.math.columbia.edu/tag/034X}{Tag 034X}]{stacks}). This implies that every divisor on $X$ splits into prime divisors (which correspond to closed points of $X$), and thus the function $\<x_1, ..., x_d\> \mapsto [x_1] + ... + [x_d]$ is surjective. Dedekind domains are special cases of UFDs, so we have also demonstrated that the function $\<x_1, ..., x_d\> \mapsto [x_1] + ... + [x_d]$ is injective.
            \end{proof}
        \begin{corollary}[$X \cong \Div_{X/k}^{\eff, (1)}$]
            $k$-rational points of $X$ are precisely the degree-$1$ effective divisors.
        \end{corollary}
        \begin{convention}
            From this point on, we shall write $\Div_{X/k}^{\eff, (d)}$ instead of $X^{(d)}$ whenever we would like to put emphasis on the fact that points of $X^{(d)}$ are effective divisors of degree $d$ on $X$ (such as in \ref{prop: the_unramified_abel_jacobi_map_is_a_projective_bundle}), and \textit{vice versa}, we shall write $X^{(d)}$ when symmetry is of importance, like in theorem \ref{theorem: unramified_abelian_geometric_class_field_theory}.
        \end{convention}
        
        \begin{definition}[The Abel-Jacobi map] \label{def: the_abel_jacobi_map}
            Let $Y$ be a geometrically connected smooth projective curve over some field $C$. Then, the \textbf{$d^{th}$ Abel-Jacobi map} associated to $Y/C$ is the morphism of smooth proper varieties:
                $$\AJ_{Y/C}^{(d)}: \Div_{Y/C}^{\eff, (d)} \to \Bun_{\GL_1}^{(d)}(Y)$$
            which section-wise (i.e. at each field extension $C'/C$) associates to each degree-$d$ effective divisor $D \in |\Div_{Y_{C'}/C'}^{\eff, (d)}|$ to its corresponding invertible quasi-coherent $\calO_{Y_{C'}}$-ideal $\calI_D \in |\Pic_{Y_{C'}/C'}^{(d)}|$.
        \end{definition}
        \begin{remark}[What does the Abel-Jacobi map do ?] \label{remark: abel_jacobi_map}
            
        \end{remark}
        \begin{convention}[Genus of the curve] \label{conv: genus_of_the_curve}
            From now on, denote the \href{https://stacks.math.columbia.edu/tag/0BY6}{\underline{genus}} of our curve $X$ by $g$.
        \end{convention}
        \begin{proposition}[The Abel-Jacobi map is a projective bundle] \label{prop: the_unramified_abel_jacobi_map_is_a_projective_bundle}
            If $d \geq 2g - 1$ then every Abel-Jacobi map $\AJ_{X/k}^{(d)}: \Div_{X/k}^{\eff, (d)} \to \Bun_{\GL_1}^{(d)}(X)$ will be a surjective smooth projective morphism with fibres\footnote{Note that these are precisely the geometric fibres, since $k$ is algebraically closed.} over $k$-rational points isomorphic to $\P^{d - g}_k$.
        \end{proposition}
            \begin{proof}
                Let us note, first of all, that thanks due to a descent-theoretic feature of the \'etale topology, namely \cite[\href{https://stacks.math.columbia.edu/tag/024V}{Tag 024V}]{stacks}, it shall suffice to demonstrate that the \textit{set-theoretic} fibres of the Abel-Jacobi maps are in bijection with the set of $k$-rational points of $\P_k^{d - g}$; for the same reason, it also suffices to only show that the Abel-Jacobi map is surjective at each point $k$-rational point $\calL \in |\Bun_{\GL_1}^{(d)}(X)|$. We shall proceed in steps, for the sake of clarity.
                    \begin{enumerate}
                        \item \textbf{(Projectivity):} Let us first apply the Riemann-Roch Theorem (cf. \cite[\href{https://stacks.math.columbia.edu/tag/0BS6}{Tag 0BS6}]{stacks}; note that the theorem is applicable to our situation because $X$ is a geometrically connected smooth projective curve and therefore a Gorenstein\footnote{Smooth schemes are Gorenstein because their stalks are regular local rings.} scheme of equidimension $1$ over a field), which tells us that should $\E \in \Vect_{X/k}^n$ be a be a locally free quasi-coherent $\calO_{X/k}$-module of constant rank $n$, then:
                            $$\chi(X, \E) = \deg(\E) - \frac12\rank(\E) \deg(\omega_{X/k})$$
                        where $\omega_{X/k}$ denotes the dualising sheaf (which is a line bundle due also to the Riemann-Roch Theorem) and $\chi(X, \E)$ denotes the Euler characteristic of $\E$ as a coherent sheaf on the proper $k$-scheme $X$. Because $\deg(\omega_{X/k}) = 2g - 2$ (thanks to \cite[\href{https://stacks.math.columbia.edu/tag/0C19}{Tag 0C19}]{stacks}, which is applicable in this situation because $X$ is a proper Gorenstein $k$-scheme such that $H^0_{\Zar}(X, \calO_{X/k}) \cong k$), the above tells us that for any degree-$d$ line bundle $\calL \in |\Bun_{\GL_1}^{(d)}(X)|$, we have:
                            $$\chi(X, \calL) = \deg(\calL) - \frac12 \rank(\calL) \deg(\omega_{X/k}) = d - \frac12 \cdot 1 \cdot (2g - 2) = d - g + 1$$
                        Line bundles are particular cases of coherent sheaves with support dimension $\leq 0$, and since $X$ is proper over a field, we can apply \cite[\href{https://stacks.math.columbia.edu/tag/0AYT}{Tag 0AYT}]{stacks} to get that:
                            $$\dim_k H^0_{\Zar}(X, \calL) = \chi(X, \calL) = d - g + 1$$
                        This implies that the (set-theoretic) fibres of $\AJ_{X/k}^{(d)}$ over $k$-rational points $\calL \in |\Bun_{\GL_1}^{(d)}(X)|$ are isomorphic to $\P^{d - g}_k$ (empty if $d < g$).
                        \item \textbf{(Smoothness):} $\Bun_{\GL_1}^{(d)}(X)$ is smooth (cf. remark \ref{remark: geometry_of_the_picard_stack}), hence it is regular, and $X^{(d)}$ is also smooth (cf. lemma \ref{lemma: smoothness_of_symmetric_powers}) and hence it is Cohen-Macaulay (because regular local rings are Cohen-Macaulay \textit{a priori}; cf. \cite[\href{https://stacks.math.columbia.edu/tag/00NQ}{Tag 00NQ}]{stacks}). Moreover, we have shown above that given any $\calL \in \Bun_{\GL_1}^{(d)}(X)$, the corresponding fibre of the Abel-Jacobi map is isomorphic to $\P^{d - g}_k$, and since $\dim \Bun_{\GL_1}^{(d)}(X) = g$ (cf. remark \ref{remark: geometry_of_the_picard_stack}) while $\dim X^{(d)} = \dim \Div_{X/k}^{\eff, (d)} = d$ (cf. remark \ref{remark: moduli_space_of_effective_divisors}), we have:
                            $$\dim \P^{d - g}_k = \dim (\AJ_{X/k}^{(d)})^{-1}(\calL) = \dim \Div_{X/k}^{\eff, (d)} - \dim \Bun_{\GL_1}^{(d)}(X) = d - g$$
                        The Miracle Flatness Theorem (cf. \cite[\href{https://stacks.math.columbia.edu/tag/00R4}{Tag 00R4}]{stacks}) can then be applied, which tells us that the Abel-Jacobi map is flat everywhere, and because the fibres are isomorphic to $\P^{d - g}_k$, which is smooth over $\Spec k$, this means that the Abel-Jacobi map is also smooth everywhere.
                        \item \textbf{(Surjectivity):} Finally, let us demonstrate that the canonical map $H^0_{\Zar}(X, \calL) \to H^0(X, \calL|_D)$ is surjective
                    \end{enumerate}
            \end{proof}
        \begin{corollary}[Unramified Galois representations induced by the Abel-Jacobi map] \label{coro: unramified_galois_representations_induced_by_the_abel_jacobi_map}
            Because $\AJ_{X/k}^{(d)}$ is proper and smooth, it is proper, flat, and of finite presentation, so by proposition \ref{prop: etale_homotopy_exact_sequence} there is an induced \'etale homotopy sequence as follows:
                $$\pi_1((\P^{d - g}_k)_{\fet}) \to \pi_1((\Div_{X/k}^{\eff, (d)})_{\fet}) \to \pi_1((\Bun_{\GL_1}^{(d)}(X))_{\fet}) \to 1$$
            Since $\P^{d - g}_k$ is \'etale-simply connected (this is a consequence of $k$ being algebraically closed; cf. example \ref{example: etale_fundamental_group_of_a_curve}), one thus obtains an equivalence between the categories of continuous $\ell$-adic characters of $\pi_1((\Div_{X/k}^{\eff, (d)})_{\fet})$ and of $\pi_1((\Bun_{\GL_1}^{(d)}(X))_{\fet})$ as below, wherein $(\AJ_{X/k}^{(d)})^*$ is the pullback of $\ell$-adic sheaves along the Abel-Jacobi map $\AJ_{X/k}^{(d)}: \Div_{X/k}^{\eff, (d)} \to \Bun_{\GL_1}^{(d)}(X)$:
                $$\Rep^1_{\bar{\Q}_{\ell}}(\pi_1((\Div_{X/k}^{\eff, (d)})_{\fet})) \cong \Rep^1_{\bar{\Q}_{\ell}}(\pi_1((\Bun_{\GL_1}^{(d)}(X))_{\fet}))$$
                $$\chi \mapsto \chi \circ (\AJ_{X/k}^{(d)})^*$$
        \end{corollary}
        \begin{definition}[Prime divisors and Weil divisors] \label{def: prime_divisors_and_weil_divisors}
            For our purposes, a \textbf{prime divisor} on $X$ (as in convention \ref{conv: base_curve}) shall be a closed point, and a \textbf{Weil divisor} shall be a formal finite linear combination of prime divisors. A Weil divisor is said to be \textbf{tame} if its coefficients are all equal to either $0$ or $1$ (i.e. if it is either empty or a pure sum of prime divisors).
        \end{definition}
        \begin{convention}[Line bundles with prescribed trivialisations] \label{conv: line_bundles_with_prescribed_trivialisations}
            If $D$ is a tame Weil divisor on $X$ then we shall (abusively) denote by $\Bun_{\GL_1}(X \setminus D)$ the moduli space of line bundles $\calL \in \Bun_{\GL_1}(X)$ with a trivialisation $\calL|_D \cong \calO_X|_D$.
        \end{convention}
        \begin{proposition}[The tamely ramified Abel-Jacobi map is a vector bundle] \label{prop: the_tamely_ramified_abel_jacobi_map_is_a_vector_bundle}
            Let $D$ be a tame Weil divisor on $X$. If $d \geq 2g - 1$ then every Abel-Jacobi map $\AJ_{(X \setminus D)/k}^{(d)}: \Div_{(X \setminus D)/k}^{\eff, (d)} \to \Bun_{\GL_1}^{(d)}(X \setminus D)$ will be a vector bundle with fibres over $k$-rational points being isomorphic to $\A^{(d - g + 1) - \deg D}_k$.
        \end{proposition}
            \begin{proof}
                
            \end{proof}
        \begin{corollary}[Tamely ramified Galois representations induced by the Abel-Jacobi map] \label{coro: tamely_ramified_galois_representations_induced_by_the_abel_jacobi_map}
            Like in corollary \ref{coro: unramified_galois_representations_induced_by_the_abel_jacobi_map}, we can make use of proposition \ref{prop: etale_homotopy_exact_sequence} (along with the fact that finite-dimensional affine spaces are \'etale-simply connected in characteristic $0$; cf. example \ref{example: etale_fundamental_group_of_the_affine_line}) to show that for $D$ any tame Weil divisor on $X$, there an equivalence between the categories of continuous $\ell$-adic characters of $\pi_1((\Div_{(X \setminus D)/k}^{\eff, (d)})_{\fet})$ and of $\pi_1((\Bun_{\GL_1}^{(d)}(X \setminus D))_{\fet})$ as below, wherein $(\AJ_{(X \setminus D)/k}^{(d)})^*$ is the pullback of $\ell$-adic sheaves along the Abel-Jacobi map $\AJ_{(X \setminus D)/k}^{(d)}: \Div_{(X \setminus D)/k}^{\eff, (d)} \to \Bun_{\GL_1}^{(d)}(X \setminus D)$:
                $$\Rep^1_{\bar{\Q}_{\ell}}(\pi_1((\Div_{(X \setminus D)/k}^{\eff, (d)})_{\fet})) \cong \Rep^1_{\bar{\Q}_{\ell}}(\pi_1((\Bun_{\GL_1}^{(d)}(X \setminus D))_{\fet}))$$
                $$\chi \mapsto \chi \circ (\AJ_{(X \setminus D)/k}^{(d)})^*$$
        \end{corollary}
    
     \subsection{Hecke eigensheaves and geometric class field theory}
        \begin{convention}[Symmetric powers of line bundles] \label{conv: symmetric_powers_of_line_bundles}
            \noindent
            \begin{itemize}
                \item Write $\sigma^{(d)}: X^d \to X^{(d)}$ for the canonical quotient map and set $\Delta_X^{(d)} := \sigma^{(d)} \circ \Delta_X^d$. Next, for each $\ell$-adic local system $\calL \in \Shv_{\underline{\bar{\Q}_{\ell}}}(X)$, we can construct an $\ell$-adic local system $\calL^{(d)} \in \Shv_{\underline{\bar{\Q}_{\ell}}}(X^{(d)})$ given by $\calL^{(d)} \cong ((\Delta_X^{(d)})_*\calL)^{\Sigma_d}$. 
                \item In addition, write $\tilde{h}_X^{(d)}: X \x X^{(d)} \to X^{(d + 1)}$ for the map given by $(-[x], D) \mapsto [x] + D$, and likewise, write $\cev{h}_X^{(d)}: X \x \Bun_{\GL_1}^{(d)}(X) \to \Bun_{\GL_1}^{(d + 1)}(X)$ for the map given by $\cev{h}_X^{(d)}(-[x], \E) \cong \E(-[x])$. Also, let the map $\cev{H}_X^{(d)}: \Bun_{\GL_1}^{(1)}(X) \x \Bun_{\GL_1}^{(d)}(X) \to \Bun_{\GL_1}^{(d + 1)}(X)$ be given by $\cev{H}_X^{(d)}(\calL', \calL) \cong \calL' \tensor_{\calO_X} \calL$.
            \end{itemize}
        \end{convention}
        \begin{definition}[Hecke eigensheaves] \label{def: hecke_eigensheaves}
            A \textit{non-zero} $\ell$-adic local system $\E \in \Shv_{\underline{\bar{\Q}_{\ell}}}^{\ad, 1}(\Bun_{\GL_1}(X))$ is called a \textbf{Hecke eigensheaf} (of rank $1$) if and only if there exists an $\ell$-adic sheaf $\calL \in \Shv_{\underline{\bar{\Q}_{\ell}}}^{\ad, 1}(X)$ (called the \textbf{eigenvalue} of $\E$) such that:
                $$\cev{h}_X^*(\E) \cong \calL \boxtimes \E$$
        \end{definition}
        \begin{remark}
            It is easy to see that Hecke eigensheaves form a full symmetric monoidal subcategory of $\Shv_{\underline{\bar{\Q}_{\ell}}}^{\ad, 1}(\Bun_{\GL_1}(X))$, which we shall denote by $\Eig\Shv_{\underline{\bar{\Q}_{\ell}}}^1(\Bun_{\GL_1}(X))$.
        \end{remark}
        
        Lemma \ref{lemma: hecke_eigensheaves_extend_to_lower_degrees}, which is regarding the \say{globality} of Hecke eigensheaves is merely a technicality in service of theorem \ref{theorem: unramified_abelian_geometric_class_field_theory}. The reader can safely skip ahead and refer back to it later.
        \begin{lemma}[Hecke eigensheaves extend to lower degrees] \label{lemma: hecke_eigensheaves_extend_to_lower_degrees}
            Let $\E$ be a Hecke eigensheaf on $\bigcup_{d \geq d_0 + 1} \Bun_{\GL_1}^{(d)}(X)$ (for any $d_0 \geq 0$) with eigenvalue $\calL \in \Shv_{\underline{\bar{\Q}_{\ell}}}^{\ad, 1}(X)$. Then, $\E$ can be extended uniquely to a Hecke eigensheaf on $\bigcup_{d \geq d_0} \Bun_{\GL_1}^{(d)}(X)$, also with eigenvalue $\calL$.
        \end{lemma}
            \begin{proof}
                Consider the following commutative diagram, where $\cev{h}_x^{(d)}$ is given by $\calL \mapsto \calL(-[x])$:
                    $$
                        \begin{tikzcd}
                        	{\Spec k \x \Bun_{\GL_1}^{(d)}(X)} & {\Bun_{\GL_1}^{(d)}(X)} \\
                        	{X \x \Bun_{\GL_1}^{(d)}(X)} & {\Bun_{\GL_1}^{(d)}(X)}
                        	\arrow["{\cev{h}_X^{(d)}}", from=2-1, to=2-2]
                        	\arrow["{\cev{h}_x^{(d)}}", from=1-1, to=1-2]
                        	\arrow["{x \x \id}"', from=1-1, to=2-1]
                        	\arrow[Rightarrow, no head, from=1-2, to=2-2]
                        \end{tikzcd}
                    $$
                By definition, we have an isomorphism $(\cev{h}_x^{(d)})^*\E \cong \calL \boxtimes \E$ for any Hecke eigensheaf $\E$ on $\Bun_{\GL_1}^{(d)}(X)$ with eigenvalue $\calL$, which induces an isomorphism $(\cev{h}_x^{(d)})^*\E \cong x^*\calL \boxtimes \E$ at each geometric point $x \in X$; but since $x \in X$ is a geometric point, $x^*\calL$ is - by definition - nothing but the stalk $\calL_x$, which is isomorphic to $\bar{\Q}_{\ell}$, since $\calL$ is an $\ell$-adic local system of rank $1$ on $X$. Next, fix an arbitrary finite set of geometric points $x_1, x_2, ..., x_n \in X$ and consider the following for any Hecke eigensheaf $\E_{\calL^{(d + n)}}$ on $\bigcup_{d \geq 2g - 1} \Bun_{\GL_1}^{(d)}(X)$ that corresponds to $\calL^{(d + n)} \in \Shv_{\underline{\bar{\Q}_{\ell}}}^{\ad, 1}(X^{(d + n)})$:
                    $$(\cev{h}_{x_1}^* \circ ... \circ \cev{h}_{x_n}^*)(\E_{\calL^{(d + n)}}) \cong \bigotimes_{i = 1}^n (\calL_{x_i} \boxtimes \E_{\calL^{(d)}})$$
                which implies that:
                    $$\bigotimes_{i = 1}^n \calL_{x_i}^{\tensor (-1)} \boxtimes (\cev{h}_{x_1}^* \circ ... \circ \cev{h}_{x_n}^*)(\E_{\calL^{(d + n)}}) \cong \E_{\calL^{(d)}}$$
                If we now take $d := 2g - 1 - n$, we will get the following equation on $\Spec k \x \Bun_{\GL_1}^{2g - 1 - n}(X)$, which yields us a \textit{unique} Hecke eigensheaf $\E_{\calL^{(2g - 1 - n)}}$ on $\bigcup_{d \geq 2g - 1 - n} \Bun_{\GL_1}^{(d)}(X)$ from $\E_{\calL^{(2g - 1)}}$ (which we have already):
                    $$\E_{\calL^{(2g - 1 - n)}} \cong \bigotimes_{i = 1}^n \calL_{x_i}^{\tensor (-1)} \boxtimes (\cev{h}_{x_1}^* \circ ... \circ \cev{h}_{x_n}^*)(\E_{\calL^{(2g - 1)}})$$
            \end{proof}
            
        Let us now begin our discussion of geometric class field theory with the unramified case. The proof of the tamely ramified case (cf. theorem \ref{theorem: tamely_ramified_abelian_geometric_class_field_theory}), as it turns out, will only include a slight modification of the proof of the unramified case (cf. theorem \ref{theorem: unramified_abelian_geometric_class_field_theory}).
        \begin{theorem}[Unramified abelian geometric class field theory] \label{theorem: unramified_abelian_geometric_class_field_theory}
            There exists a canonical equivalence between the groupoid of rank-$1$ $\ell$-adic local systems on $X$ and the groupoid of ($\ell$-adic) Hecke eigensheaves of rank $1$ on $\Bun_{\GL_1}(X)$:
                $$\Autom: \Shv_{\underline{\bar{\Q}_{\ell}}}^{\ad, 1}(X) \to \Eig\Shv_{\underline{\bar{\Q}_{\ell}}}^1(\Bun_{\GL_1}(X))$$
            which maps each $\ell$-adic local system $\calL \in \Shv_{\underline{\bar{\Q}_{\ell}}}^{\ad, 1}(X)$ to a Hecke eigensheaf $\Autom(\calL) \in \Eig\Shv_{\underline{\bar{\Q}_{\ell}}}^1(\Bun_{\GL_1}(X))$ with eigenvalue $\calL$.
        \end{theorem}
            \begin{proof}
                Our strategy for this proof is to explicitly construct - for each $\calL \in \Shv_{\underline{\bar{\Q}_{\ell}}}^{\ad, 1}(X)$ - the corresponding Hecke eigensheaf $\Autom(\calL)$, and this will involve three steps:
                    \begin{enumerate}
                        \item \textbf{(A \say{Hecke eigensheaf} on $X^{(d)}$):} In this step we shall construct an $\ell$-local system on $X^{(d)}$ (hence on $\bigcup_{d \geq 2g - 1} X^{(d)}$); cf. proposition \ref{prop: the_unramified_abel_jacobi_map_is_a_projective_bundle} satisfying an analogue of the Hecke eigensheaf property (cf. definition \ref{def: hecke_eigensheaves}) with the purpose in mind being that by pushing this local system forward using the Abel-Jacobi map, one shall obtain a legitimate Hecke eigensheaf on $\Bun_{\GL_1}^{(d)}(X)$ (hence on $\bigcup_{d \geq 2g - 1} \Bun_{\GL_1}^{(d)}(X)$).
                        
                        The first observation that one can make is that there is an isomorphism $(\sigma^{(d)})^*\calL^{(d)} \cong \calL^{\boxtimes d}$ of $\ell$-adic local systems on $X^d$. Next, notice how there are commutative diagrams of the following form, wherein the maps $\tilde{h}_X^{(d)}$ are given by $(-[x], D) \mapsto [x] + D$ (cf. remark \ref{remark: adding_effective_divisors} and convention \ref{conv: symmetric_powers_of_line_bundles}):
                            $$
                                \begin{tikzcd}
                                	{X \x X^d} & {X^{(d + 1)}} \\
                                	{X \x X^{(d)}}
                                	\arrow["{\id_X \x \sigma^{(d)}}"', from=1-1, to=2-1]
                                	\arrow["{\tilde{h}_X^{(d)}}"', dashed, from=2-1, to=1-2]
                                	\arrow["{\sigma^{(d + 1)}}", from=1-1, to=1-2]
                                \end{tikzcd}
                            $$
                        We thus have, as follows, a $\Sigma_d$-equivariant analogue on $X \x X^{(d)}$ of the Hecke eigensheaf property for all $\calL \in \Shv_{\underline{\bar{\Q}_{\ell}}}^{\ad, 1}(X)$:
                            $$(\tilde{h}_X^{(d)})^* \calL^{(d + 1)} \cong \calL \boxtimes \calL^{(d)}$$
                        \item \textbf{(A Hecke eigensheaf on $\Bun_{\GL_1}^{(d)}(X)$):} Now, recall from corollary \ref{coro: unramified_galois_representations_induced_by_the_abel_jacobi_map} that for every $d \geq 2g - 1$ and example \ref{example: gluing_constructible_sheaves}, there exists an adjoint equivalence:
                            $$
                                \begin{tikzcd}
                                	{\Shv_{\underline{\bar{\Q}_{\ell}}}^{\ad, 1}(X^{(d)})} & {\Shv_{\underline{\bar{\Q}_{\ell}}}^{\ad, 1}(\Bun_{\GL_1}^{(d)}(X))}
                                	\arrow[""{name=0, anchor=center, inner sep=0}, "{(\AJ_{X/k}^{(d)})_*}"', bend right, from=1-1, to=1-2]
                                	\arrow[""{name=1, anchor=center, inner sep=0}, "{(\AJ_{X/k}^{(d)})^*}"', bend right, from=1-2, to=1-1]
                                	\arrow["\dashv"{anchor=center, rotate=-90}, draw=none, from=1, to=0]
                                \end{tikzcd}
                            $$ 
                        From this, one infers that every $\ell$-adic local system $\calF \in \Shv_{\underline{\bar{\Q}_{\ell}}}^{\ad, 1}(X^{(d)})$ has the form $(\AJ_{X/k}^{(d)})^* \E$ for some \textit{unique} $\ell$-adic local system $\E \in \Shv_{\underline{\bar{\Q}_{\ell}}}^{\ad, 1}(\Bun_{\GL_1}^{(d)}(X))$, meaning that there exists $\ell$-adic local systems $\E_{\calL^{(d)}}, \E_{\calL^{(d + 1)}} \in \Shv_{\underline{\bar{\Q}_{\ell}}}^{\ad, 1}(\Bun_{\GL_1}^{(d)}(X))$ corresponding to $\calL^{(d)}, \calL^{(d + 1)} \in \Shv_{\underline{\bar{\Q}_{\ell}}}^{\ad, 1}(X^{(d)})$ respectively (and ultimately, to $\calL \in \Shv_{\underline{\bar{\Q}_{\ell}}}^{\ad, 1}(X)$) satisfying the following equation in $\Shv_{\underline{\bar{\Q}_{\ell}}}^{\ad, 1}(X \x X^{(d)})$:
                            $$(\tilde{h}_X^{(d)})^* (\AJ_{X/k}^{(d + 1)})^* \E_{\calL^{(d + 1)}} \cong \calL \boxtimes (\AJ_{X/k}^{(d)})^*\E_{\calL^{(d)}}$$
                        Next, consider the following commutative diagram:
                            $$
                                \begin{tikzcd}
                                	{X \x X^{(d)}} & {X^{(d + 1)}} \\
                                	{X \x \Bun_{\GL_1}^{(d)}(X)} & {\Bun_{\GL_1}^{(d + 1)}(X)}
                                	\arrow["{\cev{h}_X^{(d)}}", from=2-1, to=2-2]
                                	\arrow["{\id_X \x \AJ_{X/k}^{(d)}}"', from=1-1, to=2-1]
                                	\arrow["{\AJ^{(d + 1)}}", from=1-2, to=2-2]
                                	\arrow["{\tilde{h}_X^{(d)}}", from=1-1, to=1-2]
                                \end{tikzcd}
                            $$
                        which induces the following equations in $\Shv_{\underline{\bar{\Q}_{\ell}}}^{\ad, 1}(X \x X^{(d)})$ for all $\calL \in \Shv_{\underline{\bar{\Q}_{\ell}}}^{\ad, 1}(X)$:
                            $$
                                \begin{aligned}
                                    (\id_X \x \AJ_{X/k}^{(d)})^* (\cev{h}_X^{(d)})^* \E_{\calL^{(d + 1)}} & \cong (\tilde{h}_X^{(d)})^* (\AJ^{(d + 1)})^* \E_{\calL^{(d + 1)}}
                                    \\
                                    & \cong \calL \boxtimes (\AJ_{X/k}^{(d)})^*\E_{\calL^{(d)}}
                                    \\
                                    & \cong (\id_X \x \AJ_{X/k}^{(d)})^*(\calL \boxtimes \E_{\calL^{(d)}})
                                \end{aligned}
                            $$
                        and since $(\id_X \x \AJ_{X/k}^{(d)})^*$ is an invertible functor (cf. corollary \ref{coro: unramified_galois_representations_induced_by_the_abel_jacobi_map}), we have, furthermore, the following equation in $\Shv_{\underline{\bar{\Q}_{\ell}}}^{\ad, 1}(X \x \Bun_{\GL_1}^{(d)}(X))$, which is precisely the Hecke eigensheaf property from definition \ref{def: hecke_eigensheaves}:
                            $$(\cev{h}_X^{(d)})^* \E_{\calL^{(d + 1)}} \cong \calL \boxtimes \E_{\calL^{(d)}}$$
                        We have thus obtained a \textit{unique} Hecke eigensheaf on $\bigcup_{d \geq 2g - 1} \Bun_{\GL_1}^{(d)}(X)$ from an arbitrary $\ell$-adic local system $\calL \in \Shv_{\underline{\bar{\Q}_{\ell}}}^{\ad, 1}(X)$.
                        \item \textbf{(A Hecke eigensheaf on $\Bun_{\GL_1}(X)$):} Finally, in order extend the Hecke eigensheaf that we have constructed on $\bigcup_{d \geq 2g - 1} \Bun_{\GL_1}^{(d)}(X)$ to $\Bun_{\GL_1}(X)$ (i.e. to degrees $d < 2g - 1$), simply apply lemma \ref{lemma: hecke_eigensheaves_extend_to_lower_degrees}.
                    \end{enumerate}
            \end{proof}
        
        \begin{definition}[(Un)ramified Galois characters] \label{def: (un)ramified_galois_characters}
            Let $D := \sum_{i = 1}^n e_i \p_i$ be a Weil divisor on $X$. Additionally, fix an $\ell$-adic character $\chi \in \Rep^1_{\bar{\Q}_{\ell}}(\Gal(\bar{K}/K))$ (or equivalently, an $\ell$-adic local system of rank $1$ on $X$, thanks to theorem \ref{theorem: galois_representations_are_sheaves_on_X}).
                \begin{itemize}
                    \item \textbf{(Ramification indices):} The \textbf{local ramification index} of $\chi$, for which we shall write $\ram_{\p_i}(\chi)$, is the smallest natural number $r_i$ such that $\chi$ becomes trivial on all higher ramification groups $G_{K_{\p_i}}^{r}$ with $r > r_i$; the \textbf{global ramification index} of $\chi$, denoted by $\ram(\chi)$, is the supremum of its local ramification indices. 
                    \item \textbf{(Unramified and ramified $\ell$-adic Galois characters):} $\chi$ is said to be \textbf{unramified} if $\ram(\chi) \leq 0$, and \textbf{tamely ramified} if $\ram(\chi) \leq 1$.
                \end{itemize}
        \end{definition}
        \begin{theorem}[Tamely ramified abelian geometric class field theory] \label{theorem: tamely_ramified_abelian_geometric_class_field_theory}
            Suppose that $D$ is a tame Weil divisor on $X$. Then, there exists a canonical equivalence between the groupoid $\Shv_{\underline{\bar{\Q}_{\ell}}}^{\ad, 1}(X \setminus D)$ of rank-$1$ $\ell$-adic local systems on $X \setminus D$ which correspond (via theorem \ref{theorem: galois_representations_are_sheaves_on_X}) to tamely ramified representations of $\pi_1^{\ab}((X \setminus D)_{\fet})$, and that of Hecke eigensheaves of rank $1$ on $\Bun_{\GL_1}(X \setminus D)$:
                $$\Autom_D: \Shv_{\underline{\bar{\Q}_{\ell}}}^{\ad, 1}(X \setminus D) \to \Eig\Shv_{\underline{\bar{\Q}_{\ell}}}^1(\Bun_{\GL_1}(X \setminus D))$$
            which maps each $\ell$-adic local system $\calL \in \Shv_{\underline{\bar{\Q}_{\ell}}}^{\ad, 1}(X)$ to a Hecke eigensheaf $\Autom(\calL) \in \Eig\Shv_{\underline{\bar{\Q}_{\ell}}}^1(\Bun_{\GL_1}(X))$ with eigenvalue $\calL$.
        \end{theorem}
            \begin{proof}
                \noindent
                \begin{enumerate}
                    \item \textbf{(Characteristic $0$):} For this case, one can simply mimic the proof of theorem \ref{theorem: unramified_abelian_geometric_class_field_theory}, owing to the fact that field extensions are always separable in characteristic $0$, while extensions in prime characteristics can be purely inseparable, a phenomenon that causes $\A_k^{(d - g + 1) - \deg D}$ (as in proposition \ref{prop: the_tamely_ramified_abel_jacobi_map_is_a_vector_bundle}) to not be \'etale-simply connected in positive characteristics, which causes problems because corollary \ref{coro: tamely_ramified_galois_representations_induced_by_the_abel_jacobi_map} can not then be used to equate $\ell$-adic local systems on $X^{(d)}$ and on $\Bun_{\GL_1}^{(d)}(X)$. 
                    \item \textbf{(Characteristics $p > 0$):} 
                \end{enumerate}
            \end{proof}
        
        \begin{corollary}[Geometric Langlands for $\GL_1$] \label{coro: geometric_langlands_for_GL1}
            By putting theorem \ref{theorem: galois_representations_are_sheaves_on_X} (respectively theorem \ref{theorem: tamely_ramified_abelian_geometric_class_field_theory}) and theorem \ref{theorem: unramified_abelian_geometric_class_field_theory} together, one gets a canonical equivalence of categories as follows:
                $$\Rep^1_{\bar{\Q}_{\ell}}(\pi_1^{\ab}((X \setminus D)_{\fet}))^{\cont} \cong \Eig\Shv_{\underline{\bar{\Q}_{\ell}}}^1(\Bun_{\GL_1}(X \setminus D))$$
                $$\chi \mapsto \Autom_D(\chi)$$
            wherein $D$ is a tame Weil divisor on $X$ (the unramified version corresponds to taking $D \cong \varnothing$). What this essentially tells us is that $1$-dimensional continuous $\ell$-adic Galois representations are the same as automorphic forms associated to $\GL_1$, and as such, the combination of theorem \ref{theorem: galois_representations_are_sheaves_on_X} and theorem \ref{theorem: unramified_abelian_geometric_class_field_theory} can be understood as a geometrisation of the Global Langlands Correspondence for the (connnected reductive) group $\GL_1$\footnote{Incidentally, this is why it is commonly asserted that the Langlands Correspondence for $\GL_1$ \say{is just class field theory}.}. 
        \end{corollary}
        
    \subsection{Grothendieck's Sheaf-Function Dictionary and Hecke characters}
        As a final step, let us decategorify the left-hand side of the equivalence:
            $$\Rep^1_{\bar{\Q}_{\ell}}(\pi_1^{\ab}((X \setminus D)_{\fet}))^{\cont} \cong \Eig\Shv_{\underline{\bar{\Q}_{\ell}}}^1(\Bun_{\GL_1}(X \setminus D))$$
        from corollary \ref{coro: geometric_langlands_for_GL1} to obtain a correspondence between continuous $\ell$-adic characters of $\pi_1^{\ab}(X_{\fet})$ and so-called Hecke characters\footnote{Auf Deutsch: \say{\textit{Die Gr\"o{\ss}encharaktere}}.}, using Grothendieck's Sheaf-Function Dictionary. Doing so will yield us the usual version of Artin Reciprocity for global function fields over perfect fields of positive characteristics, in terms of towers of finite abelian extensions and so-called Hecke characters (see lemma \ref{lemma: sheaf_function_correspondence}).
        
        We begin with the notion of Weil sheaves, which will become necessary later on (cf. lemma \ref{lemma: sheaf_function_correspondence}) when we move on to computing the trace of the Frobenius endomorphism on $\ell$-adic sheaves.
        \begin{convention}[Specialising to function fields of positive characteristics]
            From now on, the base field $k$ from convention \ref{conv: automorphic_side_conventions} shall be of some prime characteristic $p$. Note that since $k$ is algebraically closed, it is perfect by virtue of containing all $p^{th}$ power roots.
        \end{convention}
        \begin{definition}[Weil sheaves] \label{def: weil_sheaves}
            A \textbf{Weil sheaf} on $X$ is an $\ell$-adic sheaf $\calF \in \Shv_{\underline{\bar{\Q}_{\ell}}}^{\ad, 1}(X)$ that is Frobenius-equivariant, i.e. $\Frob_{X/k}^*\calF \cong \calF$.
        \end{definition}
        \begin{proposition}[Hecke eigensheaves are Weil sheaves] \label{prop: hecke_eigensheaves_are_weil_sheaves}
            Let $D$ be a tame Weil divisor on $X$. Then, any $\ell$-adic local system of rank $1$ on $X$ (or equivalently, thanks to theorems \ref{theorem: unramified_abelian_geometric_class_field_theory} and \ref{theorem: tamely_ramified_abelian_geometric_class_field_theory}, a Hecke eigensheaf of rank $1$ on $\Bun_{\GL_1}(X \setminus D)$) is an instance of a Weil sheaf on $X$.
        \end{proposition}
            \begin{proof}
                \cite[Proposition 5.19]{tendler_2010_geometric_class_field_theory_original}.
            \end{proof}
        
        Let us now move on to Grothendieck's Sheaf-Function Dictionary/Corresondence\footnote{En Français: \say{\textit{La Correspondance Faisceaux-Fonctions de Grothendieck}}.}, which allows us to formally realise the notion of Hecke eigensheaves (as in definition \ref{def: hecke_eigensheaves}) as a categorification of the classical notion of Hecke characters (cf. lemma \ref{lemma: sheaf_function_correspondence}). In order to do this, we must introduce the central notion of traces of Frobenii.
        \begin{definition}[Traces of Frobenii] \label{def: traces_of_frobenii}
            For each geometric point $x \in X(k)$, we define the trace of Frobenius on an $\ell$-adic local system $\calL \in \Shv_{\underline{\bar{\Q}_{\ell}}}^{\ad, 1}(X)$ to be the trace of the Frobenius endomorphism on the stalk $\calL_x$ (which is computed in the usual manner) and write:
                $$\trace(\Frob_{X/k}^*, \calL) := \trace(\Frob_x^*, \calL_x)$$
        \end{definition}
        \begin{convention}[Global function field of $X$] \label{conv: global_function_field}
            Let us write $K$ for the global function field of $X$ (i.e the stalk of the structure sheaf of $X$ at the unique generic point), $\scrO_K$ for its ring of integers, and $\A_K$ for the corresponding ring of ad\`eles. For more details on these constructions, we refer the reader to \cite[Section VI.1]{neukirch_2010_algebraic_number_theory}.
        \end{convention}
        \begin{lemma}[Grothendieck's Sheaf-Function Correspondence] \label{lemma: sheaf_function_correspondence}
            Let $D$ be a tame Weil divisor on $X$. Then, there is a canonical equivalence between the category of Hecke eigensheaves of rank $1$ on $\Bun_{\GL_1}(X \setminus D)$ and that of continuous $\ell$-adic characters of the ad\`elic double quotient $\GL_1(K)\backslash\GL_1(\A_K)/\GL_1(\scrO_K)$, so-called $\ell$-adic \textbf{Hecke characters}:
                $$\Eig\Shv_{\underline{\bar{\Q}_{\ell}}}^1(\Bun_{\GL_1}(X \setminus D)) \cong \scrA_{\GL_1}(X \setminus D)$$
        \end{lemma}
            \begin{proof}
                Suppose that we are willing to take the special case of the Weil Uniformisation Theorem for $\Bun_{\GL_1}(X)$ (cf. \cite[Proposition 3.8]{tendler_2010_geometric_class_field_theory_original}\footnote{For details on Weil Uniformisation for $\Bun_G(X)$, with $G$ being a general connected reductive group, see \cite{heinloth2009uniformization}.}) for granted, which states that the group of $k$-rational points of $\Bun_{\GL_1}(X \setminus D)$ is nothing but the ad\`elic double-quotient $\GL_1(K)\backslash\GL_1(\A_K)/\GL_1(\scrO_K)$. Next, pick a geometric point $x \in X(k)$. By making use of proposition \ref{prop: hecke_eigensheaves_are_weil_sheaves} and theorem \ref{theorem: unramified_abelian_geometric_class_field_theory} (respectively, theorem \ref{theorem: tamely_ramified_abelian_geometric_class_field_theory}) in tandem, we can consider the trace of Frobenius on the eigenvalue $\Autom_D^{-1}(\E) \in \Shv_{\underline{\bar{\Q}_{\ell}}}^{\ad, 1}(X \setminus D)$ of each Hecke eigensheaf $\E \in \Eig\Shv_{\underline{\bar{\Q}_{\ell}}}^1(\Bun_{\GL_1}(X \setminus D))$:
                    $$\rho(x, \E) := \trace(\Frob_x^*, \Autom_D^{-1}(\E))$$
                To show that taking traces of Frobenii in this manner will yield us a well-defined Hecke character $\rho(x, \E) \in \scrA_{\GL_1}(X \setminus D)$, recall first of all that by taking stalks of $\ell$-adic local systems $\calL \in \Shv_{\underline{\bar{\Q}_{\ell}}}^{\ad, 1}(X \setminus D)$ at geometric points $x \in X(k)$, one obtains a unique continuous $\ell$-adic character of $\pi_1((X \setminus D)_{\fet})$ (cf. theorem \ref{theorem: galois_representations_are_sheaves_on_X}); through corollary \ref{coro: unramified_galois_representations_induced_by_the_abel_jacobi_map} (respectively corollary \ref{coro: tamely_ramified_galois_representations_induced_by_the_abel_jacobi_map}), this means that one also gets a unique continuous $\ell$-adic character of $\pi_1((\Bun_{\GL_1}^{(1)}(X \setminus D))_{\fet})$. 
            \end{proof}
            
        \begin{theorem}[Artin Reciprocity for global function fields] \label{theorem: artin_reciprocity_for_global_function_fields}
            Fix a tame Weil divisor $D$ on $X$. There thus exists a canonical bijection as below, such that for all places $v \not \in D$, one has $\chi(\Frob_v) = \rho(\varpi_v)$:
                $$\scrA_{\GL_1}(X \setminus D) \cong \Rep^1_{\bar{\Q}_{\ell}}(\Gal(K^{\ab}/K))^{\cont}$$
                $$\rho \mapsto \chi$$
        \end{theorem}
            \begin{proof}
                The existence of a canonical bijection as stated comes directly from a combination of corollary \ref{coro: geometric_langlands_for_GL1} and lemma \ref{lemma: sheaf_function_correspondence}, so let us focus on the \say{Frobenius-compatibility} condition $\chi(\Frob_v) = \rho(\varpi_v)$.
            \end{proof}
    
    \begin{appendices}
        %\section{Abelian varieties}
    Since the theory of abelian variety is too rich for us to present in any amount of substantial details, we shall recommend that the reader consult \cite{bhatt_abelian_varieties} instead. Regardless, we shall collect here a list of necessary machineries that shall help us understand the role that abelian varieties (particularly, Jacobians) play in the establishment of geometric class field theory.
    
    Let us begin with the definition of abelian varieties.
    \begin{definition}[Abelian varieties] \label{def: abelian_varieties}
        Let $S$ be a base scheme. An \textbf{abelian $S$-scheme} is thus a group $S$-scheme that is smooth, proper and has geometrically connected fibres, which we refer to as \textbf{abelian varieties}.
    \end{definition}
    \begin{remark}[Some basic properties of abelian varieties] \label{remark: basic_properties_of_abelian_varieties}
        The following are important basic properties of abelian varieties that the reader should keep in mind:
        \begin{itemize}
            \item \textbf{(Stability under base change):} Smoothness, properness, and geometric connectedness are all preserved by pullbacks, so pullbacks of abelian schemes are also abelian schemes. In fact, for each fixed base $S$, the category of abelian $S$-schemes is a symmetric monoidal full subcategory of the category of $S$-schemes that are smooth, proper, and with geometrically connected fibres.  
            \item \textbf{(Abelian group structure):} Abelian schemes, like their name suggests, are actually abelian group schemes, although this is a somewhat non-trivial phenomenon (see \cite[Proposition 2.1 and Corollaries 2.2, 2.3, and 2.4]{bhatt_abelian_varieties}). 
            \item \textbf{(Projectivity):} Abelian schemes over fields (i.e. abelian varieties) are not just proper, but actually projective. This is one more a non-trivial fact, and one proof is \cite[Theorem 11.1]{bhatt_abelian_varieties}.
        \end{itemize}
    \end{remark}
    \begin{example}[Some instances of abelian varieties]
        \noindent
        \begin{itemize}
            \item \textbf{(Elliptic curves):} Elliptic curves are nothing but abelian varieties of (pure) dimension $1$. 
            \item \textbf{(Abelian varieties of higher dimensions):} Algebraic groups over characteristic $0$ are smooth \textit{a priori} (cf. \cite[\href{https://stacks.math.columbia.edu/tag/047N}{Tag 047N}]{stacks}), so any proper characteristic-$0$ group scheme with geometrically connected fibres (e.g. $\Proj\left(\frac{\Z\left[x, y, z, \frac{1}{\Delta}\right]}{(y^2z - x^3 - axz^2 - bz^3)}\right)$, where $a, b \in \Z$ are such that $\Delta := -16(4a^3 + 27b^2)$ is non-zero) is automatically an abelian variety. In particular, a complex abelian variety of dimension $n$, through Serre's GAGA theorem, corresponds to a compact and connected complex Lie group of the form $\bbC^n/\Lambda$ for some full-rank lattice $\Lambda$ (cf. \cite[Remark 1.9]{bhatt_abelian_varieties}); the converse is not necessarily true.
            
            Over positive characteristics, one can use the fact that an algebraic group over a perfect field is smooth if it is geometrically reduced (cf. \cite[\href{https://stacks.math.columbia.edu/tag/047P}{Tag 047P}]{stacks}) to single out the class of abelian varieties from the class of algebraic groups. Similar to above, one might consider $\Proj\left(\frac{\F_p\left[x, y, z, \frac{1}{\Delta}\right]}{(y^2z - x^3 - axz^2 - bz^3)}\right)$ for $p \not = 2, 3$ and $\Delta \not \equiv 0 \pmod{p}$.
        \end{itemize}
    \end{example}
    
    Like abelian locally compact groups, abelian varieties and their duals are related to one another by an analogue of the Fourier transform for said topological groups, namely the Fourier-Mukai transform. In fact, if one were to think of the various derived categories of sheaves over abelian varieties as generalisations of the algebras of, say, $L^2$-functions on abelian locally compact groups, then the notion of Fourier-Mukai transforms can be thought of as a direct categorification of the notion of Fourier transforms as integral transforms. Before we can establish a link between abelian varieties and their duals, we must first know how dual abelian varieties are constructed at the level of objects. 
    \begin{proposition}[Existence and uniqueness of dual abelian varieties] \label{prop: dual_abelian_varieties}
        Let $A$ be an abelian variety over some algebraically closed field $k$. Then, there exists a unique abelian variety over $\Spec k$ (up to natural isomorphisms, of course), commonly denoted by $A^{\vee}$ and known as the \textbf{dual} of $A$\footnote{In the sense of dual abelian group.} whose functor of points is the presheaf which assigns to each $S \in \Sch_{/\Spec k}$ the data of:
            \begin{itemize}
                \item a line bundle $\calL \in \Bun_{\GL_1}(S \x_{\Spec k} A)$ such that for every (geometric) point $s \in S$, $(s \x \id_A)^*\calL \in \Bun_{\GL_1}(A)$, and
                \item a \say{rigidifying} isomorphism $(\id_S \x 0_A)^*\calL \cong \calO_{S/k}$.
            \end{itemize}
        Furthermore, the assignment of abelian varieties to their duals is functorial, and any abelian variety $A$ is canonically isomorphic to its double dual $A^{\vee \vee}$.
    \end{proposition}
        \begin{proof}
            \cite[Sections 8 and 13]{mumford_1970_abelian_varieties} and \cite[Subsection 15.2 and Theorem 16.2]{bhatt_abelian_varieties}.
        \end{proof}
    Now, let us define the Fourier-Mukai transform and utilise it for the sake of establishing a categorical relationship between abelian varieties and their duals.
    \begin{convention}
        From now on, if $k$ is a field and then $\Sch_{/\Spec k}^{\ft}$ shall denote the category of schemes that are of finite type over $\Spec k$. Futhermore, by $\QCoh, \Coh$, etc. we shall actually mean the corresponding unbounded derived categories (perhaps with decorations such as $b, \leq 0, \geq 0$, etc.), and likewise for the various functors .
    \end{convention}
    \begin{definition}[Integral transforms] \label{def: integral_transforms}
        For any $X, Y \in \Sch_{/\Spec k}^{\ft}$, the \textbf{integral transform} with \textbf{kernel} $K \in \QCoh(X \x Y)$ the functor:
            $$\Phi_K: \QCoh(X) \to \QCoh(Y)$$
            $$\calF \mapsto (\pr_2)_*\left( \pr_1^* \calF \tensor_{\calO_{X \x Y}} K \right)$$
    \end{definition}
    \begin{proposition}[Convolution of sheaves] \label{prop: convolution_of_sheaves}
        
    \end{proposition}
        
        %\section{What about local class field theory ?}
    \end{appendices}
	
	\printbibliography

\end{document}