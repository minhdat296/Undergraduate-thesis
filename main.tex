\input{preambles}

\input{commands}

\begin{document}

	\title{\textbf{MATH518: Thesis
	\\
	Geometric unramified abelian class field theory}}
	
	\author{Dat Minh Ha (UCID: 30067407)\\Supervisor: Jerrod Smith}
	\maketitle
	
	\begin{abstract}
	    The Langlands Programme is a network of many deep conjectures (and recently, some theorem!) with far-reaching consequences, notably in the realms of algebraic number theory and representation theory. At its core, it is about \textbf{reciprocity}, the idea that Galois groups should admit descriptions in terms of canonical constructions. As a matter of fact, the starting point of the Langlands Programme is what we nowadays call \textbf{class field theory}, the very topic of this thesis. More specifically, we are interested in what is known as \textbf{unramified abelian class field theory}, the simplest version of class field theory, which we shall approach via algebraic geometry. This is not the traditional approach to class field theory, but it will help us understand why the study of Galois groups naturally requires representation theory, which as a consequence, helps us makes sense of class field theory being the same as the Langlands Correspondence for the group $\GL_1$.
	\end{abstract}
	
	{
      \hypersetup{} 
      %\dominitoc
      \tableofcontents %sort sections alphabetically
    }
    
    \section{Introduction}
        \subsection{The Automorphic Side}
            Geometric class field theory is the idea that the study of function fields $K$ over $\F_q$ (i.e. finite extensions of $\F_q(t)$) can be formulated purely in terms of the machineries of algebraic geometry. More specifically, it is the idea that given Galois extensions $L/K$, one can decribe the Galois groups $\Gal(L/K)$ in purely geometric terms, via a gadget called the \textbf{\'etale fundamental group}. This, however, is only a meaningful endeavour should we know the underlying geometric space whose \'etale fundamental group would end up being the Galois groups that we are interested in.
            
            Our starting point is the following result, which explains why one might even suspect any sort of involvement of algebraic geometry in the first place:
            \begin{lemma}[Varieties and field extensions] \label{lemma: varieties_and_field_extensions}
                \cite[\href{https://stacks.math.columbia.edu/tag/0BXN}{Tag 0BXN}]{stacks} Let $k$ be a field. Then, $\trdeg K_X = \dim X$ for all varieties $X/k$, and there exists a canonical equivalence of categories as follows:
                    $$\{\text{Finite-type field extensions $k/k$ and $k$-algebra homomorphisms}\}^{\op}$$
                    $$\cong$$
                    $$\{\text{Varieties $X/k$ and \href{https://stacks.math.columbia.edu/tag/01RI}{\underline{dominant}} \href{https://stacks.math.columbia.edu/tag/01RR}{\underline{rational}} maps}\}$$
            \end{lemma}
                \begin{proof}
                    If $f: X \to Y$ is a dominant map of varieties and if $\eta_X$ and $\eta_Y$ are the unique generic points of $X$ and $Y$ (unique because varieties are integral by definition, and every integral scheme \textit{a priori} has a unique generic point), then $f(\eta_X) = \eta_Y$ per the definition of dominant morphisms. The residue field at generic points are precisely the function fields (consider the stalk of the structure sheaves at the generic points to see why this is the case), so we have obtained a map of function fields $K_Y \to K_X$. We leave the proof of finiteness up to the reader.
                \end{proof}
            Through lemma \ref{lemma: varieties_and_field_extensions}, one obtains the following regarding the relationship between curves (i.e. algebraic varieties of dimension $1$) and their function fields (which \textit{a priori} are of transcedence degree $1$ over the ground field) with little difficulty:
            \begin{proposition}[Curves and function fields] \label{prop: curves_and_function_fields}
                \cite[\href{https://stacks.math.columbia.edu/tag/0BY1}{Tag 0BY1}]{stacks} For any field $k$, one has the following canonical equivalences of categories:
                    $$\{\text{Field extensions $K/k$ of transcendence degree $1$ and $k$-algebra homomorphisms}\}^{\op}$$
                    $$\cong$$
                    $$\{\text{Curves $X/k$ and dominant rational maps}\}$$
                    $$\cong$$
                    $$\{\text{Non-singular projective curves $X/k$ and dominant rational maps}\}$$
            \end{proposition}
                \begin{proof}
                    The first equivalence is an obvious consequence of lemma \ref{lemma: varieties_and_field_extensions}. To show that the second equivalence holds, note firstly that there is an evident fully faithful functor from the third category to the second; we shall need to show that this functor is also essentially surjective. For this, simply recall that for each curve $X/k$, there exists a non-singular projective curve $\tilde{X}/k$ that is birational to $X/k$, namely $X^{\nu} \cup \{\infty_1, ..., \infty_n\}$, the normalisation $X^{\nu}/k$ of $X/k$ with finitely many extra points (recall also that any normal Noetherian scheme of dimension $\leq 1$ is \textit{a priori} non-singular; cf. \cite[\href{https://stacks.math.columbia.edu/tag/0BX2}{Tag 0BX2}]{stacks}).
                \end{proof}
                
            Through proposition \ref{prop: curves_and_function_fields}, we obtain the first crucial tool for the geometrisation of class field theory.
            \begin{corollary}[Galois covers of curves and Galois extensions] \label{coro: galois_covers_of_curves_and_galois_extensions}
                Let $k$ be a field. If $X$ is a non-singular projective curve over $\Spec k$ with function field $K$, then there is a canonical equivalence:
                    $$({}^{K/}\Fld^{\fin, \Gal})^{\op} \cong \Sch_{/X}^{\fet, \Gal}$$
                between the category of finite Galois extensions of $K$ and finite \'etale-Galois covers of $X$. 
            \end{corollary}
            The importance of corollary \ref{coro: galois_covers_of_curves_and_galois_extensions} can not be understated: what it tells us is essentially that the Galois group $\Gal(K^{\ab}/K)$ is nothing but the \'etale fundamental group of $\Sch_{/X}^{\fet, \Gal}$ (more on this later, after we have introduced the \'etale fundamental group). This, already, is one side of the Artin's Reciprocity Law, which we shall refer to as \say{\textbf{The Automorphic Side}} per popular conventions.
            
            One last remark that we shall make is that because we are working in the \'etale topology, everything is necessarily unramified (as \'etale morphisms are unramified smooth morphisms).
        
        \subsection{The Spectral Side}
    
    \section{Grothendieck's Galois Theory}
        \subsection{The \'etale fundamental group}
        
        \subsection{\texorpdfstring{$\ell$}{}-adic sheaves}
            
        \subsection{Grothendieck's Sheaf-Function Correspondence}
    
    \section{Geometric reciprocity for global function fields}
        \subsection{The Abel-Jacobi map}
    
        \subsection{Hecke eigensheaves}
        
        \subsection{Deligne-Artin Reciprocity}
        
        \subsection{What about local class field theory ?}
	
	\printbibliography

\end{document}