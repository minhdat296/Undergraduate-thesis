\documentclass[a4paper, 11pt]{article}

%\usepackage{minitoc}
%\setcounter{tocdepth}{}
%\setcounter{minitocdepth}{}
%\setcounter{secnumdepth}{}

\usepackage[english]{babel}
\usepackage[utf8]{inputenc}
%\usepackage{mathpazo}
%\usepackage{euler}
\usepackage{eucal}
\usepackage{bbm}
\usepackage{bm}
\usepackage{csquotes}
\usepackage[nottoc]{tocbibind}
\usepackage{appendix}
\usepackage{float}
\usepackage[T1]{fontenc}
\usepackage[
    left = \flqq{},% 
    right = \frqq{},% 
    leftsub = \flq{},% 
    rightsub = \frq{} %
]{dirtytalk}

\usepackage{imakeidx}
\makeindex

\usepackage[dvipsnames]{xcolor}
\usepackage{hyperref}
    \hypersetup{
        colorlinks=true,
        linkcolor=teal,
        filecolor=pink,      
        urlcolor=teal,
        citecolor=magenta
    }
\usepackage{comment}

% You would set the PDF title, author, etc. with package options or
% \hypersetup.

\usepackage[backend=biber, style=alphabetic, sorting=nty]{biblatex}
    \addbibresource{bibliography.bib}

\raggedbottom

\usepackage{mathrsfs}
\usepackage{mathtools} 
\mathtoolsset{showonlyrefs} 
\usepackage{amssymb}
\usepackage{amsthm}
\renewcommand\qedsymbol{$\blacksquare$}
\usepackage{tikz-cd}
\usepackage{tikz}
\usepackage{setspace}
\usepackage[version=3]{mhchem}
\parskip=0.1in
\usepackage[margin=25mm]{geometry}

\usepackage{listings, lstautogobble}
\lstset{
	language=matlab,
	basicstyle=\scriptsize\ttfamily,
	commentstyle=\ttfamily\itshape\color{gray},
	stringstyle=\ttfamily,
	showstringspaces=false,
	breaklines=true,
	frameround=ffff,
	frame=single,
	rulecolor=\color{black},
	autogobble=true
}

\usepackage{todonotes,tocloft,xpatch,hyperref}

% This is based on classicthesis chapter definition
\let\oldsec=\section
\renewcommand*{\section}{\secdef{\Sec}{\SecS}}
\newcommand\SecS[1]{\oldsec*{#1}}%
\newcommand\Sec[2][]{\oldsec[\texorpdfstring{#1}{#1}]{#2}}%

\newcounter{istodo}[section]

% http://tex.stackexchange.com/a/61267/11984
\makeatletter
%\xapptocmd{\Sec}{\addtocontents{tdo}{\protect\todoline{\thesection}{#1}{}}}{}{}
\newcommand{\todoline}[1]{\@ifnextchar\Endoftdo{}{\@todoline{#1}}}
\newcommand{\@todoline}[3]{%
	\@ifnextchar\todoline{}
	{\contentsline{section}{\numberline{#1}#2}{#3}{}{}}%
}
\let\l@todo\l@subsection
\newcommand{\Endoftdo}{}

\AtEndDocument{\addtocontents{tdo}{\string\Endoftdo}}
\makeatother

\usepackage{lipsum}

%   Reduce the margin of the summary:
\def\changemargin#1#2{\list{}{\rightmargin#2\leftmargin#1}\item[]}
\let\endchangemargin=\endlist 

%   Generate the environment for the abstract:
%\newcommand\summaryname{Abstract}
%\newenvironment{abstract}%
    %{\small\begin{center}%
    %\bfseries{\summaryname} \end{center}}

\newtheorem{theorem}{Theorem}[section]
    \numberwithin{theorem}{subsection}
\newtheorem{proposition}{Proposition}[section]
    \numberwithin{proposition}{subsection}
\newtheorem{lemma}{Lemma}[section]
    \numberwithin{lemma}{subsection}
\newtheorem{claim}{Claim}[section]
    \numberwithin{claim}{subsection}

\theoremstyle{definition}
    \newtheorem{definition}{Definition}[section]
        \numberwithin{definition}{subsection}

\theoremstyle{remark}
    \newtheorem{remark}{Remark}[section]
        \numberwithin{remark}{subsection}
    \newtheorem{example}{Example}[section]
        \numberwithin{example}{subsection}    
    \newtheorem{convention}{Convention}[section]
        \numberwithin{convention}{subsection}
    \newtheorem{corollary}{Corollary}[section]
        \numberwithin{corollary}{subsection}
\usepackage{soul}

\renewcommand{\cong}{\simeq}
\newcommand{\ladjoint}{\dashv}
\newcommand{\radjoint}{\vdash}
\newcommand{\<}{\langle}
\renewcommand{\>}{\rangle}
\newcommand{\ndiv}{\hspace{-2pt}\not|\hspace{5pt}}
\newcommand{\cond}{\blacktriangle}
\newcommand{\solid}{\blacksquare}
\newcommand{\ot}{\leftarrow}
\renewcommand{\-}{\text{-}}
\renewcommand{\mapsto}{\leadsto}
\renewcommand{\leq}{\leqslant}
\renewcommand{\geq}{\geqslant}
\renewcommand{\setminus}{\smallsetminus}
\makeatletter
\DeclareRobustCommand{\cev}[1]{%
  {\mathpalette\do@cev{#1}}%
}
\newcommand{\do@cev}[2]{%
  \vbox{\offinterlineskip
    \sbox\z@{$\m@th#1 x$}%
    \ialign{##\cr
      \hidewidth\reflectbox{$\m@th#1\vec{}\mkern4mu$}\hidewidth\cr
      \noalign{\kern-\ht\z@}
      $\m@th#1#2$\cr
    }%
  }%
}
\makeatother

\newcommand{\N}{\mathbb{N}}
\newcommand{\Z}{\mathbb{Z}}
\newcommand{\Q}{\mathbb{Q}}
\newcommand{\R}{\mathbb{R}}
\newcommand{\bbC}{\mathbb{C}}
\NewDocumentCommand{\x}{e{_^}}{%
  \mathbin{\mathop{\times}\displaylimits
    \IfValueT{#1}{_{#1}}
    \IfValueT{#2}{^{#2}}
  }%
}
\NewDocumentCommand{\pushout}{e{_^}}{%
  \mathbin{\mathop{\sqcup}\displaylimits
    \IfValueT{#1}{_{#1}}
    \IfValueT{#2}{^{#2}}
  }%
}
\newcommand{\supp}{\operatorname{supp}}
\newcommand{\im}{\operatorname{im}}
\newcommand{\coker}{\operatorname{coker}}
\newcommand{\id}{\mathrm{id}}
\newcommand{\chara}{\operatorname{char}}
\newcommand{\trdeg}{\operatorname{trdeg}}
\newcommand{\rank}{\operatorname{rank}}
\newcommand{\trace}{\operatorname{tr}}
\newcommand{\length}{\operatorname{length}}
\newcommand{\height}{\operatorname{height}}
\renewcommand{\span}{\operatorname{span}}
\newcommand{\e}{\epsilon}
\newcommand{\p}{\mathfrak{p}}
\newcommand{\q}{\mathfrak{q}}
\newcommand{\m}{\mathfrak{m}}
\newcommand{\n}{\mathfrak{n}}
\newcommand{\calF}{\mathcal{F}}
\newcommand{\calG}{\mathcal{G}}
\newcommand{\calO}{\mathcal{O}}
\newcommand{\F}{\mathbb{F}}
\DeclareMathOperator{\lcm}{lcm}
\newcommand{\gr}{\operatorname{gr}}
\newcommand{\vol}{\mathrm{vol}}

\newcommand{\GL}{\operatorname{GL}}
\newcommand{\SL}{\operatorname{SL}}
\newcommand{\Sp}{\operatorname{Sp}}
\newcommand{\GSp}{\operatorname{GSp}}
\newcommand{\GSpin}{\operatorname{GSpin}}
\newcommand{\opO}{\operatorname{O}}
\newcommand{\SO}{\operatorname{SO}}
\newcommand{\SU}{\operatorname{SU}}
\newcommand{\opU}{\operatorname{U}}
\newcommand{\Spec}{\mathrm{Spec}}
\newcommand{\Spf}{\mathrm{Spf}}
\newcommand{\Spm}{\mathrm{Spm}}
\newcommand{\Spv}{\mathrm{Spv}}
\newcommand{\Spa}{\mathrm{Spa}}
\newcommand{\Spd}{\mathrm{Spd}}
\newcommand{\Proj}{\mathrm{Proj}}
\newcommand{\Gr}{\mathrm{Gr}}
\newcommand{\Hecke}{\mathrm{Hecke}}
\newcommand{\Sht}{\mathrm{Sht}}
\newcommand{\Quot}{\mathrm{Quot}}
\newcommand{\Hilb}{\mathrm{Hilb}}
\newcommand{\Pic}{\mathrm{Pic}}
\newcommand{\Div}{\mathrm{Div}}
\newcommand{\Jac}{\mathrm{Jac}}
\newcommand{\Alb}{\mathrm{Alb}} %albanese variety
\newcommand{\Bun}{\mathrm{Bun}}
\newcommand{\loopspace}{\mathbf{\Omega}}
\newcommand{\suspension}{\mathbf{\Sigma}}
\newcommand{\tangent}{\mathrm{T}} %tangent space
\newcommand{\Eig}{\mathrm{Eig}}

\newcommand{\Ring}{\mathrm{Ring}}
\newcommand{\Cring}{\mathrm{CRing}}
\newcommand{\Alg}{\mathrm{Alg}}
\newcommand{\Leib}{\mathrm{Leib}} %leibniz algebras
\newcommand{\Fld}{\mathrm{Fld}}
\newcommand{\Sets}{\mathrm{Sets}}
\newcommand{\Cat}{\mathrm{Cat}}
\newcommand{\Grp}{\mathrm{Grp}}
\newcommand{\Ab}{\mathrm{Ab}}
\newcommand{\Sch}{\mathrm{Sch}}
\newcommand{\Coh}{\mathrm{Coh}}
\newcommand{\QCoh}{\mathrm{QCoh}}
\newcommand{\Desc}{\mathrm{Desc}}
\newcommand{\Sh}{\mathrm{Sh}}
\newcommand{\Psh}{\mathrm{PSh}}
\newcommand{\Fib}{\mathrm{Fib}}
\renewcommand{\mod}{\-\mathrm{mod}}
\newcommand{\bimod}{\-\mathrm{bimod}}
\newcommand{\Vect}{\mathrm{Vect}}
\newcommand{\Rep}{\mathrm{Rep}}
\newcommand{\Grpd}{\mathrm{Grpd}}
\newcommand{\Arr}{\mathrm{Arr}}
\newcommand{\Esp}{\mathrm{Esp}}
\newcommand{\Ob}{\mathrm{Ob}}
\newcommand{\Mor}{\mathrm{Mor}}
\newcommand{\Mfd}{\mathrm{Mfd}}
%\newcommand{\LR}{\mathrm{LR}}
%\newcommand{\RSpc}{\mathrm{RSpc}}
\newcommand{\Spc}{\mathrm{Spc}}
\newcommand{\Top}{\mathrm{Top}}
\newcommand{\Topos}{\mathrm{Topos}}
\newcommand{\Nil}{\mathfrak{Nil}}
\newcommand{\J}{\mathfrak{J}}
\newcommand{\Stk}{\mathrm{Stk}}
\newcommand{\Pre}{\mathrm{Pre}}
\newcommand{\simp}{\mathbf{\Delta}}
\newcommand{\Ind}{\mathrm{Ind}}
\newcommand{\Pro}{\mathrm{Pro}}
\newcommand{\Mon}{\mathrm{Mon}}
\newcommand{\Comm}{\mathrm{Comm}}
\newcommand{\Fin}{\mathrm{Fin}}
\newcommand{\Assoc}{\mathrm{Assoc}}
\newcommand{\Co}{\mathrm{Co}}
\newcommand{\Comp}{\mathrm{Comp}} %compact hausdorff spaces
\newcommand{\Stone}{\mathrm{Stone}} %stone spaces
\newcommand{\sfExt}{\mathrm{Ext}} %extremely disconnected spaces
\newcommand{\Ouv}{\mathrm{Ouv}}
\newcommand{\Str}{\mathrm{Str}}
\newcommand{\Func}{\mathrm{Func}}
\newcommand{\Crys}{\mathrm{Crys}}
\newcommand{\LocSys}{\mathrm{LocSys}}
\newcommand{\Sieves}{\mathrm{Sieves}}
\newcommand{\pt}{\mathrm{pt}}
\newcommand{\Graphs}{\mathrm{Graphs}}
\newcommand{\Lie}{\mathrm{Lie}}
\newcommand{\Env}{\mathrm{Env}}
\newcommand{\Ho}{\mathrm{Ho}}
\newcommand{\rmD}{\mathrm{D}}
\newcommand{\Cov}{\mathrm{Cov}}
\newcommand{\Frames}{\mathrm{Frames}}
\newcommand{\Locales}{\mathrm{Locales}}
\newcommand{\Span}{\mathrm{Span}}
\newcommand{\Corr}{\mathrm{Corr}}
\newcommand{\Monad}{\mathrm{Monad}}
\newcommand{\Var}{\mathrm{Var}}
\newcommand{\sfN}{\mathrm{N}} %nerve
\newcommand{\Dia}{\mathrm{Dia}}
\newcommand{\co}{\mathrm{co}}
\newcommand{\ev}{\mathrm{ev}}
\newcommand{\bi}{\mathrm{bi}}
\newcommand{\Nat}{\mathrm{Nat}}
\newcommand{\Hopf}{\mathrm{Hopf}}
\newcommand{\Dmod}{\mathrm{D}\mod}
\newcommand{\Perv}{\mathrm{Perv}}
\newcommand{\Sph}{\mathrm{Sph}}
\newcommand{\Moduli}{\mathrm{Moduli}}
\newcommand{\Pseudo}{\mathrm{Pseudo}}
\newcommand{\Lax}{\mathrm{Lax}}
\newcommand{\Strict}{\mathrm{Strict}}
\newcommand{\Opd}{\mathrm{Opd}} %operads
\newcommand{\Shv}{\mathrm{Shv}}
\newcommand{\Char}{\mathrm{Char}} %CharShv = character sheaves
\newcommand{\Huber}{\mathrm{Huber}}
\newcommand{\Tate}{\mathrm{Tate}}
\newcommand{\Ad}{\mathrm{Ad}} %adic spaces
\newcommand{\Perfd}{\mathrm{Perfd}} %perfectoid spaces
\newcommand{\Sub}{\mathrm{Sub}} %subobjects
\newcommand{\Ideals}{\mathrm{Ideals}}
\newcommand{\Isoc}{\mathrm{Isoc}}
\newcommand{\Ban}{\-\mathrm{Ban}} %Banach spaces
\newcommand{\Fre}{\-\mathrm{Fre}} %Frechet spaces
\newcommand{\Ch}{\mathrm{Ch}} %chain complexes
\newcommand{\Mot}{\mathrm{Mot}} %motives
\newcommand{\KL}{\mathrm{KL}} %category of Kazhdan-Lusztig modules
\newcommand{\Pres}{\mathrm{Pres}} %presentable categories

\newcommand{\Aut}{\mathrm{Aut}}
\newcommand{\Inn}{\mathrm{Inn}}
\newcommand{\Out}{\mathrm{Out}}
\newcommand{\frakgl}{\mathfrak{gl}}
\newcommand{\der}{\mathfrak{der}} %derivations on Lie algebras
\newcommand{\inn}{\mathfrak{inn}} %inner derivations
\newcommand{\out}{\mathfrak{out}} %outer derivations
\newcommand{\Stab}{\mathrm{Stab}}
\newcommand{\Cent}{\mathrm{Cent}}
\newcommand{\Conj}{\mathrm{Conj}}
\newcommand{\Gal}{\mathrm{Gal}}
\newcommand{\bfG}{\mathbf{G}} %absolute Galois group
\newcommand{\Frac}{\mathrm{Frac}}
\newcommand{\Ann}{\mathrm{Ann}}
\newcommand{\Val}{\mathrm{Val}}
\newcommand{\Chow}{\mathrm{Chow}}
\newcommand{\Sym}{\mathrm{Sym}}
\newcommand{\End}{\mathrm{End}}
\newcommand{\Mat}{\mathrm{Mat}}
\newcommand{\Diff}{\mathrm{Diff}}
\newcommand{\Autom}{\mathrm{Autom}}

\newcommand{\colim}{\operatorname{colim} \:}
\renewcommand{\lim}{\operatorname{lim} \:}
\newcommand{\toto}{\rightrightarrows}
%\newcommand{\tensor}{\otimes}
\NewDocumentCommand{\tensor}{e{_^}}{%
  \mathbin{\mathop{\otimes}\displaylimits
    \IfValueT{#1}{_{#1}}
    \IfValueT{#2}{^{#2}}
  }%
}
\newcommand{\eq}{\operatorname{eq}}
\newcommand{\coeq}{\operatorname{coeq}}
\newcommand{\Hom}{\mathrm{Hom}}
\newcommand{\Maps}{\mathrm{Maps}}
\newcommand{\Tor}{\mathrm{Tor}}
\newcommand{\Ext}{\mathrm{Ext}}
\newcommand{\Isom}{\mathrm{Isom}}
\newcommand{\stalk}{\mathbf{stalk}}
\newcommand{\RKE}{\operatorname{RKE}}
\newcommand{\LKE}{\operatorname{LKE}}
\newcommand{\oblv}{\mathbf{oblv}}
\newcommand{\const}{\mathbf{const}}
%\newcommand{\forget}{\mathbf{forget}}
\newcommand{\adrep}{\mathbf{ad}} %adjoint representation
\newcommand{\NL}{\mathbf{NL}} %naive cotangent complex
\newcommand{\bfL}{\mathbf{L}} %cotangent complex
\newcommand{\pr}{\operatorname{pr}}
\newcommand{\Der}{\mathbf{Der}}
\newcommand{\Frob}{\mathrm{Frob}} %Frobenius
\newcommand{\frob}{\mathrm{frob}} %trace of Frobenius
\newcommand{\bfpt}{\mathbf{pt}}
\newcommand{\bfloc}{\mathbf{loc}}
\newcommand{\1}{\mathbbm{1}}
\newcommand{\2}{\mathbbm{2}}
\newcommand{\Jet}{\mathbf{Jet}}
\newcommand{\Split}{\mathbf{Split}}
\newcommand{\Sq}{\mathbf{Sq}}
\newcommand{\Zero}{\mathbf{Z}}
\newcommand{\SqZ}{\Sq\Zero}
\newcommand{\frakLie}{\mathfrak{Lie}}
\newcommand{\y}{\mathbf{y}} %yoneda
\newcommand{\Sm}{\mathrm{Sm}}
\newcommand{\AJ}{\mathrm{AJ}} %abel-jacobi map
\newcommand{\act}{\mathrm{act}}
\newcommand{\ram}{\mathrm{ram}} %ramification index
\newcommand{\inv}{\mathrm{inv}}

\newcommand{\bbU}{\mathbb{U}}
\newcommand{\V}{\mathbb{V}}
\newcommand{\U}{\mathrm{U}}
\newcommand{\rmI}{\mathrm{I}} %augmentation ideal
\newcommand{\bfV}{\mathbf{V}}
\newcommand{\C}{\mathcal{C}}
\newcommand{\D}{\mathcal{D}}
\newcommand{\T}{\mathbf{T}} %Tate modules
\newcommand{\calM}{\mathcal{M}}
\newcommand{\calN}{\mathcal{N}}
\newcommand{\calP}{\mathcal{P}}
\newcommand{\calQ}{\mathcal{Q}}
\newcommand{\A}{\mathbb{A}}
\renewcommand{\P}{\mathbb{P}}
\newcommand{\calL}{\mathcal{L}}
\newcommand{\E}{\mathcal{E}}
\renewcommand{\H}{\mathbf{H}}
\newcommand{\calX}{\mathcal{X}}
\newcommand{\calY}{\mathcal{Y}}
\newcommand{\calZ}{\mathcal{Z}}
\newcommand{\scrX}{\mathscr{X}}
\newcommand{\scrY}{\mathscr{Y}}
\newcommand{\scrZ}{\mathscr{Z}}
\newcommand{\calA}{\mathcal{A}}
\newcommand{\calB}{\mathcal{B}}
\newcommand{\sfT}{\mathrm{T}}
\renewcommand{\S}{\mathcal{S}}
\newcommand{\B}{\mathbb{B}}
\newcommand{\bbD}{\mathbb{D}}
\newcommand{\G}{\mathbb{G}}
\newcommand{\horn}{\mathbf{\Lambda}}
\renewcommand{\L}{\mathbb{L}}
\renewcommand{\a}{\mathfrak{a}}
\renewcommand{\b}{\mathfrak{b}}
\renewcommand{\t}{\mathfrak{t}}
\renewcommand{\r}{\mathfrak{r}}
\newcommand{\bbX}{\mathbb{X}}
\newcommand{\g}{\mathfrak{g}}
\newcommand{\h}{\mathfrak{h}}
\renewcommand{\k}{\mathfrak{k}}
\newcommand{\del}{\partial}
\newcommand{\bbE}{\mathbb{E}}
\newcommand{\scrO}{\mathscr{O}}
\newcommand{\bbO}{\mathbb{O}}
\newcommand{\scrA}{\mathscr{A}}
\newcommand{\scrB}{\mathscr{B}}
\newcommand{\scrF}{\mathscr{F}}
\newcommand{\scrG}{\mathscr{G}}
\newcommand{\scrM}{\mathscr{M}}
\newcommand{\scrN}{\mathscr{N}}
\newcommand{\scrP}{\mathscr{P}}
\newcommand{\frakS}{\mathfrak{S}}
\newcommand{\calI}{\mathcal{I}}
\newcommand{\calJ}{\mathcal{J}}
\newcommand{\calK}{\mathcal{K}}
\newcommand{\scrV}{\mathscr{V}}
\newcommand{\bbS}{\mathbb{S}}
\newcommand{\scrH}{\mathscr{H}}
\newcommand{\bfB}{\mathbf{B}}
\newcommand{\W}{\mathbf{W}}
%\newcommand{\bfA}{\mathbf{A}}
\renewcommand{\O}{\mathbb{O}}
\newcommand{\calV}{\mathcal{V}}
\newcommand{\scrR}{\mathscr{R}} %radical
\newcommand{\rmZ}{\mathrm{Z}} %centre of algebra
\newcommand{\bfGamma}{\mathbf{\Gamma}}
\newcommand{\scrU}{\mathscr{U}}
\newcommand{\rmW}{\mathrm{W}} %Weil group

\newcommand{\aff}{\mathrm{aff}}
\newcommand{\ft}{\mathrm{ft}} %finite type
\newcommand{\fp}{\mathrm{fp}} %finite presentation
\newcommand{\aft}{\mathrm{aft}}
\newcommand{\lft}{\mathrm{lft}}
\newcommand{\laft}{\mathrm{laft}}
\newcommand{\cmpt}{\mathrm{cmpt}}
\newcommand{\qc}{\mathrm{qc}}
\newcommand{\qs}{\mathrm{qs}}
\newcommand{\lcmpt}{\mathrm{lcmpt}}
%\newcommand{\conv}{\mathrm{conv}}
\newcommand{\red}{\mathrm{red}}
\newcommand{\fin}{\mathrm{fin}}
\newcommand{\gen}{\mathrm{gen}}
\newcommand{\petit}{\mathrm{petit}}
\newcommand{\gros}{\mathrm{gros}}
\newcommand{\loc}{\mathrm{loc}}
\newcommand{\glob}{\mathrm{glob}}
\newcommand{\ringed}{\mathrm{ringed}}
\newcommand{\qcoh}{\mathrm{qcoh}}
\newcommand{\cl}{\mathrm{cl}}
\newcommand{\et}{\mathrm{\acute{e}t}}
\newcommand{\fet}{\mathrm{f\acute{e}t}}
\newcommand{\profet}{\mathrm{prof\acute{e}t}}
\newcommand{\proet}{\mathrm{pro\acute{e}t}}
\newcommand{\Zar}{\mathrm{Zar}}
\newcommand{\fppf}{\mathrm{fppf}}
\newcommand{\fpqc}{\mathrm{fpqc}}
\newcommand{\smooth}{\mathrm{sm}}
\newcommand{\sh}{\mathrm{sh}}
\newcommand{\op}{\mathrm{op}}
\newcommand{\open}{\mathrm{open}}
\newcommand{\closed}{\mathrm{closed}}
\newcommand{\geom}{\mathrm{geom}}
\newcommand{\alg}{\mathrm{alg}}
\newcommand{\sober}{\mathrm{sober}}
\newcommand{\dR}{\mathrm{dR}}
\newcommand{\rad}{\mathrm{rad}}
\newcommand{\discrete}{\mathrm{discrete}}
%\newcommand{\add}{\mathrm{add}}
%\newcommand{\lin}{\mathrm{lin}}
\newcommand{\Krull}{\mathrm{Krull}}
\newcommand{\qis}{\mathrm{qis}} %quasi-isomorphism
\newcommand{\ho}{\mathrm{ho}} %homotopy equivalence
\newcommand{\sep}{\mathrm{sep}}
\newcommand{\unr}{\mathrm{unr}}
\newcommand{\tame}{\mathrm{tame}}
\newcommand{\wild}{\mathrm{wild}}
\newcommand{\nil}{\mathrm{nil}}
\newcommand{\defm}{\mathrm{defm}}
\newcommand{\Art}{\mathrm{Art}}
\newcommand{\Noeth}{\mathrm{Noeth}}
\newcommand{\affd}{\mathrm{affd}}
%\newcommand{\adic}{\mathrm{adic}}
\newcommand{\pre}{\mathrm{pre}}
\newcommand{\perf}{\mathrm{perf}}
\newcommand{\perfd}{\mathrm{perfd}}
\newcommand{\rat}{\mathrm{rat}}
\newcommand{\cont}{\mathrm{cont}}
\newcommand{\dg}{\mathrm{dg}}
\newcommand{\almost}{\mathrm{a}}
%\newcommand{\stab}{\mathrm{stab}}
\newcommand{\heart}{\heartsuit}
\newcommand{\proj}{\mathrm{proj}}
\newcommand{\qproj}{\mathrm{qproj}}
\newcommand{\pd}{\mathrm{pd}}
\newcommand{\crys}{\mathrm{crys}}
\newcommand{\prisma}{\mathrm{prisma}}
\newcommand{\FF}{\mathrm{FF}}
\newcommand{\sph}{\mathrm{sph}}
\newcommand{\lax}{\mathrm{lax}}
\newcommand{\weak}{\mathrm{weak}}
\newcommand{\strict}{\mathrm{strict}}
\newcommand{\mon}{\mathrm{mon}}
\newcommand{\sym}{\mathrm{sym}}
\newcommand{\lisse}{\mathrm{lisse}}
\newcommand{\an}{\mathrm{an}}
\newcommand{\ad}{\mathrm{ad}}
\newcommand{\sch}{\mathrm{sch}}
\newcommand{\rig}{\mathrm{rig}}
\newcommand{\pol}{\mathrm{pol}}
\newcommand{\plat}{\mathrm{flat}}
\newcommand{\proper}{\mathrm{proper}}
\newcommand{\compl}{\mathrm{compl}}
\newcommand{\non}{\mathrm{non}}
\newcommand{\access}{\mathrm{access}}
\newcommand{\comp}{\mathrm{comp}}
\newcommand{\tstructure}{\mathrm{t}} %t-structures
\newcommand{\pure}{\mathrm{pure}} %pure motives
\newcommand{\mixed}{\mathrm{mixed}} %mixed motives
\newcommand{\num}{\mathrm{num}} %numerical motives
\newcommand{\ess}{\mathrm{ess}}
\newcommand{\topological}{\mathrm{top}}
\newcommand{\convex}{\mathrm{cv}}
\newcommand{\ab}{\mathrm{ab}} %abelian extensions
\newcommand{\surj}{\mathrm{surj}} %coverage on sets generated by surjections
\newcommand{\eff}{\mathrm{eff}} %effective Cartier divisors
\newcommand{\Weil}{\mathrm{Weil}} %weil divisors
\newcommand{\lex}{\mathrm{lex}}
\newcommand{\rex}{\mathrm{rex}}
\newcommand{\AR}{\mathrm{A\-R}}

%prism custom command
\usepackage{relsize}
\usepackage[bbgreekl]{mathbbol}
\usepackage{amsfonts}
\DeclareSymbolFontAlphabet{\mathbb}{AMSb} %to ensure that the meaning of \mathbb does not change
\DeclareSymbolFontAlphabet{\mathbbl}{bbold}
\newcommand{\prism}{{\mathlarger{\mathbbl{\Delta}}}}

\begin{document}

	\title{\textbf{MATH518: Proposal
	\\
	Geometric unramified abelian class field theory}}
	
	\author{Dat Minh Ha (UCID: 30067407)\\Supervisor: Jerrod Smith}
	\maketitle
	
	\begin{abstract}
	    The Langlands Programme is a network of many deep conjectures (and recently, some theorem!) with far-reaching consequences, notably in the realms of algebraic number theory and representation theory. At its core, it is about \textbf{reciprocity}, the idea that Galois groups should admit descriptions in terms of canonical constructions. As a matter of fact, the starting point of the Langlands Programme is what we nowadays call \textbf{class field theory}, the very topic of this thesis. More specifically, we are interested in what is known as \textbf{unramified abelian class field theory}, the simplest version of class field theory, which we shall approach via algebraic geometry. This is not the traditional approach to class field theory, but it will help us understand why the study of Galois groups naturally requires representation theory, which as a consequence, helps us makes sense of class field theory being the same as the Langlands Correspondence for the group $\GL_1$.
	\end{abstract}
	    
	\section{Objectives}
	    Our main objective shall be to give a proof of Deligne's geometrisation the \textbf{unramified global reciprocity law}, which asserts that for an appropriate there exists an equivalence between the categories of $\ell$-adic local systems $\calF$ of rank $1$ on an appropriate algebraic curve $X$, and that of so-called Hecke eigensheaves of rank $1$ on $\Bun_{\GL_1}(X)$, which categorify the classical notion of automorphic forms. In undertaking this project, our goal is not only to learn class field theory, but also to understand the fundamental reason behind the necessity of representation theory in the study of Galois representations (as stated in the abstract). In addition, we would like to understand the applicability of the main result to various explicit examples of the underlying curve $X$ (e.g. $X$ being $\P^1$ or some elliptic curve, which would be projective, smooth, and connected by definition), as well as its non-applicability to certain inappropriate instances of $X$ (e.g. one may consider $X$ being $\A^1$ - which fails to be projective - or some singular algebraic curve).
	
	\section{Approach}
	    The paper will be organised into two main sections, detailing what we shall call the \textbf{Galois Side} and the \textbf{Automorphic Side} of geometric class field theory. We shall elaborate on our approach to these sections here.

        \subsection{The Galois Side}
            Geometric class field theory is the idea that the study of function fields $K$ over $\F_q$ (i.e. finite extensions of $\F_q(t)$) can be formulated purely in terms of the machineries of algebraic geometry. More specifically, it is the idea that given Galois extensions $L/K$, one can decribe the Galois groups $\Gal(L/K)$ in purely geometric terms, via a gadget called the \textbf{\'etale fundamental group}. This, however, is only a meaningful endeavour should we know the underlying geometric space whose \'etale fundamental group would end up being the Galois groups that we are interested in.
            
            Our starting point is the following result, which explains why one might even suspect any sort of involvement of algebraic geometry in the first place:
            \begin{lemma}[Varieties and field extensions] \label{lemma: varieties_and_field_extensions}
                \cite[\href{https://stacks.math.columbia.edu/tag/0BXN}{Tag 0BXN}]{stacks} Let $k$ be a field. Then, $\trdeg K_X = \dim X$ for all varieties $X/k$, and there exists a canonical equivalence of categories as follows:
                    $$\{\text{Finite-type field extensions $K/k$ and $k$-algebra homomorphisms}\}^{\op}$$
                    $$\cong$$
                    $$\{\text{Varieties $X/k$ and \href{https://stacks.math.columbia.edu/tag/01RI}{\underline{dominant}} \href{https://stacks.math.columbia.edu/tag/01RR}{\underline{rational}} maps}\}$$
            \end{lemma}
            Through lemma \ref{lemma: varieties_and_field_extensions}, one obtains the following regarding the relationship between curves (i.e. algebraic varieties of dimension $1$) and their function fields (which \textit{a priori} are of transcedence degree $1$ over the ground field) with little difficulty:
            \begin{proposition}[Curves and function fields] \label{prop: curves_and_function_fields}
                \cite[\href{https://stacks.math.columbia.edu/tag/0BY1}{Tag 0BY1}]{stacks} For any field $k$, one has the following canonical equivalences of categories:
                    $$\{\text{Field extensions $K/k$ of transcendence degree $1$ and $k$-algebra homomorphisms}\}^{\op}$$
                    $$\cong$$
                    $$\{\text{Curves $X/k$ and dominant rational maps}\}$$
                    $$\cong$$
                    $$\{\text{Non-singular projective curves $X/k$ and dominant rational maps}\}$$
            \end{proposition}
            Through proposition \ref{prop: curves_and_function_fields}, we obtain the first crucial tool for the geometrisation of class field theory.
            \begin{corollary}[Galois covers of curves and Galois extensions] \label{coro: galois_covers_of_curves_and_galois_extensions}
                Let $k$ be a field. If $X$ is a connected non-singular projective curve over $\Spec k$ with function field $K$, then there is a canonical equivalence:
                    $$({}^{K/}\Fld^{\fin, \Gal})^{\op} \cong \Sch_{/X}^{\fet, \Gal}$$
                between the category of finite Galois extensions of $K$ and finite \'etale-Galois covers of $X$ (i.e. finite \'etale covers generated by Galois morphisms, which are necessarily dominant rational maps such that the associated function field extensions are Galois). 
            \end{corollary}
            The importance of corollary \ref{coro: galois_covers_of_curves_and_galois_extensions} can not be understated: what it tells us is essentially that the Galois group $\Gal(K^{\ab}/K)$ is nothing but the \'etale fundamental group of $\Sch_{/X}^{\fet, \Gal}$ (more on this later, after we have introduced the \'etale fundamental group). This, already, is one side of the Artin's Reciprocity Law, which we shall refer to as \say{\textbf{The Galois Side}} per popular conventions. Actually, this is a bit of a lie: instead of formulating geometric class field theory directly in terms of the \'etale fundamental group $\pi_1^{\ab}(X_{\fet})$ (or rather, its continuous $\ell$-adic characters), we will be phrasing things in terms of $\ell$-adic local systems of rank $1$ on $X$; the main result on the Galois Side shall establish the categorification of continuous representations of $\pi_1^{\ab}(X_{\fet})$ to $\ell$-adic local systems on $X$ via a canonical equivalence of categories:
                $$\Rep^1_{\overline{\Q_{\ell}}}(\pi_1^{\ab}(X_{\fet}))^{\cont} \cong \LocSys^1_{\overline{\Q_{\ell}}}(X)$$
            
            One last remark pertaining to the Galois Side is that because we are working in the \'etale topology, everything is necessarily unramified, since \'etale morphisms are unramified smooth morphisms by definition.
        
        \subsection{The Automorphic Side}
            Let us now move on to what is known as the \say{\textbf{Automorphic Side}}, and we shall begin with the notion of \textbf{Hecke eigensheaves}. To introduce these gadgets, however, we will first need to discuss the \textbf{Hecke correspondence}, the categorification of the action of the Hecke algebra on the space of functions satisfying certain smoothness and growth conditions on the double ad\`elic quotient $\GL_1(K) \backslash \GL_1(\A_K) / \GL_1(\scrO_K)$ (i.e. automorphic forms). For now, let us think of the Hecke correspondence as a span, i.e. a diagram of the form:
                $$
                    \begin{tikzcd}
                    	& {\Hecke_{\GL_1}(X)} \\
                    	{\Bun_{\GL_1}(X)} && {X \x \Bun_{\GL_1}(X)}
                    	\arrow["{\cev{h}}"', from=1-2, to=2-1]
                    	\arrow["{\supp_X \x \vec{h}}", from=1-2, to=2-3]
                    \end{tikzcd}
                $$
            If we were to generically denote a given category of \say{good sheaf theory} by $\Shv(-)$\footnote{Eventually, we will be interested particularly in $\ell$-adic sheaves of rank $1$, which shall be denoted by $\Shv_{\overline{\Q_{\ell}}}^1(-)$, but more on this later. In the wider context of the Geometric Langlands Programme, $\Shv(-)$ might mean perverse sheaves, or when we are working over $\bbC$, D-modules; we shall not touch on these sheaf theories.}, then the Hecke correspondence induces the following sheaf pull-push diagram:
                $$
                    \begin{tikzcd}
                    	& {\Shv(\Hecke_{\GL_1}(X))} \\
                    	{\Shv(\Bun_{\GL_1}(X))} && {\Shv(X \x \Bun_{\GL_1}(X))}
                    	\arrow["{\cev{h}^*}", from=2-1, to=1-2]
                    	\arrow["{(\supp_X \x \vec{h})_*}", from=1-2, to=2-3]
                    \end{tikzcd}
                $$
            and by composing the two functors in the obvious manner, one gets a new functor:
                $$\scrH_X: \Shv(\Bun_{\GL_1}(X)) \to \Shv(X \x \Bun_{\GL_1}(X))$$
            This is commonly known as the \textbf{Hecke functor} or the \textbf{Hecke operator} (should we want to put emphasis on the spectral nature of $\scrH_X$), and it is of central importance to us. However, before we can explain why this is the case, observe that the Hecke operator $\scrH_X$ is actually \say{global} in a sense: the fibre of $\supp_X \x \vec{h}$ over any given point $x \in X$ is the \say{local} Hecke correspondence:
                $$
                    \begin{tikzcd}
                    	& {\Hecke_{\GL_1}(x)} \\
                    	{\Bun_{\GL_1}(X)} && {\Bun_{\GL_1}(X)}
                    	\arrow["{\cev{h}}"', from=1-2, to=2-1]
                    	\arrow["{\vec{h}}", from=1-2, to=2-3]
                    \end{tikzcd}
                $$
            and one can thus define the local Hecke operator at $x \in X$ as:
                $$\scrH_x := \vec{h}_* \cev{h}^*$$
            Arguably, this is more akin to the classical Hecke operator, as it is an endofunctor on $\Shv(\Bun_{\GL_1}(X))$ as opposed to a functor $\scrH_X: \Shv(\Bun_{\GL_1}(X)) \to \Shv(X \x \Bun_{\GL_1}(X))$; henceforth, we will be thinking of the global Hecke operator $\scrH_X$ as a family $\{\scrH_x\}_{x \in X}$ of local Hecke operators parametrised by points $x \in X$. 
            
            It is now an essential technicality that we work with $\ell$-adic local systems of rank $1$ (for which we shall write $\Shv_{\overline{\Q_{\ell}}}^1(-)$) instead of simply with a generic sheaf theory $\Shv(-)$ as we have until this moment. Via the Hecke operators, one can define the Hecke eigensheaves that we eluded to earlier: as the name suggests, these are nothing but sheaves $\E \in \Shv_{\overline{\Q_{\ell}}}^1(\Bun_{\GL_1}(X))$ that are \say{eigenvectors} of the \textit{global} Hecke operators $\scrH_X$, i.e. for each such $\E$, there exists $\calL \in \Shv_{\overline{\Q_{\ell}}}^1(X)$ such that\footnote{Here, $\boxtimes$ denotes the Deligne tensor product in the $1$-category of finite abelian categories and right-exact functors; in this particular case, we shall be content with $\calL \boxtimes \E \cong \pr_1^*\calL \tensor \pr_2^*\E$.}:
                $$\scrH_X(\E) \cong \calL \boxtimes \E$$
            Now, because $\calL$ is an $\ell$-adic local system of rank $1$ on $X$, its stalk $\calL_x$ at any point $x \in X$ is nothing but $\overline{\Q_{\ell}}$. Consequently, the corresponding local Hecke operators $\scrH_x$ admit $(\overline{\Q_{\ell}})_x \in \Shv(X)$ - the skyscraper sheaf with value $\overline{\Q_{\ell}}$ and supported at $x \in X$ - as an \say{eigenvalue}:
                $$\scrH_x(\E) \cong (\overline{\Q_{\ell}})_x \boxtimes \E$$
            (and thus one may think of $\calL$ as a family of eigenvalues of $\E$ parametrised by points $x \in X$). It is easy to see that Hecke eigensheaves form a full symmetric monoidal subcategory of $\Shv_{\overline{\Q_{\ell}}}^1(\Bun_{\GL_1}(X))$, which we shall denote by $\Eig^1_{\overline{\Q_{\ell}}}(\Bun_{\GL_1}(X))$. 
            
            At this point, we can state and prove the main theorem of this section, which establishes a canonical equivalence between the category of rank-$1$ $\ell$-adic local systems on $X$ and the category of ($\ell$-adic) Hecke eigensheaves of rank $1$ on $\Bun_{\GL_1(X)}$:
                $$\LocSys_{\overline{\Q_{\ell}}}^1(X) \cong \Eig^1_{\overline{\Q_{\ell}}}(\Bun_{\GL_1}(X))$$
            which maps each local system $\calL \in \LocSys_{\overline{\Q_{\ell}}}^1(X)$ to a Hecke eigensheaf $\Aut_{\calL} \in \Eig^1_{\overline{\Q_{\ell}}}(\Bun_{\GL_1}(X))$ with eigenvalue $\calL$. Finally, by putting the Galois Side and Automorphic Side together, one obtains a canonical equivalence of categories as follows:
                $$\Rep_{\overline{\Q_{\ell}}}^1(\pi_1^{\ab}(X_{\fet}))^{\cont} \cong \Eig^1_{\overline{\Q_{\ell}}}(\Bun_{\GL_1}(X))$$
            This is the version of global class field theory that we seek, and it tells us that $1$-dimensional continuous $\ell$-adic Galois representations are the same as automorphic forms associated to $\GL_1$.
	    
        \subsection{References}
    	    \cite{tendler_2015_geometric_class_field_theory} (originally \cite{tendler_2010_geometric_class_field_theory_original}) shall be our main reference, but we will also want to keep the traditional approach to (global) class field theory in mind, for which our references will be \cite[Chapter VI]{neukirch_2010_algebraic_number_theory} and \cite[Chapter VIII]{neukirch_1999_cohomology_of_number_field}. For materials on sheaf theory and algebraic geometry (in particular, Grothendieck's Galois Theory), we shall defer to \cite{stacks}. 
    	    
	\printbibliography

\end{document}