\input{preambles}
\usepackage{soul}

\input{commands}

\begin{document}

	\title{\textbf{MATH518: Weekly timeline}}
	
	\author{Dat Minh Ha}
	\maketitle
	
	\begin{convention}
	    \noindent
	    \begin{itemize}
	        \item For us, $K$ shall be a global field over $\F_q$ with ring of integers $\scrO_K$.
	        \item Throughout, $X$ shall be a smooth projective connected curve over $\Spec \F_q$ with a fixed geometric point $\bar{x}: \Spec \bar{\F}_q \to X$.
	        \item $\pi_1^{\ab}(X_{\et})$ shall denote the abelianisation of the \'etale fundamental group of $X$ (based at $\bar{x}$).
	        \item $\Bun_{\GL_1}$ shall denote the double ad\`elic quotient $K^{\x}\setminus\A_K^{\x}/\scrO_K^{\x} \cong \GL_1(K)\setminus\GL_1(\A_K)/\GL_1(\scrO_K)$.
	    \end{itemize}
	\end{convention}
	
	\begin{enumerate}
	    \item \textbf{(Week 1):} 
	        \begin{itemize}
	            \item \st{Write down timeline.}
	            \item \st{Check if schemes are allowed.}
	            \item \st{Write proposal.}
	        \end{itemize}
	    \item \textbf{(Week 2):} 
	        \begin{itemize}
	            \item Write introduction and outline.
	            \item Write about the \'etale fundamental group and Grothendieck's Galois Theory.
	        \end{itemize}
	    \item \textbf{(Week 3 \& 4):} 
	        \begin{itemize}
	            \item Write about the \'etale fundamental group and Grothendieck's Galois Theory.
	            \item Write about the correspondence between $\ell$-local systems and continuous $\ell$-adic characters of $\pi_1^{\et}(X)^{\ab}$. 
	        \end{itemize}
	    \item \textbf{(Week 5 \& 6):} Write about traces of Frobenii and Grothendieck's Sheaf-Function Correspondence.
	    \item \textbf{(Week 7):}
	        \begin{itemize}
	            \item Write about the generalised Picard group and a rough comparision between traditional and geometric class field theory.
	            \item Discuss Hecke eigensheaves.
	        \end{itemize}
	    \item \textbf{(Week 8 \& 9):} Prove the Geometric Global Reciprocity Theorem.
	    \item \textbf{(Week 10):}
	    \item \textbf{(Week 11):}
	    \item \textbf{(Week 12):}
	    \item \textbf{(Week 13):}
	\end{enumerate}
	
	%\printbibliography

\end{document}